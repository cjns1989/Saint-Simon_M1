\PassOptionsToPackage{unicode=true}{hyperref} % options for packages loaded elsewhere
\PassOptionsToPackage{hyphens}{url}
%
\documentclass[oneside,12pt,french,]{extbook} % cjns1989 - 27112019 - added the oneside option: so that the text jumps left & right when reading on a tablet/ereader
\usepackage{lmodern}
\usepackage{amssymb,amsmath}
\usepackage{ifxetex,ifluatex}
\usepackage{fixltx2e} % provides \textsubscript
\ifnum 0\ifxetex 1\fi\ifluatex 1\fi=0 % if pdftex
  \usepackage[T1]{fontenc}
  \usepackage[utf8]{inputenc}
  \usepackage{textcomp} % provides euro and other symbols
\else % if luatex or xelatex
  \usepackage{unicode-math}
  \defaultfontfeatures{Ligatures=TeX,Scale=MatchLowercase}
%   \setmainfont[]{EBGaramond-Regular}
    \setmainfont[Numbers={OldStyle,Proportional}]{EBGaramond-Regular}      % cjns1989 - 20191129 - old style numbers 
\fi
% use upquote if available, for straight quotes in verbatim environments
\IfFileExists{upquote.sty}{\usepackage{upquote}}{}
% use microtype if available
\IfFileExists{microtype.sty}{%
\usepackage[]{microtype}
\UseMicrotypeSet[protrusion]{basicmath} % disable protrusion for tt fonts
}{}
\usepackage{hyperref}
\hypersetup{
            pdftitle={SAINT-SIMON},
            pdfauthor={Mémoires - Tome I},
            pdfborder={0 0 0},
            breaklinks=true}
\urlstyle{same}  % don't use monospace font for urls
\usepackage[papersize={4.80 in, 6.40  in},left=.5 in,right=.5 in]{geometry}
\setlength{\emergencystretch}{3em}  % prevent overfull lines
\providecommand{\tightlist}{%
  \setlength{\itemsep}{0pt}\setlength{\parskip}{0pt}}
\setcounter{secnumdepth}{0}

% set default figure placement to htbp
\makeatletter
\def\fps@figure{htbp}
\makeatother

\usepackage{ragged2e}
\usepackage{epigraph}
\renewcommand{\textflush}{flushepinormal}

\usepackage{indentfirst}
\usepackage{relsize}

\usepackage{fancyhdr}
\pagestyle{fancy}
\fancyhf{}
\fancyhead[R]{\thepage}
\renewcommand{\headrulewidth}{0pt}
\usepackage{quoting}
\usepackage{ragged2e}

\newlength\mylen
\settowidth\mylen{...................}

\usepackage{stackengine}
\usepackage{graphicx}
\def\asterism{\par\vspace{1em}{\centering\scalebox{.9}{%
  \stackon[-0.6pt]{\bfseries*~*}{\bfseries*}}\par}\vspace{.8em}\par}

\usepackage{titlesec}
\titleformat{\chapter}[display]
  {\normalfont\bfseries\filcenter}{}{0pt}{\Large}
\titleformat{\section}[display]
  {\normalfont\bfseries\filcenter}{}{0pt}{\Large}
\titleformat{\subsection}[display]
  {\normalfont\bfseries\filcenter}{}{0pt}{\Large}

\setcounter{secnumdepth}{1}
\ifnum 0\ifxetex 1\fi\ifluatex 1\fi=0 % if pdftex
  \usepackage[shorthands=off,main=french]{babel}
\else
  % load polyglossia as late as possible as it *could* call bidi if RTL lang (e.g. Hebrew or Arabic)
%   \usepackage{polyglossia}
%   \setmainlanguage[]{french}
%   \usepackage[french]{babel} % cjns1989 - 1.43 version of polyglossia on this system does not allow disabling the autospacing feature
\fi

\title{SAINT-SIMON}
\author{Mémoires - Tome I}
\date{}

\begin{document}
\maketitle

\hypertarget{avis-des-uxe9diteurs.}{%
\chapter{AVIS DES ÉDITEURS.}\label{avis-des-uxe9diteurs.}}

Cette édition des Mémoires de Saint-Simon n'est pas la reproduction de
l'édition de 1829-1830, ni d'aucune des éditions suivantes\,; le texte
en a été établi d'après une collation exacte des manuscrits originaux,
qui appartiennent à M. le duc de Saint-Simon, collation faite en entier
par M. Chéruel, et il n'est presque point de page qui n'ait donné lieu à
quelque rectification. On peut se former une idée de la nature et de
l'importance de ces restitutions ou corrections diverses, d'après
l'examen comparatif qui a été publié, et qu'on pourra rendre plus
complet un jour. Celte édition mérite donc d'être considérée comme la
véritable édition \emph{princeps} des Mémoires de Saint-Simon.

Indépendamment de cet avantage fondamental d'un texte fidèle et tout à
fait exact, cette édition en a d autres accessoires qui la recommandent
au public. Elle contient\,: une Introduction par M. Sainte-Beuve, dans
laquelle il traite du mérite et des caractères essentiels des Mémoires
de Saint-Simon\,; une nouvelle table alphabétique des matières et des
noms propres, si indispensable pour un tel ouvrage\,; un portrait
authentique de l'auteur\,; nous disons authentique, car le portrait
donné dans d'autres éditions n'était pas le sien \,; un fac-simile de
son écriture reproduisant une page de son testament\,; et enfin le
testament même, que nous sommes autorisés à publier, soit en entier,
soit par extraits.

\hypertarget{introduction-de-sainte-beuve.}{%
\chapter{INTRODUCTION DE
SAINTE-BEUVE.}\label{introduction-de-sainte-beuve.}}

On vient tard à parler maintenant de Saint-Simon et de ses Mémoires\,;
il semble qu'on ait tout dit, et bien dit, à ce sujet. Il est
impossible, en effet, qu'il y ait eu depuis plus de vingt-cinq ans une
sorte de concours ouvert pour apprécier ces admirables tableaux
d'histoire et leur auteur, sans que toutes les idées justes, toutes les
louanges méritées et les réserves nécessaires se soient produites\,: il
ne peut être question ici que de rappeler et de fixer avec netteté
quelques-uns des points principaux, acquis désormais et incontestables.

Saint-Simon est le plus grand peintre de son siècle, de ce siècle de
Louis XIV dans son entier épanouissement. Jusqu'à lui on ne se doutait
pas de tout ce que pouvaient fournir d'intérêt, de vie, de drame mouvant
et sans cesse renouvelé, les événements, les scènes de la Cour, les
mariages, les morts, les revirements soudains ou même le train habituel
de chaque jour, les déceptions ou les espérances se reflétant sur des
physionomies innombrables dont pas une ne se ressemble, les flux et
reflux d'ambitions contraires animant plus ou moins visiblement tous ces
personnages, et les groupes ou \emph{pelotons} qu'ils formaient entre
eux dans la grande galerie de Versailles, pêle-mêle apparent, mais qui
désormais, grâce à lui, n'est plus confus, et qui nous livre ses
combinaisons et ses contrastes\,: jusqu'à Saint-Simon on n'avait que des
aperçus et des esquisses légères de tout cela\,; le premier il a donné,
avec l'infinité des détails, une impression vaste des ensembles. Si
quelqu'un a rendu possible de repeupler en idée Versailles et de le
repeupler sans ennui, c'est lui. On ne peut que lui appliquer ce que
Buffon a dit de la terre au printemps\,: « Tout fourmille de vie.\,»
Mais en même temps il produit un singulier effet par rapport aux temps
et aux règnes qu'il n'a pas embrassés\,; au sortir de sa lecture,
lorsqu'on ouvre un livre d'histoire ou même de Mémoires, on court risque
de trouver tout maigre et pâle, et pauvre\,: toute époque qui n'a pas eu
son Saint-Simon paraît d'abord comme déserte et muette, et décolorée\,;
elle a je ne sais quoi d'inhabité\,; on sent et l'on regrette tout ce
qui y manque et tout ce qui ne s'en est point transmis. Très-peu de
parties de notre histoire (si on l'essaye) résistent à cette épreuve, et
échappent à ce contre-coup\,; car les peintres de cette sorte sont
rares, et il n'y a même eu jusqu'ici, à ce degré de verve et d'ampleur,
qu'un Saint-Simon. Ce n'est pas à dire qu'on n'ait pas eu avant lui de
très-belles formes de Mémoires et très-variées\,: il serait le premier à
protester contre une injustice qui diminuerait ses devanciers, lui qui
s'est inspiré d'eux, il le déclare, et de leur exemple, pour y puiser le
goût de l'histoire, de l'histoire animée et vivante. C'étaient des
peintres aussi, au milieu de leurs narrations un peu gênées, mais d'une
gaucherie charmante et naïve, que les Ville-Hardouin et les Joinville.
Les Froissart, les Commynes étaient arrivés déjà à la science et à l'art
avec des grâces restées simples. Quelle génération d'écrivains de plume
et d'épée n'avaient point produite les guerres du
{\textsc{xvi}}\textsuperscript{e} siècle, un Montluc, un Tavannes, un
d'Aubigné, un Brantôme\,! Que de paroles originales et toutes de source,
et quelle diversité d'accents dans les témoignages\,! Sully, au milieu
de ses pesanteurs, a bien des parties réellement belles, d'une solidité
attachante, et que le sourire de Henri IV éclaire. Et la Fronde, quelle
moisson nouvelle de récits de toutes sortes, quelle brusque volée
d'historiens inattendus elle a enfantés parmi ses propres acteurs, en
tête desquels Retz se détache et brille entre tous comme le plus grand
peintre avant Saint-Simon\,! Mais cette génération d'auteurs de
Mémoires, issus de la Fronde, s'arrête à peu près au seuil du règne
véritable de Louis XIV. A partir de là on n'a que des esquisses rapides,
inachevées, qu'ont tracées des plumes élégantes et fines, mais un peu
paresseuses, Choisy, M\textsuperscript{me} de La Fayette, La Fare,
M\textsuperscript{me} de Caylus. Ils mettent en goût, et ils ne tiennent
pas\,; ils commencent, et ils vous laissent en chemin. Or, il n'y a rien
qui fasse moins défaut et qui vous laisse moins, il n'y a rien de moins
paresseux et qui se décourage moins vite que Saint-Simon. Il s'adonne à
l'histoire au sortir de l'enfance comme à un travail, comme à une
mission. Ce n'est pas au courant de la plume qu'il s'amuse a se
ressouvenir de loin et en vieillissant, comme fait Retz\,; méthode
toujours scabreuse, source inévitable de confusions et de méprises. Il
amasse jour par jour, il écrit chaque soir\,; il commence dès dix-neuf
ans sous la tente, et il continue sans relâche à Versailles et partout.
Il s'informe sans cesse comme un Hérodote. Sur les généalogies il en
remontrerait au Père Anselme. Il raisonne du passé comme un
Boulainvilliers. Dans le présent il est à tout, il a vent de toutes les
pistes, et en tient registre incontinent. Toutes les heures qu'il peut
dérober, il les emploie\,; et puis vieux, retiré dans sa terre, il
coordonne cette masse de matériaux, il la met en corps de récit, en un
corps unique et continu, se bornant à la distribuer par paragraphes
distincts, avec des titres en marge \footnote{Saint-Simon, dans le texte
  original, n'établit point de chapitres proprement dits ni aucune
  division\,; il était d'une haleine infatigable\,; on a bien été
  obligé, en imprimant, de faire des chapitres de longueur à peu près
  égale pour soulager l'attention du lecteur\,; mais on a eu soin, dans
  la présente édition, de ne composer les sommaires qu'avec les termes
  mis en marge par Saint-Simon, et on a reproduit, autant qu'on l'a pu,
  ces mêmes termes de la marge au haut des pages dans le titre courant.}\,;
et ce long texte immense, il le recopie \emph{tout de sa main} avec une
netteté, une exactitude minutieuse, qualités authentiques qu'on n'a pas
assez remarquées, sans quoi on eût plus religieusement respecté son
ordre et sa marche, son style et sa phrase, qui peut bien être négligée
et redondante, mais où rien (je parle des Mémoires et non des notes)
n'est jeté au hasard.

Comment cette vocation historique si prononcée se forma-t-elle, et se
rencontra-t-elle ainsi toute née au sein de la Cour et dans un si jeune
âge\,? Et d'où sortait donc ce mousquetaire de dix-neuf ans, si résolu
dès le premier jour à transmettre les choses de son temps dans toutes
leurs complications et leurs circonstances\,?

Son père, sans un tel fils, serait resté un de ces favoris comblés, mais
obscurs, que l'histoire nomme tout au plus en passant, mais dont elle ne
s'occupe pas. Jeune page, il avait su plaire à Louis XIII par quelques
attentions et de l'adresse à la chasse, en lui présentant commodément
son cheval de rechange ou en rendant le cor après s'en être proprement
servi. Sans doute il avait bonne mine\,; il avait certainement de la
discrétion et de l'honneur. À la manière dont Saint-Simon nous parle de
son père, et même si l'on en rabat un peu, on voit en celui-ci un homme
de qualité, fidèle, assez désintéressé, reconnaissant, et, en tout,
d'une étoffe morale peu commune à la Cour. Son attitude envers Richelieu
est digne en même temps que sensée\,: il n'est ni hostile, ni servile.
On découvre même dans le père de Saint-Simon une qualité dont ne sera
pas privé son fils, une sorte d'humeur qui, au besoin, devient de
l'aigreur\,; c'est pour s'être livré à un mouvement de cette nature
qu'il tomba dans une demi-disgrâce à l'âge de trente et un ans et quitta
la Cour pour se retirer en son gouvernement de Blaye où il demeura
jusqu'à la mort du cardinal. Si j'avais à définir en deux mots le père
de Saint-Simon, je dirais que c'était un favori, mais que ce n'était pas
un courtisan\,: car il avait de l'honneur et de l'humeur.

C'est de ce père déjà vieux et remarié en secondes noces avec une
personne jeune, mais non plus de la première jeunesse, que naquit
Saint-Simon en janvier 1675. On a cité comme une singularité et un
prodige, dans un livre imprimé du vivant même du père \footnote{\emph{Tableau
  de Vamour considéré dans l'état du mariage} (Amsterdam, 1687), page
  134.}, qu'il ait eu cet enfant à l'âge de soixante et douze ans\,; il
n'en avait en réalité que soixante-huit. Il lui transmit ses propres
qualités très-marquées avec je ne sais quoi de fixe et d'opiniâtre\,: la
probité, la fierté, la hauteur de cœur, et des instincts de race forte
sous une brève stature. Le jeune Saint-Simon fut donc élevé auprès d'une
mère, personne de mérite, et d'un père qui aimait à se souvenir du passé
et à raconter mainte anecdote de la vieille Cour\,: de bonne heure il
dut lui sembler qu'il n'y avait rien de plus beau que de se ressouvenir.
Sa vocation pour l'histoire se prononça dès l'enfance, en même temps
qu'il restait indifférent et froid pour les belles-lettres proprement
dites. Il lisait sans doute aussi avec Vidée d'imiter les grands
exemples qu'il voyait retracés, et de devenir quelque chose\,; mais au
fond son plus cher désir et son ambition étaient plutôt \emph{d'être de
quelque chose} afin de savoir le mieux qu'il pourrait les affaires de
son temps et de les écrire. Cette vocation d'écrivain, qui se dégage et
s'affiche pour nous si manifestement aujourd'hui, était cependant
d'abord secrète et comme masquée et affublée de toutes les prétentions
de l'homme de cour, du grand seigneur, du duc et pair, et des autres
ambitions accessoires qui convenaient alors à un personnage de son rang.

Saint-Simon, en entrant dans le monde à l'âge de dix-neuf ans, dénote
bien ses instincts et ses goûts. Dès le lendemain de la bataille de
Nerwinde (juillet 1693) à laquelle il prend part comme capitaine dans le
Royal-Roussillon, il en fait un bulletin détaillé pour sa mère et
quelques amis. Ce récit a de la netteté, de la fermeté\,; le caractère
en est simple\,; on y sent l'amour du vrai. Le style n'a rien de cette
fougue et de ces irrégularités qu'il aura quelquefois, mais qu'il n'a
pas toujours et nécessairement chez Saint-Simon. A force de le vouloir
définir dans toutes ses diversités et ses exubérances, il ne faut pas
non plus se faire de ce style un monstre. Très-souvent il n'est que
l'expression la plus directe et la plus vive, telle qu'elle échappe à un
esprit plein de son objet.

L'année suivante (1694), dans les loisirs d'un camp en Allemagne, il
commence décidément ses Mémoires qu'il mettra soixante ans entiers à
poursuivre et à parachever. Il y fut excité « par le plaisir qu'il prit,
dit-il, à la lecture de ceux du maréchal de Bassompierre.\,»
Bassompierre avait dit pourtant un mot des plus injurieux pour le père
de Saint-Simon\,: cela n'empêche pas le fils de trouver ses Mémoires
très-curieux, « quoique dégoûtants par leur vanité.\,»

Le jeune Saint-Simon est vertueux\,; il a des mœurs, de la religion\,;
il a surtout d'instinct le goût des honnêtes gens. Ce goût se déclare
d'abord d'une manière singulière et presque bizarre par l'élan qui le
porte tout droit vers le duc de Beauvilliers, le plus honnête homme de
la Cour, pour lui aller demander une de ses filles en mariage, ou
l'aînée, ou la cadette, il n'en a vu aucune, peu lui importe laquelle\,;
peu lui importe la dot\,; ce qu'il veut épouser, c'est la famille\,;
c'est le duc et la duchesse de Beauvilliers dont il est épris. Cette
poursuite de mariage qu'il expose avec une vivacité si expressive a pour
effet, même en échouant, de le lier étroitement, avec le duc de
Beauvilliers et avec ce côté probe et sérieux de la Cour. C'est par là
qu'il se rattachera bientôt aux vertueuses espérances que donnera le duc
de Bourgogne.

Une liaison fort différente et qui semble jurer avec celle-ci, mais qui
datait de l'enfance, c'est la familiarité et l'amitié de Saint-Simon
avec le duc d'Orléans, le futur Régent. Là encore toutefois la marque de
l'honnêteté se fait sentir\,; c'est par les bons côtés du Prince, par
ses parties louables, intègres et tant calomniées que Saint-Simon lui
demeurera attaché inviolablement\,; c'est à cette noble moitié de sa
nature qu'il fera énergiquement appel dans les situations critiques
déplorables où il le verra tombé\,; et, dans ce perpétuel contact avec
le plus généreux et le plus spirituel des débauchés, il se préservera de
toute souillure.

Avec le goût des honnêtes gens, il a l'antipathie non moins prompte et
non moins instinctive contre les coquins, les hypocrites, les âmes
basses et mercenaires, les courtisans plats et uniquement intéressés. Il
les reconnaît, il les devine à distance, il les dénonce et les
démasque\,; il semble, à la manière dont il les tire au jour et les
dévisage, y prendre un plaisir amer et s'y acharner. On se rappelle, dès
les premiers chapitres des Mémoires, ce portrait presque effrayant du
magistrat pharisien, du faux Caton, de ce premier président de Harlay,
dont sous des dehors austères il nous fait le type achevé du profond
hypocrite.

Mais il avait à s'en plaindre, dira-t-on, et ici, comme en bien des cas,
en peignant les hommes il obéit à des préventions haineuses et à une
humeur méchante\,: je vais tout d'abord à l'objection. Selon moi, et
après une étude dix fois refaite de Saint-Simon, je me suis formé de lui
cette idée\,: il est doué par nature d'un sens particulier et presque
excessif d'observation, de sagacité, de vue intérieure, qui perce et
sonde les hommes, et démêle les intérêts et les intentions sur les
visages\,: il offre en lui un exemple tout à fait merveilleux et
phénoménal de cette disposition innée. Mais un tel don, une telle
faculté est périlleuse si l'on s'y abandonne, et elle est sujette à
outrer sa poursuite et à passer le but. Les tentations ne sont jamais
pour les hommes que dans le sens de leurs passions\,: on n'est pas tenté
de ce qu'on n'aime pas. Dès le début, Saint-Simon fils d'un père
antique, et, sous sa jeune mine, un peu antique lui-même, n'a pas de
goût vif pour les femmes, pour le jeu, le vin et les autres plaisirs\,:
mais il est glorieux\,; il tient au vieux culte\,; il se fait un idéal
de vertu patriotique qu'il combine avec son orgueil personnel et ses
préjugés de rang. Et avec cela il est artiste, et il l'est doublement\,:
il a un coup d'œil et un \emph{flair}\footnote{Je n'emploie le mot que
  parce que lui-même me le fournit. Il dit quelque part, à l'occasion
  des joies secrètes et des mille ambitions flatteuses mises en
  mouvement par une mort de prince\,: « Tout cela, et tout à la fois, se
  sentait \emph{comme au nez}.\,»} qui, dans cette foule dorée et cette
cohue apparente de Versailles, vont trouver à se satisfaire amplement et
à se repaître\,; et puis, écrivain en secret, écrivain avec délices et
dans le mystère, le soir, à huis clos, le verrou tiré, il va jeter sur
le papier avec feu et flamme ce qu'il a observé tout le jour, ce qu'il a
senti sur ces hommes qu'il a bien vus, qu'il a trop vus, mais qu'il a
pris sur un point qui souvent le touchait et l'intéressait. Il y a là
des chances d'erreur et d'excès jusque dans le vrai. Il est périlleux,
même pour un honnête homme, s'il est passionné, de sentir qu'il écrit
sans contrôle, et qu'il peint son monde sans confrontation. Je ne parle
en ce moment que de ce qu'il a observé lui-même et directement\,: car,
pour ce qu'il n'a su que par ouï-dire et ce qu'il a recueilli par
conversation, il y aura d'autres chances d'erreur encore qui s'y
mêleront.

Quoique Saint-Simon ne paraisse pas avoir été homme à mettre de la
critique proprement dite dans l'emploi et le résultat de ses recherches,
et qu'il ne semble avoir guère fait que verser sur sa première
observation toute chaude et toute vive une expression ardente et à
l'avenant, son soin ne portant ensuite que sur la manière de coordonner
tout cela, il n'est pas sans s'être adressé des objections graves sur la
tentation à laquelle il était exposé et dont l'avertissait sans doute le
singulier plaisir qu'il trouvait à y céder. Religieux par principes et
chrétien sincère, il se fit des scrupules de conscience, ou du moins il
tint à les empêcher de naître et à se mettre en règle contre les remords
et les faiblesses qui pourraient un jour lui venir à ses derniers
instants. S'il lui avait fallu jeter au feu ses Mémoires, croyant avoir
fait un long péché, quel dommage, quel arrachement de cœur\,! Il songea
assez naïvement à prévenir ce danger. Le Discours préliminaire qu'il a
mis en tête nous témoigne de sa préoccupation de chrétien, qui cherche à
se démontrer qu'on a droit historiquement de tout dire sur le compte du
prochain, et qui voudrait bien concilier la charité avec la médisance.
Une lettre écrite à l'abbé de Rancé \footnote{Nous reproduisons cette
  lettre en tête des Mémoires\,: elle en est la première préface.}, et
par laquelle il le consultait presque au début sur la mesure à observer
dans la rédaction de ses Mémoires, atteste encore mieux cette pensée de
prévoyance\,; il semble s'être fait donner par l'austère abbé une
absolution plénière, une fois pour toutes. Saint-Simon, dans son
apologie, admet ou suppose toujours deux choses\,: c'est, d'une part,
qu'il ne dit que la vérité, et, de l'autre, qu'il n'est pas impartial,
qu'il ne se pique pas de l'être, et, qu'en laissant la louange ou le
blâme \emph{aller de source} à l'égard de ceux pour qui il est
diversement affecté, il obéit à ses inclinations et à sa façon
impétueuse de sentir\,: et, avec cela, il se flatte de tenir en main la
balance. Dans le récit de ce premier procès au nom de la Duché-Pairie
contre M. de Luxembourg, il y a un moment où l'avocat de celui-ci ayant
osé révoquer en doute la loyauté royaliste des adversaires, Saint-Simon,
qui assistait à l'audience, assis dans une lanterne ou tribune entre les
ducs de La Rochefoucauld et d'Estrées, s'élance au dehors, criant à
l'imposture et demandant justice de ce coquin\,: « M. de La
Rochefoucauld, dit-il, me retint à mi-corps et me fit taire. Je
m'enfonçai de dépit plus encore contre lui que contre l'avocat. Mon
mouvement avait excité une rumeur.\,» Or, quand on est sujet à ces
mouvements-là, non-seulement à l'audience et dans une occasion
extraordinaire, mais encore dans l'habitude de la vie et même en
écrivant, il y a chance non pour qu'on se trompe peut-être sur
l'intention mauvaise de l'adversaire, mais au moins pour qu'on
outre-passe quelquefois le ton et qu'on sorte de la mesure. On a de ces
élans où l'on a besoin d'être retenu à \emph{mi-corps}. J'indique la
précaution à prendre en lisant Saint-Simon\,; il peut bien souvent y
avoir quelque réduction à faire dans le relief et dans les couleurs.

On a fort cherché depuis quelque temps à relever des erreurs de fait
dans les Mémoires de Saint-Simon, et l'on n'a pas eu de peine à en
rassembler un certain nombre. Il fait juger et condamner Fargues, un
ancien frondeur, par le premier président de Lamoignon, et Fargues fut
jugé par l'intendant Machault. Il dit de M\textsuperscript{lle} de
Beauvais, mariée au comte de Soissons, qu'elle était fille naturelle, et
l'on a retrouvé et l'on produit le contrat de mariage des parents. Il
fait de De Saumery un argus impitoyable et un espion farouche auprès du
duc de Bourgogne, et l'on sait, par une lettre de ce jeune prince à
Fénelon, que c'était un homme dévoué et sûr. Quelques-unes de ces
rectifications auront place dans la présente édition, et seront
indiquées en leur lieu. Dans le domaine de la littérature, j'ai moi-même
à signaler une inexactitude et une méprise. Saint-Simon impute à Racine,
en présence de Louis XIV et de M\textsuperscript{me} de Maintenon, une
distraction maladroite qui lui aurait fait mal parler de Scarron. Au
contraire, c'est Despréaux qui eut plus d'une fois cette distraction
plaisante, dans laquelle le critique s'échappait, tandis que Racine,
meilleur courtisan, lui faisait tous les signes du monde sans qu'il les
comprit. Tranchons sur cela. La question de la vérité des Mémoires de
Saint-Simon n'est pas et ne saurait être circonscrite dans le cercle des
observations de ce genre, même quand les erreurs se trouveraient cent
fois plus nombreuses. Qu'on veuille bien se rendre compte de la manière
dont les Mémoires, tels que les siens, ont été et sont nécessairement
composés. Il y a entre les façons infinies d'écrire l'histoire deux
divisions principales qui tiennent à la nature des sources auxquelles on
puise. Il y a une sorte d'histoire qui se fonde sur les pièces mêmes et
les instruments d'État, les papiers diplomatiques, les correspondances
des ambassadeurs, les rapports militaires, les documents originaux de
toute espèce. Nous avons un récent et un excellent exemple de cette
méthode de composition historique dans l'ouvrage de M. Thiers, qui se
pourrait proprement intituler\,: \emph{Histoire administrative et
militaire du Consulat et de l'Empire}. Et puis, il y a une histoire
d'une tout autre physionomie, l'histoire \emph{morale} contemporaine
écrite par des acteurs et des témoins. On vit dans une époque, à la Cour
si c'est à une époque de cour\,; on y passe sa vie à regarder, à
écouter, et, quand on est Saint-Simon, à écouter et à regarder avec une
curiosité, une avidité sans pareille, à tout boire et dévorer des
oreilles et des yeux. On entend dire beaucoup de choses\,; on s'adresse
le mieux qu'on peut pour en savoir encore davantage\,; si l'on veut
remonter en arrière, on consulte les vieillards, les disgraciés, les
solitaires en retraite, les subalternes aussi, les anciens valets de
chambre. Il est bien difficile que dans ce qu'on ne voit point soi-même
il ne se mêle un peu de crédulité, quand elle est dans le sens de nos
inclinations et aussi de notre talent à exprimer les choses. On ne fait
souvent que répéter ce qu'on a entendu\,; on ne peut aller vérifier chez
les notaires. Dans ce qu'on voit par soi-même, et avec les hommes à qui
l'on a affaire en face et qu'on juge, oh\,! ici l'on va plus sûrement\,;
si l'on a le don d'observation et la faculté dont j'ai parlé, on va
loin, on pénètre\,; et si à ce premier don d'observer se joint un talent
pour le moins égal d'exprimer et de peindre, on fait des tableaux, des
tableaux vivants et par conséquent vrais, qui donnent la sensation,
l'illusion de la chose même, qui remettent en présence d'une nature
humaine et d'une société en action qu'on croyait évanouie. Est-ce à dire
qu'un autre observateur et un autre peintre placé à côté du premier,
mais à un point de vue différent, ne présenterait pas une autre peinture
qui aurait d'autres couleurs, et peut-être aussi quelques autres traits
de dessin\,? Non, sans doute\,; autant de peintres, autant de
tableaux\,; autant d'imaginations, autant de miroirs\,; mais l'essentiel
est qu'au moins il y ait par époque un de ces grands peintres, un de ces
immenses miroirs réfléchissants\,; car, lui absent, il n'y aura plus de
tableaux du tout\,; la vie de cette époque, avec le sentiment de la
réalité, aura disparu, et vous pourrez ensuite faire et composer à
loisir toutes vos belles narrations avec vos pièces dites positives et
même avec vos tableaux d'histoire, arrangés après coup et
symétriquement, et peignés comme on en voit, ces histoires, si vraies
qu'elles soient quant aux résultats politiques, seront artificielles, et
on le sentira\,; et vous aurez beau faire, vous ne ferez pas qu'on ait
vécu dans ce temps que vous racontez.

Avec Saint-Simon on a vécu en plein siècle de Louis XIV\,; là est sa
grande vérité. Est-ce que par lui nous ne connaissons pas (mais je dis
connaître comme si nous les avions vus), et dans les traits même de leur
physionomie et dans les moindres nuances, tous ces personnages et les
plus marquants et les secondaires, et ceux qui ne font que passer et
figurer\,? Nous en savions les noms, qui n'avaient pour nous qu'une
signification bien vague\,: les personnes, aujourd'hui, nous sont
familières et présentes. Je prends au hasard les premiers que je
rencontre\,: Louville, ce gentilhomme attaché au duc d'Anjou, au futur
roi d'Espagne, et qui aura bientôt un rôle politique, • Saint-Simon se
sert de lui tout d'abord pour faire sa demande d'une entrevue à M. de
Beauvilliers\,; il raconte ce qu'est Louville, et il ajoute tout
courant\,: « Louville était d'ailleurs homme d'infiniment d'esprit, et
qui, avec une imagination qui le rendait toujours neuf et de la plus
excellente compagnie, avait toute la lumière et le sens des grandes
affaires, et des plus solides et des meilleurs conseils.\,» Louville
reviendra mainte fois dans les Mémoires\,; lui-même il a laissé les
siens\,: vous pouvez les lire si vous en avez le temps\,; mais, en
attendant, on a sur l'homme et sur sa nuance distinctive et neuve les
choses dites, les choses essentielles et fines, et comme personne autre
n'aurait su nous les dire. • M. de Luxembourg a été un adversaire de
Saint-Simon\,; il a été sa partie devant le Parlement, après avoir été
son général à l'armée\,; il a été l'objet de sa première grande colère,
de sa première levée de boucliers comme duc et pair. Est-ce à dire que
son portrait par Saint-Simon en sera moins vrai, de cette vérité qui
saisit, et qui, d'ailleurs, se rapporte bien à ce que disent les
contemporains, mais en serrant l'homme de plus près qu'ils n'ont fait\,?

« \ldots. À soixante-sept ans, il s'en croyait vingt-cinq, et vivait
comme un homme qui n'en a pas davantage. Au défaut de bonnes fortunes,
dont son âge et sa figure l'excluoient, il y suppléait par de l'argent,
et l'intimité de son fils et de lui, de M. le prince de Conti et
d'Albergotti, portait presque toute sur des mœurs communes et des
parties secrètes qu'ils faisaient ensemble avec des filles. Tout le faix
des marches et des ordres de subsistances portait toutes les campagnes
sur Puységur, qui même dégrossissait les projets. Rien de plus juste que
le coup d'œil de M. de Luxembourg, rien de plus brillant, de plus avisé,
de plus prévoyant que lui devant les ennemis, ou un jour de bataille,
avec une audace, une flatterie, et en même temps un sang-froid qui lui
laissait tout voir et tout prévoir au milieu du plus grand feu, et du
danger et du succès les plus imminent, et c'était là où il était grand.
Pour le reste, la paresse même\,: peu de promenades sans grande
nécessité\,; du jeu, de la conversation avec ses familiers, et tous les
soirs un souper avec un très-petit nombre, presque toujours le même, et
si on était voisin de quelque ville, on avait soin que le sexe y fût
agréablement mêlé. Alors il était inaccessible à tout, et s'il arrivait
quelque chose de pressé, c'était à Puységur à y donner ordre. Telle
était à l'armée la vie de ce grand général, et telle encore à Paris, où
la cour et le grand monde occupaient ses journées, et les soirs ses
plaisirs. À la fin, l'âge, le tempérament, la conformation le
trahirent\ldots.\,»

Est-ce que vous croyez que M. de Luxembourg ainsi présenté dans un
brillant de héros et dans ses vices est calomnié\,? Bien moins connu,
bien moins en vue, vous avez dès les premières pages le vieux Montal, «
ce grand vieillard de quatre-vingts ans qui avait perdu un œil à la
guerre, où il avait été couvert de coups,\,» et qui se vit injustement
mis de côté dans une promotion nombreuse de maréchaux\,: « Tout cria
pour lui hors lui-même\,; sa modestie et sa sagesse le firent
admirer.\,» Il continua de servir avec dévouement et de commander avec
honneur jusqu'à sa mort. Ce Montal, tel qu'un Montluc innocent et pur,
se dresse devant nous en pied, de toute sa hauteur, et ne s'oublie plus.
Saint-Simon ne peut rencontrer ainsi une figure qui le mérite sans s'en
emparer et la faire revivre. Et ceux même qui sembleraient le mériter
moins et qui seraient des visages effacés chez d'autres, il leur rend
cette originalité, cette empreinte individuelle qui, à certain degré,
est dans chaque être. Rien qu'à les regarder, il leur ôte de leur
insipidité, il a surpris leur étincelle. Prétendre compter chez lui ces
sortes de portraits, ce serait compter les sables de la mer, avec cette
différence qu'ici les grains de sable ne se ressemblent pas. On ne peut
porter l'œil sur une page des Mémoires sans qu'il en sorte une
physionomie. Dès ce premier volume on a (et je parle des moindres)
Crécy, Montgommery, et Cavoye, et Lassay, et Chandenier\,; qui donc les
distinguerait sans lui\,? et ce Dangeau si comique à le bien voir, qui a
reconquis notre estime par ses humbles services de gazetier auprès de la
postérité, mais qui n'en reste pas moins à jamais orné et chamarré,
comme d'un ordre de plus, de la description si complète et si
divertissante qu'a faite de lui Saint-Simon. Que s'il arrive aux plus
grandes figures, son pinceau s'y égale aussitôt et s'y proportionne. Ce
Fénelon qu'il ne connaissait que de vue, mais qu'il avait tant observé à
travers les ducs de Beauvilliers et de Chevreuse, quel incomparable
portrait il en a donné\,! voyez le en regard de celui de Godet, l'évêque
de Chartres, si creusé dans un autre sens. S'il y a du trop dans l'un et
dans l'autre, que ce trop-là aide à penser, à réfléchir, et comme, après
même l'avoir réduit, on en connaît mieux les personnages que si l'on
était resté dans les lignes d'en deça et à la superficie\,? Et quand il
aura à peindre des femmes, il a de ces grâces légères, de ces images et
de ces suavités primitives, presque homériques (voir le portrait de la
duchesse de Bourgogne), que les peintres de femmes proprement dits, les
malicieux et coquets Hamilton n'égalent pas. Mais avec Saint-Simon on ne
peut se mettre à citer et à vouloir choisir\,: ce n'est pas un livre que
le sien, c'est tout un monde. Que si on le veut absolument, on peut
retrancher et supprimer en idée quelques-uns de ces portraits qui sont
suspects, et où il entre visiblement de la haine\,; le personnage du duc
du Maine est dans ce cas. En général, toutefois, le talent de
Saint-Simon est plus impartial que sa volonté, et s'il y a une grande
qualité dans celui qu'il hait, il ne peut s'empêcher de la produire. Et
puis, oserai-je dire toute ma pensée et ma conviction\,? ce n'est pas
une bonne marque à mes yeux pour un homme que d'être très-maltraité et
défiguré par Saint-Simon\,: il ne s'indigne jamais si fort que contre
ceux à qui il a manqué de certaines fibres. Ce qu'il méprise avant tout,
ce sont les gens « en qui le servile surnage toujours, \,» ou ceux
encore à qui la duplicité est un instrument familier. Quant aux autres,
il a beau être sévère et dur, il a des compensations. Mais je ne parle
que de portraits et il y a bien autre chose chez lui, il y a le drame et
la scène, les groupes et les entrelacements sans fin des personnages, il
y a l'action\,; et c'est ainsi qu'il est arrivé à ces grandes fresques
historiques parmi lesquelles il est impossible de ne pas signaler les
deux plus capitales, celle de la mort de Monseigneur et du
bouleversement d'intérêts et d'espérances qui s'opère à vue d'œil cette
nuit-là dans tout ce peuple de princes et de courtisans, et cette autre
scène non moins merveilleuse du lit de justice au Parlement sous la
Régence pour la dégradation des bâtards, le plus beau jour de la vie de
Saint-Simon et où il savoure à longs traits sa vengeance. Mais, dans ce
dernier cas, le peintre est trop intéressé et devient comme féroce\,: la
mesure de l'art est dépassée. Quoi qu'il en soit des remarques à faire,
ce n'est certes pas exagérer que de dire que Saint-Simon est le Rubens
du commencement du {\textsc{xviii}}\textsuperscript{e} siècle.

La vie de Saint-Simon n'existe guère pour nous en dehors de ses
Mémoires\,; il y a raconté, et sans trop les amplifier (excepté pour les
disputes et procès nobiliaires), les événements qui le concernent. À
défaut de la fille du duc de Beauvilliers, il se maria à la fille aînée
du maréchal de Lorges\,; la \emph{bonté} et la \emph{vérité} du
maréchal, de ce neveu et de cet élève favori de Turenne, l'attiraient,
et l'air aimable et noble de sa fille, je ne sais quoi de majestueux,
tempéré de douceur naturelle, le fixa. Il lui dut un bonheur domestique
constant et vécut avec elle dans une parfaite fidélité. Il n'avait que
vingt ans alors, était duc et pair de France, gouverneur de Blaye,
gouverneur et grand bailli de Senlis, et commandait un régiment de
cavalerie\,: « Il sait, • disait le \emph{Mercure galant} dans une
longue notice sur ce mariage et sur ses pompes, envoyée probablement par
lui-même, • il sait tout ce qu'un homme de qualité doit savoir, et
Madame sa mère, dont le mérite est connu, l'a fait particulièrement
instruire des devoirs d'un bon chrétien. \,» • « J'oubliais à vous dire,
ajoutait le même gazetier en finissant, que la mariée est blonde et
d'une taille des plus belles\,; qu'elle a le teint d'une finesse
extraordinaire et d'une blancheur à éblouir\,; les yeux doux, assez
grands et bien fendus, le nez un peu long et qui relève sa physionomie,
une bouche gracieuse, les joues pleines, le visage ovale, et une gorge
qui ne peut être ni mieux taillée ni plus belle. Tout cela ensemble
forme un air modeste et de grandeur qui imprime du respect\,: elle a
d'ailleurs toute la beauté d'âme qu'une personne de qualité doit avoir,
et elle ira de pair en mérite avec M. le duc de Saint-Simon son époux,
l'un des plus sages et des plus accomplis seigneurs de la Cour.\,»
Saint-Simon a parlé en bien des endroits de sa femme, et toujours avec
un sentiment touchant de respect et d'affection, l'opposant à tant
d'autres femmes ou inutiles, ou ambitieuses quand elles sont capables,
et la louant en termes charmants de « la perfection d'un sens exquis et
juste en tout, mais doux et tranquille, et qui, loin de faire apercevoir
ce qu'il vaut, semble toujours l'ignorer soi-même, avec une uniformité
de toute la vie de modestie, d'agrément et de vertu.\,»

On a de Saint-Simon et de sa femme vers cette époque de leur mariage,
deux beaux portraits par Rigaud, que possède M. le duc actuel de
Saint-Simon. Le portrait gravé de Saint-Simon est joint à la présente
édition et remplace avantageusement l'ancien portrait qu'on voyait dans
la première, lequel n'était pas bon, et avait de plus l'inconvénient de
n'être réellement pas le sien, mais celui de son père. Il s'est fait
quelquefois de ces méprises.

Quoiqu'il faille prendre garde de trop raisonner sur les portraits, et
que l'air de jeunesse du nouvel époux jure un peu avec l'idée que
donnent ses Mémoires, on remarque pourtant que sa figure et sa
physionomie sont assez bien celles de son œuvre\,; la figure est fine\,;
l'œil assez doux peut se courroucer et devenir terrible. Il a le nez un
peu en l'air et assez mutin, la bouche maligne et d'où le trait n'a pas
de peine à partir. Mais l'idée de force, qui est si essentielle au
talent de Saint-Simon, reste absente, et elle est sans doute dissimulée
par la jeunesse.

Saint-Simon avait servi à la guerre convenablement et avec application
pendant plusieurs campagnes. Après la paix de Ryswick, le régiment de
cavalerie dont il était mestre de camp fut réformé, et il se trouva sans
commandement et mis à la suite. Lorsque la guerre de la Succession
commença (1702), voyant de nouvelles promotions se faire dans lesquelles
figuraient de moins anciens que lui et y étant oublié, il songea à se
retirer du service, consulta plusieurs amis, trois maréchaux et trois
hommes de Cour, et sur leur avis unanime « qu'un duc et pair de sa
naissance, établi d'ailleurs comme il était et ayant femme et enfants,
n'allait point servir comme un \emph{haut-le-pied} dans les armées et y
voir tant de gens si différents de ce qu'il était, et, qui pis est, de
ce qu'il y avait été, tous avec des emplois et des régiments,\,» il
donna, comme nous dirions, sa démission\,; il écrivit au roi une lettre
respectueuse et courte, dans laquelle, sans alléguer d'autre raison que
celle de sa santé, il lui marquait le déplaisir qu'il avait de quitter
son service. « Eh bien\,! monsieur, voilà encore un homme qui nous
quitte,\,» dit le roi au secrétaire d'État de la guerre Chamillart, en
lui répétant les termes de la lettre\,; et il ne le pardonna point de
plusieurs années à Saint-Simon, qui put bien avoir encore quelquefois
l'honneur d'être nommé pour le bougeoir au petit coucher, mais qui fut
rayé \emph{in petto} de tout acheminement à une faveur réelle, si jamais
il avait été en passe d'en obtenir. Il avait vingt-sept ans.

Un ou deux ans après, à l'occasion d'une quête que Saint-Simon ne voulut
point laisser faire à la duchesse sa femme, ni aux autres duchesses,
comme étant préjudiciable au rang des ducs vis-à-vis des princes, le roi
se fâcha, et un orage gronda sur l'opiniâtre et le récalcitrant\,: «
C'est une chose étrange, dit à ce propos Louis XIV, que depuis qu'il a
quitté le service, M. de Saint-Simon ne songe qu'à étudier les rangs et
à faire des procès à tout le monde.\,» Saint-Simon averti se décida à
demander au roi une audience particulière dans son cabinet\,; il
l'obtint, il s'expliqua, il crut avoir au moins en partie ramené le roi
sur son compte, et les minutieux détails qu'il nous donne sur cette
scène, et qui en font toucher au doigt chaque circonstance, montrent
assez que pour lui l'inconvénient d'avoir été dans le cas de demander
l'audience est bien compensé par le curieux plaisir d'y avoir observé de
plus près le maître, et par cet autre plaisir inséparable du premier, de
tout peindre et raconter.

Peu après, à l'occasion de l'ambassade de Rome, qu'il fut près d'avoir
un peu à son corps défendant et qui manqua, M\textsuperscript{me} de
Maintenon exprimait sur Saint-Simon un avis qui ne démentait point son
bon sens\,: elle le disait « glorieux, frondeur et plein de vues.\,»
Plein de \emph{vues}, c'est-à-dire de projets systématiques et plus ou
moins aventurés. Cette opinion, dans laquelle M\textsuperscript{me} de
Maintenon resta invariable, atteste l'antipathie des natures et n'était
pas propre à donner au roi une autre idée que celle qu'il avait déjà sur
ce courtisan médiocrement docile. Plus on accordait à un homme de son
âge du sérieux, de la lecture et de l'instruction en lui attribuant ce
caractère indépendant, plus on le rendait impossible dans le cadre
d'alors et inconciliable. Les envieux et ceux qui lui voulaient nuire
trouvaient leur compte en le louant\,: on le faisait passer, par sa
liberté de parole et sa hauteur, pour un homme d'esprit plus à craindre
qu'à employer et dangereux. Il avait beau se surveiller, il avait des
silences expressifs et éloquents, ou des énergies d'expression qui
emportaient la pièce\,; « il lui échappait d'abondance de cœur des
raisonnements et des blâmes.\,» Quand on le lit aujourd'hui, on n'a pas
de peine à se figurer ce qu'il devait paraître alors. Une telle nature
de \emph{grand écrivain posthume}\footnote{Expression de M. Villemaiu.}
ne laissait pas de transpirer de son vivant\,; elle s'échappait par
éclats\,; il avait ses détentes, et l'on conçoit très-bien que Louis
XIV, à qui il se plaignait un jour des mauvais propos de ses ennemis,
lui ait répondu\,: « Mais aussi, monsieur, c'est que vous parlez et que
vous blâmez, voilà ce qui fait qu'on parle contre vous.\,» Et un autre
jour\,: « Mais il faut tenir votre langue.\,»

Cependant, le secret auteur des Mémoires gagnait à ces contre-temps de
la fortune. Saint-Simon, libre et vacant, et, sauf la faveur avec le roi
perdue sans remède, nageant d'ailleurs en pleine Cour, sur bien des
récifs cachés, mais sans rien d'une disgrâce apparente, intimement lié
avec plusieurs des ministres d'État, était plus que personne en position
et à l'affût pour tout savoir et pour tout écrire. Sa liaison
particulière avec les ducs de Chevreuse et de Beauvilliers, avec
celui-ci surtout, « sans qui il ne faisait rien,\,» ne le confinait pas
de ce côté, et il l'a dit très-joliment en faisant le portrait de l'abbé
de Polignac, l'aimable et brillant séducteur dont ils furent les
dupes\,: « Malheureusement pour moi, la charité ne me tenait pas
renfermé dans une bouteille comme les deux ducs.\,» Il rayonnait dans
tous les sens, avait des ouvertures sur les cabales les plus opposées,
et par amis, femmes jeunes ou vieilles, ou même valets, était tenu au
courant, jour par jour, de tout ce qui se passait en plus d'une sphère.
Tous ces bruits, toutes ces intelligences qui circulent rapidement dans
les Cours et s'y dispersent, ne tombaient point chez lui en pure
perte\,; il en faisait amas pour nous et réservoir. Dans un précieux
chapitre où il nous expose son procédé de conduite et son système
d'information\,: « Je me suis donc trouvé instruit journellement,
dit-il, de toutes choses par des canaux purs, directs et certains, et de
toutes choses grandes et petites. Ma curiosité, indépendamment d'autres
raisons, y trouvait fort son compte\,; et il faut avouer que, personnage
ou nul, ce n'est que de cette sorte de nourriture que l'on vit dans les
Cours, sans laquelle on n'y fait que languir.\,»

L'ambitieux pourtant ne laissait pas sa part d'espérances\,: il était
jeune\,; le roi était vieux\,; Louis XIV vivant, il n'y avait rien à
faire\,; mais après lui le champ était ouvert et prêtait aux
perspectives. Saint-Simon s'appliquait donc en secret dès lors à
réformer l'État\,; et comme il faisait chaque chose avec suite et en
poussant jusqu'au bout sans se pouvoir déprendre, il avait tout écrit,
ses plans, ses voies et moyens, ses combinaisons de Conseils substitués
à la toute-puissance des secrétaires d'État\,; il avait, lui aussi, son
royaume de Salente tout prêt, et sa République de Platon en
portefeuille, avec cela de particulier qu'en homme précis il avait déjà
écrit les noms des gens qu'il croyait bons à mettre en place, les
appointements, la dépense, en un mot la chose minutée et supposée
faite\,: et un jour que le duc de Chevreuse venait le voir pour gémir
avec lui des maux de l'État et discourir des remèdes possibles, il n'eut
d'autre réponse à faire qu'à ouvrir son armoire et à lui montrer ses
cahiers tout dressés.

Il y eut un moment tout à fait brillant et souriant dans la carrière de
cour de Saint-Simon sous Louis XIV\,: ce fut l'intervalle de temps qui
s'écoula entre la mort de Monseigneur (14 avril 1711) et celle du duc de
Bourgogne (18 février 1712), ce court espace de dix mois dans lequel ce
dernier fut Dauphin et héritier présomptif du trône. Saint-Simon, après
avoir échappé à bien des crocs-en-jambe, à bien des noirceurs et des
scélératesses calomnieuses qui avaient failli par moments lui faire
quitter de dégoût la partie et abandonner Versailles, s'était assez bien
remis dans l'esprit du roi\,; la duchesse de Saint-Simon, aimée et
honorée de tous, était dame d'honneur de la duchesse de Berry, et
lui-même s'avançait chaque jour par de sérieux entretiens en tête à
tête, sur les matières d'État et sur les personnes, dans la confiance
solide du nouveau Dauphin. Il travaillait confidentiellement avec lui.
S'il eut jamais espérance de faire accepter en entier sa théorie
politique, son idéal de gouvernement, ce fut alors. Il semble à le lire,
qu'il n'existât aucun désaccord, aucun point de dissentiment entre lui
et le jeune prince qui allait comme de lui-même au-devant de ses idées
et de ses maximes\,: dès la première ouverture qu'il lui fit, tout se
passa entre eux comme en vertu d'une harmonie préétablie.

Quelle était cette théorie politique de Saint-Simon et ce plan de
réforme\,? Il nous l'a exposé assez longuement, et dans ses
conversations avec le duc du Bourgogne, et depuis dans celles qu'il eut
avec le duc d'Orléans à la veille de la mort de Louis XIV et de la
Régence. Si l'on va au fond et qu'on dégage le système des mille détails
d'étiquette qui le compliquent et qui le compromettent à nos yeux par
une teinte de ridicule, on y saisit une inspiration qui, dans
Saint-Simon, fait honneur sinon au politique pratique, du moins au
citoyen et à l'historien publiciste. Il sent la plaie et la faiblesse
morale de la France au sortir des mains de Louis XIV\,; tout a été
abaissé, nivelé, réduit à l'état d'individu, il n'y a que le roi de
grand. Il ne faut pas demander à Saint-Simon de penser au peuple dans le
sens moderne\,; il ne le voit pas, il ne le distingue pas de la populace
ignorante et à jamais incapable. Reste la bourgeoisie qui fait la tête
de ce peuple et qu'il voit déjà ambitieuse, habile, insolente, égoïste
et repue, gouvernant le royaume par la personne des commis et
secrétaires d'État, ou usurpant et singeant par les légistes une fausse
autorité souveraine dans les Parlements. Quant à la noblesse dont il
est, et sur laquelle seule il compte pour la générosité du sang et le
dévouement à la patrie, il s'indigne de la trouver abaissée, dénaturée
et comme dégradée par la politique des rois, et surtout du dernier\,: en
accusant même presque exclusivement Louis XIV, il ne se dit pas assez
que l'œuvre par lui consommée a été la politique constante des rois
depuis Philippe Auguste, en y comprenant Henri IV et ce Louis XIII qu'il
admire tant. Il s'indigne donc de voir « que cette noblesse française si
célèbre, si illustre, est devenue un peuple presque de la même sorte que
le peuple même, et seulement distingué de lui en ce que le peuple a la
liberté de tout travail, de tout négoce, des armes même, au lieu que la
noblesse est devenue un autre peuple qui n'a d'autre choix que de
croupir dans une mortelle et ruineuse oisiveté qui la rend à charge et
méprisée, ou d'aller à la guerre se faire tuer à travers les insultes
des commis des secrétaires d'État et des secrétaires des intendants.\,»
Il la voudrait relever, restaurer en ses anciens emplois, en ses charges
et services utiles, avec tous les degrés et échelons de gentilhomme, de
seigneur, de duc et pair. Les Pairs surtout, en qui il a mis toutes ses
complaisances, et dont il fait la clef de voûte dans le vrai système,
lui semblent devoir être (comme ils l'ont jadis été, selon lui,) les
conseillers nécessaires du roi, les copartageants de sa souveraineté. Il
n'a cessé de rêver là-dessus, et il a sa reconstitution de la monarchie
française toute prête. Certes, si un prince était capable d'entrer dans
quelques-unes de ces vues à la fois courageuses, patriotiques, mais
étroites, hautaines et rétrospectives, il semble que ç'ait été le duc de
Bourgogne tel qu'on nous le présente, avec ce mélange de bonnes
intentions, d'effort sur lui-même, d'éducation laborieuse et
industrieuse, de principes et de doctrine en serre-chaude. On ne refait
point l'histoire par hypothèse. Le duc de Bourgogne n'a pas régné, et la
monarchie française, lancée à travers les révolutions, a suivi un tout
autre cours que celui qu'il méditait de lui faire prendre. Quand on lit
aujourd'hui Saint-Simon après les événements accomplis et en présence de
la démocratie débordante et triomphante (quelles que soient ses formes
de couronnement et de triomphe), on se demande plus que jamais avec
doute, ou plutôt on se dit sans hésiter sur la réponse\,:

Est-ce qu'il y avait moyen de refaire ainsi après Louis XIV, après
Richelieu, après Louis XI, les fondements de la monarchie française, de
la refaire une monarchie \emph{constitutionnelle aristocratique} avec
toutes les hiérarchies de rang\,? Une telle reconstruction par les bases
était-elle possible quand déjà allaient se dérouler de plus en plus par
des pentes larges et rapides les conséquences du nivellement
universel\,? Et enfin cela était-il d'accord avec le génie de la nation,
avec le génie de cette noblesse même qui aimait à sa manière à être un
peuple, un peuple de gentilshommes\,?

La seule réponse, encore une fois, est dans les faits accomplis\,: à
Saint-Simon reste l'honneur d'avoir résisté à l'abaissement et à
l'anéantissement de son Ordre, de s'être roidi contre la platitude et la
servilité courtisanesque. Sa théorie est comme une convulsion, un
dernier effort suprême de la noblesse agonisante pour ressaisir ce qui
va passer à ce tiers état qui est tout, et qui, le jour venu, dans la
plénitude de son installation, sera même le Prince.

La mort subite du duc de Bourgogne vint porter le plus rude coup à
Saint-Simon et briser la perspective la plus flatteuse qu'un homme de sa
nature et de sa trempe pût envisager, moins encore d'être au pouvoir par
lui-même que de voir se réaliser ses idées et ses vues, cette chimère du
bien public qu'il confondait avec ses propres satisfactions d'orgueil.
Le duc de Bourgogne mort à trente ans, Saint-Simon, qui n'en avait que
trente-sept, restait fort considérable et fort compté par sa liaison
intime et noblement professée en toute circonstance avec le duc
d'Orléans, que toutes les calomnies et les cabales ne pouvaient empêcher
de devenir, après la mort de Louis XIV et de ses héritiers en âge de
régner, le personnage principal du royaume.

Les plans que Saint-Simon développa au duc d'Orléans pour une réforme du
gouvernement ne furent qu'en partie suivis. L'idée des Conseils à
substituer aux secrétaires d'État pour l'administration des affaires,
était de lui\,; mais elle ne fut pas exécutée et appliquée comme il
l'entendait. Une des mesures qu'il proposait avec le plus de confiance,
eût été de convoquer les États généraux au début de la Régence\,; il y
voyait un instrument commode duquel on pouvait se servir pour obtenir
bien des réformes, et sur qui on en rejetterait la responsabilité par
manière d'excuse. Il y avait à profiter, selon lui, de l'\emph{erreur
populaire} qui attribuait à ce corps un grand pouvoir, et on pouvait
favoriser cette erreur innocente sans en redouter les suites. Ici
Saint-Simon se trompait peut-être de date comme en d'autres cas, et il
ne se rendait pas bien compte de l'effet et de la fermentation
qu'eussent produits des États généraux en 1716\,; la machine dont il
voulait qu'on jouât pouvait devenir dangereuse à manier. On va vite en
France, et, à défaut de l'abbé Sieyès pour théoricien, on avait déjà
l'abbé de Saint-Pierre qui aurait trouvé des traducteurs plus éloquents
que lui pour sa pensée et des interprètes. Et Montesquieu n'avait-il pas
alors vingt-cinq ans\,?

À dater de ce moment (1715), les Mémoires de Saint-Simon changent un peu
de caractère. Membre du conseil de Régence, il est devenu un des
personnages du gouvernement, et, bien que rarement ses avis prévalent,
il est continuellement admis à les donner et ne s'en fait pas faute\,;
on a des entretiens sans nombre où la matière déborde sous sa plume
comme elle abondait sur ses lèvres\,; l'intérêt, qui se trouve toujours
dans de certaines scènes et dans d'admirables portraits des acteurs, y
languit par trop de plénitude et de regorgement. Le règne de Louis XIV
où il était contenu allait mieux à Saint-Simon que cette demi-faveur de
la Régence, où il a beaucoup plus d'espace sans avoir pour cela d'action
bien décisive. Il ne fut point ministre parce qu'il ne le voulut pas\,;
il aurait pu l'être à un instant ou à un autre, mais il se pliait peu
aux combinaisons diverses et n'en augurait rien de bon\,; il ne trouvait
point dans le duc d'Orléans l'homme qu'il aurait voulu et qu'il avait
tant espéré et regretté dans le duc de Bourgogne\,; il lui reprochait
précisément d'être l'homme des transactions et des \emph{moyens termes},
et le Prince, à son tour, disait de son ardent et peu commode ami «
qu'il était immuable comme Dieu et d'une suite enragée,\,» c'est-à-dire
tout d'une pièce. À un certain jour (1721), Saint-Simon, dans un intérêt
de famille, désira l'ambassade d'Espagne, et il l'eut aussitôt. Cette
mission fut plus honorifique que politique, et il l'a racontée fort au
long \footnote{Moins au long toutefois qu'il n'a semblé jusqu'ici,
  d'après les éditions précédentes\,: car, dans la première qui a servi
  aux réimpressions, on a jugé à propos de transposer, du tome III\^{}o
  au {\textsc{XIX}}\textsuperscript{e}, plus de 100 pages relatives aux
  grandesses d'Espagne, et on en a bourré le récit de l'ambassade de
  Saint-Simon.}. Ce fut son dernier acte de représentation. La mort
subite du Régent (1723) vint peu après l'avertir de ce que la mort du
duc de Bourgogne lui avait déjà dit si éloquemment au cœur, que les
choses du monde sont périssables, et qu'il faut, quand on est chrétien,
penser à mieux. La politique craintive de Fleury aida à lui redoubler le
conseil. L'évêque de Fréjus, dans une visite à M\textsuperscript{me} de
Saint-Simon, lui fit entendre qu'on saurait son mari avec plus de
plaisir à Paris qu'à Versailles. Saint-Simon pensait trop haut pour ce
ministère à voix basse que méditait Fleury. Il ne se le fit pas dire
deux fois, et dès ce moment il renonça à la Cour, vécut plus
habituellement dans ses terres et s'occupa de la rédaction défînitive de
ses Mémoires. Il ne mourut qu'en 1755, le 2 mars, à quatre-vingts ans.

Il tournait depuis longtemps le dos au nouveau siècle, et il habitait
dans ses souvenirs. Il mourut quand Voltaire régnait, quand
l'\emph{Encyclopédie} avait commencé, quand Jean-Jacques Rousseau avait
paru, quand Montesquieu ayant produit tous ses ouvrages venait de mourir
lui-même. Que pensait-il, que pouvait-il penser de toutes ces nouveautés
éclatantes\,? On a souvent cité un mot dédaigneux sur Voltaire, qu'il
appelle Arouet, « fils d'un notaire qui l'a été de mon père et de
moi\ldots.\,» On en conclut un peu trop vite, à mon sens, le mépris de
Saint-Simon pour les gens de lettres et les gens d'esprit qui n'étaient
pas de sa classe. Saint-Simon, dans ses Mémoires, se montre bien plus
attentif qu'on ne le suppose à ce qui concerne les gens de lettres et
les gens d'esprit de son temps\,; mais ce sont ceux du siècle de Louis
XIV\,; c'est Racine, c'est La Fontaine, c'est La Bruyère, c'est
Despréaux, c'est Nicole, il n'en oublie aucun à la rencontre. Il a sur
Bossuet de grandes paroles, sur M\textsuperscript{me} de Sévigné il en a
d'une grâce et d'une légèreté délicieuse. Il sait rappeler au besoin
cette vieille bourgeoise du Marais si connue par le sel de ses bons
mots, M\textsuperscript{me} Cornuel. Tels sont les gens d'esprit aux
yeux de Saint-Simon. Quant à Voltaire, il en parle, il est vrai, comme
d'un aventurier d'esprit et d'un libertin\,: on en voit assez les
raisons sans les faire, de sa part, plus générales et plus injurieuses à
la classe des gens de lettres qu'elles ne le sont en effet.

On a remarqué comme une chose singulière que tandis que Saint-Simon
parle de tout le monde, il est assez peu question de lui dans les
Mémoires du temps. Ici encore il est besoin de s'entendre. De quels
Mémoires s'agit-il\,? Il y en a très-peu sur la fin du règne de Louis
XIV. Saint-Simon alors était fort jeune et n'avait aucun rôle
apparent\,: son principal rôle, c'était celui qu'il se donnait d'être le
champion de la Duché-Pairie et le plus pointilleux de son Ordre sur les
rangs. C'est ainsi qu'on lit dans une des lettres de Madame, mère du
Régent\,:

« En France et en Angleterre, les ducs et les lords ont un orgueil
tellement excessif qu'ils croient être au-dessus de tout\,; si on les
laissent faire, ils se regarderaient comme supérieurs aux princes du
sang, et la plupart d'entre eux ne sont pas même véritablement nobles
\emph{(Gare\,! voilà un autre excès qui commence).} J'ai une fois
joliment repris un de nos ducs. Comme il se mettait à la table du roi
devant le prince des Deux-Ponts, je dis tout haut\,: « D'où vient que
monsieur le duc de Saint-Simon presse tant le prince des Deux-Ponts\,?
A-t-il envie de le prier de prendre un de ses fils pour page\,?\,» Tout
le monde se mit si fort à rire qu'il fallut qu'il s'en allât.\,»

Si un jour il se publie des Mémoires sur la Régence, si les Mémoires
politiques du duc d'Antin et d'autres encore qui doivent être dans les
archives de l'État paraissent, il y sera certainement fort question de
Saint-Simon.

Saint-Simon, à qui ne le voyait qu'en passant et à la rencontre dans ce
grand monde, devait faire l'effet, je me l'imagine aisément, d'un
personnage remuant, pressé, mystérieux, échauffé, affairé, toujours dans
les confidences et les tête-à-tête, quelquefois très-amusant dans ses
veines et charmant à de certaines heures, et à d'autres heures assez
intempestif et incommode. Le maréchal de Belle-Isle le comparait vieux,
pour sa conversation, au plus intéressant et au plus agréable des
dictionnaires. Après sa retraite de la Cour, il venait quelquefois à
Paris, et allait en visite chez la duchesse de La Vallière ou la
duchesse de Mancini (toutes deux Noailles)\,: là, on raconte que, par
une liberté de vieillard et de grand seigneur devenu campagnard, et pour
se mettre plus à l'aise, il posait sa perruque sur un fauteuil, et
\emph{sa tête fumait. }• On se figure bien en effet cette tête à vue
d'œil fumante, que tant de passions échauffaient.

Les Mémoires imprimés du marquis d'Argenson contiennent (page 178) un
jugement défavorable sur Saint-Simon. Ce jugement a été arrangé et
modifié à plaisir, comme tout le style en général dans ces éditions de
d'Argenson. Je veux donner ici le vrai texte du passage tel qu'il se lit
dans le manuscrit \footnote{Manuscrits de la Bibliothèque du Louvre,
  dans le volume de d'Argenson qui est consacré à ses Mémoires
  personnels, au paragraphe 19.}. Si injurieux qu'en soient les termes
pour Saint-Simon, ce n'est pas tant à lui que ce jugement fera tort qu'à
celui qui s'y est abandonné\,; et d'ailleurs on peut, jusqu'à un certain
point, en contrôler l'exactitude, et cela en vaut la peine avant que
quelqu'un s'en empare, ce qui aurait lieu au premier jour\,; on ne
manquerait pas de crier à la découverte et de s'en faire une arme contre
Saint-Simon\,:

« Le duc de Saint-Simon, écrivait d'Argenson à la date de 1722, est de
nos ennemis, parce qu'il a voulu grand mal à mon père, le taxant
d'ingratitude, et voici quel en a été le lieu. Il prétend qu'il a plus
contribué que personne à mettre mon père en place de ministre, et que
mon père ne lui a pas tenu les choses qu'il lui avait promises comme
pot-de-vin du marché\,; or quelles étaient ces choses\,? Ce petit
\emph{broudillon} voulait qu'on fit le procès à M. le duc du Maine,
qu'on lui fît couper la tête, et le duc de Saint-Simon devait avoir la
grande maîtrise de l'artillerie. •Voyez un peu quel caractère odieux,
injuste et anthropophage de ce petit dévot sans génie, plein
d'amour-propre, et ne servant d'ailleurs aucunement à la guerre\,!

« Mon père voyant les choses pacifiées, les bâtards réduits, punis,
envoyés en prison ou exil, et tout leur parti débellé, ce qui fut une
des grandes opérations de son ministère, il ne voulut pas aller plus
loin, ni mêler des intérêts particuliers aux motifs des grands coups
qu'il frappa.

« De là le petit duc et sa séquelle en ont voulu mal de mort à mon père
et l'ont traité d'ingrat, comme si la reconnaissance, qui est une vertu,
devait se prouver par des crimes\,; et cette haine d'une telle légitime
rejaillit sur les pauvres enfants qui s'en\ldots\ldots.'\,»

Si la haine ou l'humeur éclate quelque part, c'est assurément dans cette
injurieuse boutade bien plus que dans tout ce que Saint-Simon a écrit
sur les d'Argenson. A l'égard du duc du Maine, Saint-Simon en effet a eu
le tort de le trop craindre, même après qu'il était déraciné et
abattu\,; mais quant à juger \emph{avec haine} le garde des sceaux et
ancien lieutenant de police d'Argenson, c'est ce qu'il n'a pas fait. Les
différents endroits où il parle de lui sont d'admirables pages
d'histoire\,; le marquis n'a pas parlé de son père en des termes plus
expressifs et mieux caractérisés que ne le fait Saint-Simon, qui n'y a
pas mis d'ailleurs les ombres trop fortes\,: tant il est vrai que le
talent de celui-ci le porte, nonobstant l'affection, à la vérité et à
une sorte de justice quand il est en face d'un mérite réel et sévère,
digne des pinceaux de l'histoire.

Je ne relèverai pas les autres injures de ce passage tout brutal\,:
Saint-Simon y est appelé un dévot \emph{sans génie}. Saint-Simon n'avait
pas, il est vrai, le génie politique\,; bien peu l'ont, et le marquis
d'Argenson, avec tout son mérite comme philosophe et comme
administrateur secondaire, n'en était lui-même nullement doué. Pour être
un politique, indépendamment des vues et des idées justes qui sont
nécessaires, mais qu'il ne faut avoir encore qu'à propos et modérément,
sans une fertilité trop confuse, il ne convient pas de porter avec soi
de ces humeurs brusques qui gâtent tout, et de ces antipathies des
hommes qui créent à chaque pas des incompatibilités. Le génie de
Saint-Simon, qui devait éclater après lui, rentrait tout entier dans la
sphère des Lettres\,: en somme, ce qu'il a dû être, il l'a été.

Il y a à dire à sa dévotion. Elle était sincère et dès lors
respectable\,; mais elle ne semble pas avoir été aussi éclairée qu'elle
aurait pu l'être. Après chaque mécompte ou chagrin, Saint-Simon s'en
allait droit à la Trappe chercher une consolation, comme on va, dans une
blessure, au chirurgien\,; mais il en revenait sans avoir modifié son
fond et sans travailler à corriger son esprit. Il se livrait à toutes
ses passions intellectuelles et à ses aversions morales sans scrupule,
et sauf à se mettre en règle à de certains temps réguliers et à s'en
purger la conscience, prêt à recommencer aussitôt après. Cette manière
un peu machinale et brusque de considérer le remède religieux, sans en
introduire la vertu et l'efficace dans la suite même de sa conduite et
de sa vie, annonce une nature qui avait reçu par une foi robuste la
tradition des croyances plutôt qu'elle ne s'en était pénétrée et imbue
par des réflexions lumineuses. En tout, Saint-Simon est plutôt supérieur
comme artiste que comme homme\,; c'est un immense et prodigieux talent,
plus qu'une haute et complète intelligence.

Après la mort de Saint-Simon, ses Mémoires eurent bien des vicissitudes.
Ils sortirent des mains de sa famille pour devenir des espèces de
prisonniers d'État\,; on craignait les divulgations indiscrètes. On voit
que Duclos et Marmontel en eurent connaissance, et en firent un ample
usage dans leurs travaux d'historiographes. M. de Choiseul, pendant son
ministère, en prêta des volumes à M\textsuperscript{me} du Deffand qui
en écrivit ses impressions à Horace Walpole auquel elle aurait voulu
également les prêter et les faire lire\,: « Nous faisons une lecture
l'après-dîner, lui mandait-elle (21 novembre 1770), les Mémoires de M.
de Saint-Simon, où il m'est impossible de ne pas vous regretter\,;
\emph{vous auriez des plaisirs indicibles}.\,» Elle dit encore à un
autre endroit (2 décembre)\,: « Les Mémoires de Saint-Simon m'amusent
toujours, et comme j'aime à les lire en compagnie, cette lecture durera
longtemps, elle vous amuserait, quoique le style en soit abominable, les
portraits mal faits\,; l'auteur n'était point un homme d'esprit\,; mais
comme il était au fait de tout, les choses qu'il raconte sont curieuses
et intéressantes\,; je voudrais bien pouvoir vous procurer cette
lecture.\,»

Elle y revient pourtant et corrige ce qui peut étonner dans ce premier
jugement tumultueux (9 janvier 1771). « Je suis désespérée de ne pouvoir
pas vous faire lire les Mémoires de Saint-Simon\,: le dernier volume,
que je ne fais qu'achever, m'a causé des plaisirs infinis\,; \emph{il
vous mettrait hors de vous}. \,» Je le crois bien que ces Mémoires de
Saint-Simon vous \emph{mettent hors de vous}\,; ils vous transportent au
cœur d'un autre siècle.

Voltaire sur sa fin avait, dit-on, formé le projet « de réfuter tout ce
que le duc de Saint-Simon, dans ses Mémoires encore secrets, avait
accordé à la prévention et à la haine.\,» Voltaire, en cela, voyait où
était le défaut de ces redoutables Mémoires, et aussi, en les voulant
infirmer à l'avance, il semblait pressentir où était le danger pour lui,
pour son \emph{Siècle de Louis XIV}, de la part de ce grand rival, et
que, lorsque de tels tableaux paraîtraient, ils éteindraient les
esquisses les plus brillantes qui n'auraient été que provisoires.

À partir de 1784, la publicité commença à se prendre aux Mémoires de
Saint-Simon, mais timidement, à la dérobée, par anecdotes décousues et
par morceaux. De 1788 à 1791, puis plus tard en 1818, il en parut
successivement des extraits plus ou moins volumineux, tronqués et
compilés. La marquise de Créquy, à propos d'une de ces premières
compilations, écrivait à Sénac de Meilhan \footnote{\emph{Lettres
  inédites de la marquise de Créquy à Sénac de Meilhan}, qui sont sous
  presse (Potier, libraire-éditeur).} (7 février 1787)\,: « Les Mémoires
de Saint-Simon sont entre les mains du censeur\,; de six volumes on en
fera à peine trois, et c'est encore assez.\,» Et, un peu plus tard (25
septembre 1788)\,: « Je vous annonce que les Mémoires de Saint-Simon
paraissent, mais très-mutilés, si j'en juge par ce que j'ai vu en trois
gros tapons verts, et il y en avait six. M\textsuperscript{me} de Turpin
mourut, j'en demeurai là, cela est mal écrit, mais le goût que nous
avons pour le siècle de Louis XIV nous en rend les détails précieux.\,»

Il est curieux de voir comme chacun s'accorde à dire que c'est \emph{mal
écrit}, que les portraits sont \emph{mal faits}, en ajoutant toutefois
que c'est intéressant. M\textsuperscript{me} du Deffand elle-même, la
seule qui ait lu à la source, apprécie l'amusement plus que la portée de
ces Mémoires. La forme de Saint-Simon tranchait trop avec les habitudes
du style écrit au {\textsc{xviii}}\textsuperscript{e} siècle, et on en
parlait à peu près comme Fénelon a parlé du style de Molière et de cette
« multitude de métaphores qui approchent du galimatias.\,» Tout ce beau
monde d'alors avait fait, plus ou moins, sa rhétorique dans Voltaire.

L'inconvénient de ces publications tronquées, comme aussi des extraits
mis au jour par Lemontey et portant sur les Notes manuscrites annexées
au Journal de Dangeau, c'était de ne donner idée que de ce qu'on
appelait la causticité de Saint-Simon, en dérobant tout à fait un autre
côté de sa manière, qui est la grandeur. Cette grandeur, qui, nonobstant
tout accroc de détail, allait à revêtir d'une imposante majesté l'époque
entière de Louis XIV, et qui était la première vérité du tableau, ne
pouvait se dévoiler que par la considération des ensembles et dans la
suite même de ce corps incomparable d'annales. C'est donc la totalité
des Mémoires qu'il fallait donner dans leur forme originale et
authentique. L'édition de 1829 y a pourvu. La sensation produite par les
premiers volumes fut très-vive\,: ce fut le plus grand succès depuis
celui des romans de Walter Scott. Un rideau se levait tout d'un coup de
dessus la plus belle époque monarchique de la France, et l'on assistait
à tout comme si l'on y était. Ce succès toutefois, coupé par la
Révolution de 1830, se passa dans le monde proprement dit, encore plus
que dans le public\,; celui-ci n'y arriva qu'un peu plus tard et
graduellement.

Aujourd'hui il restait à faire un progrès important et, à vrai dire,
décisif pour l'honneur de Saint-Simon écrivain. Cette première édition,
si goûtée, avait été faite d'après un singulier principe et sur un
sous-entendu étrange\,: c'est que Saint-Simon, parce qu'il a sa phrase à
lui et qui n'est ni académique, ni celle de tout le monde, écrivait au
hasard, ne savait pas écrire (comme le disaient les marquises de Créquy
et du Duffand), et qu'il était nécessaire de temps en temps, dans son
intérêt et dans celui du lecteur, de le corriger. D'autres relèveront
dans cette première édition des noms historiques estropiés, des
généalogies mal comprises et rendues inintelligibles, des pages du
manuscrit sautées, des transpositions et des déplacements qui ôtent tout
leur sens à d'autres passages où Saint-Simon s'en réfère à ce qu'il a
déjà dit\,; pour moi, je suis surtout choqué et inquiet des libertés
qu'on a prises avec la langue et le style d'un maître. M. de
Chateaubriand, dans un jour de mauvaise humeur contre le plus grand
auteur de Mémoires, a dit\,: « Il écrit \emph{à la diable} pour
l'immortalité.\,» Et d'autres, entrant dans cette jalousie de
Chateaubriand et comme pour la caresser, ont été jusqu'à dire de
Saint-Simon qu'il était « le premier des barbares.\,» Il faut bien
s'entendre sur le style de Saint-Simon, il n'est pas le même en tous
endroits et à toute heure. Lorsque Saint-Simon écrit des Notes et
commentaires sur le Journal de Dangeau, il écrit comme on fait pour des
notes, à la volée, tassant et pressant les mots, voulant tout dire à la
fois et dans le moindre espace. J'ai comparé ailleurs cette pétulance et
cette précipitation des choses sous sa plume « à une source abondante
qui veut sortir par un goulot trop étroit et qui s'y étrangle.\,»
Toutefois, même dans ces brusques croquis de Notes, tels qu'on les a
imprimés jusqu'ici, il y a bien des fautes qui tiennent à une copie
inexacte. Dans ses Mémoires, Saint-Simon reprend ses premiers jets de
portraits, les développe, et se donne tout espace. Quand il raconte des
conversations, il lui arrive de reproduire le ton, l'empressement,
l'afflux de paroles, les redondances, les ellipses. Habituellement et
toujours, il a, dans sa vivacité à concevoir et à peindre, le besoin
d'embrasser et d'offrir mille choses à la fois, ce qui fait que chaque
membre de sa phrase pousse une branche qui en fait naître une troisième,
et de cette quantité de branchages qui s'entre-croisent, il se forme à
chaque instant un arbre des plus touffus. Mais il ne faut pas croire que
cette production comme naturelle n'ait pas sa raison d'être, sa majesté
et souvent sa grâce. C'est à quoi l'édition de 1829, qui a servi depuis
aux réimpressions, n'avait pas eu égard\,: à première vue, on y a
considéré les phrases de Saint-Simon comme des \emph{à peu près} de
grand seigneur, et chemin faisant, sans parti pris d'ailleurs, on les a
traitées en conséquence \footnote{Un seul petit exemple. Dès la seconde
  page, Saint-Simon nous montre sa mère qui lui donne dès l'enfance de
  sages conseils et qui lui représente la nécessité, à lui fils tardif
  d'un vieux favori oublié, d'être par lui-même un homme de mérite,
  puisqu'il entre dans un monde où il n'aura point d'amis pour le
  produire et l'appuyer\,: « Elle ajoutoit, dit-il, le défaut de tous
  proches, oncles, tantes, cousins germains, qui me laissoit comme dans
  l'abandon à moi-même, et augmentoit le besoin de savoir en faire un
  bon usage sans secours et sans appui\,; ses deux frères obscurs, et
  l'aîné ruiné et plaideur de sa famille, et le seul frère de mon père
  sans enfants et son aîné de huit ans.\,» Or, ne trouvant pas la phrase
  assez claire dans son tour un peu latin, l'édition de 1829 a dit\,: «
  Elle ajoutoit le défaut de tous proches, oncles, tantes, cousins
  germains, qui me laissoit comme dans l'abandon à moi-même, et
  augmentoit le besoin de savoir en faire un bon usage, me
  \emph{trouvant} sans secours et sans appui\,; ses deux frères
  \emph{étant} obscurs, et l'aîné ruiné et plaideur de sa famille, et le
  seul frère de mon père \emph{étant} sans enfants et son aîné de huit
  ans.\,» \emph{Me trouvant} et deux fois \emph{étant} sont ajoutés.
  Ainsi dès les premiers pas, comme si la phrase de Saint-Simon ne
  marchait pas toute seule, on lui prêtait un bâton et deux béquilles.}.

Respectons le texte des grands écrivains, respectons leur style. Sachons
enfin comprendre que la nature est pleine de variétés et de moules
divers\,; il y a une infinité de formelle talents. Éditeurs ou
critiques, pourquoi nous faire strictement grammairiens et n'avoir qu'un
seul patron\,? Et ici, dans ce cas particulier de Saint-Simon, comme
nous avons affaire de plus et très-essentiellement à un peintre, il faut
aussi bien comprendre (et c'est sur quoi j'ai dû insister en
commençant), quel est le genre de vérité qu'on est en droit surtout de
lui demander et d'attendre de lui, sa nature et son tempérament
d'observateur et d'écrivain étant connus. L'exactitude dans certains
faits particuliers est moins ce qui importe et ce qu'on doit chercher
qu'une \emph{vérité d'impression} dans laquelle il convient de faire une
large part à la sensibilité et aux affections de celui qui regarde et
qui exprime. Le paysage, en se réfléchissant dans ce lac aux bords
sourcilleux et aux ondes un peu amères, dans ce lac humain mobile et
toujours plus ou moins prestigieux, s'y teint certainement de la couleur
de ses eaux. Une autre forme de talent, je l'ai dit, un autre miroir
magique eût reproduit des effets différents\,; et toutefois celui-ci est
vrai, il est sincère, il l'est au plus haut degré, dans l'acception
morale et pittoresque. C'est ce qu'on ne saurait trop maintenir, et
Saint-Simon n'a eu que raison quand il a conclu de la sorte en se
jugeant\,: « Ces Mémoires sont de source, de la première main\,: leur
vérité, leur authenticité ne peut être révoquée en doute, et je crois
pouvoir dire qu'il n'y en a point eu jusqu'ici qui aient compris plus de
différentes matières, plus approfondies, plus détaillées, ni qui forment
un groupe plus instructif ni plus curieux.\,» La postérité, après avoir
bien écouté ce qui s'est dit et se dira encore pour et contre, ne
saurait, je le crois, conclure autrement.

Sainte-Beuve.

\hypertarget{lettre-uxe9crite-par-saint-simon-uxe0-m.-de-rancuxe9-abbuxe9-de-la-trappe-en-le-consultant-sur-ses-muxe9moires.}{%
\chapter{LETTRE ÉCRITE PAR SAINT-SIMON À M. DE RANCÉ, ABBÉ DE LA TRAPPE,
EN LE CONSULTANT SUR SES
MÉMOIRES.}\label{lettre-uxe9crite-par-saint-simon-uxe0-m.-de-rancuxe9-abbuxe9-de-la-trappe-en-le-consultant-sur-ses-muxe9moires.}}

« Versailles, le 29 mars 1699.

« Il faut, Monsieur, que je sois bien convaincu que vous avez pour moi
une bonté extrême, pour oser prendre la liberté que je fais en vous
envoyant par la voie de M. du Charmel les papiers dont j'eus l'honneur
de vous parler, en mon dernier voyage, lorsque vous me permîtes de le
faire. Je vous dis lors qu'il {[}y{]} avait déjà quelque temps que je
travaillais à des espèces de Mémoires de ma vie qui comprenaient tout ce
qui a un rapport particulier à moi, et aussi un peu en général et
superficiellement une espèce de relation des événements de ces temps,
principalement des choses de la Cour \footnote{On voit quelle étoit à
  cette date de 1699, c'est-à-dire quand il n'avoit encore que
  vingt-quatre ans, la première idée de Saint-Simon en rédigeant ses
  Mémoires\,: mettre avec grand détail ce qui le concernait, et assez
  superficiellement ce qui regardoit les autres. Mais, une fois à
  l'œuvre et à mesure qu'il y mordait, son dessein s'est accru et
  l'accessoire est devenu le principal\,; le peintre, en présence de sa
  toile et de ses modèles, n'y a pas tenu et s'est donné carrière.}\,;
et comme je m'y suis proposé une exacte vérité, aussi m'y suis-je lâché
à la dire bonne et mauvaise, toute telle qu'elle m'a semblé sur les uns
et les autres, songeant à satisfaire mes inclinations et passions en
tout ce que la vérité m'a permis de dire, attendu que travaillant pour
moi et bien peu des miens pendant ma vie, et pour qui voudra après ma
mort, je ne me suis arrêté à ménager personne par aucune
considération\,; mais voyant cette espèce d'ouvrage qui va grossissant
tous les jours avec quelque complaisance de le laisser après moi, et
aussi ne voulant point être exposé aux scrupules qui me convieraient à
la fin de ma vie de le brûler comme ç'avait été mon premier projet, et
même plus tôt, à cause de tout ce qu'il y a contre la réputation de
mille gens, et cela d'autant plus irréparablement que la vérité s'y
rencontre tout entière et que la passion n'a fait qu'animer le style, je
me suis résolu à vous en importuner de quelques morceaux, pour vous
supplier par iceux de juger de la pile et de me vouloir prescrire une
règle pour dire toujours la vérité sans blesser ma conscience, et pour
me donner de salutaires conseils sur la manière que j'aurai à tenir en
écrivant des choses qui me touchent particulièrement et plus
sensiblement que les autres. J'ai donc choisi la relation de notre
procès contre MM. de Luxembourg père et fils, qui a produit des
rencontres qui m'ont touché de presque toutes les plus vives passions
d'une manière autant ou plus sensible que je l'aie été en ma vie, et qui
est exprimée en un style qui le fait bien remarquer. C'est, je crois,
tout ce qu'il y a de plus âpre et de plus amer en mes Mémoires, mais, au
moins, y ai-je tâché d'être fidèle à la plus exacte vérité. Je l'ai
copiée d'iceux, où elle est écrite éparse çà et là selon l'ordre des
temps auxquels nous avons plaidé, et mise ensemble\,; et, au lieu d'y
parler à découvert comme dans mes Mémoires, je me nomme dans cette copie
comme les autres, pour la pouvoir garder et m'en servir sans que j'en
paroisse manifestement l'auteur. J'y ai joint aussi deux portraits pour
servir d'échantillon au reste, quoique en bien celui de M. d'Aguesseau
pût suffisamment servir à ceux de ce genre, duquel il y en a bien moins
qu'en mal. Je vous supplie très humblement de vouloir garder ce que je
vous envoie, jus qu'à ce que je l'aille moi-même chercher, espérant
avoir ce délice tout aussitôt après Pâques, et vous porter en même temps
quelques cahiers des Mémoires mêmes. Je me flatte donc, qu'au milieu de
tous vos maux, de toutes les peines que vous cause ce changement heureux
de votre grand et merveilleux monastère, vous aurez la charité
d'examiner ce que je vous envoie, d'y penser devant Dieu, et de dicter
ces avis, règles et salutaires conseils que j'ose vous demander, afin
que, demeurant écrits, ils ne me passent point de la mémoire et que j'y
puisse avoir toute ma vie recours. Je crois qu'il serait inutile de vous
demander des précautions sur le secret et sur le ton de voix dont on
vous lira ces papiers pour qu'on ne puisse rien entendre hors de votre
chambre\,: eux-mêmes vous en feront souvenir suffisamment. Il ne me
reste plus rien à ajouter ici, sinon de vous demander pardon cent et
cent fois de la distraction que cela vous causera de tant de saintes et
d'admirables occupations dont vous vous nourrissez sans relâche, et de
vous assurer que je suis, Monsieur, plus que personne au monde, pénétré
de respect, d'attachement et de reconnaissance pour vous, et à jamais
votre très humble et très obéissant serviteur.

« \emph{P. S.} M. du Charmel ne sait point ce que c'est que ces papiers
\footnote{On aura remarqué combien, dans cette lettre, qui porte la date
  de l'extrême fin du XVIIe siècle, le style est comme en entier du
  XVIe, et non seulement quelques mots, mais les tours et le ton
  général. Il semble que c'était, à ce point de départ, l'habitude
  naturelle de Saint-Simon dans la familiarité épistolaire, et quand la
  passion n'excitoit pas son talent.}.\,»

\hypertarget{introduction.-savoir-sil-est-permis-duxe9crire-et-de-lire-lhistoire-singuliuxe8rement-celle-de-son-temps.}{%
\chapter{INTRODUCTION. SAVOIR S'IL EST PERMIS D'ÉCRIRE ET DE LIRE
L'HISTOIRE, SINGULIÈREMENT CELLE DE SON
TEMPS.}\label{introduction.-savoir-sil-est-permis-duxe9crire-et-de-lire-lhistoire-singuliuxe8rement-celle-de-son-temps.}}

Juillet 1743.

L'histoire a été dans tous les siècles une étude si recommandée, qu'on
croirait perdre son temps d'en recueillir les suffrages, aussi
importants par le poids de leurs auteurs que par leur nombre. Dans l'un
et dans l'autre on ne prétend compter que les catholiques, et on sera
encore assez fort\,; il ne s'en trouvera même aucun de quelque autorité
dans l'Église qui ait laissé par écrit aucun doute sur ce point. Outre
les personnages que leur savoir et leur piété ont rendus célèbres, on
compte plusieurs saints qui ont écrit des chroniques et des histoires
non seulement saintes, mais entièrement profanes, et dont les ouvrages
sont révérés de la postérité, à qui ils ont été fort utiles. On omet par
respect les livres historiques de l'Écriture. Mais, si on n'ose mêler en
ce genre le Créateur avec ses créatures, on ne peut aussi se dispenser
de reconnaître que, dès que le Saint-Esprit n'a pas dédaigné d'être
auteur d'histoires dont tout le tissu appartient en gros à ce monde, et
serait appelé profane comme toutes les autres histoires du monde, si
elles n'avaient pas le Saint-Esprit pour auteur, c'est un préjugé bien
décisif qu'il est permis aux chrétiens d'en écrire et d'en lire. Si on
objecte que les histoires de ce genre qui ont le Saint-Esprit pour
auteur se reportent toutes à des objets plus relevés, et bien que
réelles et véritables en effet, ne laissent pas d'être des figures de ce
qu'il devait arriver, et cachent de grandes merveilles sous ces voiles,
il ne laissera pas de demeurer véritable qu'il y en a de grands endroits
qui ne sont simplement que des histoires, ce qui autorise toutes les
autres que les hommes ont faites depuis, et continueront de faire\,;
mais encore, que dès qu'il a plu à l'Esprit saint de voiler et de
figurer les plus grandes choses sous des événements en apparence
naturels, historiques, et en effet arrivés, ce même Esprit n'a pas
réprouvé l'histoire, puisqu'il lui a plu de s'en servir pour
l'instruction de ses créatures et de son Église. Ces propositions, qui
ne se peuvent impugner avec de la bonne foi, sont d'une transcendance en
faveur de l'histoire à ne souffrir aucune réplique. Mais sans se
départir d'un si divin appui, cherchons d'ailleurs ce que la vérité, la
raison, la nécessité et l'usage approuvé dans tous les siècles, pourront
fournir sur ce prétendu problème.

Que sait-on qu'on n'ait point appris\,? Car il ne s'agit pas ici des
prophètes et des dons surnaturels, mais de la voie commune que la
Providence a marquée à tous les hommes. Le travail est une suite et la
peine du péché de notre premier père\,; on n'entretient le corps que par
le travail du corps, la sueur et les œuvres des mains\,; on n'éclaire
l'esprit que par un autre genre de travail, qui est l'étude\,; et comme
il faut des maîtres, pour le moins des exemples sous les yeux, pour
apprendre à faire les œuvres des mains dans chaque art ou métier, à plus
forte raison en faut-il pour les sciences et les disciplines si
diverses, propres à l'esprit, sur lesquelles l'inspection des yeux et
des sens n'ont aucune prise.

Si ces leçons d'autrui sont nécessaires à l'esprit pour lui apprendre ce
qui est de son ressort, il n'y a point de science où il s'en puisse
moins passer que pour l'histoire. Encore que pour les autres disciplines
il soit indispensable d'y avoir au moins quelque introducteur, il est
pourtant arrivé à des esprits d'une ouverture extraordinaire d'atteindre
eux mêmes, sans autre secours que celui de ce commencement, à divers
degrés, même quelques-uns aux plus relevés, des disciplines où ils
n'avaient reçu qu'une assez légère introduction\,; parce que avec
l'application et la lumière de leur esprit, ils s'étaient guidés de
degré en degré, pour atteindre plus haut, et, par de premières
découvertes, se frayer la route à de nouvelles, à les constater, à les
rectifier et à parvenir ainsi au sommet de la science par eux cultivée,
après en avoir appris d'autrui les premières règles et les premières
notions. C'est que les arts et les sciences ont un enchaînement de
règles, des proportions, des gradations qui se suivent nécessairement,
et qui ne sont, par conséquent, pas impossibles à trouver successivement
par un esprit lumineux, solide et appliqué, qui en a reçu d'autrui les
premiers éléments et comme la clef, quoique ce soit une chose
extrêmement rare, et que pour presque la totalité il faille être conduit
d'échelon en échelon par les diverses connaissances et les divers
progrès de la main d'un habile maître, qui sait proportionner ses leçons
à l'avancement qu'il remarque dans ceux qu'il instruit.

Mais l'histoire est d'un genre entièrement différent de toutes les
autres connaissances. Bien que tous les événements généraux et
particuliers qui la composent soient cause l'un de l'autre, et que tout
y soit lié ensemble par un enchaînement si singulier que la rupture d'un
chaînon ferait manquer, ou pour le moins changer, l'événement qui le
suit\,; il est pourtant vrai qu'à la différence des arts, surtout des
sciences, où un degré, une découverte, conduit à un autre certain, à
l'exclusion de tout autre, nul événement général ou particulier
historique n'annonce nécessairement ce qu'il causera, et fort souvent
fera très raisonnablement présumer au contraire. Par conséquent, point
de principes ni de clef, point d'éléments, de règle ni d'introduction
qui, une fois bien compris par un esprit, pour lumineux, solide et
appliqué qu'il soit, puisse le conduire de soi-même aux événements
divers de l'histoire\,; d'où résulte la nécessité d'un maître
continuellement à son côté, qui conduise de fait en fait par un récit
lié dont la lecture apprenne ce qui sans elle serait toujours
nécessairement et absolument ignoré.

C'est ce récit qui s'appelle l'histoire, et l'histoire comprend tous les
événements qui se sont passés dans tous les siècles et dans tous les
lieux. Mais si elle s'en tenait à l'exposition nue et sèche de ces
événements, elle deviendrait un faix inutile et accablant\,: inutile,
parce que peu importerait à l'instruction d'avoir la mémoire chargée de
faits inanimés, et qui n'apprennent que des faits secs et pesants à
l'esprit, à qui nul enchaînement ne les range et ne les rappelle\,;
accablant, par un fatras sans ordre et sans lumière qui puisse conduire
à plus qu'à plier sous la pesanteur d'un amas de faits détachés et sans
liaison l'un à l'autre, dont on ne peut faire aucun usage utile ni
raisonnable.

Ainsi pour être utile, il faut que le récit des faits découvre leurs
origines, leurs causes, leurs suites et leurs liaisons des uns aux
autres, ce qui ne se peut faire que par l'exposition des actions des
personnages qui ont eu part à ces choses\,; et comme sans cela les faits
demeureraient un chaos tel qu'il a été dit, autant en serait-il des
actions de ces personnages si on s'en tenait à la simple exposition de
leurs actions, par conséquent de toute l'histoire, si on ne faisait
connaître quels ont été ces personnages, ce qui les a engagés à la part
qu'ils ont eue aux faits qu'on raconte, et le rapport d'union ou
d'opposition qu'il y a eu entre eux. Plus donc on a de lumière
là-dessus, et plus les faits deviennent intelligibles, plus l'histoire
devient curieuse et intéressante, plus on instruit par les exemples des
mœurs et des causes des événements. C'est ce qui rend nécessaire de
découvrir les intérêts, les vices, les vertus, les passions, les haines,
les amitiés, et tous les autres ressorts tant principaux que incidents
des intrigues, des cabales et des actions publiques et particulières qui
ont part aux événements qu'on écrit, et toutes les divisions, les
branches, les cascades qui deviennent les sources et les causes d'autres
intrigues et qui forment d'autres événements.

Pour une juste exécution, il faut que l'auteur d'une histoire générale
ou particulière possède à fond sa matière par une profonde lecture, par
une exacte confrontation, par une juste comparaison d'auteurs les plus
judicieusement choisis, et par une sage et savante critique, le tout
accompagné de beaucoup de lumière et de discernement. J'appelle histoire
générale celle qui l'est en effet par son étendue de plusieurs nations
ou de plusieurs siècles de l'Église, ou d'une même nations mais de
plusieurs règnes, ou d'un fait ecclésiastique éloigné et fort étendu.
J'appelle histoire particulière celle du temps et du pays où on vit.
Celle-là, étant moins vaste, et se passant sous les yeux de l'auteur,
doit être beaucoup plus étendue en détails et en circonstances, et avoir
pour but de mettre son lecteur au milieu des acteurs de tout ce qu'il
raconte, en sorte qu'il croie moins lire une histoire ou des Mémoires,
qu'être lui-même dans le secret de tout ce qui lui est représenté, et
spectateur de tout ce qui est raconté. C'est en ce genre d'écrire que
l'exactitude la plus scrupuleuse sur la vérité de chaque chose et de
chaque trait doit se garder également de haine et d'affection, de
vouloir expliquer ce qu'on n'a pu découvrir, et de prêter des vues, des
motifs, des caractères, et de grossir ou diminuer, ce qui est également
dangereux et facile si l'auteur n'est homme droit, vrai, franc, plein
d'honneur et de probité, et fort en garde contre les pièges du
sentiment, du goût et de l'imagination, très singulièrement si cet
auteur se trouve écrire de source par avoir eu part par lui-même, ou par
ses amis immédiats de qui il aura été instruit, aux choses qu'il
raconte\,; et c'est en ce dernier cas où tout amour-propre, toute
inclination, toute aversion, et toute espèce d'intérêt doit disparaître
devant la plus petite et la moins importante vérité, qui est l'âme et la
justification de toute histoire, et qui ne doit jamais pour quoi que ce
puisse être souffrir la moindre ternissure, et être toujours exposée
toute pure et tout entière.

Mais un chrétien, et qui veut l'être, peut-il écrire et lire
l'histoire\,? Les faits secs, il est vrai, accablent inutilement\,;
ajoutez-y les actions nues des personnages qui y ont eu part, il ne s'y
trouvera pas plus d'instruction, et le chaos en sera qu'augmenté sans
aucun fruit. Quoi donc, les caractères, les intrigues, les cabales de
ces personnages pour entendre les causes et les suites des événements\,?
Il est vrai que sans cela ils demeureraient inintelligibles, et
qu'autant vaudrait-il ignorer ce qui charge sans apprendre, et par
conséquent, sans instruire. Mais la charité peut-elle s'accommoder du
récit de tant de passions et de vices, de la révélation de tant de
ressorts criminels, de tant de vues honteuses, et du démasquement de
tant de personnes, pour qui sans cela on aurait conservé de l'estime, ou
dont on aurait ignoré les vices et les défauts\,? Une innocente
ignorance n'est-elle pas préférable à une instruction si éloignée de la
charité\,? et que peut-on penser de celui qui, non content de celle
qu'il a prise par lui-même ou par les autres, la transmet à la
postérité, et lui révèle tant de choses de ses frères, ou méprisables ou
souvent criminelles\,?

Voilà, ce me semble, l'objection dans toute sa force. Elle disparaîtrait
par la seule citation de ce qui a été dit, au commencement de ce
discours, de l'exemple du Saint-Esprit\,; mais on s'est proposé de la
détruire, même sans s'avantager de l'autorité divine, après laquelle il
n'est plus permis de raisonner quand elle a décidé, comme on croit
qu'elle l'a fait sur la question qu'on agite.

Ne se permettre aucune histoire au deçà de ce que l'Écriture nous en
apprend, c'est se jeter dans les ténèbres palpables d'Égypte. Du côté de
la religion, on renonce à savoir ce que c'est que tradition, et y
renoncer n'implique-t-il pas blasphème\,? c'est ignorer les dogmes et la
discipline, en ignorant les conciles oecuméniques qui ont défini les
dogmes et établi la discipline, et mettre sur la même ligne les saints
défenseurs de la foi, les uns par leur lumière et leurs travaux, les
autres par leur courage et leur martyre, et les hérésiarques et les
persécuteurs. C'est se priver de l'admirable spectacle des premiers
siècles de l'Église, et l'édification de ses colonnes, de l'instruction
de ses premiers docteurs, de la sainte horreur de la première vie
ascétique et solitaire, de la merveille de cette économie qui a établi,
étendu et fait triompher l'Église au milieu des contradictions et des
persécutions de toutes les sortes, de peur de voir en même temps la
scélératesse, la cruauté et les crimes des hérésiarques et de leurs
principaux appuis, l'ambition, les vices et les barbaries des évêques,
et de ceux des plus grands sièges, et de là jusqu'à nous, ce qui s'est
passé de mémorable dans l'Église pour le dogme, sur les dernières
hérésies, sur la discipline et le culte, et de peur de voir le désordre
et l'ignorance, l'avarice et l'ambition de tant et tant des plus
principaux membres du clergé. Ce qui résulterait de cette ignorance est
plus aisé à penser qu'à représenter. Tout en est palpable, et saute de
soi-même aux yeux.

Si donc il ne paraît pas sensé de ne vouloir pas être instruit de ces
choses qui intéressent si fort un chrétien comme chrétien, comment le
pourra-t-on être indépendamment de l'histoire profane, qui a une liaison
si intime et si nécessaire avec celle de l'Église qu'elles ne peuvent
pour être entendues être séparées l'une de l'autre\,? C'est un mélange
et un enchaînement qui, pour une cause ou pour une autre, se perpétue de
siècle en siècle jusqu'au nôtre, et qui rend impossible la connaissance
d'aucune partie de l'une, sans acquérir en même temps celle de l'autre
qui lui répond pour le temps. Si donc un chrétien, à qui tout ce qui
appartient à la religion est cher à proportion de son attachement pour
elle, ne peut être indifférent sur les divers événements qui ont agité
l'Église dans tous les temps, il ne peut aussi éviter de s'instruire en
parallèle de toute l'histoire profane, qui y a un si indispensable et un
si continuel rapport.

Mais mettant même à part ce rapport, puisqu'en effet il se trouve de
longs morceaux d'histoire qui n'en ont point avec celle de l'Église,
pourrait-on sans honte se faire un scrupule de savoir ce qu'a été la
Grèce, ce qu'ont été les Romains, l'histoire de ces fameuses républiques
et de leurs personnages principaux\,? Oserait-on ignorer par scrupule
les divers degrés de leurs changements, de leur décadence, de leur
chute, ceux de l'élévation des États qui se sont formés de leurs débris,
l'origine et la fondation des monarchies de notre Europe, et de celle
des Sarrasins, puis des Turcs, enfin la succession des siècles et des
règnes, et leurs événements principaux jusqu'à nous\,? Voilà en gros
pour l'histoire générale. Venons maintenant à ce qui regarde celle du
temps et du pays où l'on vit.

Si l'on convient que le scrupule qui retiendrait dans l'entière
ignorance de l'histoire générale serait la plus grossière ineptie, et
qui jetterait dans les inconvénients les plus honteux et les plus
lourds, il sera difficile de se persuader qu'aucun scrupule doive ou
puisse admettre l'ignorance de l'histoire particulière du temps et du
pays où on vit, qui est bien plus intéressante que la générale, et qui
touche bien autrement l'instruction de notre conduite et de nos mœurs.

J'entends le scrupuleux répondre que l'éloignement des temps et des
lieux affranchit la charité en quelque sorte sur les vices de
personnages étrangers, reculés, dont on ne connaît ni les personnes ni
les races, et à qui il n'est plus d'hommes qui puissent prendre quelque
part\,; bien différents de ceux de notre pays et de notre âge que nous
connaissons tous par leurs noms, par leur conduite, par leurs familles,
par leurs amis, pour qui on a pu conserver de l'estime, qui même en ont
pu mériter par quelques endroits, et pour qui on fait souvent plus que
la perdre par la levée du rideau qui les couvrait.

L'objection n'est pas différente de celle qui a été déjà présentée\,:
les raisons qui la détruisent ne seront pas différentes aussi des
premières dont on s'est servi\,; mais pour couper court, ne craignons
point le sarcasme, et un sarcasme que j'ai vu très littéralement et très
exactement réalisé par des personnes dont le nom et le rang distingué
sont parfaitement connus. Ce n'était pas scrupule, mais ignorance
d'éducation, puis de négligence, et d'abandon au tourbillon du jeu et
des plaisirs, au milieu même de la cour. D'où que vienne cette
ignorance, sa grossièreté est la même\,; revenons à son effet. Quelle
surprise de s'entendre demander qui était ce Monseigneur qu'on a ouï
nommer et dire qu'il était mort à Meudon\,? Qui était le père du roi,
par où et comment le roi et le roi d'Espagne sont-ils parents Qu'est-ce
que c'était que Monsieur, et que M. et M\textsuperscript{me} la duchesse
de Berry\,? De qui feu M. le duc d'Orléans régent était-il fils\,? Quand
on en est là, on peut juger si les notions remontent plus haut, ou
descendent aux personnages et aux actions du règne qui ne fait que de
passer, et quel abîme de ténèbres sur ce qui précède. Voilà néanmoins
l'effet de l'ignorance d'éducation et de tourbillon, qu'il est aisé de
réparer par de la conversation et de la lecture, mais qui, fondée sur le
scrupule, ne se peut plus guérir. Il est si imbécile, il blesse
tellement le bon sens et la raison naturelle, que la démonstration de
l'erreur de cette idée se fait tellement de soi-même et d'une façon si
rapide à la simple exposition, qu'elle efface tout ce qui s'y peut
répondre, et tarit tout ce qu'on aurait à opposer.

En effet, est-on obligé d'ignorer les Guise, les rois et la cour de leur
temps, de peur d'apprendre leurs horreurs et leurs crimes\,? les
Richelieu et les Mazarin pour ignorer les mouvements que leur ambition a
causés, et les vices et les défauts qui se sont déployés dans les
cabales et les intrigues de leur temps\,? Se taira-t-on M. le Prince
pour éviter ses révoltes et leurs accompagnements\,; M. de Turenne et
ses proches pour ne pas voir les plus insignes perfidies les plus
immensément récompensées\,? Et vivant parmi la postérité de ce qui a
figuré dans ces temps dont je parle, s'exposera-t-on avec le moindre
sens à ignorer d'où ils viennent, d'où leur fortune, quels ils sont, et
aux grossiers et continuels inconvénients qui en résultent\,?
N'aura-t-on nulle idée de M\textsuperscript{me} de Montespan et de ses
funestes suites, de peur de savoir les péchés de leur élévation\,? N'en
aura-t-on point aucune de M\textsuperscript{me} de Maintenon et du
prodige de son règne, de peur des infamies de ses premiers temps, de
l'ignominie et des malheurs de sa grandeur, des maux qui en ont inondé
la France\,? Il en est de même des personnages qui ont figuré sous ce
long règne, et de ses fertiles événements dont le long gouvernement a
changé toute l'ancienne face du royaume. Demeurera-t-on, par obligation
de conscience, sans oser s'instruire des causes d'une si funeste
mutation, dans le scrupule d'y découvrir l'intérêt et les ressorts de
ces grands ministres qui, sortis de la boue, se sont faits les seuls
existants et ont renversé toutes choses\,? Enfin se cachera-t-on
jusqu'au présent pour ne point voir les désordres personnels d'un
régent, les forfaits d'un premier ministre, les barbaries et
l'imbécillité du successeur, les faussetés, les bévues, l'ambition sans
bornes les crimes de celui qui vient de passer, dont la jalousie et
l'insuffisance plongent aujourd'hui l'État dans la situation la plus
dangereuse, et dans la plus ruineuse confusion\,? Qui pourrait résister
à un problème si insensé, je dis si radicalement impossible\,? Qui n'en
serait pas révolté\,? Ces scrupuleux persuaderont-ils que Dieu demande
ce qui est opposé à lui-même puisqu'il est lumière et vérité,
c'est-à-dire que l'on s'aveugle en faveur du mensonge, de peur de voir
la vérité\,; qu'il a donné des yeux pour les tenir exactement fermés sur
tous les événements et les personnages du monde\,; du sens et de la
raison, pour n'en faire d'autre usage que de les abrutir, et pour nous
rendre pleinement grossiers, stupides, ridicules, et parfaitement
incapables d'être soufferts parmi les plus charitables même des autres
hommes\,?

Rendons au Créateur un culte plus raisonnable, et ne mettons point le
salut que le Rédempteur nous a acquis au prix indigne de l'abrutissement
absolu, et du parfait impossible. Il est trop bon pour vouloir l'un, et
trop juste pour exiger l'autre. Fuyons la folie des extrémités qui n'ont
d'issue que les abîmes, et, avec saint Paul, ne craignons pas de mettre
notre sagesse sous la mesure de la sobriété, mais de la pousser au delà
de ses justes bornes. Servons-nous donc des facultés qu'il a plu à Dieu
de nous donner, et ne croyons pas que la charité défende de voir toutes
sortes de vérités, et de juger des événements qui arrivent, et de tout
ce qui en est l'accompagnement. Nous nous devons pour le moins autant de
charité qu'aux autres\,; nous devons donc nous instruire pour n'être pas
des hébétés, des stupides, des dupes continuelles. Nous ne devons pas
craindre, mais chercher à connaître les hommes bons et mauvais pour
n'être pas trompés, et sur un sage discernement régler notre conduite et
notre commerce, puisque l'une et l'autre est nécessairement avec eux, et
dans une réciproque dépendance les uns des autres. Faisons-nous un
miroir de cette connaissance pour former et régler nos mœurs, fuir,
éviter, abhorrer ce qui doit l'être, aimer, estimer, servir ce qui le
mérite, et s'en approcher par l'imitation et par une noble ou sainte
émulation. Connaissons donc tant que nous pourrons la valeur des gens et
le prix des choses\,; la grande étude est de ne s'y pas méprendre au
milieu d'un monde la plupart si soigneusement masqué\,; et comprenons
que la connaissance est toujours bonne, mais que le bien ou le mal
consistent dans l'usage que l'on en fait. C'est là où il faut mettre le
scrupule, et où la morale chrétienne, l'étendue de la charité, en un mot
la loi nouvelle doivent sans cesse éclairer et contenir nos pas, et non
pas le jeter sur les connaissances dont on ne peut trop acquérir.

Les mauvais, qui dans ce monde ont déjà tant d'avantages sur les bons,
en auraient un autre bien étrange contre eux, s'il n'était pas permis
aux bons de les discerner, de les connaître, par conséquent de s'en
garer, d'en avertir à même fin, de recueillir ce qu'ils sont, ce qu'ils
ont fait à propos des événements de la vie, et, s'ils ont peu ou
beaucoup figuré, de les faire passer tels qu'ils sont et qu'ils ont été
à la postérité, en lui transmettant l'histoire de leur temps. Et d'autre
part, quant à ce monde, les bons seraient bien maltraités de demeurer,
comme bêtes brutes, exposés aux mauvais sans connaissance, par
conséquent sans défense, et leur vertu enterrée avec eux. Par là toute
vérité éteinte, tout exemple inutile, toute instruction impossible, et
toute providence restreinte dans la foi, mais anéantie aux yeux des
hommes.

Distinguons donc ce que la charité commande d'avec ce qu'elle ne
commande pas, et d'avec ce qu'elle ne veut pas commander, parce qu'elle
ne veut commander rien de préjudiciable, et que sa lumière ne peut être
la mère de l'aveuglement. La charité qui commande d'aimer son prochain
comme soi-même décide par cela seul la question. Par ce commandement
elle défend les contentions, les querelles, les injures, les haines, les
calomnies, les médisances, les railleries piquantes, les mépris. Tout
cela regarde les sentiments intérieurs qu'on doit réprimer en soi-même,
et les effets extérieurs de ces choses défendues dans l'exercice du
commerce et de la société. Elle défend de nuire et de faire, même de
souhaiter, du mal à personne\,; mais quelque absolu que paroisse un
commandement si étendu, il faut toujours reconnaître qu'il a ses bornes
et ses exceptions. La même charité qui impose toutes ces obligations
n'impose pas celle de ne pas voir les choses et les gens tels qu'ils
sont.

Elle n'ordonne pas, sous prétexte d'aimer les personnes parce que ce
sont nos frères, d'aimer en eux leurs défauts, leurs vices, leurs
mauvais desseins, leurs crimes\,; elle n'ordonne pas de s'y exposer\,;
elle ne défend pas, mais elle veut même qu'on en avertisse ceux qu'ils
menacent, même qu'ils regardent, pour qu'ils puissent s'en garantir, et
elle ne défend pas de prendre tous les moyens légitimes pour s'en mettre
à couvert.

Tout est plein de cette pratique chez les saints les plus révérés, et
les plus illustres qui n'ont pas même épargné les découvertes des faits
les plus fâcheux ni les invectives les plus amères, contre les méchants
particuliers dont ils ont eu à se défendre ou qu'ils ont cru devoir
décrier\,; et quand je dis les méchants particuliers, cette expression
n'est que pour exclure la généralité vague, montrer qu'ils s'en sont
pris aux personnes de leur temps, et quelquefois les plus élevées dans
l'Église ou dans le monde. La raison de cette conduite est évidente,
c'est que la charité n'est destinée que pour le bien, et autant qu'on le
peut conserver, pour les personnes\,; mais dès qu'elle devient
préjudiciable au bien, et qu'il ne s'agit plus que de personnes et de
personnes, il est clair qu'elle est due aux bons aux dépens des mauvais,
à qui il n'est pas permis de laisser le champ libre, d'opprimer et de
nuire aux bons, faute de les avertir, de les défendre, de publier autant
qu'il le faut, les artifices, les mauvais desseins, la conduite
dangereuse, les crimes même des mauvais, qui, si on les laissait faire,
deviendraient les maîtres de toutes leurs entreprises, et réussiraient
sûrement, toujours contre les bons, et qui malgré ces secours les
accablent si souvent.

De cet éclaircissement il en résulte un autre\,: c'est que le chrétien à
qui la charité défend de mal parler et de nuire à son prochain, et dans
toute l'étendue qui vient d'être rapportée, est par elle-même obligée à
tout le contraire en certains cas, différents encore de ceux qui
viennent d'être remarqués. Ceux qui ont la confiance des généraux, des
ministres, encore plus ceux qui ont celle des princes, ne doivent pas
leur laisser ignorer les mœurs, la conduite, les actions des hommes. Ils
sont obligés de les leur faire connaître tels qu'ils sont, pour les
garantir de pièges, de surprises, et surtout de mauvais choix. C'est une
charité due à ceux qui gouvernent, et qui regarde très principalement le
public qui doit être toujours préféré au particulier. Les conducteurs de
la chose publique, en tout ou en partie, sont trop occupés d'affaires,
trop circonvenus, trop flattés, trop aisément abusés et trompés par le
grand intérêt de le faire pour pouvoir bien démêler et discerner. Ils
sont sages de se faire éclairer sur les personnes, et heureux lorsqu'ils
trouvent des amis vrais et fidèles qui les empêchent d'être séduits\,;
et le public, ou la portion du public qui en est gouvernée, a grande
obligation à ces conseillers éclairés qui les préservent de tant de
sortes d'administrations, dans lesquelles il a toujours tant à souffrir
quand elles sont commises en de mauvaises mains\,; et il ne suffit pas à
ceux qui ont l'oreille de ces puissants du siècle d'attendre qu'ils les
consultent sur certaines personnes mauvaises\,: ils doivent prévenir
leur goût, leur facilité, les embûches qui leur sont dressées, et les
prévenir à temps d'y tomber. Ils se doivent estimer placés pour cela
dans la confiance de ces maîtres du siècle\,; et ceux-là même qui ont
celle de ces favoris à portée de tout dire ne doivent pas négliger de
les éclairer, et de se rendre ainsi utiles à la société. Il en est de
même envers les proches et les amis.

S'il est évident, comme on vient de le montrer, que la charité permet de
se défendre et d'attaquer même les méchants\,; si elle veut que les bons
soient avertis et soutenus\,; si elle exige que ceux qui sont établis en
des administrations publiques soient éclairés sans ménagement sur les
personnes et sur les choses, quoique toutes ces démarches ne se puissent
faire sans nuire d'une façon très directe et très radicale à la
réputation et à la fortune, à plus forte raison la charité ne défend pas
d'écrire, et par conséquent de lire, les histoires générales et
particulières. Outre les raisons qui ont ouvert ce discours, et après
lesquelles on pourrait n'en pas alléguer d'autres, il en faut donner de
nouvelles qui achèvent de lever tout scrupule là-dessus. Je laisse les
histoires générales pour me borner aux particulières de son pays et de
son temps\,; parce que, si j'achève de démontrer que ces dernières sont
licites, la même preuve servira encore plus fortement pour les histoires
générales. Mais il faut se souvenir des conditions qui ont été proposées
pour écrire.

Écrire l'histoire de son pays et de son temps, c'est repasser dans son
esprit avec beaucoup de réflexion tout ce qu'on a vu, manié, ou su
d'original, sans reproche, qui s'est passé sur le théâtre du monde, et
les diverses machines, souvent les riens apparents, qui ont mû les
ressorts des événements qui ont eu le plus de suite et qui en ont
enfanté d'autres\,; c'est se montrer à soi-même pied à pied le néant du
monde, de ses craintes, de ses désirs, de ses espérances, de ses
disgrâces, de ses fortunes, de ses travaux\,; c'est se convaincre du
rien de tout par la courte et rapide durée de toutes ces choses et de la
vie des hommes\,; c'est se rappeler un vif souvenir que nul des heureux
du monde ne l'a été, et que la félicité, ni même la tranquillité, ne
peut se trouver ici-bas\,; c'est mettre en évidence que, s'il était
possible que cette multitude de gens de qui on fait une nécessaire
mention avait pu lire dans l'avenir le succès de leurs peines, de leurs
sueurs, de leurs soins, de leurs intrigues, tous, à une douzaine près
tout au plus, se seraient arrêtés tout court dès l'entrée de leur vie,
et auraient abandonné leurs vues et leurs plus chères prétentions\,; et
que de cette douzaine encore, leur mort, qui termine le bonheur qu'ils
s'étaient proposé, n'a fait qu'augmenter leurs regrets par le
redoublement de leurs attaches, et rend pour eux comme non avenu tout ce
à quoi ils étaient parvenus. Si les livres de piété représentent cette
morale, si capable de faire mépriser tout ce qui se passe ici-bas, d'une
manière plus expresse et plus argumentée, il faut convenir que cette
théorie, pour belle qu'elle puisse être, ne fait pas les mêmes
impressions que les faits et les réflexions qui naissent de leur
lecture. Ce fruit que l'auteur en tire le premier, se recueille aussi,
par ses lecteurs\,; ils y joignent de plus l'instruction de l'histoire
qu'ils ignoraient. Cette instruction forme ceux qui ont à vivre dans le
commerce du monde, et plus encore s'ils sont portés en celui des
affaires. Les exemples dont ils se sont remplis les conduisent et les
préservent d'autant plus aisément, qu'ils vivent dans les mêmes lieux où
ces choses se sont passées, et dans un temps encore trop proche pour que
ce ne soient pas les mêmes mœurs, et le même genre de vie, de commerce
et d'affaires. Ce sont des avis et des conseils qu'ils reçoivent de
chaque coup de pinceau à l'égard des personnages, et de chaque événement
par le récit des occasions et des mouvements qui l'ont produit\,; mais
des avis et des conseils pris de la chose et des gens par eux-mêmes qui
les lisent, et qu'ils reçoivent avec d'autant plus de facilité qu'ils
sont tous nus, et n'ont ni la sécheresse, ni l'autorité, ni le dégoût,
qui rebutent et qui font échouer si ordinairement les conseils et les
avis de ceux qui se mêlent d'en vouloir donner. Je ne vois donc rien de
plus utile que cette double et si agréable manière de s'instruire par la
lecture de l'histoire de son temps et de son pays, ni conséquemment de
plus permis que de l'écrire. Et dans quelle ignorance profonde ne
serait-on pas, dans quelles ténèbres sur l'instruction et sur la
conduite de la vie si on n'avait pas ces histoires\,? Aussi voit-on que
la Providence a permis qu'elles n'ont presque point manqué, nonobstant
les pertes infinies qu'on a faites dans tous les temps par la négligence
de les faire passer d'âge en âge en les transcrivant avant l'impression,
et depuis par les gênes que l'intérêt y a mises, par les incendies et
par mille autres accidents.

L'histoire a un avantage à l'égard de la charité sur les occasions où on
vient de voir qu'elle permet, et quelquefois qu'elle prescrit,
d'attaquer et de révéler les mauvais. C'est que l'histoire n'attaque et
ne révèle que des gens morts, et morts depuis trop longtemps pour que
personne prenne part en eux. Ainsi la réputation, la fortune et
l'intérêt des vivants n'y sont en rien altérés, et la vérité paraît sans
inconvénient dans toute sa pureté. La raison de cela est claire\,: celui
qui écrit l'histoire de son temps, qui ne s'attache qu'au vrai, qui ne
ménage personne, se garde bien de la montrer. Que n'aurait-il point à
craindre de tant de gens puissants, offensés en personne, ou dans leurs
plus proches par les vérités les plus certaines, et en même temps les
plus cruelles\,! Il faudrait donc qu'un écrivain eût perdu le sens pour
laisser soupçonner seulement qu'il écrit. Son ouvrage doit mûrir sous la
clef et les plus sûres serrures, passer ainsi à ses héritiers, qui
feront sagement de laisser couler plus d'une génération ou deux, et de
ne laisser paraître l'ouvrage que lorsque le temps l'aura mis à l'abri
des ressentiments. Alors ce temps ne sera pas assez éloigné pour avoir
jeté des ténèbres. On a lu avec plaisir, fruit et sûreté beaucoup de
diverses histoires et Mémoires de la minorité de Louis XIV aussitôt
après sa mort, et il en est de même d'âge en âge. Qui est-ce qui se
soucie maintenant des personnages qui y sont dépeints, et qui prend part
aujourd'hui aux actions et aux manèges qui y sont racontés\,? Rien n'y
blesse donc plus la charité mais tout y instruit et répand une lumière
qui éclaire tous ceux qui les lisent. S'étendre davantage sur ces
vérités serait s'exercer vainement à prouver qu'il est jour quand le
soleil luit. On espère du moins qu'on aura levé tous les scrupules.

\hypertarget{chapitre-premier.}{%
\chapter{CHAPITRE PREMIER.}\label{chapitre-premier.}}

1691

1692

~

\relsize{-1}

{\textsc{Année 1691. Où et comment ces Mémoires commencés.}} {\textsc{-
Ma première liaison avec M. le duc de Chartres.}} {\textsc{- Maupertuis,
capitaine des mousquetaires gris}} \footnote{Il y avait deux compagnies
  de mousquetaires dans la maison du roi\,: les mousquetaires noirs et
  les mousquetaires gris, qui tiraient leur nom de la couleur de leurs
  chevaux.} {\textsc{\,; sa fortune et son caractère.}} {\textsc{- Année
1692. Ma première campagne, mousquetaire gris.}} {\textsc{- Siège de
Namur par le roi en personne.}} {\textsc{- Reddition de Namur.}}
{\textsc{- Solitude de Marlaigne.}} {\textsc{- Poudre cachée par les
jésuites.}} {\textsc{- Bataille navale de la Hogue.}} {\textsc{- Danger
de badiner avec des armes.}} {\textsc{- Coetquen et sa mort.}}
\relsize{1}

~

Je suis né la nuit du 15 au 16 janvier 1675, de Claude, duc de
Saint-Simon, pair de France, et de sa seconde femme Charlotte de
L'Aubépine, unique de ce lit. De Diane de Budos, première femme de mon
père, il avait eu une seule fille et point de garçon. Il l'avait mariée
au duc de Brissac, pair de France, frère unique de la duchesse de
Villeroy. Elle était morte en 1684, sans enfants, depuis longtemps
séparée d'un mari qui ne la méritait pas, et par son testament m'avait
fait son légataire universel.

Je portais le nom de vidame \footnote{Les vidames étaient des seigneurs
  qui tenaient des terres d'un évêché, à condition de défendre le
  temporel de l'évêque et de commander ses troupes. Il y avait quatre
  principaux vidames dans l'ancienne France\,: ceux de Laon, d'Amiens,
  du Mans et de Chartres.} de Chartres, et je fus élevé avec un grand
soin et une grande application. Ma mère, qui avait beaucoup de vertu et
infiniment d'esprit de suite et de sens, se donna des soins continuels à
me former le corps et l'esprit. Elle craignit pour moi le sort des
jeunes gens qui se croient leur fortune faite et qui se trouvent leurs
maîtres de bonne heure. Mon père, né en 1606, ne pouvait vivre assez
pour me parer ce malheur, et ma mère me répétait sans cesse la nécessité
pressante où se trouverait de valoir, quelque chose un jeune homme
entrant seul dans le monde, de son chef, fils d'un favori de Louis XIII,
dont tous les amis étaient morts ou hors d'état de l'aider, et d'une
mère qui, dès sa jeunesse, élevée chez la vieille duchesse d'Angoulême,
sa parente, grand'mère maternelle du duc de Guise, et mariée à un
vieillard, n'avait jamais vu que leurs vieux amis et amies, et n'avait
pu s'en faire de son âge. Elle ajoutait le défaut de tous proches,
oncles, tantes, cousins germains, qui me laissaient comme dans l'abandon
à moi-même, et augmentait le besoin de savoir en faire un bon usage,
sans secours et sans appui\,; ses deux frères obscurs, et l'aîné ruiné
et plaideur de sa famille, et le seul frère de mon père sans enfants et
son aîné de huit ans.

En même temps, elle s'appliquait à m'élever le courage, et à m'exciter
de me rendre tel que je pusse réparer par moi-même des vides aussi
difficiles à surmonter. Elle réussit à m'en donner un grand désir. Mon
goût pour l'étude et les sciences ne le seconda pas, mais celui qui est
comme né avec moi pour la lecture et pour l'histoire, et conséquemment
de faire et de devenir quelque chose par l'émulation et les exemples que
j'y trouvais, suppléa à cette froideur pour les lettres\,; et j'ai
toujours pensé que si on m'avait fait moins perdre de temps à celles-ci,
et qu'on m'eût fait faire une étude sérieuse de celle-là, j'aurais pu y
devenir quelque chose.

Cette lecture de l'histoire et surtout des Mémoires particuliers de la
nôtre, des derniers temps depuis François Ier, que je faisais de
moi-même, me firent naître l'envie d'écrire aussi ceux de ce que je
verrais, dans le désir et dans l'espérance d'être de quelque chose et de
savoir le mieux que je pourrais les affaires de mon temps. Les
inconvénients ne laissèrent pas de se présenter à mon esprit\,; mais la
résolution bien ferme d'en garder le secret à moi tout seul me lut
remédier à tout. Je les commençai donc en juillet 1694, étant mestre de
camp \footnote{Le titre de \emph{mestre de camp} répondait à celui de
  colonel.} d'un régiment de cavalerie de mon nom, dans le camp de
Guinsheim sur le Vieux-Rhin, en l'armée commandée par le maréchal-duc de
Lorges.

En 1691, j'étais en philosophie et commençais à monter à cheval à
l'académie des sieurs de Mémon et Rochefort, et je commençais aussi à
m'ennuyer beaucoup des maîtres et de l'étude, et à désirer fort d'entrer
dans le service. Le siège de Mons, formé par le roi en personne, à la
première pointe du printemps, y avait attiré presque tous les jeunes
gens de mon âge pour leur première campagne\,; et ce qui me piquait le
plus, M. le duc de Chartres y faisait la sienne. J'avais été comme élevé
avec lui, plus jeune que lui de huit mois, et si l'âge permet cette
expression entre jeunes gens si inégaux, l'amitié nous unissait
ensemble. Je pris donc ma résolution de me tirer de l'enfance, et je
supprime les ruses dont je me servis pour y réussir. Je m'adressai à ma
mère\,; je reconnus bientôt qu'elle m'amusait. J'eus recours à mon père
à qui je fis accroire que le roi, ayant fait un grand siège cette année,
se reposerait la prochaine. Je trompai ma mère qui ne découvrit ce que
j'avais tramé que sur le point de l'exécution, et que j'avais monté mon
père à ne se laisser point entamer.

Le roi s'était roidi à n'excepter aucun de ceux qui entraient dans le
service, excepté les seuls princes du sang et ses bâtards, de la
nécessité de passer une année dans une de ses deux compagnies des
mousquetaires, à leur choix, et de là, à apprendre plus ou moins
longtemps à obéir, ou à la tête d'une compagnie de cavalerie, ou
subalterne dans son régiment d'infanterie qu'il distinguait et
affectionnait sur tous autres, avant de donner l'agrément d'acheter un
régiment de cavalerie ou d'infanterie, suivant que chacun s'y était
destiné. Mon père me mena donc à Versailles où il n'avait encore pu
aller depuis son retour de Blaye, où il avait pensé mourir. Ma mère l'y
était allée trouver en poste et l'avait ramené encore fort mal, en sorte
qu'il avait été jusqu'alors sans avoir pu voir le roi. En lui faisant sa
révérence, il me présenta pour être mousquetaire, le jour de Saint-Simon
Saint-Jude, à midi et demi, comme il sortait du conseil.

Sa Majesté lui fit l'honneur de l'embrasser par trois fois, et comme il
fut question de moi, le roi, me trouvant petit et l'air délicat, lui dit
que j'étais encore bien jeune, sur quoi mon père répondit que je l'en
servirais plus longtemps. Là-dessus, le roi lui demanda en laquelle des
deux compagnies il voulait me mettre, et mon père choisit la première, à
cause de Maupertuis, son ami particulier, qui en était capitaine. Outre
le soin qu'il s'en promettait pour moi, il n'ignorait pas l'attention
avec laquelle le roi s'informait à ces deux capitaines des jeunes gens
distingués qui étaient dans leurs compagnies, surtout à Maupertuis, et
combien leurs témoignages influaient sur les premières opinions que le
roi en prenait, et dont les conséquences avaient tant de suites. Mon
père ne se trompa pas, et j'ai eu lieu d'attribuer aux bons offices de
Maupertuis la première bonne opinion que le roi prit de moi.

Ce Maupertuis se disait de la maison de Melun et le disait de bonne
foi\,; car il était la vérité et l'honneur et la probité même, et c'est
ce qui lui avait acquis la confiance du roi. Cependant il n'était rien
moins que Melun, ni reconnu par aucun de cette grande maison. Il était
arrivé par les degrés, de maréchal des logis des mousquetaires jusqu'à
les commander en chef et à devenir officier général\,; son équité, sa
bonté, sa valeur lui en avaient acquis l'estime. Les vétilles, les
pointilles de toute espèce d'exactitude et de précision, et une vivacité
qui d'un rien faisait un crime, et de la meilleure foi du monde, l'y
faisaient moins aimer. C'était par là qu'il avait su plaire au roi qui
lui avait souvent donné des emplois de confiance. Il fut chargé, à la
dernière disgrâce de M. de Lauzun, de le conduire à Pignerol, et, bien
des années après, de l'en ramener à Bourbon deux fois de suite, lorsque
l'intérêt de sa liberté et celui de M. du Maine y joignirent
M\textsuperscript{me} de Montespan et cet illustre malheureux, qui y
céda les dons immenses de Mademoiselle à M. du Maine pour changer
seulement sa prison en exil. L'exactitude de Maupertuis dans tous ces
divers temps qu'il fut sous sa garde le mit tellement au désespoir qu'il
ne l'a oublié de sa vie. C'était d'ailleurs un très homme de bien, poli,
modeste et respectueux.

Trois mois après que je fus mousquetaire, c'est-à-dire en mars de
l'année suivante, le roi fut à Compiègne faire la revue de sa maison et
de la gendarmerie, et je montai une fois la garde chez le roi. Ce petit
voyage donna lieu de parler d'un plus grand. Ma joie en fut extrême\,;
mais mon père, qui n'y avait pas compté, se repentit bien de m'avoir cru
et me le fit sentir. Ma mère, après un peu de dépit et de bouderie de
m'être ainsi enrôlé par mon père malgré elle, ne laissa pas de lui faire
entendre raison et de me faire un équipage de trente-cinq chevaux ou
mulets, et de quoi vivre honorablement chez moi soir et matin. Ce ne fut
pas sans un fâcheux contretemps, précisément arrivé vingt jours avant
mon départ. Un nommé Tessé, intendant de mon père, qui demeurait chez
lui depuis plusieurs années, disparut tout à coup et lui emporta
cinquante mille livres qui se trouvèrent dues à tous les marchands dont
il avait produit de fausses quittances dans ses comptes. C'était un
petit homme, doux, affable, entendu, qui avait montré du bien, qui avait
des amis, avocat au parlement de Paris, et avocat du roi au bureau des
finances de Poitiers.

Le roi partit {[}le 10 mai 1692{]} avec les dames, et je fis le voyage à
cheval avec la troupe et tout le service comme les autres mousquetaires
pendant les mois qu'il dura. J'y fus accompagné de deux gentilshommes\,:
l'un, ancien de la maison, avait été mon gouverneur, et d'un autre qui
était écuyer de ma mère. L'armée du roi se forma au camp de Gevry. Celle
de M. de Luxembourg l'y joignait presque. Les dames étaient à Mons, à
deux lieues de là. Le roi les fit venir en son camp où il les régala,
puis leur fit voir la plus superbe revue qui ait peut-être jamais été
faite, de ces deux armées rangées sur deux lignes, la droite de M. de
Luxembourg touchant la gauche du roi et tenant trois lieues d'étendue.

Après dix jours de séjour à Gevry, les deux armées se séparèrent et
marchèrent. Deux jours après le siège de Namur fut déclaré, où le roi
arriva en cinq jours de marche. Monseigneur \footnote{Louis de France,
  fils de Louis XIV et de Marie-Thérèse, né le 1er novembre 1661, mort
  le 14 avril 1711. Il est toujours désigné, dans les \emph{Mémoires de
  Saint-Simon}, sous le nom de \emph{Monseigneur}.}, Monsieur
\footnote{Philippe de France, duc d'Orléans, second fils de Louis XIII
  et d'Anne d'Autriche\,; il était né le 21 septembre 1640, et mourut le
  9 juin 1701.}, M. le Prince \footnote{Henri-Jules de Bourbon, prince
  de Condé, né le 9 juillet 1643, mort le 1er avril 1709. Il était fils
  du grand Condé. Le chef de la maison de Condé porte toujours, dans ces
  Mémoires, le nom de \emph{M. le Prince}.} et le maréchal d'Humières,
tous quatre, l'un sous l'autre par degrés, commandaient l'armée sous le
roi, et M. de Luxembourg, seul général de la sienne, couvrait le siège
et faisait l'observation. Les dames étaient cependant allées à Dinant.
Au troisième jour de marche, M. le Prince fut détaché pour aller
investir la ville de Namur. Le célèbre Vauban, l'âme de tous les sièges
que le roi a faits, emporta que la ville serait attaquée séparément du
château contre le baron de Bressé, qui voulait qu'on fît le siège de
tous les deux à là fois, et c'était lui qui avait fortifié la place. Un
fort mécontentement lui avait fait quitter depuis peu le service
d'Espagne, non sans laisser quelques nuages sur sa réputation de s'être
aussitôt jeté en celui de France. Il s'était distingué par sa valeur et
sa capacité\,; il était excellent ingénieur et très bon officier
général. Il eut, en entrant au service du roi le grade de lieutenant
général et un grand traitement pécuniaire. C'était un homme de basse
mine, modeste, réservé, dont la physionomie ne promettait rien, mais qui
acquit bientôt la confiance du roi et toute l'estime militaire.

M. le Prince, le maréchal d'Humières et le marquis de Boufflers eurent
chacun une attaque. Il n'y eut rien de grande remarque pendant les dix
jours que ce siège dura. Le onzième de tranchée ouverte, la chamade fut
battue, et la capitulation telle, à peu près, que les assiégés la
désirèrent. Ils se retirèrent au château, et il fut convenu de part et
d'autre qu'il ne serait point attaqué par la ville, et que la ville
serait en pleine sûreté du château qui ne tirerait pas un seul coup
dessus. Pendant ce siège, le roi fut toujours campé, et le temps fut
très chaud et d'une sérénité constante depuis le départ de Paris. On n'y
perdit personne de remarque que Cramaillon, jeune ingénieur de grande
espérance, et d'ailleurs bon officier, que Vauban regretta fort. Le
comte de Toulouse reçut une légère contusion au bras tout proche du roi,
qui, d'un lieu éminent et pourtant assez éloigné, voyait attaquer en
plein jour une demi-lune qui fut emportée par un détachement des plus
anciens des deux compagnies de mousquetaires.

Jonvelle, gentilhomme, mais d'ailleurs soldat de fortune, d'honneur et
de valeur, mourut de maladie pendant ce siège. Il était lieutenant
général et capitaine de la deuxième compagnie des mousquetaires\,; il
avait plus de quatre-vingts ans, et fut fort regretté du roi et de sa
compagnie. Toutes les deux se joignirent pour lui rendre les derniers
devoirs militaires. Sa compagnie fut à l'instant donnée à M. de Vins qui
la commandait sous lui, beau-frère de M. de Pomponne, et qui, maréchal
de camp en l'armée d'Italie, commandait lors un gros corps pour couvrir
la Provence, où il servit très utilement, et fut l'année suivante
lieutenant général.

L'armée changea de camp pour le siège du château. En arrivant chacun
dans le lieu qui lui était marqué, le régiment d'infanterie du roi
trouva son terrain occupé par un petit corps des ennemis qui s'y
retranchaient, d'où il résulta à l'instant un petit combat particulier
assez rude. M. de Soubise, lieutenant général de jour, y courut et s'y
distingua. Le régiment du roi acquit beaucoup d'honneur avec peu de
perte, et les ennemis furent bientôt chassés. Le roi en fut très aise
par son affection pour ce régiment qu'il a toujours particulièrement
tenu pour sien entre toutes ses troupes.

Ses tentes et celles de toute la cour furent dressées dans un beau pré à
cinq cents pas du monastère de Marlaigne. Le beau temps se tourna en
pluies, de l'abondance et de la continuité desquelles personne de
l'armée n'avait vu d'exemple, et qui donnèrent une grande réputation à
saint Médard, dont la fête est au 8 juin. Il plut tout ce jour-là à
verse, et on prétend que le temps qu'il fait ce jour-là dure quarante
jours de suite. Le hasard fit que cela arriva cette année. Les soldats,
au désespoir de ce déluge, firent des imprécations contre ce saint, en
recherchèrent des images et les rompirent et brûlèrent tant qu'ils en
trouvèrent. Ces pluies devinrent une plaie pour le siège. Les tentes du
roi n'étaient communicables que par des chaussées de fascines qu'il
fallait renouveler tous les jours, à mesure qu'elles s'enfonçaient\,;
les camps et les quartiers n'étaient pas plus accessibles\,; les
tranchées pleines d'eau et de boue, il fallait souvent trois jours pour
remuer le canon d'une batterie à une autre. Les chariots devinrent
inutiles, en sorte que les transports des bombes, boulets, etc., ne
purent se faire qu'à dos de mulets et de chevaux tirés de tous les
équipages de l'armée et de la cour, sans le secours desquels il aurait
été impossible. Ce même inconvénient des chemins priva l'armée de M. de
Luxembourg de l'usage des voitures. Elle périssait faute de grains, et
cet extrême inconvénient ne put trouver de remède que par l'ordre que le
roi donna à sa maison de prendre tous les jours par détachement des sacs
de grains en croupe, et de les porter en un village où ils étaient reçus
et comptés par des officiers de l'armée de M. de Luxembourg. Quoique la
maison du roi n'eût presque aucun repos pendant ce siège pour porter les
fascines, fournir les diverses gardes et les autres services
journaliers, ce surcroît lui fut donné, parce que la cavalerie servait
continuellement aussi, et en était aux feuilles d'arbres presque pour
tout fourrage.

Cette considération ne satisfit point la maison du roi, accoutumée à
toutes sortes de distinction. Elle se plaignit avec amertume. Le roi se
roidit et voulut être obéi. Il fallut donc le faire. Le premier jour, le
détachement des gens d'armes et des chevau-légers de la garde, arrivé de
grand matin au dépôt des sacs, se mit à murmurer et, s'échauffant de
propos les uns les autres, vinrent jusqu'à jeter les sacs et à refuser
tout net d'en porter. Crenay, dans la brigade duquel j'étais, m'avait
demandé poliment si je voulais bien être du détachement pour les sacs,
sinon qu'il me commanderait pour quelque autre\,; j'acceptai les sacs,
parce que je sentis que cela ferait ma cour par tout le bruit qui
s'était déjà fait là-dessus. En effet j'arrivai avec le détachement des
mousquetaires au moment du refus des troupes rouges, et je chargeai mon
sac à leur vue. Marin, brigadier de cavalerie et lieutenant des gardes
du corps, qui était là pour faire charger les sacs par ordre, m'aperçut
en même temps, et, plein de colère du refus qu'il venait d'essuyer,
s'écria, me touchant en me montrant et me nommant\,: « que puisque je ne
trouvais pas ce service au-dessous de moi, les gens d'armes et les
chevau-légers ne seraient ni déshonorés ni gâtés de m'imiter.\,» Ce
propos, joint à l'air sévère de Marin, fit un effet si prompt qu'à
l'instant ce fut sans un mot de réplique à qui de ces troupes rouges se
chargerait le plus tôt de sacs. Et oncques depuis il n'y eut plus
là-dessus la plus légère difficulté. Marin vit partir le détachement
chargé, et alla aussitôt rendre compte au roi de ce qui s'y était passé
et de l'effet de mon exemple. Ce fut un service qui m'attira plusieurs
discours obligeants du roi, qui chercha toujours pendant le reste du
siège à me dire quelque chose avec bonté toutes les fois qu'il me
voyait, ce dont je fus d'autant plus obligé à Marin que je ne le
connaissais en façon du monde.

Le vingt-septième jour de tranchée ouverte, qui était le mardi 1er
juillet 1692, le prince de Barbançon, gouverneur de la place, battit la
chamade, et certes il était temps pour les assiégeants à bout de
fatigues et de moyens par l'excès du mauvais temps qui ne cessait point,
et qui avait rendu tout fondrière. Jusqu'aux chevaux du roi vivaient de
feuilles, et aucun de cette nombreuse cavalerie de troupes et
d'équipages ne s'en est jamais bien remis. Il est certain que sans la
présence du roi dont la vigilance était l'âme du siège, et qui, sans
l'exiger, faisait faire l'impossible (tant le désir de lui plaire et de
se distinguer était extrême), on n'en serait jamais venu à bout\,; et
encore demeura-t-il fort incertain de ce qui en serait arrivé si la
place eût encore tenu dix jours, comme il n'y eut pas deux avis qu'elle
le pouvait. Les fatigues de corps et d'esprit que le roi essuya en ce
siège lui causèrent la plus douloureuse goutte qu'il eût encore
ressentie, mais qui de son lit ne l'empêcha pas de pourvoir à tout, et
de tenir pour le dedans et le dehors ses conseils comme à Versailles,
ainsi qu'il avait fait pendant tout le siège.

M. d'Elbœuf, lieutenant général, et M. le Duc, maréchal de camp, étaient
de tranchée lors de la chamade, M. d'Elbœuf mena les otages au roi, qui
eut bientôt réglé une capitulation honorable. Le jour que la garnison
sortit, le plus pluvieux qu'il eût fait encore, le roi, accompagné de
Monseigneur et de Monsieur, fut à mi-chemin de l'armée de M. de
Luxembourg, où ce général vint recevoir ses ordres pour le reste de la
campagne. Le prince d'Orange avait mis toute sa science et ses ruses
pour le déposter pendant le siège sur lequel il brûlait de tomber\,;
mais il eut affaire à un homme qui lui avait déjà montré qu'en matière
de guerre il en savait plus que lui, et qui continua à le lui montrer le
reste de sa vie.

Pendant cette légère course du roi, le prince de Barbançon sortit par la
brèche à la tête de sa garnison qui était encore de deux mille hommes,
qui défila devant M. le Prince et le maréchal d'Humières, entre deux
haies des régiments des gardes françaises et suisses et du régiment
d'infanterie du roi. Barbançon fit un assez mauvais compliment à M. le
Prince, et parut au désespoir de la perte de son gouvernement. Il en
était aussi grand bailli, et il en tirait cent mille livres de rente. Il
ne les regretta pas longtemps, et il fut tué l'été d'après à la bataille
de Neerwinden.

La place, une des plus fortes des Pays-Bas, avait la gloire de n'avoir
jamais changé de maître. Aussi eut-elle grand regret au sien, et les
habitants ne pouvaient contenir leurs larmes. Jusqu'aux solitaires de
Marlaigne en furent profondément touchés, jusque-là qu'ils ne purent
déguiser leur douleur, encore que le roi, touché de la perte de leur blé
qu'ils avaient retiré dans Namur, leur en eût fait donner le double et
de plus une abondante aumône. Ses égards à ne les point troubler furent
pareils. Ils ne logèrent que le cardinal de Bouillon, le comte de
Grammont, le P. de La Chaise, confesseur du roi, et son frère, capitaine
de la porte\,; et le roi ne permit le passage du canon à travers leur
parc qu'à la dernière extrémité, et quand il ne fut plus possible de le
pouvoir conduire par ailleurs. Malgré tant de bontés, ils ne pouvaient
regarder un Français après la prise de la place, et un d'eux refusa une
bouteille de bière à un huissier de l'antichambre du roi, qui se renomma
de sa charge et qui offrit inutilement de l'échanger contre une de vin
de Champagne.

Marlaigne est un monastère sur une petite et agréable éminence, dans une
belle forêt tout environnée de haute futaie, avec un grand parc, fondé
par les archiducs Albert et Isabelle pour une solitude de carmes
déchaussés, telle que ces religieux en ont dans chacune de leurs
provinces, où ceux de leur ordre se retirent de temps en temps, pour un
an ou deux et jamais plus de trois, par permission de leurs supérieurs.
Ils y vivent en perpétuel silence dans des cellules plus pauvres, mais
telles à peu près que celles des chartreux, mais en commun pour le
réfectoire qui est très frugal, dans un jeûne presque continuel, assidus
à l'office, et partageant d'ailleurs leur temps entre le travail des
mains et la contemplation. Ils ont quatre chambrettes, un petit jardin
et une petite chapelle chacun, avec la plus grande abondance des plus
belles et des meilleures eaux de source que j'aie jamais bues, dans leur
maison, autour et dans leur parc, et la plupart jaillissantes. Ce parc
est tout haut et bas avec beaucoup de futaies et clos de murs. Il est
extrêmement vaste. Là dedans sont répandues huit ou dix maisonnettes
loin l'une de l'autre, partagées comme celles du cloître, avec un jardin
un peu plus grand et une petite cuisine. Dans chacune habite, un mois,
et rarement plus, un religieux de la maison qui s'y retire par
permission du supérieur qui seul le visite de fois à autre. La vie y est
plus austère que dans la maison et dans une séparation entière. Ils
viennent tous à l'office le dimanche, emportent leur provision du
couvent, préparent seuls leur manger durant la semaine, ne sortent
jamais de leur petite demeure, y disent leur messe qu'ils sonnent et que
le voisin qui entend la cloche vient répondre, et s'en retournent sans
se dire un mot. La prière, la contemplation, le travail de leur petit
ménage, et à faire des paniers, partagent leur temps, à l'imitation des
anciennes laures \footnote{Cellules des solitaires dans l'Orient,
  formant une sorte de village\,; ce furent les premiers monastères.}.

Il arriva une chose à Namur, après sa prise, qui fit du bruit, et qui
aurait pu avoir de fâcheuses suites avec un autre prince que le roi.
Avant qu'il entrât dans la ville, où pendant le siège du château il
n'aurait pas été convenable qu'il eût été, on visita tout avec
exactitude, quoique par la capitulation les mines, les magasins, et tout
en un mot eût été montré. Lorsque, dans une dernière visite après la
prise du château, on la voulut faire chez les jésuites, ils ouvrirent
tout, en marquant toutefois leur surprise, et quelque chose de plus, de
ce qu'on ne s'en fiait pas à leur témoignage. Mais en fouillant partout
où ils ne s'attendaient pas, on trouva leurs souterrains pleins de
poudre dont ils s'étaient bien gardés de parler\,: ce qu'ils en
prétendaient faire est demeuré incertain. On enleva leur poudre, et,
comme c'étaient des jésuites, il n'en fut rien.

Le roi essuya, pendant le cours de ce siège, un cruel tire-lesse
\footnote{Vieux mot que l'on écrit ordinairement \emph{tire-laisse}. Il
  exprimait le désappointement d'un homme frustré d'une chose qu'il
  croyait ne pouvoir lui manquer.}. Il avait en mer une armée navale
commandée par le célèbre Tourville, vice-amiral\,; et les Anglais une
autre jointe aux Hollandais, presque du double supérieure. Elles étaient
dans la planche, et le roi d'Angleterre sur les côtes de Normandie, prêt
à passer en Angleterre suivant le succès. Il compta si parfaitement sur
ses intelligences avec la plupart des chefs Anglais, qu'il persuada au
roi de faire donner bataille, qu'il ne crut pouvoir être douteuse par la
défection certaine de plus de la moitié des vaisseaux Anglais pendant le
combat. Tourville, si renommé par sa valeur et sa capacité, représenta
par deux courriers au roi l'extrême danger de se fier aux intelligences
du roi d'Angleterre, si souvent trompées, la prodigieuse supériorité des
ennemis, et le défaut des ports et de tout lieu de retraite si la
victoire demeurait aux Anglais, qui brûleraient sa flotte et perdraient
le reste de la marine du roi. Ses représentations furent inutiles, il
eut ordre de combattre, fort ou faible, où que ce fût. Il obéit, il fit
des prodiges que ses seconds et ses subalternes imitèrent, mais pas un
vaisseau ennemi ne mollit et ne tourna. Tourville fut accablé du nombre,
et quoiqu'il sauvât plus de navires qu'on ne pouvait espérer, tous
presque furent perdus ou brûlés après le combat dans la Hogue. Le roi
d'Angleterre, de dessus le bord de la mer, voyait le combat, et il fut
accusé d'avoir laissé échapper de la partialité en faveur de sa nation,
quoique aucun d'elle ne lui eût tenu les paroles sur lesquelles il avait
emporté de faire donner le combat.

Pontchartrain était lors secrétaire d'État, ayant le département de la
marine, ministre d'État, et en même temps contrôleur général des
finances. Ce dernier emploi l'avait fait demeurer à Paris, et il
adressait ses courriers et ses lettres pour le roi à Châteauneuf son
cousin, Phélypeaux comme lui et aussi secrétaire d'État, qui en rendait
compte au roi. Pontchartrain dépêcha un courrier avec la triste
nouvelle, mais tenue en ces premiers moments dans le dernier secret. Un
courrier de retour à Barbezieux, secrétaire d'État ayant le département
de la guerre, l'allait de hasard retrouver en ce même moment devant
Namur. Il joignit bientôt celui de Pontchartrain, moins bon courrier et
moins bien servi sur la route. Ils lièrent conversation, et celui de
terre fit tout ce qu'il put pour tirer des nouvelles de celui de la mer.
Pour en venir à bout il courut quelques heures avec lui. Ce dernier,
fatigué de tant de questions, et se doutant bien qu'il en serait gagné
de vitesse, lui dit enfin qu'il contenterait sa curiosité, s'il lui
voulait donner parole d'aller de conserve, et de ne le point devancer,
parce qu'il avait un grand intérêt de porter le premier une si bonne
nouvelle\,; et tout de suite, lui dit que Tourville a battu la flotte
ennemie, et lui raconte je ne sais combien de vaisseaux pris ou coulés à
fond. L'autre, ravi d'avoir su tirer ce secret, redouble de questions
pour se mettre bien au fait du détail qu'il voulait se bien mettre dans
la tête\,; et dès la première poste donne des deux, s'échappe et arrive
le premier, d'autant plus aisément que l'autre avait peu de hâte et lui
voulait donner le loisir de triompher.

Le premier courrier arrive, raconte son aventure à Barbezieux qui
sur-le-champ le mène au roi. Voilà une grande joie, mais une grande
surprise de la recevoir ainsi de traverse. Le roi envoie chercher
Châteauneuf, qui dit n'avoir ni lettres ni courrier, et qui ne sait ce
que cela veut dire. Quatre ou cinq heures après arrive l'autre courrier
chez Châteauneuf, qui s'empresse de lui demander des nouvelles de la
victoire qu'il apporte\,; l'autre lui dit modestement d'ouvrir ses
lettres\,; il les ouvre et trouve la défaite. L'embarras fut de l'aller
apprendre au roi, qui manda Barbezieux et lui lava la tête. Ce contraste
l'affligea fort, et la cour parut consternée. Toutefois le roi sut se
posséder, et je vis, pour la première fois, que les cours ne sont pas
longtemps dans l'affliction ni occupées de tristesse.

Le gouvernement de Namur et son comté fut donné à Guiscard. Il était
maréchal de camp, mais fort oublié et fort attaché à ses plaisirs. Il
avait le gouvernement de Sedan qu'il conserva, et qu'il avait eu de La
Bourlie, son père, sous-gouverneur du roi, et il était encore gouverneur
de Dinant qui lui fut aussi laissé. La surprise du choix fut grande,
ainsi que la douleur de ceux de Namur, accoutumés à n'avoir pour
gouverneurs que les plus grands seigneurs des Pays-Bas. Guiscard eut le
bon esprit de réparer ce qui lui manquait par tant d'affabilité et de
magnificence, par une si grande aisance dans toute la régularité du
service d'un gouvernement si jaloux, qu'il se gagna pour toujours le
cœur et la confiance de tout son gouvernement et des troupes qui s'y
succédèrent à ses ordres.

Deux jours après la sortie de la garnison ennemie, le roi s'en alla à
Dinant où étaient les dames, avec qui il retourna à Versailles. J'avais
espéré que Monseigneur achèverait la campagne, et être du détachement
des mousquetaires qui demeurerait avec lui\,; et ce ne fut pas sans
regret que je repris avec toute la compagnie le chemin de Paris. Une des
couchées de la cour fut à Marienbourg, et les mousquetaires campèrent
autour. J'avais lié une amitié intime avec le comte de Coetquen qui
était dans la même compagnie. Il savait infiniment et agréablement, et
avait beaucoup d'esprit et de douceur, qui rendait son commerce très
aimable. Avec cela assez particulier et encore plus paresseux,
extrêmement riche par sa mère, qui était une fille de Saint-Malo, et
point de père. Ce soir-là de Marienbourg, il nous devait donner à souper
à plusieurs. J'allai de bonne heure à sa tente où je le trouvai sur son
lit, d'où je le chassai en folâtrant, et me couchai dessus en sa place,
en présence de plusieurs de nous autres et de quelques officiers.
Coetquen en badinant prit son fusil qu'il comptait déchargé, et me
couche en joue. Mais la surprise fut grande lorsqu'on entendit le coup
partir. Heureusement pour moi, j'étais, en ce moment, couché tout à
plat. Trois balles passèrent à trois doigts par-dessus ma tête, et comme
le fusil était en joue un peu en montant, ces mêmes balles passèrent sur
la tête, mais fort près, à nos deux gouverneurs qui se promenaient
derrière la tente. Coetquen se trouva mal du malheur qu'il avait pensé
causer\,; nous eûmes toutes les peines du monde à le remettre, et il
n'en put bien revenir de plusieurs jours. Je rapporte ceci pour une
leçon qui doit apprendre à ne badiner jamais avec les armes.

Le pauvre garçon, pour achever de suite ce qui le regarde, ne survécut
pas longtemps. Il entra bientôt dans le régiment du roi, et sur le point
de l'aller joindre au printemps suivant, il me vint conter qu'il s'était
fait dire sa bonne aventure par une femme nommée la du Perchoir, qui en
faisait secrètement métier à Paris\,; qu'elle lui avait dit qu'il serait
noyé et bientôt. Je le grondai d'une curiosité si dangereuse et si
folle, et je me flattai de l'ignorance de ces sortes de personnes, et
que celle-là en avait jugé de la sorte sur la physionomie effectivement
triste et sinistre de mon ami, qui était très désagréablement laid. Il
partit peu de jours après et trouva un autre homme de ce métier à
Amiens, qui lui fit la même prédiction\,; et, en marchant avec le
régiment du roi pour joindre l'armée, il voulut abreuver son cheval dans
l'Escaut et s'y noya, en présence de tout le régiment, sans avoir pu
être secouru. J'y eus un extrême regret, et ce fut pour ses amis et pour
sa famille une perte irréparable. Il n'avait que deux sœurs, dont l'une
épousa le fils aîné de M. de Montchevreuil et l'autre s'était faite
religieuse au Calvaire.

Les mousquetaires m'ont entraîné trop loin\,: avant de continuer, il
faut rétrograder et n'oublier pas deux mariages faits à la cour au
commencement de cette année, le premier prodigieux, le 18 février\,;
l'autre, un mois après.

\hypertarget{chapitre-ii.}{%
\chapter{CHAPITRE II.}\label{chapitre-ii.}}

1692

~

\relsize{-1}

{\textsc{Mariage de M. le duc de Chartres.}} {\textsc{- Cause de la
préséance des princes lorrains sur les ducs à la promotion de 1688.}}
{\textsc{- Premiers commencements de l'abbé Dubois, depuis cardinal et
premier ministre.}} {\textsc{- Appartement.}} {\textsc{- Fortune de
Villars père.}} {\textsc{- Maréchale de Rochefort.}} {\textsc{- Comte et
comtesse de Mailly.}} {\textsc{- Marquis d'Arcy, et comte de
Fontaine-Martel et sa femme.}} \relsize{1}

~

Le roi, occupé de l'établissement de ses bâtards, qu'il agrandissait de
jour en jour, avait marié deux de ses filles à deux princes du sang.
M\textsuperscript{me} la princesse de Conti, seule fille du roi et de
M\textsuperscript{me} de La Vallière, était veuve sans enfants\,;
l'autre, fille aînée du roi et de M\textsuperscript{me} de Montespan,
avait épousé M. le Duc \footnote{On appelait ordinairement M. le Duc le
  fils aîné du prince de Condé. Il s'agit ici de Louis de Bourbon, né le
  11 octobre 1668, mort le 4 mars 1710.}. Il y avait longtemps que
M\textsuperscript{me} de Maintenon, encore plus que le roi, ne songeait
qu'à les élever de plus en plus\,; et que tous deux voulaient marier
M\textsuperscript{lle} de Blois, seconde fille du roi et de
M\textsuperscript{me} de Montespan, à M. le duc de Chartres. C'était le
propre et l'unique neveu du roi, et fort au-dessus des princes du sang
par son rang de petit-fils de France et par la cour que tenait Monsieur.
Le mariage des deux princes du sang, dont je viens de parler, avait
scandalisé tout le monde. Le roi ne l'ignorait pas, et il jugeait par là
de l'effet d'un mariage sans proportion plus éclatant. Il y avait déjà
quatre ans qu'il le roulait dans son esprit, et qu'il en avait pris les
premières mesures. Elles étaient d'autant plus difficiles que Monsieur
était infiniment attaché à tout ce qui était de sa grandeur, et que
Madame était d'une nation qui abhorrait la bâtardise et les
mésalliances, et d'un caractère à n'oser se promettre de lui faire
jamais goûter ce mariage.

Pour vaincre tant d'obstacles, le roi s'adressa à M. le Grand
\footnote{Ce titre désignait le grand écuyer, qui était alors Louis de
  Lorraine, comte d'Armagnac, né en 1641, mort en 1718.}, qui était de
tout temps dans sa familiarité, pour gagner le chevalier de Lorraine,
son frère, qui de tout temps aussi gouvernait Monsieur. Sa figure avait
été charmante. Le goût de Monsieur n'était pas celui des femmes, et il
ne s'en cachait même pas\,; ce même goût lui avait donné le chevalier de
Lorraine pour maître, et il le demeura toute sa vie. Les deux frères ne
demandèrent pas mieux que de faire leur cour au roi par un endroit si
sensible, et d'en profiter pour eux-mêmes en habiles gens. Cette
ouverture se faisait dans l'été 1688. Il ne restait pas au plus une
douzaine de chevaliers de l'ordre\,; chacun voyait que la promotion ne
se pouvait plus guère reculer. Les deux frères demandèrent d'en être, et
d'y précéder les ducs. Le roi, qui pour cette prétention n'avait encore
donné l'ordre à aucun Lorrain, eut peine à s'y résoudre\,; mais les deux
frères surent tenir ferme\,; ils l'emportèrent, et le chevalier de
Lorraine, ainsi payé d'avance, répondit du consentement de Monsieur au
mariage, et des moyens d'y faire venir Madame et M. le duc de Chartres.

Ce jeune prince avait été mis entre les mains de Saint-Laurent au sortir
de celles des femmes. Saint-Laurent était un homme de peu,
sous-introducteur des ambassadeurs chez Monsieur et de basse mine, mais,
pour tout dire en un mot, l'homme de son siècle le plus propre à élever
un prince et à former un grand roi. Sa bassesse l'empêcha d'avoir un
titre pour cette éducation\,; son extrême mérite l'en fit laisser seul
maître\,; et quand la bienséance exigea que le prince eût un gouverneur,
ce gouverneur ne le fut qu'en apparence, et Saint-Laurent toujours dans
la même confiance et dans la même autorité.

Il était ami du curé de Saint-Eustache et lui-même grand homme de bien.
Ce curé avait un valet qui s'appelait Dubois, et qui l'ayant été du
sieur\ldots. \footnote{Le nom est en blanc dans le manuscrit.} qui avait
été docteur de l'archevêque de Reims Le Tellier, lui avait trouvé de
l'esprit, l'avait fait étudier, et ce valet savait infiniment de
belles-lettres et même d'histoire\,; mais c'était un valet qui n'avait
rien, et qui après la mort de ce premier maître était entré chez le curé
de Saint-Eustache. Ce curé, content de ce valet pour qui il ne pouvait
rien faire, le donna à Saint-Laurent, dans l'espérance qu'il pourrait
mieux pour lui. Saint-Laurent s'en accommoda, et peu à peu s'en servit
pour l'écritoire d'étude de M. le duc de Chartres\,; de là, voulant s'en
servir à mieux, il lui fit prendre le petit collet pour le décrasser, et
de cette sorte l'introduisit à l'étude du prince pour lui aider à
préparer ses leçons, à écrire ses thèmes, à le soulager lui-même, à
chercher les mots dans le dictionnaire. Je l'ai vu mille fois dans ces
commencements, lorsque j'allais jouer avec M. de Chartres. Dans les
suites Saint-Laurent devenant infirme, Dubois faisait la leçon, et la
faisait fort bien, et néanmoins plaisant au jeune prince.

Cependant Saint-Laurent mourut et très brusquement. Dubois, par intérim,
continua à faire la leçon\,; mais depuis qu'il fut devenu presque abbé,
il avait trouvé moyen de faire sa cour au chevalier de Lorraine et au
marquis d'Effiat, premier écuyer de Monsieur, amis intimes, et ce
dernier ayant aussi beaucoup de crédit sur son maître. De faire Dubois
précepteur, cela ne se pouvait proposer de plein saut\,; mais ses
protecteurs, auxquels il eut recours, éloignèrent le choix d'un
précepteur, puis se servirent des progrès du jeune prince pour ne le
point changer de main, et laisser faire Dubois\,; enfin ils le
bombardèrent précepteur. Je ne vis jamais homme si aise ni avec plus de
raison. Cette extrême obligation, et plus encore le besoin de se
soutenir, l'attacha de plus en plus à ses protecteurs, et ce fut de lui
que le chevalier de Lorraine se servit pour gagner le consentement de M.
de Chartres à son mariage.

Dubois avait gagné sa confiance\,; il lui fut aisé en cet âge, et avec
ce peu de connaissance et d'expérience, de lui faire peur du roi et de
Monsieur, et, d'un autre côté, de lui faire voir les cieux ouverts. Tout
ce qu'il put mettre en œuvre n'alla pourtant qu'à rompre un refus\,;
mais cela suffisait au succès de l'entreprise. L'abbé Dubois ne parla à
M. de Chartres que vers le temps de l'exécution\,; Monsieur était déjà
gagné, et dès que le roi eut réponse de l'abbé Dubois, il se hâta de
brusquer l'affaire. Un jour ou deux auparavant, Madame en eut le vent.
Elle parla à M. son fils de l'indignité de ce mariage avec toute la
force dont elle ne manquait pas, et elle en tira parole qu'il n'y
consentirait point. Ainsi faiblesse envers son précepteur, faiblesse
envers sa mère, aversion d'une part, crainte de l'autre, et grand
embarras de tous côtés.

Une après-dînée de fort bonne heure que je passais dans la galerie
haute, je vis sortir M. le duc de Chartres d'une porte de derrière de
son appartement, l'air fort empêtré, triste, suivi d'un seul exempt des
gardes de Monsieur\,; et, comme je me trouvais là, je lui demandai où il
allait ainsi si vite et à cette heure-là. Il me répondit d'un air
brusque et chagrin qu'il allait chez le roi qui l'avait envoyé quérir.
Je ne jugeai pas à propos de l'accompagner, et, me tournant à mon
gouverneur, je lui dis que je conjecturais quelque chose du mariage, et
qu'il allait éclater. Il m'en avait depuis quelques jours transpiré
quelque chose, et comme je jugeai bien que les scènes seraient fortes,
la curiosité me rendit fort attentif et assidu.

M. de Chartres trouva le roi seul avec Monsieur dans son cabinet, où le
jeune prince ne savait pas devoir trouver M. son père. Le roi fit des
amitiés à M. de Chartres, lui dit qu'il voulait prendre soin de son
établissement, que la guerre allumée de tous côtés lui ôtait des
princesses qui auraient pu lui convenir\,; que, de princesses du sang,
il n'y en avait point de son âge\,; qu'il ne lui pouvait mieux témoigner
sa tendresse qu'en lui offrant sa fille dont les deux sœurs avaient
épousé deux princes du sang, que cela joindrait en lui la qualité de
gendre à celle de neveu, mais que, quelque passion qu'il eût de ce
mariage, il ne le voulait point contraindre et lui laissait là-dessus
toute liberté. Ce propos, prononcé avec cette majesté effrayante si
naturelle au roi, à un prince timide et dépourvu de réponse, le mit hors
de mesure. Il crut se tirer d'un pas si glissant en se rejetant sur
Monsieur et Madame, et répondit en balbutiant que le roi était le
maître, mais que sa volonté dépendait de la leur. « Cela est bien à
vous, répondit le roi, mais dès que vous y consentez, votre père et
votre mère ne s'y opposeront pas\,;\,» et se tournant à Monsieur\,: «
Est-il pas vrai, mon frère\,?\,» Monsieur consentit comme il l'avait
déjà fait seul avec le roi, qui tout de suite dit qu'il n'était donc
plus question que de Madame, et qui sur-le-champ l'envoya chercher\,; et
cependant se mit à causer avec Monsieur, qui tous deux ne firent pas
semblant de s'apercevoir du trouble et de l'abattement de M. de
Chartres.

Madame arriva, à qui d'entrée le roi dit qu'il comptait bien qu'elle ne
voudrait pas s'opposer à une affaire que Monsieur désirait, et que M. de
Chartres y consentait\,: que c'était son mariage avec
M\textsuperscript{lle} de Blois, qu'il avouait qu'il désirait avec
passion, et ajouta courtement les mêmes choses qu'il venait de dire à M.
le duc de Chartres, le tout d'un air imposant, mais comme hors de doute
que Madame pût n'en pas être ravie, quoique plus que certain du
contraire. Madame, qui avait compté sur le refus dont M. son fils lui
avait donné parole, qu'il lui avait même tenue autant qu'il avait pu par
sa réponse si embarrassée et si conditionnelle, se trouva prise et
muette. Elle lança deux regards furieux à Monsieur et à M. de Chartres,
dit que, puisqu'ils le voulaient bien, elle n'avait rien à y dire, fit
une courte révérence et s'en alla chez elle. M. son fils l'y suivit
incontinent, auquel, sans donner le moment de lui dire comment la chose
s'était passée, elle chanta pouille avec un torrent de larmes, et le
chassa de chez elle.

Un peu après, Monsieur, sortant de chez le roi, entra chez elle, et
excepté qu'elle ne l'en chassa pas comme son fils, elle ne le ménagea
pas davantage\,; tellement qu'il sortit de chez elle très confus, sans
avoir eu loisir de lui dire un seul mot. Toute cette scène était finie
sur les quatre heures de l'après-dînée, et le soir il y avait
appartement, ce qui arrivait l'hiver trois fois la semaine, les trois
autres jours comédie, et le dimanche rien.

Ce qu'on appelait appartement était le concours de toute la cour, depuis
sept heures du soir jusqu'à dix que le roi se mettait à table, dans le
grand appartement, depuis un des salons du bout de la grande galerie
jusque vers la tribune de la chapelle. D'abord, il y avait une
musique\,; puis des tables par toutes les pièces toutes prêtes pour
toutes sortes de jeux\,; un lansquenet où Monseigneur et Monsieur
jouaient toujours\,; un billard\,: en un mot, liberté entière de faire
des parties avec qui on voulait, et de demander des tables si elles se
trouvaient toutes remplies\,; au delà du billard, il y avait une pièce
destinée aux rafraîchissements, et tout parfaitement éclairé. Au
commencement que cela fut établi, le roi y allait et y jouait quelque
temps, mais dès lors il y avait longtemps qu'il n'y allait plus, mais il
voulait qu'on y fût assidu, et chacun s'empressait à lui plaire. Lui
cependant passait les soirées chez M\textsuperscript{me} de Maintenon à
travailler avec différents ministres les uns après les autres.

Fort peu après la musique finie, le roi envoya chercher à l'appartement
Monseigneur et Monsieur, qui jouaient déjà au lansquenet\,; Madame qui à
peine regardait une partie d'hombre auprès de laquelle elle s'était
mise\,; M. de Chartres qui jouait fort tristement aux échecs, et
M\textsuperscript{lle} de Blois qui à peine avait commencé à paraître
dans le monde, qui ce soir-là était extraordinairement parée et qui
pourtant ne savait et ne se doutait même de rien, si bien que,
naturellement fort timide et craignant horriblement le roi, elle se crut
mandée pour essuyer quelque réprimande, et était si tremblante que
M\textsuperscript{me} de Maintenon la prit sur ses genoux où elle la
tint toujours la pouvant à peine rassurer. À ce bruit de ces personnes
royales mandées chez M\textsuperscript{me} de Maintenon et
M\textsuperscript{lle} de Blois avec elle, le bruit du mariage éclata à
l'appartement, en même temps que le roi le déclara dans ce particulier.
Il ne dura que quelques moments, et les mêmes personnes revinrent à
l'appartement, où cette déclaration fut rendue publique. J'arrivai dans
ces premiers instants. Je trouvai le monde par pelotons, et un grand
étonnement régner sur tous les visages. J'en appris bientôt la cause qui
ne me surprit pas, par la rencontre que j'avais faite au commencement de
l'après-dînée.

Madame se promenait dans la galerie avec Châteauthiers, sa favorite et
digne de l'être\,; elle marchait à grands pas, son mouchoir à la main,
pleurant sans contrainte, parlant assez haut, gesticulant et
représentant bien Cérès après l'enlèvement de sa fille Proserpine, la
cherchant en fureur et la redemandant à Jupiter. Chacun, par respect,
lui laissait le champ libre et ne faisait que passer pour entrer dans
l'appartement. Monseigneur et Monsieur s'étaient remis au lansquenet. Le
premier me parut tout à son ordinaire. Jamais rien de si honteux que le
visage de Monsieur, ni de si déconcerté que toute sa personne, et ce
premier état lui dura plus d'un mois. M. son fils paraissait désolé, et
sa future dans un embarras et une tristesse extrême. Quelque jeune
qu'elle fût, quelque prodigieux que fût ce mariage, elle en voyait et en
sentait toute la scène, et en appréhendait toutes les suites. La
consternation parut générale, à un très petit nombre de gens près. Pour
les Lorrains ils triomphaient. La sodomie et le double adultère les
avaient bien servis en les servant bien eux-mêmes. Ils jouissaient de
leurs succès, comme ils en avaient toute honte bue\,; ils avaient raison
de s'applaudir.

La politique rendit donc cet appartement languissant en apparence, mais
en effet vif et curieux. Je le trouvai court dans sa durée ordinaire\,;
il finit par le souper du roi, duquel je ne voulus rien perdre. Le roi y
parut tout comme à son ordinaire. M. de Chartres était auprès de Madame
qui ne le regarda jamais, ni Monsieur. Elle avait les yeux pleins de
larmes qui tombaient de temps en temps, et qu'elle essuyait de même,
regardant tout le monde comme si elle eût cherché à voir quelle mine
chacun faisait. M. son fils avait aussi les yeux bien rouges, et tous
deux ne mangèrent presque rien. Je remarquai que le roi offrit à Madame
presque de tous les plats qui étaient devant lui, et qu'elle les refusa
tous d'un air de brusquerie qui jusqu'au bout ne rebuta point l'air
d'attention et de politesse du roi pour elle.

Il fut encore fort remarqué qu'au sortir de table et à la fin de ce
cercle debout d'un moment dans la chambre du roi, il fit à Madame une
révérence très marquée et basse, pendant laquelle elle fit une pirouette
si juste, que le roi en se relevant ne trouva plus que son dos, et
{[}elle{]} avancée d'un pas vers la porte.

Le lendemain toute la cour fut chez Monsieur, chez Madame et chez M. le
duc de Chartres, mais sans dire une parole\,; on se contentait de faire
la révérence, et tout s'y passa en parfait silence. On alla ensuite
attendre à l'ordinaire la levée du conseil dans la galerie et la messe
du roi. Madame y vint. M. son fils s'approcha d'elle comme il faisait
tous les jours pour lui baiser la main. En ce moment Madame lui appliqua
un soufflet si sonore qu'il fut entendu de quelques pas, et qui, en
présence de toute la cour, couvrit de confusion ce pauvre prince, et
combla les infinis spectateurs, dont j'étais, d'un prodigieux
étonnement. Ce même jour l'immense dot fut déclarée, et le jour suivant
le roi alla rendre visite à Monsieur et à Madame, qui se passa fort
tristement, et depuis on ne songea plus qu'aux préparatifs de la noce.

Le dimanche gras, il y eut grand bal réglé chez le roi, c'est-à-dire
ouvert par un branle, suivant lequel chacun dansa après. J'allai ce
matin-là chez Madame qui ne put se tenir de me dire, d'un ton aigre et
chagrin, que j'étais apparemment bien aise des bals qu'on allait avoir,
et que cela était de mon âge, mais qu'elle qui était vieille voudrait
déjà les voir bien loin. Mgr le duc de Bourgogne y dansa pour la
première fois, et mena le branle avec Mademoiselle. Ce fut aussi la
première fois que je dansai chez le roi, et je menai
M\textsuperscript{lle} de Sourches, fille du grand prévôt, qui dansait
très bien. Tout le monde y fut fort magnifique.

Ce fut, un peu après, les fiançailles et la signature du contrat de
mariage, dans le cabinet du roi, en présence de toute la cour. Ce même
jour la maison de la future duchesse de Chartres fui déclarée\,; le roi
lui donna un chevalier d'honneur et une dame d'atours, jusqu'alors
réservés aux filles de France, et une dame d'honneur qui répondit à une
si étrange nouveauté. M. de Villars fut chevalier d'honneur, la
maréchale de Rochefort dame d'honneur, la comtesse de Mailly dame
d'atours, et le comte de Fontaine-Martel, premier écuyer.

Villars était petit-fils d'un greffier de Coindrieu, l'homme de France
le mieux fait et de la meilleure mine. On se battait fort de son
temps\,; il était brave et adroit aux armes, et avait acquis de la
réputation fort jeune en des combats singuliers. Cela couvrit sa
naissance aux yeux de M. de Nemours, qui aimait à s'attacher des braves,
et qui le prit comme gentilhomme. Il l'estima même assez pour le prendre
pour second au duel qu'il eut contre M. de Beaufort, son beau-frère, qui
le tua, tandis que Villars avait tout l'avantage sur son adversaire.

Cette mort renvoya Villars chez lui\,; il n'y fut pas longtemps que M.
le prince de Conti se l'attacha aussi comme un gentilhomme à lui. Il
venait de quitter le petit collet. Il était faible et contrefait, et
souvent en butte aux trop fortes railleries de M. le Prince son frère\,;
il projeta de s'en tirer par un combat, et ne sachant avec qui, il
imagina d'appeler le duc d'York, maintenant le roi Jacques d'Angleterre,
qui est à Saint-Germain et qui pour lors était en France. Cette belle
idée et le souvenir du combat de M. de Nemours lui fit prendre Villars.
Il ne put tenir son projet si caché qu'il ne fût découvert, et aussitôt
rompu par la honte qui lui en fut faite, n'ayant jamais eu la plus
petite chose à démêler avec le duc d'York. Dans les suites il prit
confiance en Villars, alors que le cardinal Mazarin songea à lui donner
sa nièce. Ce fut de Villars dont il se servit, et par qui il fit ce
mariage. On sait combien il fut heureux et sage ensuite. Villars devint
le confident des deux époux et leur lien avec le cardinal, et tout cela
avec toute la sagacité et la probité possible.

Une telle situation le mit fort dans le monde, et dans un monde fort
au-dessus de lui, parmi lequel quelque fortune qu'il ait faite depuis,
il ne s'est jamais méconnu. Sa figure lui donna entrée chez les dames\,;
il était galant et discret, et cette voie ne lui fut pas inutile. Il
plut à M\textsuperscript{me} Scarron qui, sur le trône où elle sut
régner longtemps depuis, n'a jamais oublié ces sortes d'amitiés si
librement intimes. Villa fut employé auprès des princes d'Allemagne et
d'Italie, et fut après ambassadeur en Savoie, en Danemark et en Espagne,
et réussit et se fit estimer et aimer partout. Il eut ensuite une place
de conseiller d'État d'épée, et, au scandale de l'ordre du Saint-Esprit,
il fut de la promotion de 1698. Sa femme était sœur du père du maréchal
de Bellefonds, qui avait de l'esprit infiniment, plaisante, salée,
ordinairement méchante\,: tous deux fort pauvres, toujours à la cour, où
ils avaient beaucoup d'amis et d'amies considérables.

La maréchale de Rochefort était d'une autre étoffe et de la maison de
Montmorency, de la branche de Laval. Son père, second fils du maréchal
de Boisdauphin, avec très peu de bien, épousa pour sa bonne mine la
marquise de Coislin, veuve du colonel général des Suisses et mère du duc
et du chevalier de Coislin, et de l'évêque d'Orléans, premier aumônier
du roi. Elle était fille aînée du chancelier Séguier et sœur aînée de la
duchesse de Verneuil, mère en premières noces du duc de Sully et de la
duchesse du Lude. La maréchale de Rochefort naquit posthume, seule de
son lit, en 1646, et M. de Boisdauphin, frère aîné de son père, n'eut
point de postérité. Elle épousa en 1662 le marquis, depuis maréchal, de
Rochefort-Alloigny, peu de mois après que l'héritière de Souvré, sa
cousine issue de germaine, eut épousé M. de Louvois.

Cette héritière était fille du fils de M. de Courtenvaux, lequel était
fils du maréchal de Souvré et frère de la célèbre M\textsuperscript{me}
de Sablé, mère de M. de Laval, père de la maréchale de Rochefort. M. de
Rochefort, qu'elle épousa, était ami intime de M. Le Tellier et de M. de
Louvois qui lui firent rapidement sa fortune. Il mourut capitaine des
gardes du corps, gouverneur de Lorraine, et désigné général d'armée, en
allant en prendre le commandement au printemps de 1676. Il n'y avait pas
un an qu'il était maréchal de France de la promotion qui suivit la mort
de M. de Turenne. Cette même protection avait fait sa femme dame du
palais de la reine.

Elle était belle, encore plus piquante, toute faite pour la cour, pour
les galanteries, pour les intrigues, l'esprit du monde à force d'en
être, peu ou point d'ailleurs, et toute la bassesse nécessaire pour être
de tout et en quelque sorte que ce fût. M. de Louvois la trouva fort à
son gré, et elle s'accommoda fort de sa bourse et de figurer par cette
intimité. Lorsque le roi eut et changea de maîtresses, elle fut toujours
leur meilleure amie\,; et quand il lia avec M\textsuperscript{me} de
Soubise, c'était chez la maréchale qu'elle allait, et chez qui elle
attendait Bontems à porte fermée, qui la menait par des détours chez le
roi. La maréchale elle-même me l'a conté, et comme quoi elle fut un jour
embarrassée à se défaire du monde que M\textsuperscript{me} de Soubise
trouva chez elle, qui n'avait pas eu le temps de l'avertir\,; et comme
elle mourait de peur que Bontems ne s'en retournât, et que le
rendez-vous ne manquât, s'il arrivait avant qu'elle se fût défaite de sa
compagnie.

Elle fut donc amie de M\textsuperscript{me}s de La Vallière, de
Montespan et de Soubise, et surtout de la dernière, jusqu'au temps où
j'ai connu la maréchale, et le sont toujours demeurées intimement. Elle
le devint après de M\textsuperscript{me} de Maintenon, qu'elle avait
connue chez M\textsuperscript{me} de Montespan, et à qui elle s'attacha
à mesure qu'elle vit arriver et croître sa faveur. Elle était telle au
mariage de Monseigneur que le roi n'eut pas honte de la faire dame
d'atours de la nouvelle Dauphine\,; mais n'osant aussi l'y mettre en
plein, il ne put trouver mieux que la maréchale de Rochefort pour y être
en premier, et pour s'accommoder d'une compagne si étrangement inégale,
et avoir cependant pour elle toutes les déférences que sa faveur
exigeait. Elle y remplit parfaitement les espérances qu'on en avait
conçues, et sut néanmoins avec cela se concilier l'amitié et la
confiance de M\textsuperscript{me} la Dauphine jusqu'à sa mort,
quoiqu'elle ne pût souffrir M\textsuperscript{me} de Maintenon, ni
M\textsuperscript{me} de Maintenon cette pauvre princesse.

Une femme si connue du roi, et si fort à toutes mains, était son vrai
fait pour mettre auprès de M\textsuperscript{me} la duchesse de Chartres
qui entrait si fort de traverse dans une famille tellement au-dessus
d'elle, et avec une belle-mère outrée, et qui n'était pas femme à
contraindre ses mépris. Si une maréchale de France, et de cette qualité,
avait surpris le monde dans la place de dame d'atours de
M\textsuperscript{me} la Dauphine, ce fut bien un autre étonnement de la
voir dame d'honneur d'une bâtarde, petite-fille de France. Aussi se
fit-elle prier avec cette pointe de gloire qui lui prenait quelquefois,
mais qui pliait le moment d'après. Elle était fort tombée par la mort de
M. de Louvois, quoique M. de Barbezieux eût pour elle les mêmes égards
qu'avait eus son père. Tout ce qu'elle gagna à ce premier refus fut une
promesse d'être dame d'atours lorsqu'on marierait Mgr le duc de
Bourgogne.

M\textsuperscript{me} de Mailly était une demoiselle de Poitou qui
n'avait pas de chausses, fille de Saint-Hermine, cousin issu de germain
de M\textsuperscript{me} de Maintenon. Elle l'avait fait venir de sa
province demeurer chez elle à Versailles, et l'avait mariée, moitié gré,
moitié force, au comte de Mailly, second fils du marquis et de la
marquise de Mailly, héritiers de Montcavrel qui, mariés avec peu de
biens, étaient venus à bout avec l'âge, à force d'héritages et de
procès, d'avoir ce beau marquisat de Nesle, de bâtir l'hôtel de Mailly,
vis-à-vis le pont Royal, et de faire une très puissante maison. Le
marquis de Nesle, leur fils aîné, avait épousé malgré eux la dernière de
l'illustre maison de Coligny. Il était mort devant Philippsbourg en
1688, maréchal de camp, et n'avait laissé qu'un fils et fine fille.
C'était à ce fils que les marquis et marquise de Mailly voulaient
laisser leurs grands biens. Ils avaient troqué un fils et une fille, et
fait prêtre malgré lui un autre fils\,; une autre fille avait épousé
malgré eux l'aîné de la maison de Mailly.

Le comte de Mailly qui leur avait échappé, ils ne voulaient lui rien
donner ni le marier. C'était un homme de beaucoup d'ambition, qui se
présentait à tout, aimable s'il n'avait pas été si audacieux, et qui
avait le nez tourne la fortune. C'était une manière de favori de
Monseigneur. Avec ces avances il se voulut appuyer de
M\textsuperscript{me} de Maintenon pour sa fortune et pour obtenir un
patrimoine de son père\,: c'est ce qui fit le mariage en faisant espérer
monts et merveilles aux vieux Mailly qui voulaient du présent, et
sentaient en gens d'esprit que le mariage fait, on les laisserait là,
comme il arriva. Mais quand on a compté sur un mariage de cette
autorité, il ne se trouve plus de porte de derrière, et il leur fallut
sauter le bâton d'assez mauvaise grâce. La nouvelle comtesse de Mailly
avait apporté tout le gauche de sa province dont, faute d'esprit, elle
ne sut se défaire\,; et enta dessus toute la gloire de la
toute-puissante faveur de M\textsuperscript{me} de Maintenon\,: bonne
femme et sûre amie d'ailleurs, quand elle l'était noble, magnifique,
mais glorieuse à l'excès et désagréable avec le gros du monde, avec peu
de conduite et fort particulière. Les Mailly trouvèrent cette place avec
raison bien mauvaise, mais il la fallut avaler.

M. de Fontaine-Martel, de bonne et ancienne maison des Martel et des
Claire de Normandie, était un homme perdu de goutte et pauvre. Il était
frère unique du marquis d'Arcy, dernier gouverneur de M. le duc de
Chartres, qui avait acquis une grande estime par la conduite qu'il lui
avait fait tenir à la guerre et dans le monde, qui y était lui-même fort
estimé, et qui s'était fait auparavant ce dernier emploi une grande
réputation dans ses ambassades. Il était chevalier de l'ordre et
conseiller d'État d'épée, et mourut des fatigues de l'armée et de son
emploi sans avoir été marié, au printemps de 1694, à Valenciennes. Ce
fut à cette qualité de frère de M. d'Arcy que la charge fut donnée. Sa
femme était fille posthume de M. de Bordeaux, mort ambassadeur de France
en Angleterre, et de M\textsuperscript{me} de Bordeaux, qui, pour une
bourgeoise, était extrêmement du monde et amie intime de beaucoup
d'hommes et de femmes distingués. Elle avait été belle et galante\,;
elle en avait conservé le goût dans sa vieillesse, qui lui avait
conservé aussi des amies considérables. Elle avait élevé sa fille unique
dans les mêmes mœurs\,: l'une et l'autre avaient de l'esprit et du
manège. M\textsuperscript{me} de Fontaine-Martel s'était ainsi trouvée
naturellement du grand monde\,; elle était fort de la cour de Monsieur.
La place de confiance que M. d'Arcy, son beau-frère, y remplit si
dignement lui donna de la considération, et tout cela ensemble leur
valut cette lucrative charge.

Le lundi gras, toute la royale noce et les époux superbement parés se
rendirent un peu avant midi dans le cabinet du roi, et de là à la
chapelle. Elle était rangée à l'ordinaire comme pour la messe du roi,
excepté qu'entre son prie-Dieu et l'autel étaient deux carreaux pour les
mariés, qui tournaient le dos au roi. Le cardinal de Bouillon tout
revêtu y arriva en même temps de la sacristie, les maria et dit la
messe. Le poêle fut tenu par le grand maître et par le maître des
cérémonies, Blainville et Sainctot. De la chapelle on alla tout de suite
se mettre à table. Elle était en fer à cheval. Les princes et les
princesses du sang y étaient placés à droite et à gauche, suivant leur
rang, terminés par les deux bâtards du roi, et pour la première fois,
après eux la duchesse de Verneuil\,; tellement que M. de Verneuil,
bâtard d'Henri IV, devint ainsi prince du sang, tant d'années après sa
mort sans s'être jamais douté de l'être. Le duc d'Uzès le trouva si
plaisant, qu'il se mit à marcher devant elle, criant tant qu'il
pouvait\,: « Place, place à M\textsuperscript{me} Charlotte
Séguier\,!\,» Aucune duchesse ne fit sa cour à ce dîner que la duchesse
de Sully et la duchesse du Lude, fille et belle-fille de
M\textsuperscript{me} de Verneuil, ce que toutes les autres trouvèrent
si mauvais qu'elles n'osèrent plus y retourner. L'après-dînée, le roi et
la reine d'Angleterre vinrent à Versailles avec leur cour. Il y eut
grande musique, grand jeu, où le roi fut presque toujours fort paré et
fort aise, son cordon bleu par-dessus comme la veille. Le souper fut
pareil au dîner. Le roi d'Angleterre ayant la reine sa femme à sa droite
et le roi à sa gauche ayant chacun leur cadenas \footnote{Coffret de
  métal précieux contenant la cuiller, la fourchette et le couteau. Le
  cadenas était un signe distinctif des princes et des seigneurs du plus
  haut rang.}. Ensuite on mena les mariés dans l'appartement de la
nouvelle duchesse de Chartres, à qui la reine d'Angleterre donna la
chemise, et le roi d'Angleterre à M. de Chartres, après s'en être
défendu, disant qu'il était trop malheureux. La bénédiction du lit se
fit par le cardinal de Bouillon, qui se fit attendre un quart d'heure,
ce qui fit dire que ces airs-là ne valaient rien à prendre pour qui
revenait comme lui d'un long exil, où la folie qu'il avait eue de ne pas
donner la bénédiction nuptiale à M\textsuperscript{me} la duchesse s'il
n'était admis au festin royal, l'avait fait envoyer.

Le mardi gras grande toilette de M\textsuperscript{me} de Chartres, où
le roi et la reine d'Angleterre vinrent, et où le roi se trouva avec
toute la cour\,; la messe du roi ensuite\,; puis le dîner comme la
veille. On avait dès le matin renvoyé M\textsuperscript{me} de Verneuil
à Paris, trouvant qu'elle en avait eu sa suffisance. L'après-dînée, le
roi s'enferma avec le roi et la reine d'Angleterre\,; et puis grand bal
comme le précédent, excepté que la nouvelle duchesse de Chartres y fut
menée par Mgr le duc de Bourgogne. Chacun eut le même habit et la même
danseuse qu'au précédent.

Je ne puis passer sous silence une aventure fort ridicule qui arriva au
même homme à tous les deux. C'était le fils de Montbron, qui n'était pas
fait pour danser chez le roi, non plus que son père pour être chevalier
de l'ordre, qui le fut pourtant en 1688, et qui était gouverneur de
Cambrai, lieutenant général, et seul lieutenant général de Flandre, sous
un nom qu'il ne put jamais prouver être le sien. Ce jeune homme, qui
n'avait encore que peu ou point paru à la cour, menait
M\textsuperscript{lle} de Mareuil, fille de la dame d'honneur de
M\textsuperscript{me} la Duchesse (les bâtards de cette grande maison
des Mareuil) et qui, non plus que lui, ne devait pas être admise à cet
honneur. On lui avait demandé s'il dansait bien, et il avait répondu
avec une confiance qui donna envie de trouver qu'il dansait mal\,: on
eut contentement. Dès la première révérence il se déconcerta. Plus de
cadence dès les premiers pas. Il crut la rattraper et couvrit son défaut
par des airs penchés et un haut port de bras\,; ce ne fut qu'un ridicule
de plus qui excita une risée qui en vint aux éclats, et qui, malgré le
respect de la présence du roi qui avait peine à s'empêcher de rire,
dégénéra enfin en véritable huée. Le lendemain, au lieu de s'enfuir ou
de se taire, il s'excusa sur la présence du roi qui l'avait étourdi, et
promit merveilles pour le bal qui devait suivre. Il était de mes amis,
et j'en souffrais. Je l'aurais même averti si le sort tout différent que
j'avais eu ne m'eût fait craindre que mon avis n'eût pas de grâce. Dès
qu'au second bal on le vit pris à danser, voilà les uns en pied, les
plus reculés à l'escalade, et la huée si forte qu'elle fut poussée aux
battements de mains. Chacun, et le roi même, riait de tout son cœur, et
la plupart en éclats, en telle sorte, que je ne crois pas que personne
ait jamais rien essuyé de semblable. Aussi disparut-il incontinent
après, et ne se remontra-t-il de longtemps. Il eut depuis le régiment
Dauphin infanterie, et mourut tôt après sans avoir été marié, Il avait
beaucoup d'honneur et de valeur, et ce fut dommage. Ce fut le dernier de
ces faux entés sur Montbron, c'est-à-dire son père qui lui survécut.

\hypertarget{chapitre-iii.}{%
\chapter{CHAPITRE III.}\label{chapitre-iii.}}

1692

1693

~

\relsize{-1}

{\textsc{Mariage du duc du Maine.}} {\textsc{- M\textsuperscript{me} de
Saint-Vallery.}} {\textsc{- M. de Montchevreuil, sa femme et leur
fortune.}} {\textsc{- Année 1693. Duchesse douairière d'Hanovre et ses
filles sans rang, à grands airs.}} {\textsc{- Causes de sa retraite en
Allemagne et de la haute fortune de sa seconde fille.}} {\textsc{- Ma
sortie des mousquetaires pour une compagnie de cavalerie dans le
Royal-Roussillon.}} {\textsc{- Promotion de sept maréchaux de France.}}
{\textsc{- Duc de Choiseul pourquoi laissé.}} {\textsc{- Mort de
Mademoiselle et ses donations libres et forcées.}} {\textsc{-
Distinction du rang de petite-fille de France procurée par mon père.}}
\relsize{1}

~

Le mercredi des cendres mit fin à toutes ces tristes réjouissances de
commande, et on ne parla plus que de celles qu'on attendait. M. du Maine
voulut se marier. Le roi l'en détournait et lui disait franchement que
ce n'était point à des espèces comme lui à faire lignée\,; mais, pressé
par M\textsuperscript{me} de Maintenon qui l'avait élevé et qui eut
toujours pour lui le faible de nourrice, il se résolut de l'appuyer du
moins de la maison de Condé et de le marier à une fille de M. le Prince,
qui en ressentit une joie extrême. Il voyait croître de jour en jour le
rang, le crédit, les alliances des bâtards. Celle-ci ne lui était pas
nouvelle depuis le mariage de son fils, mais elle le rapprochait
doublement du roi, et venait incontinent après le mariage de M. le duc
de Chartres. Madame en fut encore bien plus aise. Elle avait
horriblement appréhendé que le roi, lui ayant enlevé son fils, ne portât
encore les yeux sur sa fille\,; et ce mariage de celle de M. le Prince
lui parut une délivrance.

Il en avait trois à choisir. Un pouce de taille de plus qu'avait la
seconde lui valut la préférence. Toutes trois étoient extrêmement
petites\,; la première était belle et pleine d'esprit et de raison.
L'incroyable contrainte, pour ne rien dire de pis, où l'humeur de M, le
Prince tenait tout ce qui était réduit sous son joug, donna un extrême
crève-cœur à cette aînée. Elle sut le supporter avec constance, avec
sagesse, avec hauteur, et se fit admirer dans toute sa conduite. Mais
elle le paya chèrement\,: cet effort lui renversa la santé, qui fut
toujours depuis languissante.

Le roi, d'accord du choix avec M. le Prince, alla à Versailles faire la
demande à M\textsuperscript{me} la Princesse dans son appartement\,; et
peu après, sur la fin du carême, les fiançailles se firent dans le
cabinet du roi. Ensuite le roi et toute la cour fut à Trianon, où il y
eut appartement et un grand souper pour quatre-vingts dames en cinq
tables, tenues chacune par le roi, Monseigneur, Monsieur, Madame, et la
nouvelle duchesse de Chartres. Le lendemain, mercredi 19 mars, le
mariage fut célébré à la messe du roi par le cardinal de Bouillon, comme
l'avait été celui de M. le duc de Chartres. Le dîner fut de même et le
souper aussi\,; après, l'appartement. Le roi d'Angleterre donna la
chemise à M. du Maine. M\textsuperscript{me} de Montespan ne parut à
rien et ne signa point à ces deux contrats de mariage. Le lendemain, la
mariée reçut toute la cour sur son lit, la princesse d'Harcourt faisant
les honneurs, choisie pour cela par le roi. M\textsuperscript{me} de
Saint-Vallery fut dame d'honneur, et Montchevreuil, qui avait été
gouverneur de M. du Maine, et qui conduisait toute sa maison, continua
dans cette dernière fonction et demeura gentilhomme de sa chambre.

M\textsuperscript{me} de Saint-Vallery était fille de Montlouet, premier
écuyer de la grande écurie, petit-fils cadet de Bullion, surintendant
des finances, et elle était veuve depuis un an d'un fils cadet de
Gamaches, chevalier de l'ordre en 1661, sans enfants, par conséquent
belle-sœur de Cayeu, depuis Gamaches, duquel il y aura occasion de dire
un mot\,; et la mère de M\textsuperscript{me} de Saint-Vallery était
Rouault, cousine germaine paternelle de Gamaches, le chevalier de
l'ordre. C'était une femme grande, belle, agréable, très bien faite, de
fort peu d'esprit, à qui la douceur et une vertu jamais démentie et une
piété solide tenaient lieu de tout le reste, et la rendirent aimable et
respectée de toute la cour, où elle ne vint que malgré elle. Aussi n'y
demeura-t-elle que le moins qu'elle put. Elle s'aperçut qu'on avait
envie de sa place où tout lui déplaisait, et que M. du Maine se
radoucissait autour d'elle, ou naturellement, ou de dessein. Il n'en
fallut pas davantage pour lui faire demander à se retirer, avec la
douleur de toute la cour, que sa beauté, sa vertu, sa modestie a le
grand air de toute sa personne avaient charmée. On mit en sa place
M\textsuperscript{me} de Manneville, femme du gouverneur de Dieppe et de
la dernière duchesse de Luynes fille du chancelier d'Aligre \footnote{La
  phrase est reproduite textuellement d'après le manuscrit autographe.
  Il y a une erreur évidente. Il faut lire probablement\,: «
  M\textsuperscript{me} de Manneville, femme du gouverneur de Dieppe et
  belle-fille de la dernière duchesse de Luynes fille du chancelier
  d'Aligre. \,» En effet, Marguerite d'Aligre, fille du chancelier,
  épousa en premières noces Charles-Bonaventure, marquis de Manneville,
  et en secondes noces, Louis-Charles d'Albert, duc de Luynes. Du
  premier mariage était né Étienne de Manneville, gouverneur de Dieppe,
  qui épousa Bonne-Angélique de Mornay-Montchevreuil, dont il s'agit
  ici.}. M\textsuperscript{me} de Manneville était fille de
Montchevreuil\,; et c'était tellement leur vrai ballot, qu'on ne
comprend pas comment elle n'y avait pas été mise d'abord.

Montchevreuil était Mornay, de bonne maison, sans esprit aucun, et gueux
comme un rat d'église. Villarceaux, de même maison que lui, était un
débauché fort riche, ainsi que l'abbé son frère, avec qui il vivait.
Villarceaux entretint longtemps M\textsuperscript{me} Scarron, et la
tenait presque tout l'été à Villarceaux. Sa femme, dont la vertu et la
douceur donnaient une sorte de respect au mari, lui devint une peine de
mener cette vie en sa présence. Il proposa à son cousin Montchevreuil de
le recevoir chez lui avec sa compagnie, et qu'il mettrait la nappe pour
tous. Cela fut accepté avec joie, et ils vécurent de la sorte nombre
d'étés à Montchevreuil. La Scarron devenue reine eut cela de bon qu'elle
aima presque tous ses vieux amis dans tous les temps de sa vie. Elle
attira Montchevreuil et sa femme à la cour où les Villarceaux trop
libertins ne se pouvaient contraindre\,; elle voulut Montchevreuil pour
un des trois témoins de son mariage avec le roi\,; elle lui procura le
gouvernement de Saint-Germain en Laye, l'attacha à M. du Maine, le fit
chevalier de l'ordre avec le fils de Villarceaux, au refus du père, en
1688, qui l'aima mieux pour son fils que pour lui-même, et mit sous la
conduite de M\textsuperscript{me} de Montchevreuil
M\textsuperscript{lle} de Blois jusqu'à son mariage avec M. le duc de
Chartres, après avoir été gouvernante des filles d'honneur de
M\textsuperscript{me} la Dauphine, emploi qu'elle prit par pauvreté.

Montchevreuil était un fort honnête homme, modeste, brave, mais des plus
épais. Sa femme, qui était Boucher-d'Orsay, était une grande créature,
maigre, jaune, qui riait niais, et montrait de longues et vilaines
dents, dévote à outrance, d'un maintien composé, et à qui il ne manquait
que la baguette pour être une parfaite fée. Sans aucun esprit, elle
avait tellement captivé M\textsuperscript{me} de Maintenon qu'elle ne
voyait que par ses yeux, et ses yeux ne voyaient jamais que des
apparences et la laissaient la dupe de tout. Elle était pourtant la
surveillante de toutes les femmes de la cour, et de son témoignage
dépendaient les distinctions ou les dégoûts et souvent par enchaînement
les fortunes. Tout jusqu'aux ministres, jusqu'aux filles du roi,
tremblait devant elle\,; on ne l'approchait que difficilement\,; un
sourire d'elle était une faveur qui se comptait pour beaucoup. Le roi
avait pour elle une considération la plus marquée. Elle était de tous
les voyages et toujours avec M\textsuperscript{me} de Maintenon.

Le mariage de M. du Maine causa une rupture entre M\textsuperscript{me}
la princesse et la duchesse d'Hanovre, sa sœur, qui avait fort désiré M.
du Maine pour une de ses filles, et qui prétendit que M. le Prince lui
avait coupé l'herbe sous le pied. Elle vivait depuis longtemps en France
avec ses deux filles déjà fort grandes. Elles n'avaient aucun rang,
n'allaient point à la cour, voyaient peu de monde et jamais
M\textsuperscript{me} la Princesse qu'en particulier. Elles ne
laissaient pas d'avoir usurpé peu à peu de marcher avec deux carrosses,
force livrée, et un faste qui ne leur convenait point à Paris. Avec ce
cortège, elle rencontra M\textsuperscript{me} de Bouillon dans les rues,
à qui les gens de l'Allemande firent quitter son chemin, et la firent
ranger avec une grande hauteur. Ce fut quelque temps après le mariage de
M. du Maine. M\textsuperscript{me} de Bouillon, fort offensée,
n'entendit point parler de M\textsuperscript{me} d'Hanovre. Sa famille
était nombreuse et lors en grande splendeur, elle-même tenait un grand
état chez elle\,; les Bouillon, piqués à l'excès, résolurent de se
venger et l'exécutèrent. Un jour qu'ils surent que M\textsuperscript{me}
d'Hanovre devait aller à la comédie, ils y allèrent tous avec
M\textsuperscript{me} de Bouillon et une nombreuse livrée. Elle avait
ordre de prendre querelle avec celle de M\textsuperscript{me} d'Hanovre,
et l'exécution fut complète\,; les gens de la dernière battus à
outrance, les harnais de ses chevaux coupés, son carrosse fort
maltraité. L'Allemande fit les hauts cris, se plaignit au roi, s'adressa
à M. le Prince, qui, mécontent de sa bouderie, n'en remua pas\,; et le
roi, qui aimait mieux les trois frères Bouillon qu'elle qui avait le
premier tort et s'était attiré cette insulte, ne voulut point s'en
mêler, en sorte qu'elle en fut pour ses plaintes, et qu'elle apprit à se
conduire plus modestement.

Elle en demeura si outrée, que dès lors elle résolut de se retirer avec
ses filles en Allemagne, et quelques mois après elle l'exécuta. Ce fut
leur fortune\,: elle maria son aînée au duc de Modène, qui venait de
quitter le chapeau de cardinal pour succéder à son frère\,; et, quelque
temps après, le prince de Salm, veuf de sa sœur, gouverneur, puis grand
maître de la maison du fils aîné de l'empereur Léopold, roi de Bohème,
puis des Romains, fit le mariage de ce prince avec Amélie, son autre
fille.

Mon année de mousquetaire s'écoulait, et mon père demanda au roi ce
qu'il lui plairait faire de moi. Sur la disposition que le roi lui en
laissa, il me destina à la cavalerie, parce qu'il l'avait souvent
commandée par commission, et le roi résolut me donner, sans acheter, une
compagnie de cavalerie dans un de ses régiments. Il fallait qu'il en
vaquât\,; quatre ou cinq mois s'écoulèrent de la sorte, et je faisais
toujours mes fonctions de mousquetaire avec assiduité. Enfin, vers le
milieu d'avril, Saint-Pouange m'envoya demander si je voudrais bien
accepter une compagnie dans le Royal-Roussillon qui venait de vaquer,
mais fort délabrée et en garnison à Mons. Je mourais de peur de ne point
faire la campagne qui s'allait ouvrir\,; ainsi je disposai mon père à
l'accepter. Je remerciai le roi qui me répondit très obligeamment. La
compagnie fut entièrement réparée en quinze jours.

J'étais à Versailles lorsque, le vendredi 27 mars, le roi fit maréchaux
de France le comte de Choiseul, le duc de Villeroy, le marquis de
Joyeuse, Tourville, le duc de Noailles, le marquis de Boufflers et
Catinat\,: le comte de Tourville et Catinat n'étaient point chevaliers
de l'ordre. M. de Boufflers était en Flandre et Catinat sur la frontière
d'Italie\,; les cinq autres à la cour ou à Paris. Le roi manda aux deux
absents de prendre dès lors le titre, le rang et les honneurs de
maréchaux de France en attendant leur serment, qui en effet n'est point
nécessaire pour leur donner le caractère. M. de Duras ne l'a prêté que
parce que les gens du roi, qui en touchent gros, s'avisèrent enfin qu'il
n'avait prêté ni celui de maréchal de France ni celui de gouverneur de
Franche-Comté, et l'obligèrent par le roi de le prêter plus de trente
ans après.

J'étais au dîner du roi ce même jour. À propos de rien, le roi regardant
la compagnie\,: « Barbezieux, dit-il, apprendra la promotion des
maréchaux de France par les chemins.\,» Personne ne répondit mot. Le roi
était mécontent de ses fréquents voyages à Paris où les plaisirs le
détournaient. Il ne fut pas fâché de lui donner ce coup de caveçon et de
faire entendre aussi le peu de part qu'il avait en la promotion.

Le roi l'avait dit au duc de Noailles en entrant au conseil, mais avec
défense d'en parler à personne, même à ses collègues. Sa joie ne se peut
exprimer, et il avait plus raison d'être aise que pas un des autres.

L'engouement du duc de Villeroy dura plusieurs années. Tourville fut
d'autant plus transporté que sa véritable modestie lui cachait sa propre
réputation, et qu'il n'imaginait pas même d'être maréchal de France si
on en faisait, quoiqu'il le méritât autant qu'aucun d'eux pour le moins,
de l'aveu général. Choiseul et Joyeuse parurent fort modérés, comme des
seigneurs qui méritaient cet honneur et l'espéraient depuis longtemps.
Ils dînaient ensemble à Paris lorsqu'un capitaine d'infanterie arriva en
poste, satisfait d'avoir ouï nommer Joyeuse à qui il l'apprit, et ne
s'était point informé des autres\,; de sorte que Choiseul fut une
demi-heure dans un état violent jusqu'à ce que le courrier arriva. Ils
allèrent le soir à Versailles et prêtèrent serment le lendemain avec les
trois autres.

Cette promotion fit une foule de mécontents, moins de droit par mérite
que pour s'en donner un par les plaintes\,; mais de tous ceux-là le
monde ne trouva mauvais que l'oubli du duc de Choiseul, de Maulevrier et
de Montal. Ce qui exclut le premier est curieux. Sa femme, sœur de La
Vallière, belle et faite en déesse, ne bougeait d'avec
M\textsuperscript{me} la princesse de Conti, dont elle était cousine
germaine et intime amie. Elle avait eu des galanteries en nombre, et qui
avaient fait grand bruit. Le roi qui craignait cette liaison étroite
avec sa fille, lui avait fait parler, puis l'avait mortifiée, ensuite
éloignée, et lui avait après toujours pardonné. La voyant incorrigible
et n'aimant pas les éclats par lui-même, il le voulut faire par le mari,
et se défaire d'elle une fois pour toutes. Il se servit pour cela de la
promotion, et chargea M. de La Rochefoucauld, ami intime du duc de,
Choiseul, de lui représenter le tort que lui faisait le désordre public
de sa femme, de le presser de la faire mettre dans un couvent, et de lui
faire entendre, s'il avait peine à s'y résoudre, que le bâton qu'il lui
destinait était à ce prix.

Ce que le roi avait prévu arriva. Le duc de Choiseul, excellent homme de
guerre, était d'ailleurs un assez pauvre homme et le meilleur homme du
monde. Quoique vieux, un peu amoureux de sa femme qui lui faisait
accroire une partie de ce qu'elle voulait, il ne put se résoudre à un
tel éclat, tellement que M. de La Rochefoucauld à bout d'éloquence fut
obligé d'en venir à la condition du bâton. Cela même gâta tout. Le duc
de Choiseul s'indigna que la récompense de ses services et de la
réputation qu'il avait justement acquise à la guerre, se trouvât
attachée à une affaire domestique qui ne regardait que lui, et refusa
avec une opiniâtreté qui ne put être vaincue. Il lui en coûta le bâton
de maréchal de France, dont le scandale public éclata. Ce qu'il y eut de
pis pour lui, c'est que sa femme bientôt après fut chassée, et qu'elle
en fit tant, que le duc de Choiseul enfin n'y put tenir, la chassa de
chez lui et s'en sépara pour toujours.

Maulevrier avait beaucoup de réputation à la guerre et il la méritait.
Elle lui avait valu l'ordre malgré M. de Louvois, un gros gouvernement
et force commandements en chef. Le roi le crut assez récompensé et le
laissa. Ce pauvre homme en conçut une si violente douleur, qu'il ne
survécut pas deux mois à la promotion de ces sept cadets. Croissy, son
frère, ministre et secrétaire d'État, en fut outré, mais il n'osa le
trop paraître.

Montal était un grand vieillard de quatre-vingts ans, qui avait perdu un
œil à la guerre, où il avait été couvert de coups. Il s'y était
infiniment distingué, et souvent en des commandements en chef
considérables. Il avait acquis beaucoup d'honneur à la bataille de
Fleurus et encore plus de gloire au combat de Steinkerque, qu'il avait
rétabli. Tout cria pour lui, hors lui-même. Sa modestie et sa sagesse le
firent admirer. Le roi même en fut touché et lui promit de réparer le
tort qu'il lui avait fait. Il s'en alla quelque peu chez lui, puis
revint et servit par les espérances qui lui avaient été données et qui
furent trompeuses jusques à sa mort.

Mademoiselle, la grande Mademoiselle, qu'on appelait {[}ainsi{]} pour la
distinguer de la fille de Monsieur, ou, pour l'appeler par son nom,
M\textsuperscript{lle} de Montpensier, fille aînée de Gaston, et seule
de son premier mariage, mourut en son palais de Luxembourg, le dimanche
5 avril, après une longue maladie de rétention d'urine, à soixante-trois
ans, la plus riche princesse particulière de l'Europe. Le roi l'avait
visitée, et elle lui avait fort recommandé M. de Joyeuse, comme son
parent, pour être fait maréchal de France. Elle cousinait et distinguait
et s'intéressait fort en ceux qui avaient l'honneur de lui appartenir,
en cela, bien que très altière, fort différente de ce que les princes du
sang sont devenus depuis à cet égard. Elle portait exactement le deuil
de parents même très médiocres et très éloignés, et disait par où et
comment ils l'étaient. Monsieur et Madame ne la quittèrent point pendant
sa maladie. Outre la liaison qui avait toujours été entre elle et
Monsieur, dans tous les temps, il muguetait sa riche succession, et fut
en effet son légataire universel. Mais les plus gros morceaux avaient
échappé.

Les Mémoires publics de cette princesse montrent à découvert sa
faiblesse pour M. de Lauzun, la folie de celui-ci de ne l'avoir pas
épousée dès qu'il en eut la permission du roi, pour le faire avec plus
de faste et d'éclat. Leur désespoir de la rétractation de la permission
du roi fut extrême, mais les donations du contrat de mariage étaient
faites et subsistèrent par d'autres actes. Monsieur, poussé par M. le
Prince, avait pressé le roi de se rétracter, mais M\textsuperscript{me}
de Montespan et M. de Louvois y eurent encore plus de part, et furent
ceux sur qui tomba toute la fureur de Mademoiselle et la rage du
favori\,; car M. de Lauzun l'était. Ce ne fut pas pour longtemps\,; il
s'échappa plus d'une fois avec le roi, plus souvent encore avec la
maîtresse, et donna beau jeu au ministre pour le perdre. Il vint à bout
de le faire arrêter et conduire à Pignerol, où il fut extrêmement
maltraité par ses ordres et y demeura dix ans. L'amour de Mademoiselle
ne se refroidit point par l'absence. On sut en profiter pour faire un
grand établissement à M. du Maine, à ses dépens et ceux de M. de Lauzun
qui en acheta sa liberté. Eu, Aumale, Dombes et d'autres terres encore
furent données à M. du Maine, au grand regret de Mademoiselle. Et ce fut
sous ce prétexte de reconnaissance que, pour élever de plus en plus les
bâtards, le roi leur fit prendre la livrée de Mademoiselle, qui était
celle de M. Gaston. Cet héritier forcé lui fut toujours fort peu
agréable, et elle était toujours sur la défensive pour le reste de ses
biens que le roi lui voulait arracher pour ce fils bien-aimé.

Les aventures incroyables de M. de Lauzun, qui avait sauvé la reine
d'Angleterre et le prince de Galles, l'avaient ramené à la cour. Il
s'était brouillé avec Mademoiselle toujours jalouse de lui, qui, même à
la mort, ne le voulut pas voir. Il avait conservé Thiers et
Saint-Fargeau de ses dons. Il laissait toujours entendre qu'il avait
épousé Mademoiselle, et il parut devant le roi, en grand manteau, qui le
trouva fort mauvais. Après son deuil il ne voulut pas reprendre sa
livrée et s'en fit une d'un brun presque noir, avec des galons bleus et
blancs, pour conserver toujours la tristesse de la perte de
Mademoiselle, dont il avait des portraits partout.

Cette princesse donna à Monseigneur sa belle maison de Choisy, qui fut
ravi d'en avoir une de plaisance où il pût aller seul quelquefois avec
qui il voudrait\,; vingt-deux mille livres à M\textsuperscript{lle} de
Bréval et du Cambout, ses filles d'honneur\,; et des legs pieux et
d'autres à ses domestiques qui répondirent peu à ces richesses.

Tous les Mémoires de guerres civiles et les siens propres l'ont trop
fait connaître pour qu'il soit nécessaire d'y rien ajouter ici. Le roi
ne lui avait jamais bien pardonné la journée de Saint-Antoine, et je
l'ai ouï lui reprocher une fois, à son souper, en plaisantant, mais un
peu fortement, d'avoir fait tirer le canon de la Bastille sur ses
troupes. Elle fut un peu embarrassée, mais elle ne s'en tira pas trop
mal.

Sa pompe funèbre se fit en entier, et son corps fut gardé plusieurs
jours, alternativement par deux heures, par une duchesse ou une
princesse et par deux dames de qualité toutes en mantes, averties, de la
part du roi, par le grand maître des cérémonies\,; à la différence des
filles de France qui en ont le double ainsi que d'évêques, en rochet et
camail, et des princesses du sang qui ne sont gardées que par leurs
domestiques. La comtesse de Soissons refusa d'y aller\,; le roi se
fâcha, la menaça de la chasser et la fit obéir.

Il y arriva une aventure fort ridicule. Au milieu de la journée et toute
la cérémonie présente, l'urne, qui était sur une crédence et qui
contenait les entrailles, se fracassa avec un bruit épouvantable et une
puanteur subite et intolérable. À l'instant voilà les dames les unes
pâmées d'effroi, les autres en fuite. Les hérauts d'armes, les
feuillants qui psalmodiaient, s'étouffaient aux portes avec la foule qui
gagnait au pied. La confusion fut extrême. La plupart gagnèrent le
jardin et les cours. C'étaient les entrailles mal embaumées qui, par
leur fermentation, avaient causé ce fracas. Tout fut parfumé et rétabli,
et cette frayeur servit de risée. Ces entrailles furent portées aux
Célestins, le cœur au Val-de-Grâce, et le corps conduit à Saint-Denis
par la duchesse de Chartres, suivie de la duchesse de La Ferté, de la
princesse d'Harcourt et de dames de qualité\,; celles de
M\textsuperscript{me} la duchesse d'Orléans suivaient dans le carrosse
de cette princesse. Les cours assistèrent au service à Saint-Denis
quelques jours après, où l'archevêque d'Alby officia. L'abbé Anselme,
grand prédicateur, fit l'oraison funèbre. Mademoiselle, fille de
Monsieur, suivie de la duchesse de Ventadour et de la princesse de
Turenne, sa fille, avait conduit le cœur\,: toutes distinctions
au-dessus des princesses du sang, par ce rang de petite-fille de France,
que mon père lui fit donner par le feu roi, étant lors seule de la
famille royale.

\hypertarget{chapitre-iv.}{%
\chapter{CHAPITRE IV.}\label{chapitre-iv.}}

1693

~

\relsize{-1}

{\textsc{Distribution des armées.}} {\textsc{- Le roi en Flandre.}}
{\textsc{- Époque de l'obéissance des maréchaux les uns aux autres par
ancienneté.}} {\textsc{- Art de M. de Turenne.}} {\textsc{- Mort de mon
père dont le roi me donne le gouvernement.}} {\textsc{- Origine première
de la fortune de mon père.}} {\textsc{- Bonté et prévoyance de Louis
XIII sur le gouvernement de Blaye.}} {\textsc{- Mon oncle et mon père
chevaliers de l'ordre en 1633 avant l'âge.}} {\textsc{- Mon père duc et
pair en janvier 1635, et comment.}} {\textsc{- Grandeur d'âme et courage
de Louis XIII à la perte de Corbie.}} {\textsc{- Réprimande à mon père
en public pour n'avoir pas écrit monseigneur au duc de Bellegarde
disgracié et exilé.}} {\textsc{- Chasteté de Louis XIII digne de saint
Louis, qui réprimande mon père.}} {\textsc{- Époque du nom de madame aux
dames d'atours filles.}} {\textsc{- Intimité jusqu'à la mort de M. le
Prince, père du héros, avec mon père, et sa cause.}} {\textsc{- Bontems
et Nyert.}} {\textsc{- Leur fortune par mon père.}} \relsize{1}

~

Le 3 mai, le roi déclara qu'il irait de Flandre commander une de ses
armées avec le nouveau maréchal de Boufflers, ayant sous lui
Monseigneur, et M. le Prince entre deux comme à Namur, M. de Luxembourg
pour l'autre armée de Flandre avec les maréchaux de Villeroy et de
Joyeuse sous lui\,; et en même temps {[}le roi nomma pour{]} les autres
armées, c'est-à-dire le maréchal de Lorges en Allemagne, le maréchal
Catinat en Italie, et le nouveau maréchal de Noailles en Catalogne.
Comme on craignait les descentes des Anglais, Monsieur eut le
commandement de toutes les côtes de l'Océan, avec des troupes en divers
lieux, le maréchal d'Humières sous lui et le duc de Chaulnes gouverneur
de Bretagne, qui y était, le maréchal d'Estrées commandant d'Aunis,
Saintonge et Poitou, et le maréchal de Bellefonds en Normandie à ses
ordres. M. le duc de Chartres eut le commandement de la cavalerie dans
l'armée de M. de Luxembourg où M. le Duc et M. le prince de Conti
servirent de lieutenants généraux. M. du Maine en servit en celle de M.
de Boufflers que le roi commandait, et fut en même temps à la tête de la
cavalerie\,; ce qui exclut le comte d'Auvergne de servir, qui en était
colonel général.

Il fut nouveau de voir des maréchaux de France obéir à d'autres.
L'inconvénient du commandement égal tour à tour avait été souvent
funeste. C'est ce qui donna lieu à la faveur de M. de Turenne, jointe à
sa grande réputation, de renouveler pour lui la charge de maréchal
général des camps et armées de France, pour le faire commander aux
maréchaux de France, et qui encore ne s'y soumirent qu'après l'exil de
MM. de Bellefonds, Humières et Créqui, et c'est depuis cette époque de
charge, que M. de Turenne, confondant avec art son nouvel état avec son
rang de prince, ôta les bâtons de ses armes et ne voulut plus être
appelé que le vicomte de Turenne. Enfin le roi régla, pour l'utilité de
son service, que les maréchaux de France s'obéiraient les uns aux autres
par ancienneté, tellement que ces maréchaux en second n'étoient
proprement à l'armée que des lieutenants généraux qui ne roulaient point
avec les autres et qui les commandaient, qui ne prenaient point jour et
qui avaient les mêmes honneurs militaires que le général de l'armée,
mais qui prenaient l'ordre de lui et ne se mêlaient de rien que sous ses
ordres et par ses ordres, et duquel ils étaient même fort rarement
consultés, et point du tout du secret de la campagne.

Ce même jour, 3 mai, sur les dix heures du soir, j'eus le malheur de
perdre mon père. Il avait quatre-vingt-sept ans, et ne s'était jamais
bien rétabli d'une grande maladie, qu'il avait eue à Blaye, il y avait
deux ans. Depuis trois semaines il avait un peu de goutte. Ma mère, qui
le voyait avancer en âge, lui proposa des arrangements domestiques qu'il
fit en bon père, et elle songeait à le faire démettre en ma faveur de sa
dignité de duc et pair. Il avait dîné avec de ses amis comme il avait
toujours compagnie. Sur le soir il se remit au lit sans aucun mal ni
accident, et pendant qu'on l'entretenait, il poussa tout à coup trois
violents soupirs tout de suite. Il était mort qu'à peine s'écriait-on
qu'il se trouvait mal\,: il n'y avait plus d'huile à la lampe.

J'en appris la triste nouvelle en revenant du coucher du roi, qui se
purgeait le lendemain. La nuit fut donnée aux justes sentiments de la
nature. Le lendemain j'allai de bon matin trouver Bontems, puis le duc
de Beauvilliers qui était en année \footnote{Le duc de Beauvilliers
  était un des quatre premiers gentilshommes de la chambre du roi\,; ces
  gentilshommes servaient par année à tour de rôle.} et dont le père
avait été l'ami du mien. M. de Beauvilliers me témoignait mille bontés
chez les princes dont il était gouverneur, et me promit de demander au
roi les gouvernements de mon père en ouvrant son rideau. Il les obtint
sur-le-champ. Bontems, fort attaché à mon père, accourut me le dire à la
tribune où j'attendais\,; puis M. de Beauvilliers lui-même, qui me dit
de me trouver à trois heures dans la galerie où il me ferait appeler et
entrer par les cabinets, à l'issue du dîner du roi.

Je trouvai la foule écoulée de sa chambre. Dès que Monsieur, qui était
debout au chevet du lit du roi, m'aperçut\,: « Ah\,! voilà, dit-il tout
haut, M. le duc de Saint-Simon.\,» J'approchai du lit et fis mon
remerciement par une profonde révérence. Le roi me demanda fort comment
ce malheur était arrivé, avec beaucoup de bonté pour mon père et pour
moi\,: il savait assaisonner ses grâces. Il me parla des sacrements que
mon père n'avait pu recevoir\,; je lui dis qu'il y avait fort peu qu'il
avait fait une retraite de plusieurs jours à Saint-Lazare où il avait
son confesseur, et où il avait fait ses dévotions, et un mot de la piété
de sa vie. Le colloque dura assez longtemps, et finit par des
exhortations à continuer d'être sage et à bien faire, et qu'il aurait
soin de moi.

Lors de la maladie de mon père à Blaye, plusieurs personnes demandèrent
au roi le gouvernement de Blaye, d'Aubigné entre autres, frère de
M\textsuperscript{me} de Maintenon, à qui il répondit plus brusquement
qu'il n'avait accoutumé\,: « Est-ce qu'il n'a pas un fils\,?\,» En
effet, le roi, qui s'était fermé à n'accorder plus de survivances,
s'était toujours fait entendre à mon père qu'il me destinait son
gouvernement. M. le Prince muguetait fort celui de Senlis qu'avait mon
oncle\,; il l'avait demandé à sa mort. Le roi le donna à mon père, et je
l'eus en même temps que celui de Blaye.

Tout ce qui avaisinait Chantilly était envié par M. le Prince. Il embla
\footnote{Vieux mot qui a le même sens que \emph{voler}.} à mon oncle la
capitainerie des chasses de Senlis et d'Hallustre en vrai Scapin. Mon
oncle, aîné de huit ans de mon père, avait eu ce gouvernement et cette
capitainerie de son père, qui était depuis longtemps dans la maison, et
depuis des siècles avec peu d'intervalles. Son grand âge et un
tremblement universel qui n'attaqua jamais sa tête ni sa santé l'avaient
retiré depuis bien des années du monde. Il passait les hivers à Paris où
il en voyait fort peu, et sept ou huit mois à sa campagne tout auprès de
Senlis. Sa femme, dont il n'avait point d'enfants, était aussi vieille
que lui. Elle était sœur du père du duc d'Uzès, et avait épousé en
premières noces le marquis de Portes, de la maison de Budos, chevalier
de l'ordre ainsi que mon oncle, et vice-amiral de France, tué au siège
de Privas. Il était frère de la connétable de Montmorency, mère de
M\textsuperscript{me} la Princesse, grand'mère de M. le Prince, et du
dernier duc de Montmorency, décapité à Toulouse. M. le Prince l'appelait
toujours sa tante, et les allait voir assez souvent de Chantilly.

Un beau jour il fut leur compter dans leur retraite que le roi,
importuné des plaintes de ceux qui se trouvaient enclavés dans les
capitaineries royales, allait rendre un édit pour les supprimer toutes,
à l'exception de celles de ses maisons qu'il habitait et des bois et
plaines qui environnaient Paris\,; que les leurs allaient donc être
supprimées\,; que cependant il espérait cette considération du roi que
si elles étaient entre ses mains il les lui conserverait\,; qu'eux-mêmes
y trouveraient doublement leur compte, parce que les capitaineries étant
conservées, ils en demeureraient toujours les maîtres comme lui-même,
pour leurs gens, leur table et leurs amis, et qu'il leur donnerait
volontiers deux ou trois cents pistoles pour cette complaisance,
quoiqu'il ne fût pas sûr de faire changer le roi là-dessus en sa faveur.
Les bonnes gens le crurent, pestèrent contre l'édit, donnèrent la
démission à M. le Prince\,; qui laissa deux cents pistoles en partant,
et se moqua d'eux. Tout le pays, qui vivait en paix et sans inquiétude
dans cette capitainerie, fut outré de douleur. Elle devint une tyrannie
entre les mains de M. le Prince, qui l'étendit encore tant qu'il put\,;
mais il est vrai qu'il laissa ceux qu'il avait ainsi escamotés les
maîtres pour eux et pour leurs domestiques le reste de leur vie.

Mon oncle avait eu le régiment de Navarre\,; il était lieutenant
général, et avait emporté le prix de la bonne mine à sa promotion de
l'ordre en 1633. Il mourut en 1690, le 25 janvier, et sa femme en avril
1695. C'était un fort grand homme, très bien fait, de grande mine, plein
de sens, de sagesse, de valeur et de probité. Mon père l'avait toujours
fort respecté et suivait fort ses avis pendant sa faveur. La marquise de
Saint-Simon était haute, intéressée et méchante, et elle trouva le moyen
de faire passer la plupart des biens de mon oncle aux ducs d'Uzès, de
faire payer à mon père et à moi une grande partie des dettes, et de
laisser les autres insolvables. Sa passion était de me marier à
M\textsuperscript{lle} d'Uzès, qui a été la première femme de M. de
Barbezieux. Je n'ai pu me refuser ce mot sur mon oncle\,; il est bien
juste de m'étendre un peu plus sur mon père.

La naissance et les biens ne vont pas toujours ensemble. Diverses
aventures de guerre et de famille avaient ruiné notre branche, et laissé
mes derniers pères avec peu de fortune et d'éclat pour leur service
militaire. Mon grand-père, qui avait suivi toutes les guerres de son
temps, et toujours passionné royaliste, s'était retiré dans ses terres,
où son peu d'aisance l'engagea de suivre la mode du temps, et de mettre
ses deux aînés pages de Louis XIII, où les gens des plus grands noms se
mettaient alors.

Le roi était passionné pour la chasse, qui était sans meute \footnote{Les
  précédents éditeurs ont mis ici le mot suite\,; il est impossible de
  lire autre chose que \emph{meute} ou \emph{route}. Ce dernier mot se
  trouvant plus bas dans la même phrase, \emph{meute} a paru préférable.}
et sans cette abondance de chiens, de piqueurs, de relais, de
commodités, que le roi son fils y a apportés, et surtout sans routes
dans les forêts. Mon père, qui remarqua l'impatience du roi à relayer,
imagina de lui tourner le cheval qu'il lui présentait, la tête à la
croupe de celui qu'il quittait. Par ce moyen, le roi, qui était dispos,
sautait de l'un sur l'autre sans mettre pied à terre, et cela était fait
en un moment. Cela lui plut, il demanda toujours ce même page à son
relais\,; il s'en informa, et peu à peu il le prit en affection.
Baradas, premier écuyer, s'étant rendu insupportable au roi par ses
hauteurs et ses humeurs arrogantes avec lui, il le chassa, et donna sa
charge à mon père. Il eut après celle de premier gentilhomme de la
chambre du roi à la mort de Blainville, qui était chevalier de l'ordre,
et avait été ambassadeur en Angleterre. Il était du nom de Warigniés,
qui est bon, mais éteint en Normandie, n'avait point été marié, et était
frère aîné du père de la comtesse de Saint-Géran, qui a été dame du
palais de la reine, et qui a tant figuré dans le monde, femme de ce
comte de Saint-Géran, chevalier de l'ordre, de qui l'état fut tant et si
longtemps disputé par un procès également étrange et curieux.

Mon père devint tout à fait favori sans autre protection que la bonté
seule du roi, et ne compta jamais avec aucun ministre, pas même avec le
cardinal de Richelieu, et c'était un de ses mérites auprès de Louis
XIII. Il m'a conté qu'avant de l'élever, et en ayant envie, il s'était
fait sourdement extrêmement informer de son personnel et de sa
naissance, car il n'avait pas été instruit à les connaître, pour voir si
cette base était digne de porter une fortune, et de ne retomber pas une
autre fois. Ce furent ses propres termes à mon père à qui il le raconta
depuis, attrapé comme il l'avait été à M. de Luynes. Il aimait les gens
de qualité, cherchait à les connaître et à les distinguer\,; aussi en
a-t-on fait le proverbe des trois places et des trois statues de
Paris\,: Henri IV avec son peuple sur le pont Neuf, Louis XIII avec les
gens de qualité à la place Royale, qui de son temps était le beau
quartier\,; et Louis XIV avec les maltôtiers dans la place des
Victoires. Celle de Vendôme, faite longtemps depuis, ne lui a guère
donné meilleure compagnie.

À la mort de M. de Luxembourg, frère du connétable de Luynes, le roi
donna le choix à mon père de sa vacance. Il avait les chevau-légers de
la garde et le gouvernement de Blaye. Mon père le supplia d'en
récompenser des seigneurs qui le méritaient plus que lui déjà comblé de
ses bienfaits. Le roi et lui insistèrent dans cette singulière
dispute\,; puis, se fâchant, lui dit que ce n'était pas à lui ni à
personne à refuser ses grâces, qu'il lui donnait vingt-quatre heures
pour choisir, et qu'il lui ordonnait de lui dire le lendemain matin le
choix qu'il aurait fait. Le matin venu, le roi le lui demanda avec
empressement. Mon père lui répondit que, puisque absolument il lui
voulait donner une des deux vacances, il croyait ne pouvoir rien faire
de plus avantageux pour lui que de le laisser choisir lui-même. Le roi
prit un air serein et le loua\,; puis lui dit que les chevau-légers
étaient brillants, mais que Blaye était solide, une place qui bridait la
Guyenne et la Saintonge, et qui, dans les troubles, faisait fort compter
avec elle\,; qu'on ne savait ce qui pouvait arriver\,; que s'il venait
après lui une guerre civile, les chevau-légers n'étaient rien, et que
Blaye le rendrait considérable, raison qui le déterminait à lui
conseiller de préférer cet établissement. C'est ainsi que mon père a eu
ce gouvernement, et que les suites ont fait voir combien Louis XIII
avait pensé juste et quelle était sa bonté, non par ce que mon père en
retira, mais par tout ce qu'il méprisa, et par la fidélité et
l'importance du service dont il s'illustra.

Lorsque M. Gaston revint de Bruxelles, par ce traité tenu si secret que
sa présence subite à la cour l'apprit aux plus clairvoyants, le roi
l'avait confié à mon père. Il lui dit en même temps qu'il avait résolu
de le faire un jour duc et pair, que sa jeunesse l'aurait encore retenu,
mais qu'ayant promis à Monsieur de faire Puylaurens, il ne pouvait se
résoudre à le faire sans lui. Ce bon maître ajouta qu'il y avait une
condition qui lui semblerait dure, c'était de faire Puylaurens le
premier, s'il en faisait encore à cette occasion. En effet, mon père
s'en trouva si choqué, qu'il balança vingt-quatre heures comme si,
n'étant pas duc, Puylaurens duc n'eût pas été bien au-dessus de lui que
simplement son ancien. Enfin il accepta, et le fut seul quinze jours
après lui. Il n'en eut pas le dégoût longtemps\,; moins d'une année
éteignit ce duché-pairie de la façon que tout le monde l'a su. Mon père
était déjà chevalier de l'ordre, deux ans auparavant, n'ayant lors que
vingt-sept ans juste, à la promotion de 1633. Mon grand-père fut nommé
avec lui. Il était vieux et retiré. Il trouva que ce n'était pas la
peine de faire connaissance avec la cour. Il chargea mon père de
demander le collier qui lui était destiné pour mon oncle, qui avait
trente-cinq ans au juste, qui en jouirait plus longtemps que lui. En
effet, il l'a porté cinquante-sept ans et mon père soixante, et sont
restés longtemps les deux seuls du feu roi. Chose sans exemple dans
aucun ordre.

Mon père eut encore les capitaineries de Saint-Germain et de Versailles,
dont il se défit au président de Maisons, par amitié pour lui\,; et fut
aussi quelque temps grand louvetier. Lorsqu'il fut fait duc et pair, il
vendit sa charge de premier gentilhomme de la chambre au duc de
Lesdiguières, pour M. de Créqui, fils de feu son second fils de
Canaples, tué mestre de camp du régiment des gardes. M. de Lesdiguières
l'exerça durant sa jeunesse, mais rarement, par son presque Continuel
séjour en son gouvernement de Dauphiné. M. de Créqui, depuis duc et
pair, ambassadeur à Rome, enfin gouverneur de Paris, fit passer sa
charge au duc de la Trémoille, mari de sa fille unique, d'où elle est
restée à sa postérité. De l'argent de cette charge mon père acquit, de
l'aîné de la maison, la terre de Saint-Simon qui n'en était jamais
sortie, depuis l'héritage de Vermandois qui nous l'avait apportée en
mariage, et la fit ériger en duché-pairie.

Il ne se contenta pas de suivre le roi en toutes ses expéditions de
guerre. Il eut plusieurs fois le commandement de la cavalerie dans les
armées, et le commandement en chef de tous les arrière-bans du royaume,
qui étaient de cinq mille gentilshommes, à qui, contre leur privilège,
il persuada de sortir les frontières du royaume. Sa valeur et sa
conduite lui acquirent beaucoup de réputation à la guerre, et l'amitié
intime du maréchal de La Meilleraye et du fameux duc de Weimar. Je puis
dire, sans craindre d'être démenti par tout ce qu'il y a d'amateurs de
ces temps-là, que sa faveur fut sans envie, qu'il fut toujours modeste
et souverainement désintéressé\,; qu'il ne demanda jamais rien pour soi,
qu'il fut l'homme le plus obligeant, le mieux faisant et le plus
généreux qui ait paru à la cour, où il causa un grand nombre de
fortunes, appuya les malheureux et lit répandre force bienfaits.

La condamnation du duc de Montmorency lui pensa coûter la sienne, pour
avoir demandé sa grâce avec trop de persévérance et de chaleur. L'éclat
que cela fit perça jusqu'à cet illustre coupable qui avait toujours été
de ses amis. Allant à l'échafaud avec le courage et la piété qui l'ont
tant fait admirer, il fit deux présents bien différents de deux tableaux
d'un grand prix du même maître \footnote{Le Carrache. (Note de
  Saint-Simon.)}, et uniques de lui en France\,: un saint Sébastien
percé de flèches, au cardinal de Richelieu\,; et une Pomone et Vertumne
(Pomone la plus belle et la plus agréable qu'on saurait voir de grandeur
naturelle), à mon père. Je l'ai encore, et je le garde précieusement.

Je serais trop long si je me mettais à raconter bien des choses que j'ai
sues de mon père, qui me font bien regretter mon âge et le sien qui ne
m'ont pas permis d'en apprendre davantage. Je me contenterai de
quelques-unes, remarquables en général. Je ne m'arrêterai point à la
fameuse journée des Dupes, où il eut le sort du cardinal de Richelieu
entre les mains, parce que je l'ai trouvée, dans\ldots{} \footnote{Le
  nom a été gratté dans le manuscrit.}, toute telle que mon père me l'a
racontée. Ce n'est pas qu'il tînt en rien au cardinal de Richelieu, mais
il crut voir un précipice dans l'humeur de la reine mère et dans le
nombre de gens qui, par elle, prétendaient tous à gouverner. Il crut
aussi, par les succès qu'avait eus le premier ministre, qu'il était bien
dangereux de changer de main dans la crise où l'État se trouvait alors
au dehors, et ces vues seules le conduisirent. Il n'est pas difficile de
croire que le cardinal lui en sut un bon gré extrême, et d'autant plus
qu'il n'y avait aucun lien entre eux. Ce qui est plus rare, c'est que,
s'il conçut quelques peines secrètes de s'être vu en ses mains et de lui
devoir l'affermissement de sa place et de sa puissance et le triomphe
sur ses ennemis, il eut la force de le cacher si bien, qu'il n'en donna
jamais la moindre marque, et mon père aussi ne lui en témoigna pas plus
d'attachement. Il arriva seulement que ce premier ministre, soupçonneux
au possible, et persuadé sur mon père par une expérience si décisive et
si gratuite, allait depuis à lui sur les ombrages qu'il prenait. Il est
souvent arrivé à mon père d'être réveillé en sursaut, en pleine nuit,
par un valet de chambre qui tirait son rideau une bougie à la main,
ayant derrière lui le cardinal de Richelieu, qui s'asseyait sur le lit
et prenait la bougie, s'écriant quelquefois qu'il était perdu, et venant
au conseil et au secours de mon père sur des avis qu'on lui avait
donnés, ou sur des prises qu'il avait eues avec le roi.

Ce fut cette journée des Dupes qui coûta au maréchal de Bassompierre
tant d'années de Bastille, qui le mirent de si mauvaise humeur contre
mon père qui en avait été la cause indirecte en sauvant et maintenant le
cardinal de Richelieu. Ce dépit qu'il montre si à découvert dans ses
curieux Mémoires, quoique d'ailleurs si dégoûtants par leur vanité, ne
peut pourtant rien alléguer contre mon père\,; et se borne à une injure,
sans aucun appui, qui ne mérite que le mépris et la compassion d'une
envie et d'une colère impuissante jusqu'à ne pouvoir rien articuler que
le mot injurieux et unique dans tout ce qui reste d'écrits de ces
temps-là.

Je ne puis passer sous silence ce que mon père m'a raconté de la
consternation qui saisit Paris et la cour lorsque, les Espagnols prirent
Corbie, après s'être rendus maîtres de toute la frontière jusque-là et
de tout le pays jusqu'à Compiègne, et du conseil qui fut tenu. Le roi
voulait qu'il y fût présent fort souvent, non pour y opiner à son âge,
mais pour le former aux affaires, le questionner en particulier sur les
partis importants à prendre, pour voir son sens et le louer ou le
reprendre et lui expliquer en quoi il avait bien ou mal pensé et
pourquoi, comme un père qui prend plaisir à former l'esprit de son fils.

Dans ce conseil le cardinal de Richelieu parla le premier. Il opina à
des partis faibles, et surtout de retraite pour le roi au delà de la
Seine, et compta d'emporter l'avis de tout ce qui était au conseil,
comme il ne manqua pas d'arriver. Le roi les laissa tout dire sans
témoigner ni impatience ni répugnance, puis leur demanda s'ils n'avaient
rien à ajouter. Comme ils eurent répondu que non, il dit que c'était
donc à lui à leur expliquer à son tour son avis. Il parla un bon quart
d'heure, réfuta le leur par les plus fortes raisons, allégua que sa
retraite ne ferait qu'achever le désordre, précipiter la fuite,
resserrer toutes les bourses et perdre toute espérance, décourager ses
troupes et ses généraux\,; puis expliqua pendant un autre quart d'heure
le plan qu'il estime devoir être suivi\,; et tout de suite se tournant à
mon père, sans plus prendre les avis, lui ordonna que tout ce qui
pourrait être prêt de ses charges le fût à le suivre le lendemain matin
vers Corbie, et que le reste le joindrait quand il pourrait. Cela dit
d'un ton à n'admettre point de réplique, se lève, sort du conseil et
laisse le cardinal et tous les autres dans le dernier étonnement. On
peut voir par l'histoire et les Mémoires de ces temps-là que ce hardi
parti fut le salut de l'État, et les succès qu'il eut. Le cardinal, tout
grand homme qu'il était, en trembla jusqu'à ce que les premières
apparences de fortune l'enhardirent à joindre le roi. Voilà un
échantillon de ce roi faible et gouverné par son premier ministre à qui
les muses et les écrivains ont donné bien de la gloire qu'ils ont
dérobée à son maître, comme l'opiniâtreté et tous les travaux du siège
de la Rochelle et l'invention et le succès inouïs de sa digue si
célèbre, tous uniquement dus au feu roi.

Si le roi savait bien aimer mon père, aussi savait-il bien le reprendre,
dont mon père m'a raconté deux occasions. Le duc de Bellegarde, grand
écuyer et premier gentilhomme de la chambre, était exilé\,; mon père
était de ses amis et premier gentilhomme de la chambre aussi, ainsi que
premier écuyer et au comble de sa faveur. Cette dernière raison et ses
charges exigeaient une grande assiduité, de manière que, faute d'autre
loisir, il se mit à écrire à M. de Bellegarde en attendant que le roi
sortît pour la chasse. Comme il finissait sa lettre, le roi sortit et le
surprit comme un homme qui se lève brusquement et qui cache un papier.
Louis XIII, qui de ses favoris plus que de tous les autres voulait tout
savoir, s'en aperçut et lui demanda ce que c'était que ce papier qu'il
ne voulait pas qu'il vît. Mon père fut embarrassé, pressé et avoua que
c'était un mot qu'il écrivait à M. de Bellegarde. « Que je voie\,! \,»
dit le roi\,; et prit le papier et le lut. « Je ne trouve point mauvais,
dit-il à mon père après avoir lu, que vous écriviez à votre ami,
quoiqu'en disgrâce, parce que je suis bien sûr que vous ne lui manderez
rien de mal à propos\,; mais ce que je trouve très mauvais, c'est que
vous lui manquiez au respect que vous devez à un duc et pair, et que,
parce qu'il est exilé, vous ne lui écriviez pas \emph{monseigneur\,;}\,»
et déchirant la lettre en deux\,: « Tenez, ajouta-t-il, voilà votre
lettre\,; elle est bien, d'ailleurs, refaites-la après la chasse, et
mettez \emph{monseigneur}, comme vous le lui devez.\,» Mon père m'a
conté que, quoique bien honteux de cette réprimande, tout en marchant,
devant du monde, il s'en était tenu quitte à bon marché, et qu'il
mourait de peur de pis pour avoir écrit à un homme en profonde disgrâce
et qui ne put revenir dans les bonnes grâces du roi.

L'autre réprimande fut sur un autre article et plus sérieuse. Le roi
était véritablement amoureux de M\textsuperscript{lle} d'Hautefort. Il
allait plus souvent chez la reine à cause d'elle, et il y était toujours
à lui parler. Il en entretenait continuellement mon père, qui vit
clairement combien il en était épris. Mon père était jeune et galant, et
il ne comprenait pas un roi si amoureux, si peu maître de le cacher, et
en même temps qui n'allait pas plus loin. Il crut que c'était
timidité\,; et, sur ce principe, un jour que le roi lui parlait avec
passion de cette fille, mon père lui témoigna la surprise que je viens
d'expliquer, et lui proposa d'être son ambassadeur et de conclure
bientôt son affaire. Le roi le laissa dire, puis prenant un air
sévère\,: « Il est vrai, lui dit-il, que je suis amoureux d'elle, que je
le sens, que je la cherche, que je parle d'elle volontiers et que j'y
pense encore davantage\,; il est vrai encore que tout cela se fait en
moi malgré moi, parce que je suis homme, et que j'ai cette faiblesse\,;
mais plus ma qualité de roi me peut donner plus de facilité à me
satisfaire qu'à un autre, plus je dois être en garde contre le péché et
le scandale. Je pardonne pour cette fois à votre jeunesse, mais qu'il ne
vous arrive jamais de me tenir un pareil discours si vous voulez que je
continue à vous aimer.\,» Ce fut pour mon père un coup de tonnerre\,;
les écailles lui tombèrent des yeux\,; l'idée de la timidité du roi dans
son amour disparut à l'éclat d'une vertu si pure et si triomphante.
C'est la même que le roi fit dame d'atours \footnote{La charge de
  \emph{dame d'atours} était une des principales de la maison de la
  reine. D'après le \emph{Traité des Offices} de Guyot, la dame d'atours
  devait donner les ordres pour tout ce qui concernait les vêtements et
  les pierreries de la reine\,; elle présidait à sa toilette et
  dirigeait les femmes de chambre chargées de l'habiller et de la
  coiffer. Aux audiences que donnait la reine, la dame d'atours se
  plaçait à sa gauche\,; elle servait la reine à son petit couvert, en
  l'absence de la dame d'honneur.} de la reine, et que sous ce prétexte
il fit appeler M\textsuperscript{me} d'Hautefort, qui à la fin fut la
seconde femme du dernier maréchal de Schomberg, duc d'Halluyn, qui n'en
eut point d'enfants. C'est depuis elle que les dames d'atours filles ont
été appelées madame.

Mon père fut heureux dans plusieurs de ses différentes sortes de
domestiques, qui firent des fortunes considérables. Tourville, qui était
un de ses gentilshommes, et celui par qui, à la journée des Dupes, il
envoya dire au cardinal de Richelieu de venir sur sa parole trouver le
roi à Versailles le soir même, était un homme fort sage et de mérite. Le
cardinal de Richelieu mariant sa nièce au fameux duc d'Enghien, M. le
Prince lui demande un gentilhomme de valeur et de confiance à mettre
auprès de M. son fils. Il lui donna Tourville, qui y fit une fortune.
Son fils, à force d'être, de l'aveu des Anglais et des Hollandais, le
plus grand homme de mer de son siècle, en fit une bien plus grande. Il
voyait mon père assidûment quand il était à Paris, et avec un respect
qui lui faisait honneur. Je me souviens de la joie de mon père quand il
fut maréchal de France et de celle qu'il lui témoigna en l'embrassant.
Il n'eut pas le temps de jouir longtemps de cette satisfaction\,; mais
avec moi, tout jeune que j'étais, ce maréchal me voyait et en toutes
occasions et en tous lieux affectait pour moi une déférence qui
m'embarrassait souvent. Ce n'est pas pour lui une petite louange.

Ce qui mit son père chez M. le prince, où il est demeuré et sa femme
jusqu'à leur mort, dans les premières places de la maison, fut la
confiance de M. le Prince le père pour le mien, et son intimité avec lui
que l'éloignement à Blaye ne diminua point. La cause en fut très
singulière. Le cardinal de Richelieu tomba très dangereusement malade à
Bordeaux, revenant du voyage qui coûta la vie au dernier duc de
Montmorency, et le roi retourna à Paris par une autre route. Ce fut
cette maladie dont on crut qu'il ne reviendrait point qui donna lieu aux
lettres du garde des sceaux de Châteauneuf et de la fameuse duchesse de
Chevreuse, par lesquelles ils se réjouissaient de sa mort prochaine.
Elles furent interceptées. Châteauneuf en perdit les sceaux et fut mis
au château d'Angoulême, d'où il ne sortit qu'après la mort du cardinal,
et la duchesse de Chevreuse s'enfuit du royaume. Dans cette extrémité du
cardinal, le roi, en peine de qui le remplacer, s'il venait à le perdre,
en raisonna souvent avec mon père, qui lui persuada M. le Prince. Cela
n'eut pas lieu parce que le cardinal guérit\,; longtemps après M. le
Prince témoigna à mon père toute sa reconnaissance de ce qu'il avait
voulu faire pour lui. Mon père se tint sur la négative et sur une
entière ignorance jusqu'à ce que M. le Prince lui dit que c'était du roi
même qu'il le savait, et cela lia entre eux une amitié qui n'a fini
qu'avec la vie de ce prince de Condé, mais qu'il ne transmit pas à sa
famille.

Entre d'autres sortes de domestiques de mon père, il eut un secrétaire
dont le fils, connu sous le nom de du Fresnoy, devint dans les suites un
des plus accrédités commis de M. de Louvois, et qui n'a jamais oublié
d'où il était parti. Sa femme fut cette M\textsuperscript{me} du Fresnoy
si connue par sa beauté conservée jusque dans la dernière vieillesse,
pour qui le crédit de M. de Louvois fit créer une charge de dame du lit
de la reine qui a fini avec elle, parce qu'avec la rage de la cour elle
ne pouvait être dame et ne voulait pas être femme de chambre.

Il eut encore deux chirurgiens domestiques qui se rendirent célèbres et
riches\,: Bien aise, par l'invention de l'opération de l'anévrisme ou de
l'artère piquée\,; Arnaud, pour celle des descentes. Sur quoi je ne puis
me tenir de raconter que depuis qu'il fut revenu chez lui, et devenu
considérable dans son métier, un jeune abbé fort débauché alla lui en
montrer une qui l'incommodait fort dans ses plaisirs. Arnaud le fit
étendre sur un lit de repos pour le visiter, puis lui dit que
l'opération était si pressée qu'il n'y avait pas un moment à perdre, ni
le temps de retourner chez lui. L'abbé, qui n'avait pas compté sur rien
de si instant, voulut capituler, mais Arnaud tint ferme et lui promit
d'avoir grand soin de lui. Aussitôt il le fait saisir par ses garçons et
avec l'opération de la descente lui en fit une autre qui n'est que trop
commune en Italie aux petits garçons dont on espère de belles voix.
Voilà l'abbé aux hauts cris, aux fureurs, aux menaces. Arnaud, sans
s'émouvoir, lui dit que s'il voulait mourir incessamment il n'avait qu'à
continuer ce vacarme, que s'il voulait guérir et vivre il fallait
surtout se calmer et se tenir dans une grande tranquillité. Il guérit et
voulait tuer Arnaud, qui s'en gara bien, et le pauvre abbé en fut pour
ses plaisirs.

Deux des quatre premiers valets de chambre durent leur fortune à mon
père\,: Bontems, dont le fils, gouverneur de Versailles et le plus
intérieur des quatre du roi, ne l'a jamais oublié\,; et Nyert, dont le
fils n'a rien fait moins que s'en souvenir. Le père de Bontems saignait
dans Paris et avait très bien saigné mon père\,; Louis XIII, quelque
temps après, eut besoin de l'être et ne se fiait pas à son premier
chirurgien, dont la main était appesantie. Mon père lui produisit
Bontems, qui continua à saigner le roi et que mon père fit premier valet
de chambre. Le père de Nyert avait une jolie voix, savait la musique, et
jouait fort bien du luth. M. de Mortemart, premier gentilhomme de la
chambre, qui en 1633 devint duc et pair, l'avait pris à lui et le mena
au voyage de Lyon et de Savoie, où mon père l'entendit plusieurs fois
chez M. de Mortemart.

\hypertarget{chapitre-v.}{%
\chapter{CHAPITRE V.}\label{chapitre-v.}}

1693

~

\relsize{-1}

{\textsc{Gloire de Louis XIII au fameux pas de Suse.}} {\textsc{-
Chavigny\,; ses trahisons\,; son étrange mort.}} {\textsc{- Retraite à
Blaye de mon père et sa cause jusqu'à la mort du cardinal de Richelieu,
et cependant employé et toujours dans la faveur.}} {\textsc{- Mort
sublime de Louis XIII qui fait mon père grand écuyer de France.}}
{\textsc{- Prophétie de Louis XIII mourant.}} {\textsc{- Scélératesse
qui prive mon père de la charge de grand écuyer et qui la donne au comte
d'Harcourt.}} {\textsc{- Fortune de Beringhen, premier écuyer.}}
{\textsc{- Premier mariage de mon père.}} {\textsc{- Sa fidélité.}}
{\textsc{- Contraste étrange de fidélité et de perfidie de mon père et
du comte d'Harcourt.}} {\textsc{- Refus héroïque de mon père.}}
{\textsc{- Quel était le marquis de Saint-Mégrin.}} {\textsc{- Origine
du bonnet que MM. de Brissac et depuis MM. de La Trémoille et de
Luxembourg ont à leurs armes.}} {\textsc{- Deuxième mariage de mon
père.}} {\textsc{- Combat de mon père contre le marquis de Vardes.}}
{\textsc{- Étrange éclat de mon père sur un endroit des Mémoires de M.
de La Rochefoucauld.}} {\textsc{- Gratitude de mon père jusqu'à sa mort
pour Louis XIII.}} \relsize{1}

~

Les diverses ruses suivies de toutes les difficultés militaires que le
fameux Charles-Emmanuel avait employées au délai d'un traité et à
l'occupation de son duché de Savoie, l'avaient mis en état de se bien
fortifier à Suse\,; d'en empêcher les approches par de prodigieux
retranchements bien gardés, si connus sous le nom des barricades de
Suse, et d'y attendre les troupes impériales et espagnoles dont l'armée
venait à son secours. Ces dispositions, favorisées par les précipices du
terrain à forcer, arrêtèrent le cardinal de Richelieu, qui ne jugea pas
à propos d'y risquer les troupes, et qui emporta l'avis de tous les
généraux à la retraite. Le roi ne la put goûter. Il s'opiniâtra à
chercher des moyens de vaincre tant et de si grands obstacles naturels
et artificiels, auxquels le duc de Savoie n'avait rien épargné. Le
cardinal, résolu de n'y pas commettre l'armée, empêchait les généraux
d'y donner aucun secours au roi, qui, s'irritant des difficultés, ne
chercha plus les ressources qu'en soi-même.

Pour le dégoûter, le cardinal y ajouta l'industrie\,: il fit en sorte
que, sous divers prétextes, le roi était laissé fort seul tous les
soirs, après s'être fatigué toute la journée à tourner le pays pour
chercher quelques passages, ce qui dura ainsi plusieurs jours. Mon père,
qui s'aperçut que les soirées paraissaient en effet longues au roi
depuis le retour de ses promenades jusqu'au coucher, s'avisa de profiter
du goût de ce prince pour la musique et lui fit entendre Nyert il s'en
amusa quelques soirs, jusqu'à ce qu'enfin, ayant trouvé un passage à
l'aide d'un paysan et encore plus de lui-même, il fit seul toute la
disposition de l'attaque et l'exécuta glorieusement le 9 mars 1629. J'ai
ouï conter à mon père, qui fut toujours auprès de sa personne, qu'il
mena lui-même ses troupes aux retranchements et qu'il les escalada à
leur tête, l'épée à la main et poussé par les épaules pour escalader sur
les roches et sur les tonneaux et sur les parapets.

Sa victoire fut complète. Suse fut emportée après, ne pouvant se
soutenir devant le vainqueur. Mais ce que je ne puis assez m'étonner de
ne trouver point dans les histoires de ces temps-là, et que mon père m'a
raconté comme l'ayant vu de ses deux yeux, c'est que le duc de Savoie
éperdu vint à la rencontre du roi, mit pied à terre, lui embrassa la
botte et lui demanda grâce et pardon\,; que le roi, sans faire aucune
mine de mettre pied à terre, lui accorda en considération de son fils et
plus encore de sa sœur, qu'il avait eu l'honneur d'épouser. Ce furent
les termes du roi à M. de Savoie.

On sait combien il tâcha d'en abuser aussitôt après qu'il se vit délivré
de la présence d'un prince qui ne devait un si grand succès qu'à sa
ferme volonté de le remporter, à ses travaux pour y parvenir et à son
épée pour en remporter tout le prix et la gloire, et combien ce duc en
fut châtié par le prompt retour du roi. Mais depuis cette humiliation de
Charles-Emmanuel, ce prince, si longuement et si dangereusement compté
dans toute l'Europe, qui s'était emparé du marquisat de Saluces pendant
les derniers désordres de la Ligue sous Henri III, qui avait donné tant
de peine à Henri IV régnant et affermi dans la paix, et qui n'avait pu
être forcé à rendre ce fameux vol à un roi si guerrier,
Charles-Emmanuel, dis-je, depuis son humiliation, ne parut plus en
public de dépit et de honte, s'enferma dans son palais, n'y vit que ses
ministres, pour les ordres seulement qu'il avait à leur donner, et son
fils des moments nécessaires, aucun de ses domestiques, que les plus
indispensables et pour le service personnel seulement dont il ne put se
passer. Il mourut enfin de honte et de douleur le 26 juillet 1630,
c'est-à-dire treize mois après. C'est ainsi que Louis XIII sut protéger
le nouveau duc de Mantoue, auparavant son sujet, et l'établir et le
maintenir dans les États que la nature et la loi lui donnaient, malgré
la maison d'Autriche, celle de Savoie et toutes leurs armées.

Pour en revenir à Nyert, le roi à son retour continua de s'en amuser.
Mon père, qui était l'homme du monde le mieux faisant, vit jour à sa
fortune. Il demanda à M. de Mortemart s'il trouverait bon que le roi le
prît à lui, et ce seigneur, qui aimait Nyert, y consentit\,; mon père ne
tarda pas à le donner au roi, et, assez peu de temps {[}après{]}, le fit
premier valet de chambre.

Il est difficile d'avoir un peu lu des histoires et des Mémoires du
règne de Louis XIII et de la minorité du roi son fils, sans y avoir vu
M. de Chavigny faire d'étranges personnages auprès du roi, du cardinal
de Richelieu, des deux reines, de Gaston, à qui, bien que secrétaire
d'État, il ne fut donné pour chancelier, malgré ce prince, que pour être
son espion domestique. Il ne se conduisit pas plus honnêtement après la
mort du roi, avec les principaux personnages, avec la reine, avec le
cardinal Mazarin, avec M. le Prince, père et fils, avec la Fronde, avec
le parlement, et ne fut fidèle à pas un des partis qu'autant que son
intérêt l'y engagea. Sa catastrophe ne le corrigea point. Ramassé par M.
le Prince, il le trompa enfin, et il fut découvert au moment qu'il s'y
attendait le moins. M. le Prince, outré de la perfidie d'un homme qu'il
avait tiré d'une situation perdue, éclata et l'envoya chercher.
Chavigny, averti de la colère de M. le Prince, dont il connaissait
l'impétuosité, fit le malade et s'enferma chez lui\,; mais M. le Prince,
outré contre lui, ne tâta point de cette nouvelle duperie, et partit de
l'hôtel de Condé suivi de l'élite de cette florissante jeunesse de la
cour qui s'était attachée à lui, et dont il était peu dont les pères et
eux-mêmes n'eussent éprouvé ce que Chavigny savait faire, et qui ne
s'étaient pas épargnés à échauffer encore M. le Prince. Il arriva, ainsi
escorté, chez Chavigny, à qui il dit ce qui l'amenait, et qui, se voyant
mis au clair, n'eut recours qu'au pardon. Mais M. le Prince, qui n'était
pas venu chez lui pour le lui accorder, lui reprocha ses trahisons sans
ménagement, et l'insulta par les termes et les injures les plus
outrageants\,; les menaces les plus méprisantes et les plus fâcheuses
comblèrent ce torrent de colère, et Chavigny de rage et du plus violent
désespoir. M. le Prince sortit après s'être soulagé de la sorte en si
bonne compagnie. Chavigny, perdu de tous côtés, se vit ruiné, perdu sans
ressource et hors d'état de pouvoir se venger. La fièvre le prit le jour
même et l'emporta trois jours après.

Tel fut l'ennemi de mon père, qui lui coûta cher par deux fois. Il était
secrétaire d'État et avait la guerre dans son département. Soit sottise,
soit malice, il pourvut fort mal les places de Picardie, dont les
Espagnols surent bien profiter en 1636 qu'ils prirent Corbie. Mon père
avait un oncle qui commandait à la Capelle, et qui demandait sans cesse
des vivres et surtout des munitions, dont il manquait absolument. Mon
père en parla plusieurs fois à Chavigny, puis au cardinal de Richelieu,
enfin au roi, sans avoir pu obtenir le moindre secours. La Capelle,
dénuée de tout, tomba comme les autres places voisines. On a vu plus
haut que le courage d'esprit et de cour de Louis XIII ne laissa pas
jouir longtemps ses ennemis de leurs avantages\,; mais naturellement
ennemi de la lâcheté, et plein encore du péril que l'État avait couru
par la prompte chute des places de Picardie, il en voulut châtier les
gouverneurs à son retour à Paris. Chavigny l'y poussait. Il était lors
dans la plus grande confiance du cardinal de Richelieu\,; il lui donna
de l'ombrage de la faveur de mon père et le fit consentir à s'en
délivrer, quoique autrefois cette même faveur l'eût sauvé. L'oncle de
mon père fut donc attaqué comme les autres. Mon père ne put souffrir
cette injustice. Il fit souvenir des efforts inutiles qu'il avait faits
pour faire envoyer des munitions à son oncle, il prouva qu'il en
manquait entièrement\,; mais le parti était pris et on aigrit le roi de
l'aigreur de mon père, qui avait éclaté contre Chavigny et parlé
hautement au cardinal qui le protégeait. Piqué à l'excès, et surtout de
trouver pour la première fois de sa vie le roi différent pour lui de ce
qu'il l'avait toujours éprouvé, il lui demanda la permission de se
retirer à Blaye, et il fut pris au mot. Il s'y en alla donc au
commencement de 1637, et il y demeura jusqu'à la mort du cardinal de
Richelieu. Dans cet éloignement, le roi lui écrivit souvent et presque
toujours en un langage qu'ils s'étaient composé pour se parler devant le
monde sans en être entendus, et j'en ai encore beaucoup de lettres, avec
un grand regret d'en ignorer le contenu.

Le goût du roi ne put être émoussé par l'absence, et la confiance
subsista telle qu'il ordonna à mon père d'aller trouver M. le Prince en
Catalogne en 1639, et de lui rendre compte en leur langage de ce qui s'y
passerait. Il s'y distingua extrêmement par sa valeur, et il fut
toujours considéré dans cette armée, non seulement comme l'ami
particulier de M. le Prince, mais comme l'homme de confiance du roi,
bien que éloigné de lui. L'année d'auparavant, il avait commandé la
cavalerie sous le même prince de Condé, au siège de Fontarabie\,; avec
la même confiance du roi et le même bonheur pour lui-même, en une
occasion où le mauvais succès ne laissa d'honneur à partager qu'entre si
peu de personnes. Mon père, toujours soutenu par ce commerce direct avec
le roi, et inintelligible à tous autres, et par deux expéditions de
suite, où il fut si honorablement employé, passa ainsi quatre ans à
Blaye et fut rappelé par une lettre du roi qu'il lui envoya par un
courrier, lors de la dernière extrémité du cardinal de Richelieu. Mon
père se rendit aussitôt à la cour où il fut mieux que jamais, mais dont
il ne put sentir la joie, par l'état où il trouva le roi, qui ne vécut
plus que quelques mois.

On sait avec quel courage, quelle solide piété, quel mépris du monde et
de toutes ses grandeurs, dont il était au comble, quelle présence et
quelle liberté d'esprit, il étonna tout ce qui fut témoin de ses
derniers jours, et avec quelle prévoyance et quelle sagesse il donna
ordre à l'administration de l'état après lui, dont il fit lire toutes
les dispositions devant tous les princes du sang, les grands, les
officiers de la couronne et les députés du parlement, mandés exprès dans
sa chambre, en présence de son conseil. Il connaissait trop les esprits
des personnes qui nécessairement après lui se trouveraient portées de
droit au timon des affaires, pour ne leur laisser la disposition que de
celles qu'il ne pouvait pas faire avant de mourir. Il dicta donc un long
écrit à Chavigny de ses dernières volontés les plus particulières, et il
y remplit tout ce qui vaquait.

Il n'y avait point de grand écuyer depuis la mort funeste de
Cinq-Mars\,; cette belle charge fut donnée à mon père\,: l'écrit dicté à
Chavigny fut lu tout haut devant tout le monde, comme les dispositions
concernant l'État l'avaient été, mais non devant le même nombre ni avec
les mêmes cérémonies. Tout ce que le roi en put défendre pour ses
obsèques le fut étroitement, et comme il s'occupait souvent de la vue de
Saint-Denis, que ses fenêtres lui découvraient de son lit, il régla
jusqu'au chemin de son convoi pour éviter le plus qu'il put à un nombre
de curés de venir à sa rencontre, et il ordonna jusqu'à l'attelage qui
devait mener son chariot avec une paix et un détachement incomparables,
un désir d'aller à Dieu, et un soin de s'occuper toujours de sa mort,
qui le fit descendre dans tous ces détails.

Mon père, éperdu de douleur, ne put répondre au roi qui lui apprit qu'il
l'avait fait grand écuyer, que par se jeter sur ses mains et les inonder
de ses larmes, ni autrement que par elles aux compliments qu'il en
reçut. Sa douleur lui déroba sans doute une infinité de grandes choses
qui, dans le détail, se passèrent dans les derniers temps de la vie de
son maître, et c'est sans doute ce qui m'a empêché de savoir par lui ce
que j'ai appris de Priolo.

C'était un noble Vénitien, né en France d'un père exilé de sa patrie, et
qui s'attacha au duc de Longueville, qui à la mort de Louis XIII venait
d'épouser en secondes noces la fille de M. le Prince qui a fait tant de
bruit dans le monde, parmi les troubles et les guerres civiles de la
jeunesse de Louis XIV. Priolo a fait une histoire latine de cette
minorité, dont l'extrême élégance est la moindre partie. On y voit un
homme extrêmement instruit et souvent acteur, traitant lui-même avec la
reine et avec le cardinal Mazarin, pour ceux à qui il était attaché, et
avec tous les personnages, dont il fait des portraits parfaitement
ressemblants. On voit dans cet ouvrage qu'il avait toute la confiance de
son parti, qu'il y influait par ses conseils, qu'il avait une
pénétration profonde, une grande probité, et l'amour de la vérité\,; et
l'exactitude à la transmettre à la postérité s'y faisait sentir partout,
jusque dans les choses les moins avantageuses, et qu'il aurait pu cacher
des fautes et des faiblesses des personnes qui il était attaché. Dès les
premières pages de son histoire, qu'il commence à la mort de Louis XIII,
et qu'il décrit courtement, mais avec des traits admirables pour ce
prince, il rapporte un fait merveilleux, et qu'il était en situation
d'avoir appris sur-le-champ de M. le Prince même et de M. de
Longueville. Parlant du roi qui mourut le lendemain \footnote{Benj.
  Prioli \emph{Ab excessu Lud. XIII de rebus Gallicis hist.} libri XII.
  \emph{Ad Ser. Pr. Et Aug.~Sen.~Reip. Venet}., 1 vol.~in-4, Carolopoli,
  typ. God. Ponceleti Ser. D. Munt. typ. (\emph{Note de Saint-Simon}.)
  --- Le texte de Priolo est un peu altéré dans cette citation. Voici la
  reproduction exacte du passage\,: « Condeum intuius, Filius tuus,
  inquit, insignem victoriam reportavit\ldots. Id ante efflatam animam
  Ludv. magis praesagium, quam menis alienatae signum dedit. Gast.
  Aurel. fratrem unicum serio monuit, etc\ldots. Quae toties
  concionatoribus intonata hic reticeo. Nullus mortalium nec antiquorum
  nec recentiorum fatum ultimum tam intrepide excepit.\,»}\,:
\emph{Condaeum intuius}, dit-il livre Ier, page 17, \emph{Filius tuus},
inquit, \emph{insignem victoriam reporiavit} (comme les prophètes, ce
qui va arriver comme déjà passé)\ldots. \emph{Id ante efflatam animam,
Ludov. magis praesagio, quam mentis alienatae signum dedit. Gast. Aurel.
fratrem unicum serio monuit, etc\ldots. Quae toties concionatoribus
intonata reticeo. Nullus mortalium nec antiquorum nec recentiorum fatum
ultimum tam intrepide excepit}.

Pour revenir à mon père et à sa nouvelle charge, il en fit les fonctions
aux obsèques du roi, et il m'a souvent dit qu'en jetant l'épée royale
dans le caveau, il fut au moment de s'y jeter lui-même. Il ne pensait
qu'à sa douleur, et ses amis le pressaient d'envoyer chercher ses
provisions de grand écuyer sans qu'ils le pussent distraire. À la fin
pourtant il y envoya\,; ce fut inutilement\,: elles n'étaient pas,
disait-on, expédiées.

Le crime rend honteux, on ne l'avoue que le plus tard qu'on peut\,;
cependant, après plusieurs mois, il apprit que Chavigny avait laissé son
nom en blanc, bien sûr que le roi, en l'état extrême où il se trouvait
lorsqu'il lui dicta ses dernières dispositions, signerait sans lire,
ainsi qu'il arriva\,; que Chavigny avait été trouver la reine, auprès de
laquelle il s'était fait un mérite de sa scélératesse, pour lui laisser
la disposition de la charge de grand écuyer, dont il remplirait le nom à
son choix, afin que celui à qui elle donnerait cet office de la
couronne, mon père ou un autre, lui en eût l'obligation entière, et
qu'elle pût s'acquérir une créature considérable par ce grand bienfait à
l'entrée de sa régence. Chavigny n'ignorait pas que l'aversion que la
reine avait pour le roi s'étendait à tout ce qu'il aimait, même sans
autre cause, et qu'avec ce détour mon père ne serait point grand écuyer.
La comtesse d'Harcourt, quoique nièce du cardinal de Richelieu, avait
depuis longtemps trouvé grâce devant elle, et les moyens de se mettre
intimement bien avec elle, ce qui a duré jusqu'à sa mort. Elle fut bien
avertie, et le comte d'Harcourt fut grand écuyer.

À cette nouvelle on peut juger de l'indignation de mon père\,; la reine
lui était trop respectable, et Chavigny trop vil\,: il envoya appeler le
comte d'Harcourt. Les exploits et la valeur de celui-ci mettaient sa
réputation au-dessus du refus d'un combat particulier, dont la cause
était si odieuse. Il avertit la reine qui leur envoya à chacun un exempt
des gardes du corps. Elle n'oublia rien pour apaiser ou plutôt pour
tromper mon père. Les amis communs s'entremirent\,; tout fut inutile, et
mon père, sans s'emporter, persévéra toujours à vouloir tirer raison de
cette iniquité l'épée à la main. Il n'y put parvenir, et les exempts des
gardes du roi demeurèrent auprès d'eux fort longtemps et tant qu'ils
furent à portée l'un de l'autre. Désespérant enfin de se pouvoir
satisfaire, mon père s'en alla à Blaye et mit en vente la seule charge
qui lui restait, qui était celle de premier écuyer.

Lors de ce grand vacarme, qui fit tant de bruit dans le monde, du
commerce et des intelligences de la reine avec l'Espagne, où la reine,
par l'ordre du roi, fut fouillée jusque dans son sein, au Val-de-Grâce,
par le chancelier Séguier, qui par sa politique conduite en cette
occasion s'assura pour toujours de la faveur de la reine sans se
commettre avec le roi ni avec le cardinal de Richelieu, tout ce qui
était le plus alors dans la confidence prit la fuite ou fut chassé.
M\textsuperscript{me} de Senecey, sa dame d'honneur, le fut chez elle à
Randan en Auvergne, et M\textsuperscript{me} de Brassac, tante
paternelle de M. de Montausier, fut mise en sa place.
M\textsuperscript{me} de Chevreuse s'enfuit en Flandre, et Beringhen,
premier valet de chambre du roi après son père, se sauva à Bruxelles.
C'était un homme d'esprit et d'intrigue, et le plus avant dans celle-là,
parce qu'il était sur le pied qu'on pouvait se fier à son secret et à sa
parole.

Dès que la reine fut veuve et régente, son premier soin fut de rappeler
et récompenser ses martyrs. M\textsuperscript{me} de Chevreuse accourut
comptant tout gouverner, et y fut trompée. M\textsuperscript{me} de
Brassac fut congédiée, et M\textsuperscript{me} de Senecey rétablie, et
pour dédommagement, la comtesse de Fleix, sa fille, eut sa survivance\,:
elles jouirent toutes deux de toute la confiance et de la plus intime
faveur de la reine le reste de sa vie, devinrent duchesses, et avec
elles, M. de Foix, fils de la comtesse de Fleix, duc et pair.

Beringhen reçut à Bruxelles un courrier de la reine, et arriva auprès
d'elle dans les premiers jours de sa puissance. Il fut de tous le plus
prodigieusement récompensé\,: je dis avec raison prodigieusement.

Henri IV, tout au commencement de son règne, lors très mal affermi,
passait pays à cheval avec une très petite suite, et s'arrêta chez un
gentilhomme pour faire repaître ses chevaux, manger un morceau, et
gagner pays\,: c'était en Normandie et il ne le connaissait point. Ce
gentilhomme le reçut le mieux qu'il put dans la surprise, et le promena
par sa maison en attendant que le dîner fût prêt. Il était curieux en
armes et en avait une chambre assez bien remplie. Henri IV se récria sur
la propreté dont elles étaient tenues, et voulut voir celui qui en avait
le soin. Le gentilhomme lui dit que c'était un Hollandais qu'il avait à
son service, et lui montra le père de Beringhen. Le roi le loua tant et
dit si souvent qu'il serait bien heureux d'avoir des armes aussi
propres, que quelques-uns de sa suite comprirent qu'il avait envie du
Hollandais et le dirent au gentilhomme. Celui-ci, ravi de faire sa cour
au roi et plaisir à son domestique, le lui offrit, et après quelques
compliments, le roi lui avoua qu'il lui faisait plaisir. Beringhen eut
le même soin des armes du roi, lui plut par là, et en eut à la fin une
charge de premier valet de chambre qu'il fit passer à son fils.

Ce fils avait perdu cette charge par sa fuite. Chamarande, père de
l'officier général, en était pourvu\,; il s'était si bien conduit que la
reine n'eut point envie de le chasser, et Beringhen lui-même en avait
encore moins. Il avait affaire à une femme qu'il avait complètement
servie, et pour laquelle il avait été perdu, et au premier ministre qui
ne connaissait les états que pour en vouloir la confusion, et qui, dans
la primeur de son règne, voulait flatter celle par qui il régnait, et
s'acquérir des créatures importantes dans son plus intérieur. Beringhen
en sut profiter, et de premier valet de chambre fugitif osa lever les
yeux sur la charge de premier écuyer, et il l'osa avec succès. La reine
en fit son affaire, et l'obtint de mon père pour quatre cent mille
livres et vingt mille livres de pension du roi, dont il n'a de sa vie
touché que la première année. Défait d'une charge qui ne faisait plus
que lui peser, et ayant perdu mon grand-père la même année que Louis
XIII, il fut longtemps à se pouvoir résoudre de recommencer à vivre avec
ses amis.

Il était fort attaché à son père et à sa mère qu'il allait voir toutes
les semaines au Plessis près de Senlis, tant que le roi demeurait à
Paris et à Saint-Germain, et ils jouirent pleinement de sa fortune.
Revenu de Blaye, son frère aîné, qui avait grand pouvoir sur son esprit,
le pressa de se marier\,; lui-même l'était dès 1634 à la sœur du duc
d'Uzès, dont il n'eut point d'enfants. Elle était veuve de M. de Portes,
du nom de Budos, vice-amiral, chevalier de l'ordre, tué au siège de
Privas, frère de la connétable de Montmorency, mère de
M\textsuperscript{me} la Princesse et du duc de Montmorency, comme je
l'ai dit plus haut. Il avait laissé deux filles extrêmement différentes,
une Lia et une Rachel. L'aînée était également laide, méchante,
glorieuse, artificieuse\,; la cadette, belle et agréable au possible,
avec une douceur, une bonté et des agréments qui ne firent que rehausser
sa vertu, et qui la firent aimer de tout le monde. Ce fut elle que mon
père choisit\,; il l'épousa chez mon oncle, à la Versine près Chantilly,
en septembre 1644\,; et M\textsuperscript{lle} de Portes, sa sœur aînée,
ne leur pardonna jamais. Ces deux sœurs étaient cousines germaines de
M\textsuperscript{me} la Princesse, mère du héros, de M. le prince de
Conti et de M\textsuperscript{me} de Longueville, avec qui, et surtout
avec M. le Prince le père et M\textsuperscript{me} la Princesse, ce
mariage lia mon père de plus en plus.

Le voisinage de la cour ne pouvait être agréable à mon père après la
perte de son maître, et sous une régente qui lui avait ravi la charge de
grand écuyer. Il songea donc bientôt à s'en retourner à Blaye, où il
vivait en grand seigneur, aimé et recherché de tout ce qu'il y avait de
plus distingué à Bordeaux et dans les provinces voisines. Il s'y retira
donc bientôt après pour n'en revenir de longtemps. Tandis qu'il y était,
les cartes se brouillèrent à diverses reprises, et enfin on vit M. le
Prince armé contre la cour, et la guerre civile allumée. M. son père
était mort, mais il avait conservé les mêmes liaisons avec mon père, et
M\textsuperscript{me} de Longueville aussi. De cette situation avec eux
et tout opposée avec la cour, ils ne doutèrent pas d'avoir Blaye en leur
disposition et par les mesures qui leur réussirent en Guyenne et dans
les provinces voisines, disposant de Blaye, ils ne comptaient pas moins
et avec raison que partager le royaume à la rivière de Loire.

Les armes levées, mon père sourd à leurs prières songea se fortifier.
Les messages et les lettres redoublèrent inutilement. Ni l'amitié, ni
l'honneur de l'alliance si proche, ni le dépit amer contre la reine ne
purent rien obtenir. À bout d'espérances, ils tentèrent les plus grandes
avances du côté de l'Espagne. La grandesse et beaucoup d'établissements
lui furent proposés directement de la part du roi d'Espagne, qui furent
également rejetés. Enfin, un second message arriva de sa part à Blaye,
muni de lettres de créance, comme la première fois, et d'une lettre de
plus à mon père avec des propositions encore plus fortes. Dès que le
porteur se fut découvert à mon père, il jugea que c'était trop, et
sur-le-champ assembla son état-major et tous les officiers de sa
garnison avec ce qui se trouva de ses amis du voisinage dans Blaye. Là,
il leur présenta l'homme du roi d'Espagne, leur montra les lettres qu'il
portait, que mon père n'avait point voulut décacheter, exposa sa mission
en sa présence, puis lui dit que, sans le respect qu'il voulait garder à
une tête couronnée, frère de la reine mère, il le ferait jeter en ce
moment même dans la Gironde avec un boulet aux pieds, mais qu'il eût à
se retirer sur-le-champ avec ses lettres et ses propositions, qui ne
tenteraient jamais un homme de bien, et qu'il retint bien, pour en
avertir où il appartenait, que si on se jouait encore à lui envoyer
quelqu'un avec des commissions semblables il ne ménagerait plus rien et
le ferait jeter dans la rivière. Aussi n'y renvoya-t-on plus.

Mais M. le Prince et tout son parti firent les hauts cris, et, ce qui
est remarquable, jamais ni lui ni les siens ne l'ont pardonné à mon
père, tant ils l'avaient belle s'ils eussent pu l'entraîner\,! Cependant
mon père fit fondre force canons, pour remplacer celui que la cour lui
demanda faute d'autre, mit cinq cents gentilshommes bien armés dans
Blaye, habilla et paya la garnison, et fut dix-huit mois comme bloqué en
cet état sans avoir jamais rien voulu prendre sur le pays. Aussi
contracta-t-il de grandes dettes dont il a été incommodé toute sa vie,
et dont je me sens encore\,; tandis que toutes celles que M. le Prince,
M. de Bouillon et bief d'autres avaient faites contre le roi et l'État,
ont été très bien payées, et plus encore par le roi même, dans la suite
des temps. Mais ce n'est pas tout\,: mon père qui avait beaucoup d'amis
dans le parlement et dans la ville de Bordeaux, était exactement averti,
toutes les marées, de tout ce qu'il s'y passait de plus secret et en
faisait part à la cour, et pendant ces malheureux temps il rendit les
plus importants services.

La cour s'était avancée à l'entrée de la Guyenne, suivie d'une armée
commandée par le comte d'Harcourt, si grandement payé d'avance pour la
bien servir, et si capable par lui-même de le bien faire, mais il était
de la maison de Lorraine et issu des Guise, et voici le contraste\,: il
ne songea qu'à profiter de l'embarras de la cour et du désordre de
l'État, pour se rendre maître de l'Alsace et de Brisach, et les joindre
à la Lorraine. Sa partie faite il se dérobe de l'armée, perce le royaume
nuit et jour et arrive aux portes de Brisach. Comme quoi il manqua de
réussir se trouve dans tous les Mémoires de ces temps-là, et n'est pas
matière aux miens. Je me contente de rapporter la belle gratitude du
grand écuyer, fait tel aux dépens de mon père, à quoi il faut encore
ajouter qu'il tira de ce crime le gouvernement d'Anjou, mis pour lui sur
le pied des grands gouvernements, pour vouloir bien rentrer dans
l'obéissance\,; et que la charge et le gouvernement, toujours sur ce
grand pied ont passé l'un et l'autre à sa postérité. Voici et en même
temps la contrepartie\,:

La reine et le cardinal Mazarin charmés de la fidélité et des importants
services de mon père, jugèrent qu'il était à propos de le récompenser
pour le bon exemple, ou peut-être de s'en assurer de plus en plus. Pour
cela ils lui écrivirent l'un et l'autre en des termes si obligeants
qu'ils faisaient sentir leur détresse, et lui dépêchèrent le marquis de
Saint-Mégrin chargé de ces lettres et, de plus, d'une autre de créance
sur ce dont il était chargé de leur part. Saint-Mégrin portait à mon
père le bâton de maréchal de France, à son choix, ou le rang de prince
étranger sous le prétexte de la maison de Vermandois, du sang de
Charlemagne, dont nous sortons au moins par une femme, sans contestation
quelconque. Mon père refusa l'un et l'autre. Saint-Mégrin, qui était son
ami, lui représenta que, le péril passé, il n'aurait rien, et qu'il y
avait de la folie à ne pas accepter une si belle offre, qui a été toute
l'ambition des Bouillon. « Je m'y attends bien, répondit résolument mon
père, et je les connais trop pour en douter. Je compte aussi que bien
des gens se moqueront de moi\,; mais il ne sera pas dit qu'un rang de
prince étranger, ni un bâton de maréchal de France terniront ma gloire
et attaqueront mon honneur. Si j'accepte on ne doutera jamais qu'on ne
m'ait retenu dans mon devoir par une grâce, et je n'y consentirai
jamais.\,»

Trois jours entiers se passèrent en cette dispute, sans jamais pouvoir
être ébranlé. Il répondit respectueusement à la reine, mais sèchement
dans le sens qu'il avait fait à Saint-Mégrin, et ajouta, pour qu'elle
n'en prît rien pour elle, qu'il ne manquerait jamais au fils ou à la
veuve de son maître. Il en manda autant au cardinal Mazarin, mais avec
hauteur. Cet Italien n'était pas fait pour admirer une action si grande.
Dira-t-on de plus qu'elle était trop au-dessus de la portée de la
reine\,? Il arriva ce que Saint-Mégrin avait prédit\,: le péril passé,
on n'y songea plus, mais mon père aussi ne fit l'honneur à l'un ni à
l'autre de les en faire souvenir. Il continua ses dépenses et ses
services avec le même zèle jusqu'à la fin des troubles. Et voilà comment
Louis XIII lui avait bien prédit tout l'usage et le grand parti qui se
pouvait tirer de Blaye, lorsqu'il lui en donna le gouvernement.

Il faut maintenant dire qui était Saint-Mégrin. Il s'appelait Esthuert,
et par une héritière de Caussade il en joignait le nom au sien. C'était
un fort homme d'honneur quoique très bien avec la reine mère. Il avait
eu les chevau-légers de la garde, et les avait cédés à son fils, qui
avait aussi ceux de la reine mère. Ce fils fut un jeune favori du temps,
avec du mérite, qui avait fort servi pour son âge, et qui avait commandé
une armée en Catalogne\,: il fut tué au combat de la porte,
Saint-Antoine. La reine en fut fort affligée et le cardinal aussi\,; ils
le firent enterrer dans l'abbaye de Saint-Denis. Sa veuve sans enfants a
été depuis duchesse de Chaulnes, femme de l'ambassadeur et gouverneur de
Bretagne\,; la sœur de ce jeune favori épousa M. du Broutay, du nom de
Quelen, dont elle a eu postérité\,; elle se remaria à La Vauguyon,
ambassadeur en Espagne et ailleurs, chevalier de l'ordre en 1688, dont
il sera bientôt parlé, et n'en a point eu d'enfants. Il prit ce nom
d'une terre de sa femme en l'épousant. Son nom était Betoulat, et il
portait celui de Fromenteau. Saint-Mégrin, père de cette femme et du
jeune favori, qu'il survécut longtemps, était gendre du maréchal de
Roquelaure, et grand sénéchal de Guyenne. Il fut chevalier de l'ordre en
1661, et mourut en 1675, à quatre-vingt-trois ans.

La majorité, le sacre, le mariage du roi, mon père les passa tous à
Blaye, et en cette dernière occasion il reçut magnifiquement la cour.
Longtemps après il revint vivre avec ses amis à Paris\,; il en avait
beaucoup, et des gens les plus considérables, fruits de sa modestie, de
n'avoir jamais fait mal à personne, et du bien tant qu'il avait pu
pendant sa faveur. De son mariage il n'eut qu'une fille unique
parfaitement belle et sage, qu'il maria au duc de Brissac, frère de la
dernière maréchale de Villeroy. Ce fut elle qui, sans y penser, affubla
MM. de Brissac de ce bonnet qu'ils ont mis, et, à leur exemple, que MM.
de La Trémoille et de Luxembourg ont imité depuis, et avec autant de
raison les uns que les autres. Ma sœur était à Brissac avec la maréchale
de La Meilleraye, tante paternelle de son mari, extrêmement glorieuse et
folle surtout de sa maison. Elle promenait souvent M\textsuperscript{me}
de Brissac dans une galerie où les trois maréchaux étaient peints avec
le célèbre comte de Brissac, fils aîné du premier des trois, et autres
ancêtres de parure, que la généalogie aurait peine à montrer. La
maréchale admirait ces grands hommes, les saluait et leur faisait faire
des révérences par sa nièce. Elle qui était jeune et plaisante avec de
l'esprit, se voulut divertir au milieu de l'ennui qu'elle éprouvait à
Brissac, et tout à coup se mit à dire à la maréchale\,: « Ma tante,
voyez donc cette bonne tête\,! il a l'air de ces anciens princes
d'Italie, et je pense que si vous cherchiez bien, il se trouverait qu'il
l'a été. --- Mais que vous avez d'esprit et de goût, ma nièce\,! s'écria
la maréchale\,; je pense en vérité que vous avez raison.\,» Elle regarde
ce vieux portrait, l'examine, ou en fait le semblant, et tout aussitôt
déclare le bonhomme un ancien prince d'Italie, et se hâte d'aller
apprendre cette découverte à son neveu qui n'en fit que rire. Peu de
jours après elle trouva inutile d'être descendue d'un ancien prince
d'Italie, si rien n'en rappelait le souvenir. Elle imagine le bonnet des
princes d'Allemagne avec quelque petite différence dérobée par la
couronne qui l'enveloppe, envoie chercher furtivement un peintre à
Angers et lui fait mettre ce bonnet aux armes de leurs carrosses. M. et
M\textsuperscript{me} de Brissac l'apprirent bientôt. Ils en rirent de
tout leur cœur. Mais le bonnet est demeuré et s'est appelé longtemps
parmi eux \emph{le bonnet de ma tante}.

Ce mariage ne fut jamais uni, le goût de M. de Brissac était trop
italien. La séparation se fit entre les mains de M. le Prince,
homologuée au parlement\,; et M. le Prince demeura dépositaire de
papiers trop importants sur ce fait au duc de Brissac, pour qu'il ne
craignît pas infiniment qu'ils fussent remis au greffe du parlement,
comme M. le Prince s'obligea de les y remettre au cas que M. de Brissac
voulût contrevenir à aucune des conditions de la séparation.

Ma sœur mourut en février 1684, et me fit son légataire universel.
M\textsuperscript{me} sa mère était morte comme elle de la petite vérole
dès 1670 (et toutes deux à Paris), comme désignée daine d'honneur de la
reine. M\textsuperscript{me} de Montausier, qui l'était, était lors
tombée dans cette étrange maladie de corps et d'esprit qui faisait
attendre sa fin tous les jours, et qui dura pourtant plus qu'on ne
pensait, et au delà de la vie de la première femme de mon père.

Quelque affligé que mon père en fût, la considération de n'avoir point
de garçon l'engagea, quoique vieux, à se remarier. Il chercha une
personne dont la beauté lui plût, dont la vertu le pût rassurer, et dont
l'âge fût le moins disproportionné au sien qu'il fût possible. Il ne
trouva toutes ces choses si difficiles à rassembler que dans ma mère,
qui était avec M\textsuperscript{lle} de Pompadour, depuis
M\textsuperscript{me} de Saint-Luc, auprès de la duchesse d'Angoulême.
Elles étaient lasses du couvent, et leurs mères n'aimaient point à les
avoir auprès d'elles. Toutes deux étaient parentes de
M\textsuperscript{me} d'Angoulême, fille de M. de La Guiche, chevalier
de l'ordre et grand maître de l'artillerie, et veuve, qui les prit avec
elle, et chez qui elles furent toutes deux mariées.

Ma mère était L'Aubépine, fille du marquis d'Hauterive, lieutenant
général des armées du roi et des états généraux des Provinces-Unies, et
colonel général des troupes françaises à leur service. La catastrophe du
garde des sceaux de Châteauneuf, son frère aîné, mis au château
d'Angoulême, lui avait coûté l'ordre, auquel il était nommé pour la
Pentecôte suivante de 1633, et le bâton de maréchal de France, qui lui
était promis. M. de Charost devant qui le cardinal de Richelieu donna
l'ordre d'arrêter les deux frères, qui avait porté le mousquet en
Hollande sous mon grand-père, comme presque toute la jeune noblesse de
ces temps-là, et qui l'appelait toujours son colonel, se déroba et vint
l'avertir comme il jouait avec les filles d'honneur de la reine. Mon
grand-père ne fit aucun semblant de rien\,; mais un moment après,
feignant un besoin pressant, il demanda permission de sortir pour un
instant, alla prendre le meilleur cheval de son écurie, et se sauva en
Hollande. Il était dans la plus intime confidence du prince d'Orange qui
lui donna le gouvernement de Breda. Il avait épousé l'héritière de
Ruffec, de la branche aînée de la maison de Voluyre, dont la mère était
sœur du père du premier duc de Mortemart\,; elle était fort riche. Mon
grand-père passa une grande partie de sa vie en Hollande, et mourut à
Paris en 1670.

Le second mariage de mon père se fit la même année en octobre. Il eut
tout lieu d'être content de son choix\,; il trouva une femme toute pour
lui, pleine de vertu, d'esprit et d'un grand sens, et qui ne songea qu'à
lui plaire et à le conserver, à prendre soin de ses affaires et à
m'élever de son mieux. Aussi ne la voulut-il que pour lui. Lorsqu'on mit
des dames du palais auprès de la reine au lieu de ses filles d'honneur,
M\textsuperscript{me} de Montespan qui aimait ses parents (c'en était
encore la mode) obtint une place pour ma mère, qui ne se doutait de rien
moins, et le lui manda. Le gentilhomme qui vint de sa part la trouva
sortie, mais on lui dit que mon père ne l'était pas. Il demanda donc à
le voir, et lui donna la lettre de M\textsuperscript{me} de Montespan
pour ma mère. Mon père l'ouvrit et tout de suite prit une plume,
remercia M\textsuperscript{me} de Montespan, et ajouta qu'à son âge il
n'avait pas pris une femme pour la cour, mais pour lui, et remit cette
réponse au gentilhomme. Ma mère, de retour, apprit la chose par mon
père. Elle y eut grand regret, mais il n'y parut jamais.

Avant de finir ce qui regarde mon père, je me souviens de deux aventures
d'éclat que j'aurais dû placer plus haut et longtemps avant son second
mariage. Un dévolu, sur un bénéfice, fut cause de la première qui fit un
procès entre un parent de M. de Vardes et un de mon père. Chacun soutint
son parent avec chaleur, et les choses allèrent, si loin, que M. le
Prince prit leurs paroles. Longtemps après et l'affaire assoupie, M. le
Prince la leur rendit comme à des gens qui n'ont plus rien à démêler.
Cette affaire leur était demeurée sur le cœur, et bien plus encore à
Vardes qui, après avoir laisser écouler quelque temps, convint avec mon
père des battre à la porte Saint-Honoré, sur le midi, lieu alors fort
désert, et que, pour que ce combat parût une rencontre et par l'heure et
par tout le reste, le carrosse de M. de Vardes couperait celui de mon
père, et que les maîtres, prenant la querelle des cochers, mettraient
pied à terre avec chacun un second et se battraient là tout de suite.
C'était pendant la régence, et en des âges fort inégaux. Le matin, mon
père alla voir M. le Prince et plusieurs des premiers magistrats de ses
amis, et finit par le Palais-Royal faire sa cour à la reine. Il affecta
d'en sortir avec le maréchal de Grammont, et d'aller avec lui faire des
visites au Marais. Comme ils descendaient ensemble le degré, mon père
feignit d'avoir oublié quelque chose en haut, s'excuse et remonte, puis
redescend, trouve La Roque Saint-Chamarant, très brave gentilhomme qui
lui était fort attaché et qui commandait son régiment de cavalerie, qui
lui devait servir de second, monte avec lui en carrosse, et s'en vont à
la porte Saint-Honoré. Vardes, qui attendait au coin d'une rue, joint le
carrosse de mon père, le frôle, le coupe\,; coups de fouet de son cocher
et riposte de celui de mon père\,; têtes aux portières\,; ils arrêtent,
et pied à terre. Ils mettent l'épée à la main. Le bonheur en voulut à
mon père\,; Vardes tomba et fut désarmé. Mon père lui voulut faire
demander la vie\,; il ne le voulut pas. Mon père lui dit qu'au moins il
le balafrerait\,; Vardes l'assura qu'il était trop généreux pour le
faire, mais qu'il se confessait vaincu. Alors mon père le releva et alla
séparer les seconds. Le carrosse de mon père se trouvant par hasard le
plus proche, Vardes parut pressé d'y monter. Mon père et La Roque
Saint-Chamarant y montèrent avec lui, et le remenèrent chez lui. Il se
trouva mal en chemin, et blessé au bras. Ils se séparèrent civilement en
braves gens, et mon père s'en alla chez lui.

M\textsuperscript{me} de Châtillon, depuis de Meckelbourg \footnote{Élisabeth-Angélique
  de Montmorency-Bouteville, sœur du maréchal de Luxembourg, avait
  épousé en premières noces Gaspard de Coligny, duc de Châtillon, et en
  secondes noces Christian-Louis, duc de Mecklenbourg. On disait au
  XVIIe siècle \emph{Meckelbourg}.}, logeait dans une des dernières
maisons, près de la porte Saint-Honoré, qui, au bruit des carrosses et
des cochers, mit la tête à la fenêtre et vit froidement tout le combat.
Il ne tarda pas à faire grand bruit. La reine, Monsieur, M. le Prince et
tout ce qu'il y avait de plus distingué, envoyèrent chez mon père, qui,
peu après, alla au Palais-Royal et trouva la reine au cercle\,: on peut
croire qu'il y essuya bien des questions et que ses réponses étaient
bien préparées. Pendant qu'il recevait tous ces compliments, Vardes
avait été conduit à la Bastille, par ordre de la reine, et y fut dix ou
douze jours. Mon père ne cessa de paraître à la cour et partout, et
d'être bien reçu partout. Telle fut la fin de cette affaire qui ne passa
jamais que pour ce qu'elle parut, et Vardes pour l'agresseur. Il eut un
grand chagrin de son triste succès, et un dépit amer de la Bastille.
Oncques depuis il n'a revu mon père qu'à la mort\,; à la vérité sa
disgrâce le tint bien des années en Languedoc. Son retour fut de peu
d'années\,; il mourut à Paris, en 1688, d'une fort longue maladie. Sur
la fin, il envoya prier mon père de l'aller voir. Il se raccommoda
parfaitement avec lui et le pria de revenir\,; mon père y retourna
souvent, et le vit toujours dans le peu qu'il vécut depuis.

L'autre aventure était pour finir comme celle-ci, mais elle se termina
plus doucement. Il parut des Mémoires de M. de La Rochefoucauld\,; mon
père fut curieux d'y voir les affaires de son temps. Il y trouva qu'il
avait promis à M. le Prince de se déclarer pour lui, qu'il lui avait
manqué de parole, et que le défaut d'avoir pu disposer de Blaye, comme
M. le Prince s'y attendait, avait fait un tort extrême à son parti.
L'attachement plus que très grand de M. de La Rochefoucauld à
M\textsuperscript{me} de Longueville n'est inconnu à personne. Cette
princesse, étant à Bordeaux, avait fait tout ce qu'elle avait pu pour
séduire mon père, par lettres\,; espérant mieux de ses grâces et de son
éloquence, elle avait fait l'impossible pour obtenir de lui une
entrevue, et demeura piquée à l'excès de n'avoir pu l'obtenir. M. de La
Rochefoucauld, ruiné, en disgrâce profonde (dont la faveur de son
heureux fils releva bien sa maison sans en avoir pu relever son père),
ne pouvait oublier l'entière différence que Blaye, assurée ou contraire,
avait mise au succès du parti, et le vengea autant qu'il put et
M\textsuperscript{me} de Longueville, par ce narré.

Mon père sentit si vivement l'atrocité de la calomnie, qu'il se jeta sur
une plume et mit à la marge\,: \emph{L'auteur en a menti}. Non content
de ce qu'il venait de faire, il s'en alla chez le libraire qu'il
découvrit, parce que cet ouvrage ne se débitait pas publiquement dans
cette première nouveauté. Il voulut voir ses exemplaires, pria, promit,
menaça et fit si bien qu'il se les fit montrer. Il prit aussitôt une
plume et mit à tous la même note marginale. On peut juger de
l'étonnement du libraire, et qu'il ne fut pas longtemps sans faire
avertir M. de La Rochefoucauld de ce qui venait d'arriver à ses
exemplaires. On peut croire aussi que ce dernier en fut outré. Cela fit
grand bruit alors, et mon père en fit plus que l'auteur ni ses amis\,:
il avait la vérité pour lui, et une vérité qui n'était encore ni oubliée
ni vieillie. Les amis s'interposèrent\,; mon père voulait une
satisfaction publique. La cour s'en mêla, et la faveur naissante du
fils, avec les excuses et les compliments, firent recevoir pour telle
celle que mon père s'était donnée sur les exemplaires et par ses
discours.

On prétendit que c'était une méprise mal fondée sur ce que
M\textsuperscript{me} la Princesse, venue furtivement à Paris pour
réclamer la protection du parlement sur la prison des princes ses
enfants, avait présenté sa requête elle-même à la porte de la
grand'chambre, appuyée sur mon oncle qui, par la proximité, n'avait pu
lui refuser cet office\,; que cela avait fait espérer qu'il suivrait le
parti, ce qu'il ne fit toutefois jamais, et qu'ayant un grand crédit sur
mon père, qui était à Blaye, il l'engagerait avec sa place dans les
mêmes intérêts. Tous ces propos furent reçus pour ce qu'ils valaient, et
les choses en demeurèrent là après cet éclat, mon père n'en pouvant
espérer davantage\,; et de l'autre côté par la difficulté de soutenir un
mensonge si fort avéré par tant de gens principaux et des premières
têtes encore vivants et qui savaient la vérité, qui n'avaient jusque-là
jamais été mise en doute. Mais il est vrai que jamais MM. de La
Rochefoucauld ne l'ont pardonné à mon père, tant il est vrai qu'on
oublie moins encore les injures qu'on fait que celles même qu'on reçoit.

Mon père passa le reste d'une longue et saine vie de corps et d'esprit,
sans aucune faveur, mais avec une considération que le roi se tenait
comme obligé de lui devoir, et qui influait sur les ministres, entre
lesquels il était ami de M. Colbert\,: la vertu était encore comptée.
Les seigneurs principaux, même fort au-dessus de son âge et les plus de
la cour, le voyaient chez lui et mangeaient quelquefois, où je les ai
vus. Il avait beaucoup d'amis parmi les personnes de tous les états, et
force connaissances qui le cultivaient, outre quelques amis intimes et
particuliers. Il les vit tous jusque dans la dernière vieillesse, et
avait tous les jours bonne chère et bonne compagnie chez lui à dîner.
Dans son gouvernement, il y était tellement le maître, que de Paris il y
commandait et disposait de tout. Si quelque place vaquait dans
l'état-major, le roi lui envoyait la liste des demandeurs\,; quelquefois
il y choisissait, d'autres fois il demandait un homme qui ne s'y
trouvait pas. Rien ne lui était refusé, jusque-là qu'il faisait ôter
ceux dont il n'était pas content, comme je l'ai vu d'un major, puis d'un
lieutenant de roi, et mettre en la place du dernier, à la prière d'un de
ses amis intimes, un officier appelé Dastor, qui avait quitté le service
depuis près de vingt ans et était retiré dans sa province. Mon père
était unique dans cette autorité, et le roi disait, qu'après les
services signalés qu'il lui avait rendus, par ce gouvernement, dans les
temps les plus fâcheux, il était juste qu'il y disposât de tout
absolument.

Jamais il ne se consola de la mort de Louis XIII, jamais il n'en parla
que les larmes aux yeux, jamais il ne le nomma que le roi son maître,
jamais il ne manqua d'aller à Saint-Denis à son service, tous les ans,
le 14 de mai, et d'en faire faire un solennel à Blaye, lorsqu'il s'y
trouvait dans ce temps-là. C'était la vénération, la reconnaissance, la
tendresse même qui s'exprimait par sa bouche toutes les fois qu'il
parlait de lui\,; et il triomphait quand il s'étendait sur ses exploits
personnels et sur ses vertus, et avant que de me présenter au roi il me
mena un 14 de mai à Saint-Denis (je ne puis finir de parler de lui par
des traits plus touchants ni plus illustres). Il était indigné d'être
tout seul à Saint-Denis. Outre sa dignité, ses charges et ses biens
qu'il devait en entier à Louis XIII, n'ayant jamais rien eu de sa
maison, c'était à ses bontés, à son amitié, au soin paternel de le
former, à sa confiance intime et entière qu'il était le plus tendrement
sensible, et c'est à cette privation, non au changement de fortune,
qu'il ne se put jamais accoutumer.

\hypertarget{chapitre-vi.}{%
\chapter{CHAPITRE VI.}\label{chapitre-vi.}}

1693

~

\relsize{-1}

{\textsc{Départ subit du roi pour Versailles, et de Monseigneur avec le
maréchal de Boufflers pour le Rhin.}} {\textsc{- Monsieur sur les
côtes.}} {\textsc{- Tilly défait.}} {\textsc{- Huy rendu au maréchal de
Villeroy.}} {\textsc{- Bataille de Neerwinden.}} \relsize{1}

~

Après avoir rendu les derniers devoirs à mon père, je m'en allai à Mons
joindre le Royal-Roussillon cavalerie, où j'étais capitaine. Montfort,
gentilhomme du pays du Maine, en était mestre de camp, qui était un
officier de distinction, et brigadier, et qui fut mis à la tête de tous
les carabiniers de l'armée, dont on faisait toujours une brigade à part
avant qu'on en eût fait un corps pour M. du Maine. Puyrobert,
gentilhomme d'Angoumois, voisin de Ruffec, en était lieutenant-colonel,
et d'Achy, du nom de Courvoisin fort connu en Picardie, y était
capitaine avec commandement de mestre de camp, après en avoir été
lieutenant-colonel. On ne saurait trois plus honnêtes gens ni plus
différents qu'ils l'étaient. Le premier était le meilleur homme du
monde, le troisième très vif et très pétulant\,; le second d'excellente
compagnie\,; le premier et le dernier surtout avec de l'esprit. Le major
était frère de Montfort, et d'ailleurs le régiment bien composé\,; ils
étaient lors, tant les royaux que plusieurs gris à douze compagnies de
cinquante cavaliers, faisant quatre escadrons. On ne peut être mieux
avec eux tous que j'y fus, et c'était à qui me préviendrait de plus
d'honnêtetés et de déférence, à quoi je répondis de manière à me les
faire continuer, de manière que d'Achy, qui commanda le régiment par
l'absence de Montfort et qui était aux couteaux tirés avec Puyrobert et
ne se voulait trouver nulle part avec lui, s'y laissa apprivoiser chez
moi, mais sans se parler l'un à l'autre. Notre brigade joignait
l'infanterie à la gauche de la première ligne, et fut composée de notre
régiment, de celui du duc de La Feuillade et de celui de Quoadt qui,
parce que Montfort était aux carabiniers, en fut le brigadier. L'armée
se forma et j'allai faire ma cour aux généraux et aux princes.

Le roi partit le 18 mai avec les dames, fit avec elles huit ou dix jours
de séjour au Quesnoy, les envoya ensuite à Namur, et s'alla mettre à la
tête de l'armée de M. de Boufflers, le 2 juin, avec laquelle il prit, le
7 du même mois, le camp de Gembloux\,; en sorte qu'il n'y avait pas demi
lieue de sa gauche à la droite de M. de Luxembourg, et qu'on allait et
venait en sûreté de l'une à l'autre. Le prince d'Orange était campé à
l'abbaye de Parc, de manière qu'il n'y pouvait recevoir de subsistances,
et qu'il n'en pouvait sortir sans avoir les deux armées du roi sur les
bras. Il s'y retrancha à la hâte et se repentit bien de s'y être laissé
acculer si promptement. On a su depuis qu'il écrivit plusieurs fois au
prince de Vaudemont, son ami intime, qu'il était perdu et qu'il n'y
avait que par un miracle qu'il en pût échapper. Son armée était
inférieure à la moindre des deux du roi, qui l'une et l'autre étaient
abondamment pourvues d'équipages, de vivres et d'artillerie, et qui,
comme on peut croire, étaient maîtresses de la campagne.

Dans une position si parfaitement à souhait pour exécuter de grandes
choses et pour avoir quatre grands mois à en pleinement profiter, le roi
déclara le 8 juin à M. de Luxembourg qu'il s'en retournait à Versailles,
qu'il envoyait Monseigneur en Allemagne avec un gros détachement et le
maréchal de Boufflers. La surprise du maréchal de Luxembourg fut sans
pareille. Il représenta au roi la facilité de forcer les retranchements
du prince d'Orange, et de le battre entièrement avec une de ses deux
armées, et de poursuivre la victoire avec l'autre avec tout l'avantage
de la saison et de n'avoir plus d'armée vis-à-vis de soi. Il combattit
par un avantage présent, si certain et si grand, l'avantage éloigné de
forcer dans Heilbronn le prince Louis de Bade\,; et combien l'Allemagne
serait aisément en proie au maréchal de Larges, si les Impériaux
envoyaient de gros détachements en Flandre, qui n'y seraient pas même
suffisants, et qui, n'y venant pas, laisseraient tous les Pays-Bas à la
discrétion de ces deux armées. Mais la résolution était prise.
Luxembourg, au désespoir de se voir échapper une si glorieuse et si
facile campagne, se mit à deux genoux devant le roi et ne put rien
obtenir. M\textsuperscript{me} de Maintenon avait inutilement tâché
d'empêcher le voyage du roi\,: elle en craignait les absences\,; une si
heureuse ouverture de campagne y aurait retenu le roi longtemps pour en
cueillir par lui-même les lauriers\,; ses larmes à leur séparation, ses
lettres après le départ furent plus puissantes et l'emportèrent sur les
plus pressantes raisons d'État, de guerre et de gloire.

Le soir de cette funeste journée, M. de Luxembourg, outré de douleur, de
retour chez lui, en fit confidence au maréchal de Villeroy, à M. le Duc
et à M. le prince de Conti et à son fils, qui tous ne le pouvaient
croire et s'exhalèrent en désespoir. Le lendemain 9 juin, qui que ce
soit ne s'en doutait encore. Le hasard fit que j'allai seul à l'ordre
chez M. de Luxembourg, comme je faisais très souvent, pour voir ce qui
se passait et ce qui se ferait le lendemain. Je fus très surpris de n'y
trouver pas une âme, et que tout était à l'armée du roi. Pensif et
arrêté sur mon cheval, je ruminais sur un fait si singulier, et je
délibérais entre m'en retourner ou pousser jusqu'à l'armée du roi,
lorsque je vis venir de notre camp M. le prince de Conti seul aussi,
suivi d'un seul page et d'un palefrenier avec un cheval de main. «
Qu'est-ce que vous faites là\,?\,» me dit-il, en me joignant, et riant
de ma surprise\,; il me dit qu'il s'en allait prendre congé du roi et
que je ferais bien d'aller avec lui en faire autant. « Que veut dire
prendre congé\,?\,» lui répondis-je. Lui tout de suite dit à son page et
à son palefrenier de le suivre un peu de loin, et m'invita d'en dire
autant au mien et à un laquais qui me suivait. Alors il me conta la
retraite du roi, mourant de rire, et malgré ma jeunesse la chamarra
bien, parce qu'il ne se défiait pas de moi. J'écoutais de toutes mes
oreilles, et mon étonnement inexprimable ne me laissait de liberté que
pour faire quelques questions. Devisant de la sorte, nous rencontrâmes
toute la généralité qui revenait. Nous les joignîmes, et tout aussitôt
les deux maréchaux, M. le Duc, M. le prince de Conti, le prince de
Tingry, Albergotti, Puységur s'écartèrent, mirent pied à terre et y
furent une bonne demi-heure à causer, on peut ajouter à pester\,; après
quoi ils remontèrent à cheval et chacun poursuivit son chemin. M. le duc
de Chartres revint plus tard, et nous ne nous y amusâmes pas pour
arriver encore à temps, moi toujours seul avec M. le prince de Conti, et
ne cessant de nous entretenir d'un événement si étrange et si peu
attendu.

Arrivés chez le roi, nous trouvâmes la surprise peinte sur tous les
visages, et l'indignation sur plusieurs. On servit presque aussitôt
après. M. le prince de Conti monta pour prendre congé, et comme le roi
descendait le degré qui tombait dans la salle du souper, le duc de La
Trémoille me dit de monter au-devant du roi pour prendre congé aussi. Je
le fis au milieu du degré. Le roi s'arrêta à moi et me fit l'honneur de
me souhaiter une heureuse campagne. Le roi à table, je rejoignis M. le
prince de Conti et nous remontâmes à cheval. Il était extrêmement poli
et avec discernement. Il me dit qu'il avait une permission à me
demander, qui ne serait pas trop honnête\,: c'était de descendre chez M.
le Prince à qui il voulait dire adieu, et franchement un peu causer avec
lui, et cependant de vouloir bien l'attendre. Il fut environ trois
quarts d'heure avec lui. En revenant au camp, nous ne fîmes que parler
de cette nouvelle qui n'avait éclaté que ce jour-là même, et le roi et
Monseigneur partirent le lendemain pour Namur, d'où Monseigneur s'en
alla en Allemagne, et le roi, accompagné des darnes, retourna à
Versailles pour ne revenir plus sur la frontière.

L'effet de cette retraite fut incroyable jusque parmi les soldats et
même parmi les peuples. Les officiers généraux ne s'en pouvaient taire
entre eux, et les officiers particuliers en parlaient tout haut avec une
licence qui ne put être connue. Les ennemis n'en purent ni n'en
voulurent contenir leur surprise et leur joie. Tout ce qui revenait des
ennemis n'était guère plus scandaleux que ce qui se disait dans les
armées, dans les villes, à la cour même par des courtisans,
ordinairement si aises de se retrouver à Versailles, mais qui se
faisaient honneur d'en être honteux, et on sut que le prince d'Orange
avait mandé à Vaudemont qu'une main qui ne l'avait jamais trompé lui man
doit la retraite du roi\,; mais que cela était si fort qu'il ne la
pouvait espérer\,; puis, par un second billet, que sa délivrance était
certaine, que c'était un miracle qui ne se pouvait imaginer, et qui
était le salut de son armée et des Pays-Bas, et l'unique par qui il pût
arriver. Parmi tous ces bruits le roi arriva avec les dames, le 25 juin,
à Versailles.

M. de Luxembourg, allant, le 14 juillet, reconnaître un fourrage de
l'abbaye d'Heylesem où il était campé, fut averti de la marche de Tilly
avec un corps de cavalerie de six mille hommes pour se poster en lieu
d'incommoder ses convois. Là-dessus notre général fit monter à cheval
dans la nuit quarante-quatre escadrons de sa droite, qui en était la
plus à portée, avec des dragons, et marcha à eux avec les princes. On ne
put arriver sur eux que le matin, parce que, avertis par un moine
d'Heylesem, ils avaient monté à cheval\,: on les trouva sur une hauteur
avec des ravines devant eux. Marsin, le chevalier du Rosel, et
Sanguinet, exempt des gardes du corps, les attaquèrent par trois
endroits avec chacun un détachement\,; et Sanguinet, pour s'être trop
pressé, fut culbuté et tué, et le duc de Montfort, qui était avec lui et
le détachement des chevau-légers, fut très dangereusement blessé de six
coups de sabre, dont il fut et demeura balafré. Thianges, qui était
accouru volontaire, y fut dangereusement blessé par les nôtres, qui, par
son habit toujours bizarre, le prirent pour être des ennemis. Ils furent
enfoncés et mis tellement en fuite, qu'on ne put presque faire de
prisonniers.

Le maréchal de Villeroy alla ensuite prendre Huy avec un gros
détachement de l'armée, que le reste couvrit avec M. de Luxembourg. Tout
fut pris en trois jours\,; on n'y perdit qu'un sous-ingénieur et
quelques soldats. J'en vis sortir une assez mauvaise garnison de
diverses troupes\,; elle passa devant le maréchal de Villeroy, et fut
fort inquiétée par nos officiers qui eurent, par la capitulation, la
liberté de rechercher leurs déserteurs. Je visitai la place où on mit un
commandant aux ordres de Guiscard, gouverneur de Namur. L'armée réunie
fit ensuite quelques camps de passage, et prit enfin celui de Lecki, à
trois lieues de Liège. En arrivant, on commanda à l'ordre quantité de
fascines par bataillon\,; ce qui fit croire qu'on allait marcher aux
lignes de Liège. Cette opinion dura tout le lendemain\,; mais le jour
suivant, 28 juillet, il y eut, dans la fin de la nuit, ordre de les
brûler et de se tenir prêts à marcher. L'armée, en effet, se mit en
mouvement de grand matin pour grande chaleur, et vint passer le défilé
de Warem, au débouché duquel elle fit halte.

Pendant ce temps-là je gagnai une grange voisine avec force officiers du
Royal-Roussillon et quelques autres de la brigade, pour manger un
morceau à l'abri du soleil. Comme nous finissions ce repas, arriva
Boissieux, cornette de ma compagnie, qui revenait de dehors avec
Lefèvre, capitaine dans notre régiment, qui de gardeur de cochons était
parvenu là à force de mérite et de grades, et qui ne savait encore lire
ni écrire, quoique vieux. C'était un des meilleurs partisans des troupes
du roi, et qui ne sortait jamais sans voir les ennemis, ou en rapporter
des nouvelles sûres. Nous l'aimions, l'estimions et le considérions
tous, et il l'était des généraux. Boissieux me dit tout joyeux que nous
allions voir les ennemis\,; qu'ils avaient reconnu leur camp au deçà de
la Gette, et qu'il se passerait sûrement une grande action. Nous le
laissâmes aux prises avec ce qu'il y avait encore à manger, et sur ces
nouvelles nous montâmes à cheval. Un moment après je rencontrai Marsin,
maréchal de camp, qui nous les confirma. Je m'en allai au moulin de
Warem, dans lequel nos principaux généraux étoient montés avec M. le Duc
et le maréchal de Joyeuse, tandis que M. de Luxembourg s'était avancé
avec M. de Chartres et M. le prince de Conti. J'y montai aussi, et après
m'être informé des nouvelles, je m'en allai rejoindre le
Royal-Roussillon.

Voici la relation que je fis le lendemain de cette bataille, que
j'envoyai à ma mère et à quelques amis\,:

Lundi 27 juillet, le maréchal de Joyeuse fut détaché du camp de Lexhy, à
trois lieues de Liège, avec Montchevreuil, lieutenant général, et
Pracomtal, maréchal de camp, deux brigades d'infanterie et quelques
régiments de cavalerie, pour aller à nos lignes joindre quelques troupes
qu'y commandait La Valette, et s'opposer aux ennemis qui avaient exigé
des contributions du côté d'Arras et de Lille. Le mardi 28, l'armée
décampa, marcha sur Warem, dont elle traversa la petite ville, et le
détachement du maréchal de Joyeuse séparément d'elle, mais les deux
maréchaux ensemble. La tête de l'armée arrivant à une demi-lieue au
delà, il vint plusieurs avis que le prince d'Orange était campé avec son
armée au deçà de la Gette, qui est une petite rivière guéable en fort
peu d'endroits, et dont les bords sont fort hauts et escarpés, et que
cette armée n'était qu'à demi-lieue de Lave ou Lo, petite ville qui a
une forteresse peu considérable dans des marais au delà de la Gette, et
fort différente de Loo, maison de plaisance du prince d'Orange, qui en
est bien loin en Hollande.

Sur ces nouvelles, M. de Luxembourg s'avança avec le maréchal de
Villeroy, M. le duc de Chartres, M. le prince de Conti et fort peu
d'autres, et quelques troupes pour tâcher de se bien assurer de la
vérité de ces rapports. Une heure et demie après il manda au maréchal de
Joyeuse, qui était resté à la tête de l'armée avec M. le Duc, et qui,
pour voir de plus loin, était monté dans le moulin à vent de Warem, de
marcher à lui avec l'armée, et d'y faire rentrer le détachement destiné
à nos lignes. M. le prince de Conti revint qui confirma les nouvelles
qu'on avait eues de la position des ennemis, et se chargea de
l'infanterie dont quelques brigades achevaient encore de passer le
défilé de Warem. L'armée marcha fort vite, faisant néanmoins de temps en
temps quelques haltes pour attendre l'infanterie, et sur les huit heures
du soir arriva à trois lieues au delà de Warem, dans une plaine où les
troupes furent mises en bataille. Peu de temps après elle se remit en
colonne, s'avança un quart de lieue plus près de l'ennemi, et passa
ainsi le reste de la nuit en colonne, tandis que l'infanterie et
l'artillerie achevèrent d'arriver\,: c'était une chose charmante que la
joie des troupes après plus de huit lieues de marche, et leur ardeur
d'aller aux ennemis, dans le camp desquels on entendit beaucoup de bruit
et de mouvement toute la nuit, ce qui fit craindre qu'ils se retiraient.

Sur les quatre heures du matin leur canon commença à se faire
entendre\,; nos batteries, disposées un peu trop loin à loin, ne purent
être prêtes qu'une heure après, qu'on commença à se canonner
vigoureusement\,; et alors on reconnut que l'affaire serait difficile.
Les ennemis occupaient toutes les hauteurs, un village à droite et un
autre village à gauche, dans lesquels ils s'étaient bien retranchés. Ils
avaient fait aussi un long retranchement avec beaucoup de petites
redoutes sur la hauteur, d'un village à l'autre jusqu'auprès d'un grand
ravin à la droite, de manière qu'il fallait aller à eux par entre les
deux villages, d'où il les fallait chasser, et qui étaient trop proches
pour laisser de quoi s'étendre, ce qui obligeait nos troupes d'être sur
plusieurs lignes et leur causait le désavantage d'être débordées surtout
sur notre gauche\,; et cependant les batteries qu'ils avaient disposées
fort près à près sur le haut de leur retranchement, entre les deux
villages, et beaucoup mieux disposées que les nôtres, fouettaient
étrangement notre cavalerie, repliée très confusément vis-à-vis, par la
raison que je viens de dire.

M. le prince de Conti, le maréchal de Villeroy et beaucoup d'infanterie
attaqua le village de notre droite, nommé Bas-Landen. Feuquières,
lieutenant général, qui ne manquait ni de capacité ni de courage, fut
accusé de n'avoir voulu faire aucun mouvement. En même temps
Montchevreuil, sous le maréchal de Joyeuse, qui tout à cheval arracha le
premier cheval de frise, attaqua le village de notre gauche appelé
Neerwinden, qui donna le nom à la bataille. Montchevreuil y fut tué, et
fut remplacé par Rubentel, autre lieutenant général, et par le duc de
Berwick, qui y fut pris. Ces deux attaques à la droite et à la gauche
furent vivement repoussées, et sans le prince de Conti le désordre
aurait été fort grand à celle de la droite. M. de Luxembourg, voyant
l'infanterie presque rebutée, fit avancer toute la cavalerie au petit
trot, comme pour forcer les retranchements du front ou d'entre les deux
villages. L'infanterie ennemie qui les bordait laissa approcher la
cavalerie plus près que la portée du pistolet, et fit dessus une
décharge si à propos, que les chevaux tournèrent bride et retournèrent
plus vite qu'ils n'étaient venus. Ralliée à peine par ses officiers et
les officiers généraux, elle fut ramenée avec la même furie, mais avec
le même malheureux succès deux fois de suite. Ce n'était pas que M. de
Luxembourg comptât de faire entrer la cavalerie dans ces retranchements
qu'on pouvait à peine escalader à pied\,; mais il espérait, par un
mouvement général et audacieux de cette cavalerie, faire abandonner ces
retranchements.

Voyant donc à ce coup sa cavalerie inutile et son infanterie repoussée
deux fois\,: celle-ci des deux villages, et la cavalerie par trois fois
des retranchements du front, et qui, durant plus de quatre heures, avait
essuyé un feu de canon terrible sans branler que pour resserrer les
rangs à mesure que des files étaient emportées, il la porta un peu plus
loin dans une espèce de petit fond, où le canon ne pouvait les
incommoder de volée, mais seulement de bonds, où elle demeura plus d'une
grosse demi-heure. Alors les trois maréchaux, les trois princes,
Albergotti et le duc de Montmorency, fils aîné de M. de Luxembourg,
qu'on appelait auparavant le prince de Tingry, se mirent ensemble dans
ce même petit fond, peu éloigné de la cavalerie, presqu'à la tête du
Royal-Roussillon. Le colloque fut vif à les voir et assez long, puis ils
se séparèrent.

Alors on fit marcher les régiments des gardes françaises et suisses par
derrière la cavalerie, M. le prince de Conti à leur tête, droit au
village de Neerwinden, à notre gauche, qu'ils attaquèrent d'abordée avec
furie. Dès qu'on vit qu'ils commençaient à emporter des jardinages et
quelques maisons retranchées, on fit avancer la maison du roi, les
carabiniers et toute la cavalerie. Chaque escadron défila par où il put,
à travers les fossés relevés, les haies, les jardins, les houblonnières,
les granges, les maisons dont on abattit ce que l'on put de murailles
pour se faire des passages\,; tandis que plus avant dans le village,
l'infanterie, de part et d'autre, attaquait et défendait avec une
vigueur extraordinaire. Cependant Harcourt, qui avait un petit corps
séparé que Guiscard avait joint, était parti de six lieues de là, soit
au bruit du canon, soit sur un ordre que M. de Luxembourg lui avait
envoyé, et commençait à paraître dans la plaine tout à la gauche, à
notre égard, de Neerwinden, mais encore fort dans l'éloignement. En même
temps notre cavalerie commença à déboucher de ce village dans la plaine
et â se remettre à mesure du désordre d'un si étrange défilé.

Tout cela ensemble ébranla les ennemis qui commencèrent à se retirer
dans le retranchement du fond et à abandonner le village, le curé duquel
eut tout ce grand et long spectacle du haut de son clocher où il s'était
grimpé. Leur cavalerie, qui n'avait point encore paru, sortit de
derrière le retranchement du front et du village, s'avança en bon ordre
dans la plaine où la nôtre débouchait, et y fit d'abord plier des
troupes d'élite, jusqu'alors invincibles, mais qui n'avaient pas eu le
loisir de se former et de se bien mettre en bataille en sortant de ces
fâcheux passages du village par où il avait fallu défiler dans la
plaine. Les gardes du prince d'Orange, ceux de M. de Vaudemont et deux
régiments Anglais en eurent l'honneur\,; mais ils ne purent entamer ni
faire perdre un pouce de terrain aux chevau-légers de la garde,
peut-être plus heureusement débouchés dans la plaine et mieux placés et
formés que les autres troupes. Leur ralliement fait en moins de rien,
elles firent bientôt merveille, tandis que le reste de la cavalerie
débouchait et se formait à mesure qu'ils sortaient du village.

M. le duc de Chartres chargea plusieurs fois à la tête de ses braves
escadrons de la maison du roi avec une présence d'esprit et une valeur
dignes de sa naissance, et il y fut une fois mêlé et y pensa demeurer
prisonnier. Le marquis d'Arcy, qui avait été son gouverneur, fut
toujours auprès de lui en cette action, avec le sang-froid d'un vieux
capitaine et tout le courage de la jeunesse, comme il avait fait à
Steinkerque. M. le Duc, à qui principalement fut imputé le parti de
cette dernière tentative des régiments des gardes françaises et suisses
pour emporter le village de Neerwinden, fut toujours entre le feu des
ennemis et le nôtre. Cependant toute notre cavalerie, passée et formée
dans la plaine, alla jusqu'à cinq différentes fois à la charge\,; et à
la fin, après une vigoureuse résistance de la cavalerie ennemie, la
poussa jusqu'à la Gette, dans laquelle elle se précipita, et où un
nombre infini fut noyé.

M. le prince de Conti, maître enfin de tout le village de Neerwinden, où
il avait reçu une contusion au côté et un coup de sabre sur la tête que
le fer de son chapeau para, se mit à la tête de quelque cavalerie, la
plus proche de la tête de ce village, avec laquelle il prit à revers en
flanc le retranchement du front, aidé par l'infanterie qui avait emporté
enfin le village de Neerwinden, et acheva de faire prendre la fuite à ce
qui était derrière ce long retranchement. Mais cette infanterie n'ayant
pu les charger aussi vite, ni la cavalerie de notre gauche qui en était
la plus éloignée, cette retraite des ennemis, quoique précipitée, ne
laissa pas d'être belle. Un peu après quatre heures ou vers cinq heures
après midi, tout fut achevé après douze heures d'action par un des plus
ardents soleils de tout l'été.

J'interromprai ici pour un moment cette relation, pour dire un mot de
moi-même. J'étais du troisième escadron du Royal-Roussillon, commandé
par le premier capitaine du régiment, très brave gentilhomme de
Picardie, que nous aimions tous, qui s'appelait Grandvilliers. Du Puy,
autre capitaine, qui était à la droite de notre escadron, me pressa de
prendre sa place par honneur, ce que je ne voulus pas faire. Il fut tué
à une de nos cinq charges. J'avais deux gentilshommes\,: l'un avait été
mon gouverneur et était homme de mérite, l'autre écuyer de ma mère, cinq
palefreniers avec des chevaux de main et un valet de chambre. Je fis
trois charges sur un excellent courtaud bai brun, que je n'avais pas
descendu depuis quatre heures du matin. Le sentant mollir, je me tournai
pour en demander un autre. Alors je m'aperçus que ces gentilshommes n'y
étaient plus. On cria à mes gens qui se trouvèrent assez près de
l'escadron, et ce valet de chambre qui s'appelait Bretonneau, que
j'avais presque de mon enfance, me demanda brusquement s'il ne me
donnerait pas un cheval aussi bien que ces deux messieurs qui avaient
disparu il y avait longtemps. Je montai un très joli cheval gris, sur
lequel je fis encore deux charges\,: j'en fus quitte en tout pour la
croupière du courtaud coupée et un agrément d'or de mon habit bleu
déchiré.

Mon ancien gouverneur m'avait suivi, mais dès la première charge son
cheval prit le mors aux dents, et l'ayant enfin rompu le portait deux
fois dans les ennemis si d'Achy ne l'eût arrêté l'une et un lieutenant
l'autre. Le cheval fut blessé, et l'homme en prit un de cavalier. Il ne
fut guère plus heureux après cette aventure. Il perdit sa perruque et
son chapeau\,; quelqu'un lui en donna un grand d'Espagnol qui avait un
chardon, auquel il ne pensa pas, et qui le fit passer par les armes des
nôtres. Enfin il gagna les équipages où il attendit le succès de la
bataille et ce que je serais devenu. Pour l'autre qui avait disparu tout
d'abord et n'avait point essuyé d'aventure, il se trouva lorsque, tout
étant plus que fini, j'allais manger un morceau avec force officiers du
régiment et de la brigade, et s'approchant de moi, se félicita hardiment
de m'avoir changé de cheval bien à propos. Cette effronterie me surprit
et m'indigna tellement que je rie lui répondis pas un mot et ne lui en
parlai jamais depuis\,; mais voyant de quel bois ce brave se chauffait,
je m'en défis dès que je fus de retour de l'armée.

Mes gens, à la halte de la veille, avaient sagement sauvé un gigot de
mouton et une bouteille de vin, sur la nouvelle d'une action prochaine.
Je l'avais expédié le matin avec nos officiers qui, comme moi, n'avaient
point eu à souper, et nous avions tous les dents bien longues lorsque
nous aperçûmes, de loin, deux chevaux de bât couverts de jaune, qui
rôdaient dans la plaine, avec deux ou trois hommes à cheval. Quelqu'un
de nous se détacha après et vit mon maître d'hôtel qu'il ramena avec son
convoi, qui nous fit à tous un plaisir extrême. Ce fut la première fois
que d'Achy et Puyrobert s'embrassèrent de bon cœur et burent de même
ensemble. Le dernier avait montré une grande et judicieuse valeur.
D'Achy en fut charmé, fit toutes les avances, et ils furent depuis
toujours amis. Ils étaient les miens l'un et l'autre, et cette
réconciliation sincère me fit un grand plaisir et à tous les officiers
du régiment. Je venais d'écrire trois mots à ma mère, avec une écritoire
et un morceau de papier que ce même valet de chambre avait eu soin de
mettre dans sa poche, et j'envoyai un laquais à ma mère tout à
l'instant\,; mais mille embarras le retardèrent et laissèrent passer à
la tendresse de ma mère vingt-quatre heures fort mauvais temps.

Quand nous eûmes mangé, je pris quelques anciens officiers avec moi pour
aller visiter tout le champ de bataille et surtout les retranchements
des ennemis. Il est incroyable qu'en si peu d'heures qu'ils eurent à les
faire, dont la nuit couvrit la plupart, ils aient pu leur donner
l'étendue qu'ils avaient entre les deux villages (ce que nous appelions
ceux du front), la hauteur de quatre pieds, des fossés larges et
profonds, la régularité partout par les flancs qu'ils y pratiquèrent et
les petites redoutes qu'ils y semèrent, avec des portes et des
ouvertures couvertes de demi-lunes de même. Les deux villages,
naturellement environnés de fortes haies et de fossés, suivant l'usage
du pays, étaient encore mieux fortifiés que tout le reste. La quantité
prodigieuse de corps dont les rues, surtout de celui de Neerwinden,
étaient plutôt comblées que jonchées, montrait bien quelle résistance on
y avait rencontrée\,; aussi, la victoire si disputée coûta cher.

On y perdit Montchevreuil, lieutenant général, gouverneur d'Arras et
lieutenant général d'Artois. Il était frère du chevalier de l'ordre, par
conséquent fort bien avec le roi, dont il avait le régiment
d'infanterie. C'était un fort honnête homme et un bon officier
général\,; Lignery, maréchal de camp et lieutenant des gardes du corps,
qui les commandait\,; milord Lucan, capitaine des gardes du corps du roi
d'Angleterre\,; le duc d'Uzès, qui eut les deux jambes emportées, et le
prince Paul de Lorraine, dernier fils de M. de Lislebonne, colonels\,;
le premier d'infanterie, l'autre de cavalerie\,; cinq brigadiers de
cavalerie\,: Saint-Simon, mon parent éloigné, de la branche de
Monbléon\,; Montfort, notre mestre de camp, à la tête des carabiniers\,;
Quoadt, notre brigadier. Je le vis tuer d'un coup de canon devant nous
dès le grand matin (le duc de La Feuillade devint par là commandant de
notre brigade et s'en acquitta avec distinction\,; il disparut un moment
après et nous fûmes plus d'une demi-heure sans le revoir\,: c'est qu'il
était allé faire sa toilette\,; il revint poudré et paré d'un beau
surtout rouge, fort brodé d'argent, et tout son ajustement et celui de
son cheval étaient magnifiques)\,; le comte de Montrevel, neveu du
lieutenant général, et Boolen, qui avait le Royal-Allemand\,; Gournay,
un des deux maréchaux de camp mis aux carabiniers\,; Rebé, qui avait
Piémont, et brigadier\,; Gassion, enseigne des gardes du corps et
brigadier, et un grand nombre d'officiers particuliers. J'y perdis le
marquis de Chanvalon, mon cousin germain, enseigne des gens d'armes de
la garde, fils unique de la sœur de ma mère, qui ne s'en est jamais
consolée.

Les blessés furent\,: M. le prince de Conti, très légèrement\,; le
maréchal de Joyeuse et le duc de Montmorency de même\,; le comte de
Luxe, son frère, dangereusement\,; le duc de La Rocheguyon, un pied
fracassé\,; le chevalier de Sillery, une jambe cassée, qui n'était là
qu'à la suite de M. le prince de Conti, dont il était écuyer\,; Fonville
et Saillant, capitaines aux gardes, dont deux autres furent tués\,; M.
de Bournonville, dans les gens d'armes de la garde, fort blessé\,; M. de
Villequier, fort légèrement.

Artagnan, major des gardes françaises et major général de l'armée, fort
bien avec M. de Luxembourg et encore mieux avec le roi, lui porta la
nouvelle et en eut le gouvernement d'Arras et la lieutenance générale
d'Artois. Le comte de Nassau-Saarbrück eut le Royal-Allemand, qui vaut
beaucoup\,; et le marquis d'Acier, devenu duc d'Uzès par la mort de son
frère, eut ses gouvernements de Saintonge et d'Angoumois, d'Angoulême et
de Saintes, et son régiment. Albergotti, favori de M. de Luxembourg,
neveu de Magalotti, lieutenant général et gouverneur de Valenciennes,
porta quelques jours après le détail. Il s'évanouit chez
M\textsuperscript{me} de Maintenon, et tout à la mode qu'il fût se fit
moquer de lui.

Les ennemis perdirent le prince de Barbançon, qui avait défendu Namur\,;
les comtes de Solars et d'Athlone, généraux d'infanterie, et plusieurs
autres officiers généraux. Le duc d'Ormond, le fils du comte d'Athlone
furent pris\,; Ruvigny l'a été et relâché dans l'instant\,; on n'a pas
fait semblant de le savoir\,; et grand nombre d'officiers particuliers.
On estime leur perte à plus de vingt mille hommes. On ne se trompera
guère si on estime notre perte à près de la moitié. Nous avons pris tout
leur canon, huit mortiers, beaucoup de charrettes d'artillerie et de
caissons, et quantité d'étendards et de drapeaux et quelques paires de
timbales. La victoire se peut dire complète.

Le prince d'Orange, étonné que le feu continuel et si bien servi de son
canon n'ébranlât point notre cavalerie, qui l'essuya six heures durant
sans branler et tout entière sur plusieurs lignes, vint aux batteries en
colère, accusant le peu de justesse de ses pointeurs. Quand il eut vu
l'effet, il tourna bride et s'écria\,: « Oh\,! l'insolente nation\,!\,»
Il combattit presque jusqu'à la fin, et l'électeur de Bavière et lui se
retirèrent par des ponts qu'ils avaient sur la Gette, quand ils virent
qu'ils ne pouvaient plus raisonnablement rien espérer. L'armée du roi
demeura longtemps comme elle se trouva, sur le terrain même où elle
avait combattu\,; et vers la nuit marcha au camp marqué tout proche, le
quartier général au village de Landen ou Land fermé. Plusieurs brigades
prises de la nuit couchèrent en colonne comme elles se trouvèrent,
marchant au camp, où elles entrèrent au jour, et la nôtre fut de ce
nombre.

\hypertarget{chapitre-vii.}{%
\chapter{CHAPITRE VII.}\label{chapitre-vii.}}

~

\relsize{-1}

{\textsc{Monseigneur, mal conseillé, n'attaque point les retranchements
d'Heilbronn, dont le maréchal de Lorges est outré.}} {\textsc{-
Monseigneur de retour du Rhin et Monsieur des côtes.}} {\textsc{- Succès
à la mer.}} {\textsc{- Siège et prise de Charleroy par le maréchal de
Villeroy.}} {\textsc{- Prise de Roses par le maréchal de Noailles.}}
{\textsc{- Bataille de la Marsaille en Piémont.}} {\textsc{- J'arrive à
Paris et j'achète un régiment de cavalerie.}} {\textsc{- Daquin, premier
médecin du roi, chassé, et Fagon en sa place.}} {\textsc{- Fortune et
mort de La Vauguyon.}} {\textsc{- Survivance de Pontchartrain.}}
{\textsc{- Saint-Malo bombardé sans dommage.}} {\textsc{- Mariage du
maréchal de Boufflers.}} {\textsc{- Dangeau, maître de l'ordre de
Saint-Lazare.}} {\textsc{- Ordre de Saint-Louis.}} \relsize{1}

~

J'allai de bonne heure au quartier général que je trouvai sortant du
village. Je fis mon compliment à M. de Luxembourg\,: il était avec les
princes, le maréchal de Villeroy et peu d'officiers généraux. Je les
suivis à la visite d'une partie du champ de bataille, et même ils se
promenèrent au delà de la Gette, où il se trouva quelques pontons. Je
leur prêtai une lunette d'approche avec laquelle nous vîmes six ou sept
escadrons des ennemis qui se retiraient fort vite encore, et passaient
sous le canon de Lave ou Lo. Je causai fort avec M. le prince de Conti
qui me montra sa contusion au côté, et qui ne me parut pas insensible à
la gloire qu'il avait acquise. Je fus ravi de celle de M. le duc de
Chartres\,; j'avais été comme élevé auprès de lui, et si l'inégalité
permet ce terme, l'amitié s'était formée et liée entre lui et moi\,:
c'était aussi celui que je voyais le plus souvent à l'armée. L'infection
du champ de bataille l'en éloigna bientôt.

Les ennemis s'étaient retirés sous Bruxelles. M. de Luxembourg fut
quelque temps à ne songer qu'au repos et à la subsistance de ses
troupes. Ce beau laurier qu'il venait de cueillir ne le mit pas à
couvert du blâme. Il en essaya plus d'un\,: celui de la bataille même,
et celui de n'en avoir pas profité. Pour la bataille, on lui reprochait
de l'avoir hasardée contre une armée si bien postée et si fortement
retranchée, et avec la sienne quoique un peu supérieure, mais fatiguée
et pour ainsi dire encore essoufflée de la longueur de la marche de la
veille\,; on l'accusait, et non sans raison, d'avoir été plus d'une fois
au moment de la perdre, et de ne l'avoir gagnée qu'à force
d'opiniâtreté, de sang et de valeur française. Sur le fruit de la
victoire, on ne se contraignit pas de dire qu'il n'avait pas voulu
l'achever de peur de terminer trop tôt une guerre qui le rendait grand
et nécessaire. La première se détruisait aisément\,: il avait des ordres
réitérés de donner bataille, et il ne pouvait imaginer que les ennemis
eussent pu en une nuit si courte fortifier leur poste déjà trop bon par
une telle étendue de retranchements si forts et si réguliers, qu'il
n'aperçut que lorsque le jour parut auquel la bataille fut livrée. Sur
l'autre accusation, je n'en sais pas assez pour en parler. Il est vrai
qu'entre quatre et cinq tout fut fini, et les ennemis partie en
retraite, partie en fuite. La Gette par là était en notre disposition.
Nous avions des pontons tout prêts. Au delà, le pays est ouvert, et il y
avait assez de jour en juillet pour les suivre de près\,; mais il est
vrai que les troupes n'en pouvaient plus de la marche de la veille et de
douze heures de combat, que les chevaux étaient à bout, ceux de trait
surtout pour le canon et les vivres, et qu'on prétendit qu'on manquait
absolument de ce dernier côté pour aller en avant, et les charrettes
composées étaient épuisées de munitions.

Cossé, prisonnier, fut renvoyé incontinent sur sa parole, et les ducs de
Berwick et d'Ormond presque aussitôt échangés. On eut grand soin de nos
blessés et le même des prisonniers qui l'étaient\,; et de bien traiter
ceux qui ne l'étaient pas et surtout de faire enlever du champ de
bataille tout ce qui n'était pas mort et qu'on put emporter.

Le maréchal de Lorges passa le Rhin et prit la ville et le château
d'Heidelberg, puis passa le Necker et prit Zuingenberg, où Vaubecourt
eut un pied cassé et le prince d'Épinay a été dangereusement blessé. La
jonction faite de Monseigneur, le maréchal de Lorges voulut attaquer
Heilbronn \,: Monseigneur y trouva de la difficulté. Le maréchal s'y est
opiniâtré, les a toutes levées et les troupes ne demandaient qu'à
donner, lorsqu'un petit conseil particulier de Saint-Pouange et de M. le
Premier \footnote{On appelait \emph{M. le Premier} le premier écuyer de
  la petite écurie du roi. C'était à cette époque Jacques-Louis de
  Beringhen.} a tout arrêté. Le maréchal s'est mis en furie, mais
Chaulny ayant été entraîné par les deux autres, et Monseigneur penchant
fort de ce côté, il n'y a pas eu moyen de le résoudre, au grand regret
des principaux généraux et de toutes les troupes. Le reste de la
campagne se passa en subsistances abondantes, et Monseigneur revint de
bonne heure avec ses trois conseillers pacifiques. Monsieur avec le
maréchal d'Humières était revenu longtemps avant lui de Pontorson, où il
s'était le plus fixé. Il avait fait un tour en Bretagne, où le duc de
Chaulnes l'avait reçu et traité avec une magnificence royale. Monsieur
eut des relais du roi à Dreux, et trouva Madame qui venait d'avoir la
petite vérole.

Tourville prit ou défit et dissipa presque toute la flotte marchande de
Smyrne dont il battit le convoi, et fit encore plusieurs moindres
expéditions, cette même campagne, qui coûtèrent fort cher aux Anglais et
aux Hollandais. Rocke qui commandait cette flotte eut près de cinquante
vaisseaux brûlés ou coulés à fond, et vingt-sept pris, tous marchands et
richement chargés\,: sur un seul de ceux qu'on prit la charge fut
estimée cinq cent mille écus, et on croit la perte des ennemis de plus
de trente millions. On prit aussi deux gros vaisseaux de guerre et on en
coula bas deux autres. Coetlogon brûla les vaisseaux Anglais qui
s'étaient retirés à Gibraltar.

Cependant les régiments vacants de Neerwinden furent donnés. Tous les
capitaines du Royal-Roussillon avec Puyrobert, lieutenant-colonel à leur
tête, m'étaient venus offrir d'écrire pour me demander, et le major,
frère de notre maréchal de camp, s'y joignit\,; ils me citèrent deux
exemples où cela avait réussi. Ils me pressèrent, et quoique je me
sentisse fort flatté et à la sortie d'une grande action, je persévérai à
leur en témoigner ma reconnaissance sans accepter leur offre. Je
regardai ce régiment comme la fortune du chevalier de Montfort dont le
frère l'avait acheté. J'en écrivis à M. de Beauvilliers, et je pressai
infiniment M. le duc de Chartres, qui commandait la cavalerie, de le
demander pour lui, qui me le fit espérer sans s'y engager tout à fait,
pour se débarrasser de pareilles prières pour les autres régiments.
Morstein, qui était bien avec lui, me dit devant lui qu'il se doutait
bien qui aurait ce régiment, et fut honteux de ce que M. de Chartres lui
répondit. Praslin le demanda et l'obtint par Barbezieux qui était son
ami. J'avais su qu'il le demandait, je le lui avais dit et en même temps
mes désirs pour notre major. Le jour que M. de Chartres le vint faire
recevoir, Praslin vint m'éveiller, dîna chez moi, s'y tint toute la
journée, et y soupa. Lui et le chevalier de Montfort se firent
merveilles. M. le comte de Toulouse eut son régiment. D'Achy qui n'en
eut point en fut outré et ne voulut ni voir Praslin ni en entendre
parler. Je fis l'impossible pour le ramener de cette folie\,; il la
poussa jusqu'à ne vouloir manger ni chez moi ni à ma halte, qu'il ne fût
bien assuré que ce dernier n'y serait pas, quoiqu'il n'oubliât rien pour
l'apprivoiser. Non seulement j'eus tout lieu de me louer de ce nouveau
mestre de camp, mais l'amitié et la confiance se mirent entre nous et
n'ont fini qu'avec lui.

Après divers camps de repos, de subsistances, d'observations, l'armée
s'approcha de Charleroi\,; le maréchal de Villeroy avec une partie de
l'armée en fit le siège et ouvrit la tranchée du 15 au 16 septembre. M.
de Luxembourg le couvrit avec l'autre partie de l'armée, de laquelle
nous étions, mais assez près pour s'aller promener souvent au siège, et
pour que les deux armées se communiquassent sans aucun besoin d'escorte.
Le prince d'Orange ne songea pas à donner la moindre inquiétude. Le
marquis d'Harcourt avec son corps un peu renforcé fut envoyé aux lignes
que gardait La Valette, vers où l'électeur de Bavière avait marché avec
un assez gros corps. Fort peu après, le prince d'Orange quitta l'armée,
et s'en alla à Breda, puis chasser à Loo et de là à la Haye. Charleroi
battit la chamade le dimanche matin 11 octobre. On y perdit fort peu de
monde, et personne de distinction que le fils aîné de Broglio, qui était
allé voir le marquis de Créqui à la tranchée. Castille, qui commandait à
Charleroi, s'est fort plaint de n'avoir point été secouru, contre la
parole que le prince d'Orange et l'électeur de Bavière lui en avaient
donnée. Il a obtenu la permission de passer par la France pour aller en
Espagne, et ne veut plus servir sous eux. Boisselot, qui défendit si
bravement Limerick en Irlande, eut le gouvernement de Charleroi
sur-le-champ.

M. de Noailles prit Roses. Un gros détachement de son armée alla joindre
le maréchal Catinat, et la gendarmerie y fut aussi de l'armée du Rhin.
M. de Savoie faisait mine d'assiéger Pignerol, et se contenta de le
bombarder, Tessé dedans, de prendre et de faire sauter le fort de
Sainte-Brigitte, après quoi il perdit une grande bataille le dimanche 4
octobre, près de l'abbaye de la Marsaglia. Clérembault en apporta la
nouvelle. Le combat dura depuis neuf heures du matin jusqu'à quatre
heures après midi. On prétend qu'ils y ont perdu dix-sept mille hommes,
trente-six pièces de canon, leurs bagages, cinquante étendards ou
drapeaux. Les deux armées se cherchaient mutuellement. Au moment que le
combat commença, M. Catinat s'aperçut que le dessein de M. de Savoie
était tout sur sa gauche. Il y porta la gendarmerie et encore d'autres
troupes qui n'y étaient pas attendues, et qui non seulement soutinrent
tout l'effort que les ennemis espéraient imprévu, mais qui les
renversèrent. Mais ce désordre se rétablit, et cette droite ennemie fit
bien mieux que leur gauche, qui fut enfoncée\,: à la fin la victoire fut
si complète, que la retraite des ennemis devint une fuite, et que M. de
Savoie fut poursuivi jusqu'à la vue de Turin. M. Catinat avait
soixante-quinze escadrons et quarante-huit bataillons, et M. de Savoie
quatre-vingts escadrons et quarante-cinq bataillons. Ceux des
religionnaires français ont combattu en désespérés et s'y sont presque
tous fait tuer.

Caprara et Louvigny ne voulaient point que M. de Savoie donnât la
bataille\,; mais il s'y est opiniâtré, en fureur d'avoir vu brûler sa
belle maison de la Verrerie par Buchevilliers deux jours auparavant. Le
roi l'avait très expressément ordonné, en représailles des feux que M.
de Savoie avait faits en Dauphiné et tout nouvellement dans la vallée de
Pragelus, sans même pardonner aux églises. Nous y avons perdu La
Hoguette, lieutenant général et très bon, force officiers de
gendarmerie, entre autres le chevalier de Druy, major fort au goût du
roi, et quelques brigadiers et colonels. Les ennemis conviennent de la
perte de douze mille hommes, dont deux mille prisonniers. Ce qui est
resté de troupes espagnoles se retira dans le duché de Milan.

Le roi envoya Chamlay concerter avec le maréchal Catinat\,: c'était son
homme de confiance de tout temps pour toutes les affaires de la guerre,
et celui de M. de Louvois\,; il le méritait par sa capacité et son
secret\,; bon citoyen, la modestie et la simplicité même, avec beaucoup
d'honneur et de probité\,; d'ailleurs homme de fort peu et qui ne s'en
cachait pas. En partant, le roi le fit grand-croix de Saint-Louis à la
place de Montchevreuil, tué à Neerwinden. Le duc de Schomberg mourut de
ses blessures\,; nous avons eu Varennes et Médavy, maréchaux de camp,
fort blessés, et Ségur, capitaine de gendarmerie, une jambe emportée,
plusieurs autres blessés. Nous eûmes huit ou neuf cents blessés et moins
de deux mille morts. Nos Irlandais s'y distinguèrent. Le roi écrivit à
MM. de Vendôme tous deux, et ne fit pas le même honneur à M. le Duc ni à
M. le prince de Conti. Il est pourtant difficile que les uns aient mieux
mérité à la Marsaille que les autres firent à Neerwinden. Cette
différence ne les rapprocha pas et scandalisa fort tout le monde.

Charleroi rendu, après une fort belle défense, par une honorable
capitulation, les trois princes s'en allèrent, et l'armée se mit dans
les quartiers de fourrages en attendant ceux d'hiver. Dès qu'ils furent
venus je ne songeai plus qu'à m'en aller, après avoir visité Tournay et
sa belle citadelle. Je trouvai les chemins et les postes en grand
désarroi, et entre autres aventures, je fus mené par un postillon sourd
et muet qui m'embourba de nuit auprès du Quesnoy. Je passai à Noyon chez
l'évêque, qui était un Clermont-Tonnerre, parent et ami de mon père,
célèbre par sa vanité et les faits et dits qui en ont été les fruits.
Toute sa maison était remplie de ses armes jusqu'aux plafonds et aux
planchers\,; des manteaux de comte et pair dans tous les lambris, sans
chapeau d'évêque\,; des clefs partout, qui sont ses armes, jusque sur le
tabernacle de sa chapelle\,; ses armes sur sa cheminée, en tableau avec
tout ce qui se peut imaginer d'ornements, tiare, armures, chapeaux,
etc., et toutes les marques des offices de la couronne\,; dans sa
galerie une carte que j'aurais prise pour un concile, sans deux
religieuses aux deux bouts\,: c'étaient les premiers et les successeurs
de sa maison\,; et deux autres grandes cartes généalogiques avec ce
titre de \emph{Descente de la très auguste maison de Clermont-Tonnerre,
des empereurs d'Orient}, et à l'autre, \emph{des empereurs d'Occident}.
Il me montra ces merveilles, que j'admirai à la hâte dans un autre sens
que lui\,; et je gagnai Paris à grand'peine. Je pensai même demeurer à
Pont-Sainte-Maxence, où tous les chevaux étaient retenus pour M. de
Luxembourg. Je dis au maître de la poste que j'en étais gouverneur,
comme il était vrai, et que je l'allais faire mettre au cachot s'il ne
me donnait des chevaux. J'aurais été bien empêché comment m'y prendre,
mais il fut assez simple pour en avoir peur et me donner des chevaux.

J'avais fait amitié à l'armée avec le chevalier du Rosel, mestre de
camp, grand partisan, et très bon officier et fort estimé. C'était
d'ailleurs un gentilhomme fort homme d'honneur. Il avait eu le régiment
du prince Paul, tué à Neerwinden. Peu de jours avant de nous séparer, il
me confia que le roi mettait en un seul corps les cent compagnies de
carabiniers qui étaient les grenadiers de la cavalerie, que ce corps se
séparait en cinq brigades avec chacune son mestre de camp et son
état-major, et que le tout était donné à M. du plaine, qui avait fait
l'impossible, et le roi aussi, pour que le comte d'Auvergne lui vendit
sa charge de colonel général de la cavalerie, à quoi rien ne l'avait pu
résoudre. Du Rosel ajouta qu'il savait qu'il avait une de ces brigades,
dont notre d'Achy eut aussi une, et qu'il aurait son régiment à vendre,
que je tâchasse de l'avoir, et que pour le droit d'avis il me demandait
vingt-six mille livres, au lieu du prix fixé de vingt-deux mille cinq
cents livres. Je trouvai l'avis salutaire et j'en remerciai fort du
Rosel. En arrivant à Paris, je trouvai la chose publique. J'écrivis à M.
de Beauvilliers, et j'eus un régiment dans les premières vingt-quatre
heures que je fus arrivé, dont je remerciai le roi en lui faisant ma
révérence d'arrivée. Je tins parole à du Rosel et lui payai vingt-six
mille livres sans que personne le sût, et nous avons été amis toute sa
vie. C'était un des galants hommes que j'aie connus\,; il avait un frère
plus avancé que lui, qui valait beaucoup aussi, quoique le cadet lui fût
supérieur et reconnu pour tel.

Je trouvai un changement à la cour qui la surprit fort. Daquin, premier
médecin du roi, créature de M\textsuperscript{me} de Montespan, n'avait
rien perdu de son crédit par l'éloignement final de la maîtresse, mais
il n'avait jamais pu prendre avec M\textsuperscript{me} de Maintenon, à
qui tout ce qui sentait cet autre côté fut toujours plus que suspect.
Daquin était grand courtisan, mais reître, avare, avide, et qui voulait
établir sa famille en toute façon. Son frère, médecin ordinaire, était
moins que rien\,: et le fils du premier médecin, qu'il poussait par le
conseil et les intendances, valait encore moins. Le roi peu à peu se
lassait de ses demandes et de ses importunités. Lorsque M. de
Saint-Georges passa de Tours à Lyon, par la mort du frère du premier
maréchal de Villeroy, commandant et lieutenant de roi de cette province
et proprement le dernier seigneur de nos jours, Daquin avait un fils
abbé, de très bonnes mœurs, de beaucoup d'esprit et de savoir, pour
lequel il osa demander Tours de plein saut, et en presser le roi avec la
dernière véhémence. Ce fut l'écueil où il se brisa\,;
M\textsuperscript{me} de Maintenon profita du dégoût où elle vit le roi
d'un homme qui demandait sans cesse, et qui avait l'effronterie de
vouloir faire son fils tout d'un coup archevêque \emph{al despetto} de
tous les abbés de la première qualité, et de tous les évêques du
royaume\,; et Tours en effet fut donné à l'abbé d'Hervault, qui avait
été longtemps auditeur de rote avec réputation, et qui y avait bien
fait. C'était un homme de condition, bien allié, et qui dans cet
archevêché a grandement soutenu tout le bien qu'il y promettait.

M\textsuperscript{me} de Maintenon, qui voulait tenir le roi par toutes
les avenues, et qui considérait celle d'un premier médecin habile et
homme d'esprit comme une des plus importantes, à mesure que le roi
viendrait à vieillir et sa santé il s'affaiblir, sapait depuis longtemps
Daquin, et saisit ce moment de la prise si forte qu'il donna sur lui et
de la colère du roi\,; elle le résolut à le chasser, et en même temps à
prendre Fagon en sa place. Ce fut un mardi, jour de la Toussaint, qui
était le jour du travail chez elle de Pontchartrain, qui outre la marine
avait Paris, la cour et la maison du roi en son département. Il eut donc
ordre d'aller le lendemain avant sept heures du matin chez Daquin lui
dire de se retirer sur-le-champ à Paris\,; que le roi lui donnait six
mille livres de pension, et à son frère médecin ordinaire, trois mille
livres pour se retirer aussi, et défense au premier médecin de voir le
roi et de lui écrire. Jamais le roi n'avait tant parlé à Daquin que la
veille à son souper et à son coucher, et n'avait paru le mieux traiter.
Ce fut donc pour lui un coup de foudre qui l'écrasa sans ressource. La
cour fut fort étonnée et ne tarda pas à s'apercevoir d'où cette foudre
partait, quand on vit, le jour des Morts, Fagon déclaré premier médecin
par le roi même qui le lui dit à son lever, et qui apprit par là la
chute de Daquin à tout le monde qui l'ignorait encore, et qu'il n'y
avait pas deux heures que Daquin lui-même l'avait apprise. Il n'était
point malfaisant, et ne laissa pas à cause de cela d'être plaint et
d'être même visité dans le court intervalle qu'il mit à s'en aller à
Paris.

Fagon était un des beaux et des bons esprits de l'Europe, curieux de
tout ce qui avait trait à son métier, grand botaniste, bon chimiste,
habile connaisseur en chirurgie, excellent médecin et grand praticien.
Il savait d'ailleurs beaucoup\,; point de meilleur physicien que lui\,;
il entendait même bien les différentes parties des mathématiques. Très
désintéressé, ami ardent, mais ennemi qui ne pardonnait point, il aimait
la vertu, l'honneur, la valeur, la science, l'application, le mérite, et
chercha toujours à l'appuyer sans autre cause ni liaison, et à tomber
aussi rudement sur tout ce qui s'y opposait, que si on lui eût été
personnellement contraire. Dangereux aussi parce qu'il se prévenait très
aisément en toutes choses, quoique fort éclairé, et qu'une fois prévenu,
il ne revenait presque jamais\,; mais s'il lui arrivait de revenir,
c'était de la meilleure foi du monde et faisait tout pour réparer le mal
que sa prévention avait causé. Il était l'ennemi le plus implacable de
ce qu'il appelait charlatans, c'est-à-dire des gens qui prétendaient
avoir des secrets et donner des remèdes, et sa prévention l'emporta
beaucoup trop loin de ce côté-là. Il aimait sa Faculté de Montpellier,
et en tout la médecine, jusqu'au culte. À son avis il n'était permis de
guérir que par la voie commune des médecins reçus dans les Facultés dont
les lois et l'ordre lui étaient sacrés\,; avec cela délié courtisan, et
connaissant parfaitement le roi, M\textsuperscript{me} de Maintenon, la
cour et le monde. Il avait été le médecin des enfants du roi, depuis que
M\textsuperscript{me} de Maintenon en avait été gouvernante\,; c'est que
leur liaison s'était formée. De cet emploi il passa aux enfants de
France, et ce fut d'où il fut tiré pour être premier médecin. Sa faveur
et sa considération, qui devinrent extrêmes, ne le sortirent jamais de
son état ni de ses mœurs, toujours respectueux et toujours à sa place.

Un autre événement surprit moins qu'il ne fit admirer les fortunes. Le
dimanche 29 novembre, le roi sortant du salut apprit, par le baron de
Beauvais, que La Vauguyon s'était tué le matin de deux coups de pistolet
dans son lit, qu'il se donna dans la gorge, après s'être défait de ses
gens sous prétexte de les envoyer à la messe. Il faut dire un mot de ces
deux hommes\,: La Vauguyon était un des plus petits et des plus pauvres
gentilshommes de France. Son nom était Bétoulat, et il porta le nom de
Fromenteau. C'était un homme parfaitement bien fait, mais plus que brun
et d'une figure espagnole. Il avait de la grâce, une voix charmante,
qu'il savait très bien accompagner du luth et de la guitare, avec cela
le langage des femmes, de l'esprit et insinuant.

Avec ces talents et d'autres plus cachés mais utiles à la galanterie, il
se fourra chez M\textsuperscript{me} de Beauvais, première femme de
chambre de la reine mère et dans sa plus intime confidence, et à qui
tout le monde faisait d'autant plus la cour qu'elle ne s'était pas mise
moins bien avec le roi, dont elle passait pour avoir eu le pucelage. Je
l'ai encore vue vieille, chassieuse et borgnesse, à la toilette de
M\textsuperscript{me} la dauphine de Bavière où toute la cour lui
faisait merveilles, parce que de temps en temps elle venait à
Versailles, où elle causait toujours avec le roi en particulier, qui
avait conservé beaucoup de considération pour elle. Son fils, qui
s'était fait appeler le baron de Beauvais, avait la capitainerie des
plaines d'autour de Paris. Il avait été élevé, au subalterne près, avec
le roi\,; il avait été de ses ballets et de ses parties, et galant,
hardi, bien fait, soutenu par sa mère et par un goût personnel du roi,
il avait tenu son coin, mêlé avec l'élite de la cour, et depuis traité
du roi toute sa vie avec une distinction qui le faisait craindre et
rechercher. Il était fin courtisan et gâté, mais ami à rompre des glaces
auprès du roi avec succès, et ennemi de même\,; d'ailleurs honnête homme
et toutefois respectueux avec les seigneurs. Je l'ai vu encore donner
les modes.

Fromenteau se fit entretenir par la Beauvais, et elle le présentait à
tout ce qui venait chez elle, qui là et ailleurs, pour lui plaire,
faisait accueil au godelureau. Peu à peu elle le fit entrer chez la
reine mère, puis chez le roi, et il devint courtisan par cette
protection. De là il s'insinua chez les ministres. Il montra de la
valeur volontaire à la guerre, et enfin il fut employé auprès de
quelques princes d'Allemagne. Peu à peu il s'éleva jusqu'au caractère
d'ambassadeur en Danemark, et il alla après ambassadeur en Espagne.
Partout on en fut content, et le roi lui donna une des trois places de
conseiller d'État d'épée, et, au scandale de sa cour, le fit chevalier
de l'ordre en 1688. Vingt ans auparavant il avait épousé la fille de
Saint-Mégrin dont j'ai parlé ci-devant à propos du voyage qu'il fit à
Blaye de la part de la cour, pendant les guerres de Bordeaux, auprès de
mon père\,; ainsi je n'ai pas besoin de répéter qui elle était, sinon
qu'elle était veuve avec un fils de M. du Broutay, du nom de Quelen, et
que cette femme était la laideur même. Par ce mariage, Fromenteau
s'était seigneurifié et avait pris le nom de comte de La Vauguyon. Tant
que les ambassades durèrent et que le fils de sa femme fut jeune, il eut
de quoi vivre\,; mais quand la mère se vit obligée de compter avec son
fils, ils se trouvèrent réduits fort à l'étroit. La Vauguyon, comblé
d'honneurs bien au delà de ses espérances, représenta souvent au roi le
misérable état de ses affaires, et n'en tirait que de rares et très
médiocres gratifications.

La pauvreté peu à peu lui tourna la tête, mais on fut très longtemps
sans s'en apercevoir. Une des premières marques qu'il en donna fut chez
M\textsuperscript{me} Pelot, veuve du premier président du parlement de
Rouen, qui avait tous les soirs un souper et un jeu uniquement pour ses
amis en petit nombre. Elle ne voyait que fort bonne compagnie, et La
Vauguyon y était presque tous les soirs. Jouant au brelan, elle lui fit
un renvi \footnote{Terme de jeu. On appelait \emph{renvi} ce que l'on
  ajoutait à la somme engagée.} qu'il ne tint pas. Elle l'en plaisanta,
et lui dit qu'elle était bien aise de voir qu'il était un poltron. La
Vauguyon ne répondit mot, mais, le jeu fini, il laissa sortir la
compagnie et quand il se vit seul avec M\textsuperscript{me} Pelot, il
ferma la porte au verrou, enfonça son chapeau dans sa tête, l'accula
contre sa cheminée, et lui mettant la tête entre ses deux poings, lui
dit qu'il ne savait ce qui le tenait qu'il ne la lui mit en compote,
pour lui apprendre à l'appeler poltron. Voilà une femme bien effrayée,
qui, entre ses deux poings, lui faisait des révérences perpendiculaires
et des compliments tant qu'elle pouvait, et l'autre toujours en furie et
en menaces. À la fin il la laissa plus morte que vive et s'en alla.
C'était une très bonne et très honnête femme, qui défendit bien à ses
gens de la laisser seule avec La Vauguyon, mais qui eut la générosité de
lui en garder le secret jusqu'après sa mort, et de le recevoir chez elle
à l'ordinaire, où il retourna comme si de rien n'eût été.

Longtemps après, rencontrant sur les deux heures après midi M. de
Courtenay, dans ce passage obscur à Fontainebleau, qui, du salon d'en
haut devant la tribune, conduit à une terrasse le long de la chapelle,
lui fit mettre l'épée à la main, quoi que l'autre lui pût dire sur le
lieu où ils étaient et sans avoir jamais eu occasion ni apparence de
démêlé. Au bruit des estocades, les passants dans ce grand salon
accoururent et les séparèrent, et appelèrent des Suisses de la salle des
gardes de l'ancien appartement de la reine mère, où il y en avait
toujours quelques-uns et qui donnait dans le salon. La Vauguyon, dès
lors chevalier de l'ordre, se débarrassa d'eux et courut chez le roi,
tourne la clef du cabinet, force l'huissier, entre, et se jette aux
pieds du roi, en lui disant qu'il venait lui apporter sa tête. Le roi,
qui sortait de table, chez qui personne n'entrait jamais que mandé, et
qui n'aimait pas les surprises, lui demanda avec émotion à qui il en
avait. La Vauguyon, toujours à genoux, lui dit qu'il a tiré l'épée dans
sa maison, insulté par M. de Courtenay, et que son honneur a été plus
fort que son devoir. Le roi eut grand'peine à s'en débarrasser, et dit
qu'il verrait à éclaircir cette affaire, et un moment après les envoya
arrêter tous deux par des exempts du grand prévôt, et mener dans leurs
chambres. Cependant on amena deux carrosses, qu'on appelait de la pompe,
qui servaient à Bontems et à divers usages pour le roi, qui étaient à
lui, mais sans armes et avaient leurs attelages. Les exempts qui les
avaient arrêtés les mirent chacun dans un de ces carrosses et l'un d'eux
avec chacun, et les conduisirent à Paris à la Bastille, où ils
demeurèrent sept ou huit mois, avec permission au bout du premier mois
d'y voir leurs amis, mais traités tous deux en tout avec une égalité
entière. On peut croire le fracas d'une telle aventure\,: personne n'y
comprenait rien. Le prince de Courtenay était un fort honnête homme,
brave, mais doux, et qui n'avait de la vie eu querelle avec personne. Il
protestait qu'il n'en avait aucune avec La Vauguyon, et qu'il l'avait
attaqué et forcé de mettre l'épée à la main, pour n'en être pas
insulté\,; d'autre part on ne se doutait point encore de l'égarement de
La Vauguyon, il protestait de même que c'était l'autre qui l'avait
attaqué et insulté\,: on ne savait donc qui croire, ni que penser.
Chacun avait ses amis, mais personne ne put goûter l'égalité si fort
affectée en tous les traitements faits à l'un et à l'autre. Enfin, faute
de meilleur éclaircissement et la faute suffisamment expiée, ils
sortirent de prison, et peu après reparurent à la cour.

Quelque temps après, une nouvelle escapade mit les choses plus au net.
Allant à Versailles, La Vauguyon rencontre un palefrenier de la livrée
de M. le Prince, menant un cheval de main tout sellé, allant vers Sèvres
et vers Paris. Il arrête, appelle, met pied à terre et demande à qui est
le cheval. Le palefrenier répond qu'il est à M. le Prince. La Vauguyon
lui dit que M. le Prince ne trouvera pas mauvais qu'il le monte, et
saute au même temps dessus. Le palefrenier bien étourdi ne sait que
faire à un homme à qui il voit un cordon bleu par-dessus son habit et
sortant de son équipage, et le suit. La Vauguyon prend le petit galop
jusqu'à la porte de la Conférence, gagne le rempart et va mettre pied à
terre à la Bastille, donne pour boire au palefrenier et le congédie. Il
monte chez le gouverneur à qui il dit qu'il a eu le malheur de déplaire
au roi et qu'il le prie de lui donner une chambre. Le gouverneur bien
surpris lui demande à son tour à voir l'ordre du roi, et sur ce qu'il
n'en a point, plus étonné encore, résiste à toutes ses prières, et par
capitulation le garde chez lui en attendant réponse de Pontchartrain, à
qui il écrit par un exprès. Pontchartrain en rend compte au roi, qui ne
sait ce que cela veut dire, et l'ordre vient au gouverneur de ne point
recevoir La Vauguyon, duquel, malgré cela, il eut encore toutes les
peines du monde à se défaire. Ce trait et cette aventure du cheval de M.
le Prince firent grand bruit et éclaircirent fort celle de M. de
Courtenay. Cependant, le roi fit dire à La Vauguyon qu'il pouvait
reparaître à la cour, et il continua d'y aller comme il faisait
auparavant, mais chacun l'évitait et on avait grand'peur de lui, quoique
le roi par bonté affectât de le traiter bien.

On peut juger que ces dérangements publics n'étaient pas sans d'autres
domestiques qui demeuraient cachés le plus qu'il était possible. Mais
ils devinrent si fâcheux à sa pauvre femme, bien plus vieille que lui et
fort retirée, qu'elle prit le parti de quitter Paris et de s'en aller
dans ses terres. Elle n'y fut pas bien longtemps, et y mourut tout à la
fin d'octobre, à la fin de cette année. Ce fut le dernier coup qui
acheva de faire tourner la tête à son mari\,: avec sa femme il perdait
toute sa subsistance\,; nul bien de soi et très peu du roi. Il ne la
survécut que d'un mois. Il avait soixante quatre ans, près de vingt ans
moins qu'elle, et n'eut jamais d'enfants. On sut que les deux dernières
années de sa vie il portait des pistolets dans sa voiture et en menaçait
souvent le cocher ou le postillon, en joue, allant et venant de
Versailles. Ce qui est certain c'est que, sans le baron de Beauvais qui
l'assistait de sa bourse et prenait fort soin de lui, il se serait
souvent trouvé aux dernières extrémités, surtout depuis le départ de sa
femme. Beauvais en parlait souvent au roi, et il est inconcevable
qu'ayant élevé cet homme au point qu'il avait fait et lui ayant toujours
témoigné une bonté particulière, il l'ait persévéramment laissé mourir
de faim et devenir fou de misère.

L'année finit par la survivance de la charge de secrétaire d'État de M.
de Pontchartrain, à M. de Maurepas, son fils, qui était conseiller aux
requêtes du palais, et n'avait pas vingt ans, borgne de la petite
vérole. Il est seul, et a perdu un aîné dont le père et la mère ne se
consolent point.

À propos de cette charge, les ennemis bombardèrent Saint-Malo presque en
même temps, sans presque autres dommages que toutes les vitres de la
ville cassées par le bruit terrible d'une espèce de machine infernale
qui s'ouvrit et sauta avant d'être à portée. M. de Chaulnes et le duc de
Coislin qui était allé présider aux états, y étaient accourus avec force
officiers de marine et beaucoup de noblesse. Le maréchal de Boufflers
épousa la fille du duc de Grammont, à Paris, et le roi donna à Dangeau
la grande maîtrise de l'ordre de Notre-Dame du Mont-Carmel et de celui
de Saint-Lazare unis, comme l'avait Nerestang lorsqu'il la remit entre
les mains du roi, qui en fit M. de Louvois son grand vicaire. L'hiver
précédent le roi avait institué l'ordre de Saint-Louis, et c'est ce qui
donna lieu à donner à un particulier la grande maîtrise de Saint-Lazare.
Ces deux ordres sont si connus que je ne m'arrêterai pas à les
expliquer\,; je remarquerai seulement que le roi, qui, faute d'assez de
récompenses effectives, était fort attentif à en faire de tout ce qui
pouvait amuser l'émulation, se montra fort jaloux de faire valoir ce
nouvel ordre de Saint-Louis en toutes les manières qui lui furent
possibles. Il déclara aussi chevalier du Saint-Esprit le marquis
d'Arquien, aux instances les plus vives du roi et de la reine de
Pologne, sa fille, auprès de laquelle il vivait, et qui n'avait jamais
pu réussir à le faire faire due.

L'année finit par l'arrivée de MM. de Vendôme de l'armée du maréchal
Catinat. On remarqua d'autant mieux combien ils furent bien reçus, qu'on
avait été plus surpris de ce que M. le Dire, quoique gendre du roi,
l'avait été médiocrement, M. le prince de Conti très froidement, et M.
de Luxembourg, comme s'il n'avait point fait parler de lui de toute la
campagne dont le roi ne l'entretint, et encore peu, que plus de quinze
jours après son arrivée.

\hypertarget{chapitre-viii.}{%
\chapter{CHAPITRE VIII.}\label{chapitre-viii.}}

1694

~

\relsize{-.8}
\linespread{0.4}

{\textsc{Année 1694. Origine de mon intime amitié avec le duc de
Beauvilliers jusqu'à sa mort.}} {\textsc{- Louville.}} {\textsc{- La
Trappe et son réformateur, et mon intime liaison avec lui.}} {\textsc{-
Son origine.}} {\textsc{- Procès de préséance de M. de Luxembourg contre
seize pairs de France ses anciens.}} {\textsc{- Branche de la maison de
Luxembourg établie en France.}} {\textsc{- M. de Luxembourg, sa branche
et sa fortune.}} {\textsc{- Ruses de M. de Luxembourg.}} {\textsc{- Ducs
à brevet.}} \relsize{+.8} \linespread{1}

~

Ma mère, qui avait eu beaucoup d'inquiétude de moi pendant toute la
campagne, désirait fort que je n'en fisse pas une seconde sans être
marié. Il fut donc fort question de cette grande affaire entre elle et
moi. Quoique fort jeune, je n'y avais pas de répugnance, mais je voulais
me marier à mon gré. Avec un établissement considérable, je me sentais
fort esseulé dans un pays où le crédit et la considération faisaient
plus que tout le reste. Fils d'un favori de Louis VIII, et d'une mère
qui n'avait vécu que pour lui, qu'il avait épousée n'étant plus jeune
elle-même, sans oncle ni tante, ni cousins germains, ni parents proches,
ni amis utiles de mon père et de ma mère, si hors de tout par leur âge,
je me trouvais extrêmement seul. Les millions ne pouvaient me tenter
d'une mésalliance, ni la mode, ni mes besoins me résoudre à m'y ployer.

Le duc de Beauvilliers s'était toujours souvenu que mon père et le sien
avaient été amis, et que lui-même avait vécu sur ce pied-là avec mon
père, autant que la différence d'âge, de lieux et de vie l'avait pu
permettre\,; et il m'avait toujours montré tant d'attention chez les
princes dont il était gouverneur, et à qui je faisais ma cour, que ce
fut à lui à qui je m'adressai, à la mort de mon père et depuis, pour
l'agrément du régiment, comme je l'ai marqué. Sa vertu, sa douceur, sa
politesse, tout m'avait épris de lui. Sa faveur alors était au plus haut
point. Il était ministre d'État depuis la mort de M. de Louvois\,; il
avait succédé fort jeune au maréchal de Villeroy dans la place de chef
du conseil des finances, et il avait eu de son père la charge de premier
gentilhomme de la chambre\,; la réputation de la duchesse de
Beauvilliers me touchait encore, et l'union intime dans laquelle ils
avaient toujours vécu. L'embarras était le bien\,: j'en avais grand
besoin pour nettoyer le mien, qui était fort en désordre, et M. de
Beauvilliers avait deux fils et huit filles. Malgré tout cela, mon goût
l'emporta, et ma mère l'approuva.

Le parti pris, je crus qu'aller droit à mon but, sans détours et sans
tiers, aurait plus de grâce\,; ma mère me remit un état bien vrai et
bien exact de mon bien et de mes dettes, des charges et des procès que
j'avais. Je le portai à Versailles, et je fis demander à M. de
Beauvilliers un temps où je pusse lui parler secrètement, à loisir et
tout à mon aise. Louville fut celui qui le lui demanda. C'était un
gentilhomme de bon lieu, dont la mère l'était aussi, la famille de
laquelle avait toujours été fort attachée à mon père et qu'il avait fort
protégée dansa faveur, et longtemps depuis par M. de Seignelay.
Louville, élevé dans ce même attachement, avait été pris, de capitaine
au régiment du roi infanterie, pour être gentilhomme de la manche de M.
le duc d'Anjou, par M. de Beauvilliers, à la recommandation de mon père,
et M. de Beauvilliers, qui l'avait fort goûté depuis, ne l'avait connu,
quoique son parent, que par mon père. Louville était d'ailleurs homme
d'infiniment d'esprit, et qui, avec une imagination qui le rendait
toujours neuf et de la plus excellente compagnie, avait toute la lumière
et le sens des grandes affaires et des plus solides et des meilleurs
conseils.

J'eus donc mon rendez-vous, à huit heures du soir, dans le cabinet de
M\textsuperscript{me} de Beauvilliers, où le duc me vint trouver seul et
sans elle. Là, je lui fis mon compliment, et sur ce qui m'amenait, et
sur ce que j'avais mieux aimé m'adresser directement à lui, que de lui
faire parler comme on fait d'ordinaire dans ces sortes d'affaires\,; et
qu'après lui avoir témoigné tout mon désir, je lui apportais un état le
plus vrai, le plus exact de mon bien et de mes affaires, sur lequel je
le suppliais de voir ce qu'il y pourrait ajouter pour rendre sa fille
heureuse avec moi\,; que c'étaient là toutes les conditions que je
voulais faire, sans vouloir ouïr parler d'aucune sorte de discussion sur
pas une autre, ni sur le plus ou le moins\,; et que toute la grâce que
je lui demandais était de m'accorder sa fille et de faire faire le
contrat de mariage tout comme il lui plairait\,; que ma mère et moi
signerions sans aucun examen.

Le duc eut sans cesse les yeux collés sur moi pendant que je lui parlai.
Il me répondit en homme pénétré de reconnaissance, et de mon désir, et
de ma franchise, et de ma confiance. Il m'expliqua l'état de sa famille,
après m'avoir demandé un peu de temps pour en parler à
M\textsuperscript{me} de Beauvilliers, et voir ensemble ce qu'ils
pourraient faire. Il me dit donc que, de ses huit filles, l'aînée était
entre quatorze et quinze ans\,; la seconde très contrefaite et nullement
mariable\,; la troisième entre douze et treize ans\,; toutes les autres,
des enfants qu'il avait à Montargis, aux Bénédictines, dont il avait
préféré la vertu et la piété qu'il y connaissait, à des couvents plus
voisins où il aurait eu le plaisir de les voir plus souvent. Il ajouta
que son aînée voulait être religieuse\,; que la dernière fois qu'il
l'avait été voir de Fontainebleau, il l'y avait trouvée plus déterminée
que jamais\,; que, pour le bien, il en avait peu\,; qu'il ne savait s'il
me conviendrait, mais qu'il me protestait qu'il n'y avait point
d'efforts qu'il ne fit pour moi de ce côté-là. Je lui répondis qu'il
voyait bien, à la proposition que je lui faisais, que ce n'était pas le
bien qui m'amenait à lui, ni même sa fille que je n'avais jamais vue,
que c'était lui qui m'avait charmé et que je voulais épouser avec
M\textsuperscript{me} de Beauvilliers. « Mais, me dit-il, si elle veut
absolument être religieuse\,? --- Alors, répliquai-je, je vous demande
la troisième.\,» À cette proposition, il me fit deux objections\,: son
âge et la justice de lui égaler l'aînée pour le bien, si le mariage de
la troisième fait, cette aînée changeait d'avis et ne voulait plus être
religieuse, et l'embarras où cela le jetterait. À la première, je
répondis par l'exemple domestique de sa belle-sœur, plus jeune encore
lorsqu'elle avait épousé le feu duc de Mortemart\,; à l'autre, qu'il me
donnât la troisième, sur le pied que l'aînée se marierait, quitte à me
donner le reste de ce qu'il aurait destiné d'abord, le jour que l'aînée
ferait profession, et que si elle changeait d'avis, je me contenterais
d'un mariage de cadette, et serais ravi que l'aînée trouvât encore mieux
que moi.

Alors, le duc levant les yeux au ciel, et presque hors de lui, me
protesta qu'il n'avait jamais été combattu de la sorte\,; qu'il lui
fallait ramasser toutes ses forces pour ne me la pas donner à l'instant.
Il s'étendit sur mon procédé avec lui, et me conjura, que la chose
réussit ou non, de le regarder désormais comme mon père, qu'il m'en
servirait en tout, et que l'obligation que j'acquérais sur lui était
telle qu'il ne pouvait moins m'offrir et me tenir que tout ce qui était
en lui de services et de conseils. Il m'embrassa en effet comme son
fils, et nous nous séparâmes de la sorte pour nous revoir à l'heure
qu'il me dirait le lendemain au lever du roi. Il m'y dit à l'oreille, en
passant, de me trouver ce même jour, à trois heures après midi, dans le
cabinet de Mgr le duc de Bourgogne, qui devait alors être au jeu de
paume et son appartement désert. Mais il se trouve toujours des fâcheux.
J'en trouvai deux, en chemin du rendez-vous, qui, étonnés de l'heure où
ils me trouvaient dans ce chemin où ils ne me voyaient aucun but,
m'importunèrent de leurs questions\,; je m'en débarrassai comme je pus,
et j'arrivai enfin au cabinet du jeune prince, où je trouvai son
gouverneur qui avait mis un valet de chambre de confiance à la porte
pour n'y laisser entrer que moi. Nous nous assîmes vis-à-vis l'un de
l'autre, la table d'étude entre nous deux. Là, j'eus la réponse la plus
tendre, mais négative, fondée sur la vocation de sa fille, sur son peu
de bien pour l'égaler à la troisième, si, le mariage fait, elle se
ravisait\,; sur ce qu'il n'était point payé de ses états, et sur le
désagrément que ce lui serait d'être le premier des ministres qui n'eût
pas le présent que le roi avait toujours fait lors du mariage de leurs
filles, et que l'état présent des affaires l'empêchait d'espérer. Tout
ce qui se peut de douleur, de regret, d'estime, de préférence, de
tendre, me fut dit\,; je répondis de même, et nous nous séparâmes, en
nous embrassant, sans pouvoir plus nous parler. Nous étions convenus
d'un secret entier qui nous faisait cacher nos conversations et les
dépayser, de sorte que, ce jour-là, j'avais conté à M. de Beauvilliers,
avant d'entrer en matière, les deux rencontres que j'avais faites\,; et
sur ce qu'il me recommanda de plus en plus le secret, je donnai le
change à Louville de ce second entretien, quoiqu'il sût le premier, et
qu'il fût un des deux hommes que j'avais rencontrés.

Le lendemain matin, au lever du roi, M. de Beauvilliers me dit à
l'oreille qu'il avait fait réflexion que Louville était homme très sûr
et notre ami intime à tous deux, et que, si je voulais lui confier notre
secret, il nous deviendrait un canal très commode et très caché. Cette
proposition me rendit la joie par l'espérance, après avoir compté tout
rompu. Je vis Louville dans la journée\,; je l'instruisis bien, et le
priai de n'oublier rien pour servir utilement la passion que j'avais de
ce mariage.

Il me procura une entrevue pour le lendemain dans ce petit salon du bout
de la galerie qui touche à l'appartement de la reine et où personne ne
passait, parce que cet appartement était fermé depuis la mort de
M\textsuperscript{me} la Dauphine. J'y trouvai M. de Beauvilliers à qui
je dis, d'un air allumé de crainte et d'espérance, que la conversation
de la veille m'avait tellement affligé, que je l'avais abrégée dans le
besoin que je me sentais d'aller passer les premiers élans de ma douleur
dans la solitude, et il était vrai\,; mais que, puisqu'il me permettait
de traiter encore cette matière, je n'y voyais que deux principales
difficultés, le bien et la vocation\,; que pour le bien, je lui
demandais en grâce de prendre cet état du mien que je lui apportais
encore, et de régler dessus tout ce qu'il voudrait. À l'égard du
couvent, je me mis à lui faire une peinture vive de ce que l'on ne prend
que trop souvent pour vocation, et qui n'est rien moins et très souvent
que préparation aux plus cuisants regrets d'avoir renoncé à ce qu'on
ignore et qu'on se peint délicieux, pour se confiner dans une prison de
corps et d'esprit qui désespère\,; à quoi j'ajoutai celle du bien et des
exemples de vertu que sa fille trouverait dans sa maison.

Le duc me parut profondément touché du motif de mon éloquence. Il me dit
qu'il en était pénétré jusqu'au fond de l'âme, qu'il me répétait, et de
tout son cœur, ce qu'il m'avait déjà dit, qu'entre M. le comte de
Toulouse et moi, s'il lui demandait sa fille, il ne balancerait pas à me
préférer, et qu'il ne se consolerait de sa vie de me perdre pour son
gendre. Il prit l'état de mon bien pour examiner avec
M\textsuperscript{me} de Beauvilliers tout ce qu'ils pourraient faire
tant sur le bien que sur le couvent\,: « Mais si c'est sa vocation,
ajouta-t-il, que voulez-vous que j'y fasse\,? Il faut en tout suivre
aveuglément la volonté de Dieu et sa loi, et il sera le protecteur de ma
famille. Lui plaire et le servir fidèlement est la seule chose désirable
et doit être l'unique fin de nos actions.\,» Après quelques autres
discours nous nous séparâmes.

Ces paroles si pieuses, si détachées, si grandes, dans un homme si
grandement occupé, augmentèrent mon respect et mon admiration, et en
même temps mon désir, s'il était possible. Je contai tout cela à
Louville, et le soir j'allai à la musique à l'appartement, où je me
plaçai en sorte que j'y pus toujours voir M. de Beauvilliers, qui était
derrière les princes. Au sortir de là je ne pus me contenir de lui dire
à l'oreille que je ne me sentais point capable de vivre heureux avec une
autre qu'avec sa fille, et, sans attendre de réponse, je m'écoulai.
Louville avait jugé à propos que je visse M\textsuperscript{me} de
Beauvilliers, à cause de la confiance entière de M. de Beauvilliers en
elle, et me dit de me trouver le lendemain chez elle, porte fermée, à
huit heures du soir. J'y trouvai Louville avec elle\,; là, après les
remerciements, elle me dit sur le bien et sur le couvent à peu près les
mêmes raisons, mais je crus apercevoir fort clairement que le bien était
un obstacle aisé à ajuster, et qui n'arrêterait pas\,; mais que la
pierre d'achoppement était la vocation. J'y répondis donc comme j'avais
fait là-dessus à M. de Beauvilliers. J'ajoutai qu'elle se trouvait entre
deux vocations\,; qu'il n'était plus question que d'examiner laquelle
des deux était la plus raisonnable, la plus ferme, la plus dangereuse à
ne pas suivre\,: l'une, d'être religieuse, l'autre, d'épouser sa
fille\,; que la sienne était sans connaissance de cause, la mienne,
après avoir parcouru toutes les filles de qualité\,; que la sienne était
sujette au changement, la mienne stable et fixée\,; qu'en forçant la
sienne on ne gâtait rien, puisqu'on la mettait dans l'état naturel et
ordinaire, et dans le sein d'une famille où elle trouverait autant ou
plus de vertu et de piété qu'à Montargis\,; que forcer la mienne
m'exposait à vivre malheureux et mal avec la femme que j'épouserais et
avec sa famille.

La duchesse fut surprise de la force de mon raisonnement et de la
prodigieuse ardeur de son alliance qui me le faisait faire. Elle me dit
que si j'avais vu les lettres de sa fille à M. l'abbé de Fénelon, je
serais convaincu de la vérité de sa vocation\,; qu'elle avait fait ce
qu'elle avait pu pour porter sa fille à venir passer sept ou huit mois
auprès d'elle pour lui faire voir la cour et le monde sans avoir pu y
réussir à moins d'une violence extrême\,; qu'au fond elle répondrait à
Dieu de la vocation de sa fille dont elle était chargée, et non de la
mienne\,; que j'étais un si bon casuiste, que je ne laissais pas de
l'embarrasser\,; qu'elle verrait encore avec M. de Beauvilliers, parce
qu'elle serait inconsolable de me perdre, et me répéta les mêmes choses
tendres et flatteuses que son mari m'avait dites, et avec la même
effusion de cœur. La duchesse de Sully qui entra, je ne sais comment,
quoique la porte fût défendue, nous interrompit là, et je m'en allai
fort triste, parce que je sentis bien que des personnes si pieuses et si
désintéressées ne se mettraient jamais au-dessus de la vocation de leur
fille.

Deux jours après, au lever du roi, M. de Beauvilliers me dit de le
suivre de loin jusque dans un passage obscur, entre la tribune et la
galerie de l'aile neuve au bout de laquelle il logeait, et ce passage
était destiné à un grand salon pour la chapelle neuve que le roi voulait
bâtir. Là, M. de Beauvilliers me rendit l'état de mon bien, et me dit
qu'il y avait vu que j'étais grand seigneur en bien comme dans le reste,
mais qu'aussi je ne pouvais différer à me marier\,; me renouvela ses
regrets et me conjura de croire que Dieu seul qui voulait sa fille pour
son épouse avait la préférence sur moi, et l'aurait sur le Dauphin même,
s'il était possible qu'il la voulût épouser\,; que si, dans les suites,
sa fille venait à changer et que je fusse libre, j'aurais la préférence
sur quiconque, et lui se trouverait au comble de ses désirs\,; que, sans
l'embarras de ses affaires, il me prêterait ou me ferait prêter, sous sa
caution, les quatre-vingt mille livres qui faisaient celui des
miennes\,; qu'il était réduit à me conseiller de chercher à me marier,
et à s'offrir d'en porter les paroles, et de faire son affaire propre
désormais de toutes les miennes. Je m'affligeai, en lui répondant, que
la nécessité de mes affaires ne me permit pas d'attendre à me marier
jusqu'à sa dernière fille, qui toutes peut-être ne seraient pas
religieuses\,: c'était en effet ma disposition. La fin de l'entretien ne
fut que protestations les plus tendres d'un intérêt et d'une amitié
intime et éternelle, et de me servir en tout et pour tout de son conseil
et de son crédit en petites et en grandes choses, et de nous regarder
désormais pour toujours l'un et l'autre comme un beau-père et un gendre
dans la plus indissoluble union. Il s'ouvrit après à Louville, et dans
son amertume il lui dit qu'il ne se consolait que dans l'espérance que
ses enfants et les miens se pourraient marier quelque jour, et il me fit
prier d'aller passer quelques jours à Paris pour lui laisser chercher
quelque trêve à sa douleur par mon absence. Nous en avions tous deux
besoin.

Je me suis peut-être trop étendu en détails sur cette affaire, mais j'ai
jugé à propos de le faire pour donner par là la clef de cette union et
de cette confiance si intime, si entière, si continuelle et en toutes
affaires si importantes de M. de Beauvilliers en moi et de ma liberté
avec lui en toutes choses qui sans cela serait tout à fait
incompréhensible dans cette extrême différence d'âge, et du caractère
secret, isolé, particulier et si mesuré ou plutôt resserré du duc de
Beauvilliers et de cet attachement que j'ai eu toujours pour lui sans
réserve ni comparaison.

Ce fut donc à chercher un autre mariage. Un hasard fit jeter des propos
à ma mère de celui de la fille aînée du maréchal-duc de Lorges avec sa
charge de capitaine des gardes du corps\,; mais la chose tomba bientôt
pour lors, et j'allai chercher à me consoler à la Trappe de
l'impossibilité de l'alliance du duc de Beauvilliers.

La Trappe est un lieu si célèbre et si connu et son réformateur si
célèbre que je ne m'étendrai point ici en portraits ni en
descriptions\,; je dirai seulement que cette abbaye est à cinq lieues de
la Ferté-au-Vidame ou Arnault, qui est le véritable nom distinctif de
cette Ferté parmi tant d'autres Ferté en France qui ont conservé le nom
générique de ce qu'elles ont été, c'est-à-dire des forts ou des
forteresses (\emph{firmitas}). Louis XIII avait voulu que mon père
achetât cette terre depuis longtemps en décret après la mort de ce La
Fin qui, après être entré dans la conspiration du duc de Biron, le
trahit d'autant plus cruellement qu'il le tint toujours en telle opinion
de sa fidélité qu'il fut cause de sa perte. La proximité de
Saint-Germain et de Versailles, dont la Ferté n'est qu'à vingt lieues,
fut caisse de cette acquisition. C'était ma seule terre bâtie où mon
père passait les automnes. Il avait fort connu M. de la Trappe dans le
monde. Il y était son ami particulier, et cette liaison se resserra de
plus en plus depuis sa retraite si voisine de chez mon père qui l'y
allait voir plusieurs jours tous les ans\,; il m'y avait mené. Quoique
enfant, pour ainsi dire encore, M. de la Trappe eut pour moi des charmes
qui m'attachèrent à lui, et la sainteté du lieu m'enchanta. Je désirai
toujours d'y retourner, et je me satisfis toutes les années et souvent
plusieurs fois, et souvent des huitaines de suite\,; je ne pouvais me
lasser d'un spectacle si grand et si touchant, ni d'admirer tout ce que
je remarquais dans celui qui l'avait dressé pour la gloire de Dieu et
pour sa propre sanctification et celle de tant d'autres. Il vit avec
bonté ces sentiments dans le fils de son ami\,; il m'aima comme son
propre enfant, et je le respectai avec la même tendresse que si je
l'eusse été. Telle fut cette liaison, singulière à mon âge, qui m'initia
dans la confiance d'un homme si grandement et si saintement distingué,
qui me lui fit donner la mienne, et dont je regretterai toujours de
n'avoir pas mieux profité.

À mon retour de la Trappe où je n'allais que clandestinement pour
dérober ces voyages aux discours du monde à mon âge, je tombai dans une
affaire qui fit grand bruit et qui eut pour moi bien des suites.

M. de Luxembourg, fier de ses succès et de l'applaudissement du monde à
ses victoires, se crut assez fort pour se porter du dix-huitième rang
d'ancienneté qu'il tenon parmi les pairs au second, et immédiatement
après M. d'Uzès. Ceux qu'il attaqua en préséance furent\,:

Henri de Lorraine, duc d'Elbœuf, gouverneur de Picardie et d'Artois\,;

Charles de Rohan, duc de Montbazon, prince de Guéméné\,;

Charles de Lévy, duc de Ventadour\,;

Duc de Vendôme, gouverneur de Provence et chevalier de l'ordre\,;

Charles duc de la Trémoille, premier gentilhomme de la chambre et
chevalier de l'ordre\,;

Maximilien de Béthune, duc de Sully, chevalier de l'ordre\,;

Charles d'Albert, duc de Chevreuse, chevalier de l'ordre, capitaine des
chevau-légers de la garde\,;

Le fils mineur de la duchesse de Lesdiguières-Gondi\,;

Henri de Cossé, duc de Brissac\,;

Charles d'Albert, dit d'Ailly, chevalier de l'ordre, gouverneur de
Bretagne, si connu par ses ambassades\,;

Armand Jean de Vignerod, dit du Plessis, duc de Richelieu et de Fronsac,
chevalier de l'ordre\,;

Louis, duc de Saint-Simon\,;

Fr.~duc de La Rochefoucauld, chevalier de l'ordre, grand maître de la
garde-robe, toujours si bien avec le roi, et grand veneur de France\,;

Jacques-Nompar de Caumont, duc de La Force\,;

Henri Grimaldi, duc de Valentinois, prince de Monaco, chevalier de
l'ordre\,;

Chabot, duc de Rohan\,;

Et de La Tour, duc de Bouillon, grand chambellan de France et gouverneur
d'Auvergne.

Avant d'entrer dans l'explication de la prétention de M. de Luxembourg,
une courte généalogie y jettera de la lumière pour la suite\,:

François de Luxembourg, fait duc V de Piney, 18 septembre 1577, et pair
de France femelle \footnote{La \emph{pairie femelle} était celle qui
  pouvait se transmettre aux femmes.}, 29 décembre 1581, mort septembre
1613.

I. Diane de Lorraine-Aumale \footnote{Première femme de François de
  Luxembourg.}, 13 novembre 1576.

\begin{enumerate}
\def\labelenumi{\Roman{enumi}.}
\setcounter{enumi}{1}
\tightlist
\item
  Marguerite de lorraine-Vaudemont \footnote{Deuxième femme de François
    de Luxembourg.}, 1599, morte sans enfants, 20 septembre 1625.
\end{enumerate}

Henri, duc de Piney, mort dernier mâle de la maison de Luxembourg à 24
ans, 23 mai 1616.

Madeleine \footnote{Femme d'Henri, duc de Piney.}, fille unique de
Guillaume, seigneur de Thoré, fils e frère des deux derniers connétables
de Montmorency, 19 juin 1597.

Marguerite de Luxembourg épousa, le 28 avril 1607, René Potier, depuis
premier duc de Tresmes, mort 1er février 1670, à 91 ans, et elle le 9
août 1645.

Marguerite-Charlotte de Luxembourg, duchesse de Piney, morte à 72 ans, à
Ligny, en novembre 1680.

I. Marie-Léon d'Albert \footnote{Premier mari de Marguerite-Charlotte de
  Luxembourg.}, seigneur de Brantes, frère du connétable de Luynes, 6
juillet 1620, mort novembre 1630.

\begin{enumerate}
\def\labelenumi{\Roman{enumi}.}
\setcounter{enumi}{1}
\tightlist
\item
  Marie-Charles-Henri de Clermont-Tonnerre \footnote{Deuxième mari de
    Marguerite-Charlotte de Luxembourg.}, mort à Ligny, juillet 1674, à
  67 ans.
\end{enumerate}

Marie-Liesse de Luxembourg, mariée à Henri de Levy, duc de Ventadour,
sans enfants. Séparés de bon gré. Il se fit prêtre et mourut chanoine de
Notre-Dame de Paris, octobre 1680, et elle se fit carmélite, septembre
1641, au monastère de Chambéry qu'elle fonda, et y mourut, janvier 1660.

Henri-Léon, duc de Piney, imbécile, diacre, enfermé à Saint-Lazare, à
Paris, où il est mort sans avoir été marié, 19 février 1697, et toujours
interdit par justice.

Marie-Charlotte, etc., religieuse professe 20 ans, et maîtresse des
novices à l'Abbaye-aux-Bois, puis sans être restituée au siècle
chanoinesse, dame du palais de la reine, assise, morte à Versailles,
sous le nom de princesse de Tingry.

Madeleine-Charlotte, née 14 août 1635, mariée 17 mars 1661, morte à
Ligny, laissant nombreuse postérité ---------------\,?

(époux)

François-Henri de Montmorency, comte de Bouteville, maréchal de France,
fait duc et pair de Piney, par nouvelles lettres en se mariant, et
joignant les noms et armes de Luxembourg aux siennes, si connu sous le
nom de maréchal --duc de Luxembourg, mort à Versailles.

Éclaircissons maintenant les personnages de cette généalogie autant
qu'il est nécessaire pour savoir en gros ce qu'ils ont été. Le trop
fameux Louis de Luxembourg, si connu sous le nom de connétable de
Saint-Paul, à qui Louis XI fit couper la tête en place de Grève à Paris,
19 décembre 1475, quoique actuellement remarié à une fille de Savoie,
sœur de la reine sa femme, avait eu trois fils de sa première femme J.
de Bar\,: Pierre, l'aîné, épousa une autre sœur de la reine et de sa
belle-mère, dont une fille unique porta un grand héritage à François de
Bourbon, comte de Vendôme, dont elle eut le premier duc de Vendôme.

Antoine, le second, fit la branche de Brienne où on va revenir, et
Charles, le troisième fils, fut évêque-duc de Laon.

Cet Antoine fut comte de Brienne, père de Charles, et celui-ci
d'Antoine, qui de la seconde fille de René, bâtard de Savoie et frère
bâtard de la mère de François Ier, qui le fit grand maître de France et
gouverneur de Provence, eut deux fils\,: Jean, comte de Brienne, et
François qui fut fait duc de Piney. La sœur aînée de leur mère avait
épousé le célèbre Anne de Montmorency, depuis connétable et duc et pair
de France.

De Jean, comte de Brienne et d'une fille de Robert de La Marck IV,
maréchal de France, duc de Bouillon, seigneur de Sedan, un fils et une
fille\,: le fils fut Charles, comte de Brienne, qui, en 1583, épousa une
sœur du fameux duc d'Épernon qui le fit faire duc à brevet \footnote{Les
  \emph{ducs à brevet} étaient ceux qui portaient le titre de duc en
  vertu d'un brevet royal ou acte privé du roi, qui n'était ni vérifié
  ni enregistré par les cours souveraines. Ce brevet ne pouvait être
  transmis à leurs fils qu'en vertu d'une autorisation spéciale du roi.
  Pour comprendre les détails que donne Saint-Simon dans les passages
  relatifs aux ducs, il est nécessaire de se rappeler qu'il y avait
  alors trois sortes de ducs\,: 1° les \emph{ducs et pairs} dont la
  dignité était héréditaire\,; les femmes mêmes pouvaient la
  transmettre, lorsque les pairies étaient femelles\,; ils avaient droit
  de siéger et de voter au parlement, lorsque les rois y tenaient leurs
  lits de justice et toutes les fois qu'il s'agissait d'affaires
  d'État\,; 2° les \emph{ducs vérifiés}, mais sans pairie, étaient ceux
  dont les terres avaient été érigées en duché et dont le titre, vérifié
  par les cours souveraines, était héréditaire de mâle en mâle par ordre
  de primogéniture. Ils avaient les mêmes droits honorifiques que les
  ducs et pairs\,; ils avaient les \emph{honneurs du Louvre},
  c'est-à-dire qu'ils pouvaient entrer en carrosse au Louvre et dans les
  autres palais royaux\,; leurs femmes avaient un tabouret chez la
  reine\,; mais les ducs vérifiés n'exerçaient aucun des droits
  politiques des ducs et pairs\,; 3° les \emph{ducs à brevet}, dont il a
  été question au commencement de cette note.} en 1587\,; il fut
chevalier du Saint-Esprit en 1597, le sixième après deux ducs et trois
gentilshommes, et mourut sans enfants en novembre 1605\,; ainsi finit sa
branche, et il était fils unique du frère aîné du premier duc de Piney.

Il faut remarquer que ce duc à brevet de Brienne avait deux sœurs,
toutes deux mariées deux fois\,: l'aînée à Louis de Plusquelec, comte de
Kerman en Bretagne, puis à Just de Pontallier, baron de Pleurs\,; la
cadette à Georges d'Amboise, seigneur d'Aubijoux et de Casaubon, puis à
Bernard de Béon, seigneur du Massés, gouverneur de Saintonge e
d'Angoumois\,; elle mourut avec postérité masculine à Bouteville, le 16
juin 1647, à quatre-vingts ans\,: il s'agira d'elle dans la suite du
procès. Son dernier mariage, qui fut une étrange mésalliance, fut
précédé de celle de la sœur de son père, mariée à Christophe Jouvenel,
si plaisamment dit des Ursins, marquis de Traisnel et pourtant chevalier
de l'ordre et gouverneur de Paris\,: nous l'allons voir suivie d'une
autre qu'on a déjà vue dans la généalogie.

Notre premier duc de Piney est fort connu par ses deux ambassades à
Rome, où il reçut tant de dégoûts\,: sa première femme était fille et
sœur des ducs d'Aumale, et la seule dont il eut des enfants. Malgré
l'énorme exemple de ses beaux-frères, il fut fidèle contre la Ligue. Sa
seconde femme était sœur de la reine Louise veuve d'Henri III, et veuve
du duc de Joyeuse, favori de ce prince. À tout prendre, ce premier duc
de Piney était un assez pauvre homme à tout ce qu'on voit de lui\,; mais
quel qu'il fût, on ne s'accoutume point en remontant à ces temps-là à ne
lui voir qu'un fils et une fille (car l'autre fille qui était cadette
fut religieuse et abbesse de Notre-Dame de Troyes, où elle mourut en
1602), on ne s'accoutume point, dis-je, à lui voir marier sa seule fille
à René Potier, et une fille de cette naissance et qui, par la mort de
son frère unique sans enfants, pouvait apporter tous les biens de cette
grande maison et la dignité de duc et pair, si rare encore, à son
mari\,; et il faut noter que le premier duc de Piney fit ce mariage dans
son château de Pongy, sa principale demeure, et où il mourut six où sept
ans après son fils unique, n'ayant que quatorze ans lors de ce mariage.

René Potier était alors uniquement bailli et gouverneur de Valois. Il ne
fut chambellan du roi et gouverneur de Châlons que l'année d'après son
mariage et même dix-huit mois, et trois ans après capitaine des gardes
du corps qu'il acheta de M. de Praslin. Il poussa après sa fortune, à
force d'années, jusqu'à devenir duc et pair à l'étrange fournée de
1663\,; et son fils, le gros duc de Gesvres, vendit sa charge de
capitaine des gardes du corps à M. de Lauzun, et acheta celle de premier
gentilhomme de la chambre qui a passé à sa postérité avec le
gouvernement de Paris qu'il eut à la mort du duc de Créqui. René Potier
dont il s'agit était fils et frère aîné de secrétaires d'État, qui, et
longtemps depuis, n'avaient pas pris le vol où ils se sont su élever. Le
secrétaire d'État était énormément riche\,; il avait été secrétaire du
roi, puis secrétaire du conseil, et avait travaillé dans les bureaux du
secrétaire d'État Villeroy. Il ne fut secrétaire d'État qu'en février
1589. Son père était conseiller au parlement, et son grand-père prévôt
des marchands, dont le père était général des monnaies, au delà duquel
on ne voit rien. Il ne faut donc pas croire que les mésalliances soient
si nouvelles en France\,; mais à la vérité elles n'étoient pas communes
alors.

Le second duc de Piney mourut si jeune qu'on ne sait quel il eût été. Le
mariage de sa fille, et presque unique héritière, fut l'effet et
l'effort de la faveur alors toute-puissante du connétable de Luynes. Le
père était mort en 1616, et la mère en 1615\,; l'autre fille n'a point
eu de postérité, et la singularité de l'issue de son mariage avec le duc
de Ventadour les a suffisamment fait connaître l'un et l'autre.

Venons présentement à notre duchesse héritière de Piney. Elle perdit son
mari au bout de dix années de mariage\,; elle avait été mariée à douze
ans, et n'en avait que vingt-deux lorsqu'elle devint veuve, puisqu'elle
en avait soixante-douze lorsqu'elle mourut en 1680. Il paraît qu'elle ne
fit pas grand cas de son premier mari ni des deux enfants qu'elle en
eut. Toute la faveur avait disparu avec le connétable de Luynes. Louis
XIII, né à Fontainebleau, 27 septembre 1601, tenu en esclavage par la
reine sa mère et ses favoris jusqu'à savoir à peine lire et écrire,
n'avait que quinze ans et demi lorsque, n'ayant que le seul Luynes à qui
pouvoir parler, il consentit à se livrer à lui pour se délivrer de
prison et d'un joug énorme, en faisant arrêter le maréchal d'Ancre qu'il
défendit à plusieurs reprises de tuer, et qu'à cet âge on lui fit croire
qu'on n'avait pu s'en dispenser. Ce même âge, joint à l'inexpérience et
à l'ignorance totale où il avait été tenu, l'abandonna à son libérateur
qui en sut si rapidement et si prodigieusement profiter, et lorsqu'il
mourut à la fin de 1621, Louis XIII, qui ne faisait qu'avoir vingt ans,
s'était déjà ouvert les yeux sur un si grand abus de sa faveur. Elle ne
put donc plus rien, et il n'est pas étrange qu'en 1630, que la duchesse
héritière de Piney devint veuve d'un frère de ce connétable, le duc de
Chaulnes, son autre frère, qui était aussi maréchal de France, et qui ne
laissait pas de figurer à force de mérite et d'établissements, ne l'ait
pu empêcher d'user de toute l'autorité de mère sur ses enfants et de
toute la liberté de veuve en se remariant. Celui qu'elle épousa était
par sa naissance un parti très digne d'elle, mais d'ailleurs il était
frère cadet du comte de Tonnerre, père de l'évêque-comte de Noyon dont
j'ai parlé plus haut, et ce comte de Tonnerre, bien qu'aîné, fit une
mésalliance qui marque qu'il avait besoin de bien. L'amour apparemment
fit faire ce second mariage, et comme il entraîna la chute du nom, du
rang et des honneurs de duchesse, ce couple s'en alla vivre chez
l'épouse dans sa magnifique terre de Ligny, où tous deux sont morts sans
en être presque jamais sortis. Il était de l'intérêt du nouvel époux de
se défaire du fils et de la fille du premier lit. Le fils en offrit les
moyens de soi-même. Il était imbécile\,; ils le firent interdire
juridiquement et enfermer à Paris, à Saint-Lazare\,; et de peur que
quelqu'un ne le fît marier, ils le firent ordonner diacre, et c'est dans
cet état et dans ce même lieu qu'il a passé sa longue vie, et qu'il est
mort. La fille n'avait guère le sens commun, mais n'était pas imbécile.
On la fit religieuse à Paris, à l'Abbaye-aux-Bois. De fois à autre elle
disait que ç'avait été malgré elle, mais elle y vécut vingt ans
professe, et y fut plusieurs années maîtresse des novices\,; ce qui ne
marque pas qu'elle eût été forcée\,; ou du moins il paraît par cet
emploi qu'elle avait consenti et pris goût à son état, puisqu'on la
chargeait d'y former des novices. Elle était encore dans cette fonction
quand M. le Prince l'en tira comme on le dira bientôt.

M. de Luxembourg, qui combla sa fortune en épousant la fille unique du
second lit, était fils unique de ce M. de Bouteville si connu par ses
duels, et qui, retiré a Bruxelles pour avoir tué en duel le comte de
Thorigny en 1627, hasarda de revenir à Paris se battre à la place Royale
contre Bussy d'Amboise, qui était Clermont-Gallerande, qu'il tua.
Bouteville avait pour second son cousin de Rosmadec, baron des
Chapelles, qui eut affaire au baron d'Harcourt, second de l'autre, qui
fut le seul qui s'en tira et qui s'en alla en Italie, se jeta dans
Casal, assiégé par les Espagnols, et y fut tué en novembre 1628. Il ne
fut point marié, et il était frère puîné du grand-père du marquis de
Beuvron père du maréchal-duc d'Harcourt. La mère de ces deux frères
était fille du maréchal de Matignon\,; il était cousin germain de ce
comte de Thorigny, fils de la Longueville, que Bouteville avait tué,
petit-fils au même maréchal de Matignon, et premier mari sans enfants de
la duchesse d'Angoulême La Guiche, fille du grand maître de
l'artillerie. Ce comte de Thorigny était frère aîné de l'autre comte de
Thorigny qui lui succéda, lequel fut père du dernier maréchal de
Matignon et du comte de Matignon, dont le fils unique a été fait duc de
Valentinois, en épousant la fille aînée du dernier prince de
Monaco-Grimaldi. MM. de Bouteville et des Chapelles furent pris se
sauvant en Flandre, et eurent la tête coupée en Grève, à Paris, par
arrêt du parlement, 22 juin 1627. Ce M. de Bouteville avait épousé en
1617 Élisabeth, fille de Jean Vienne, président en la chambre des
comptes, et d'Élisabeth Dolu, et cette M\textsuperscript{me} de
Bouteville a vu toute la fortune de son fils et les mariages de ses deux
filles. Elle a passé sa longue vie toujours retirée à la campagne, et y
est morte, en 1696, à quatre-vingt-neuf ans, et veuve depuis
soixante-neuf ans. M. de Bouteville était de la maison de Montmorency,
petit-fils d'un puîné du baron de Fosseux.

M. de Luxembourg naquit posthume six mois après la mort de son père\,;
il était fils unique, cadet de deux sœurs M\textsuperscript{me} de
Valencey, l'aînée, morte en 1684, n'a fait aucune figure par elle ni par
les siens\,; la cadette, belle, spirituelle et fort galante, peut-être
encore plus intrigante, a toute sa vie fait beaucoup de bruit dans le
monde dans ses trois états de fille, de duchesse de Châtillon, enfin de
duchesse de Meckelbourg\,; {[}elle{]} contribua fort à la fortune de son
frère avec qui elle fut toujours intimement unie, et mourut à Paris,
vingt jours après lui, et de la même maladie, ayant un an plus que lui,
et sans enfants.

Un grand nom, qui, dans les commencements de la vie du jeune Bouteville,
brillait encore de la mémoire de cette branche illustre des derniers
connétables et de l'amour que la princesse douairière de Condé portait à
son nom, beaucoup de valeur, une ambition que rien ne contraignit, de
l'esprit, mais un esprit d'intrigue, de débauche et du grand monde, lui
fit surmonter le désagrément d'une figure d'abord fort rebutante\,; mais
ce qui ne se peut comprendre de qui ne l'a point vu, une figure à
laquelle on s'accoutumait, et qui, malgré une bosse médiocre par devant,
mais très grosse et fort pointue par derrière, avec tout le reste de
l'accompagnement ordinaire des bossus, avait un feu, une noblesse et des
grâces naturelles, et qui brillaient dans ses plus simples actions. Il
s'attacha, dès en entrant dans le monde, à M. le Prince, et bientôt
après, M. le Prince s'attacha à sa sœur. Le frère, aussi peu scrupuleux
qu'elle, s'en fit un degré de fortune pour tous les deux. M. le Prince
se hâta de procurer son mariage avec le fils du maréchal de Châtillon,
jeune homme de grande espérance qui lui était fort attaché, avant que
cet amour fût bien découvert, et lui procura un brevet de duc en 1646.

Le cardinal Mazarin avait renouvelé cette sorte de dignité qui n'a que
des honneurs sans rang et sans successions, connue sous François Ier et
sous ses successeurs, mais depuis quelque temps tombée en désuétude, et
qui parut propre au premier ministre à retenir et à récompenser des gens
considérables ou qu'il voulait s'attacher\,; c'est de ceux-là qu'il
disait, « qu'il en ferait tant qu'il serait honteux de ne l'être pas, et
honteux de l'être\,;\,» et à la fin il se le fit lui-même, pour donner
plus de désir de ces brevets.

M. de Châtillon n'en jouit que trois ans, bon et paisible mari, et
toutefois fort à la mode. M. le Prince dominait la cour et le cardinal
Mazarin qu'il s'était attaché par sa réputation et ses services\,; ce
qui ne dura pas longtemps. Il assiégeait Paris, pour la cour qui en
était sortie, contre le parlement et les mécontents en 1649, lorsque le
duc de Châtillon fut tué à l'attaque du pont de Charenton et enterré à
Saint-Denis. L'amant et l'amante s'en consolèrent. La grandeur du
service que M. le Prince avait rendu au cardinal Mazarin en le ramenant
triomphant dans Paris, pesa bientôt par trop à l'un par la fierté et les
prétentions absolues de l'autre, d'où naquit la prison des princes,
pendant laquelle la princesse douairière de Condé se retira à
Châtillon-sur-Loire avec la fidèle amante de son fils, et y mourut. De
la délivrance forcée des princes aux désordres, puis à la guerre civile
qu'entreprit M. le Prince, il n'y eut presque pas d'intervalle. La
bataille du faubourg Saint-Antoine la finit, et jeta M. le Prince entre
les bras des Espagnols jusqu'à la paix des Pyrénées.

Bouteville le suivit partout. Sa valeur et ses mœurs, son, activité,
tout en lui était fait pour plaire au prince, et toutes sortes de
liaisons fortifiaient la leur. À ce retour en France,
M\textsuperscript{me} de Châtillon reprit son empire. Son frère avait
trente-trois ans. Il avait acquis de la réputation à la guerre\,; il
était devenu officier général, et avait auprès de M. le Prince le mérite
d'avoir suivi sa fortune jusqu'au bout\,; {[}ce{]} qu'il partageait avec
fort peu de gens de sa volée. Ils cherchèrent donc une récompense qui
fît honneur à M. le Prince, et une fortune à Bouteville, et ils
dénichèrent ce mariage du second lit de l'héritière de Piney avec M. de
Clermont. Elle était laide affreusement et de taille et de visage\,;
c'était une grosse vilaine harengère dans son tonneau, mais elle était
fort riche par le défaut des enfants du premier lit, dont l'état parut à
M. le Prince un chausse-pied pour faire Bouteville duc et pair. Il crut
d'abord se devoir assurer de la religieuse. Elle avait souvent murmuré
contre ses vœux. Il craignit qu'un grand mariage de sa sœur du second
lit ne la portât à un, éclat embarrassant. Il la fut trouver à sa
grille, et moyennant une dispense du pape dont il se chargea pour la
défroquer, et un tabouret de grâce ensuite, elle consentit â tout,
demeura dans ses vœux et signa tout ce qu'on voulut. Rien ne convenait
mieux au projet que de la lier de nouveau à ses vœux, et ce tabouret de
grâce devenait un échelon pour la dignité en faveur du mariage de la
sœur. Le pape accorda la dispense de bonne grâce, et la cour le tabouret
de grâce, sous le prétexte qu'étant fille du premier lit, elle aurait
succédé, au duché de Piney, à son frère sans alliance, si elle n'avait
pas été religieuse professe. On la fit dame du palais de la reine, sous
le nom de princesse de Tingry, avec une petite marque à sa coiffure du
chapitre de Poussay, dont elle se défit bientôt. À l'égard du frère, on
joua la comédie de lever son interdiction, de le tirer de Saint-Lazare,
et tout de suite de lui faire faire une donation à M. de Bouteville, par
son contrat de mariage, de tous ses biens, et une cession de sa dignité,
en considération des grandes sommes qu'il avait reçues pour cela de M.
de Bouteville, et qu'il lui avait payées. Cette clause est fort
importante au procès dont il s'agit. Aussitôt il assista au mariage de
sa sœur, et dès qu'il fut célébré, on le fit interdire de nouveau, et on
le remit à Saint-Lazare, dont il n'est pas sorti depuis.

Le mariage fait, 17 mars 1661, M. de Bouteville mit l'écu de Luxembourg
sur le tout du sien, et signa Montmorency-Luxembourg, ce que tous ses
enfants et les leurs ont toujours fait aussi. Incontinent après il
entama le procès de sa prétention pour la dignité de duc et pair de
Piney, et M. le Prince s'en servit pour lui obtenir des lettres
nouvelles d'érection de Piney en sa faveur, dans lesquelles on fit
adroitement couler la clause \emph{en tant que besoin serait}, pour lui
laisser entière sa prétention de l'ancienneté de la première création de
1581. Avec ces lettres, il fut reçu duc et pair au parlement, 22 mai
1662, et y prit le dernier rang après tous les autres pairs.

Le reste de la vie de M. de Luxembourg est assez connu. Il se trouva
enveloppé dans les affaires de la Voisin, cette devineresse, et pis
encore, accusée de poison, qui, par arrêt du parlement, fut brûlée à la
Grève {[}le 22 février 1680{]}, et qui fit sortir la comtesse de
Soissons du royaume pour la dernière fois, et la duchesse de Bouillon,
sa sœur. On reproche à M. de Luxembourg d'avoir oublié en cette occasion
une dignité qu'il avait tant ambitionnée. Il répondit sur la sellette
comme un particulier, et ne réclama aucun des privilèges de la pairie.
Il fut longtemps à la Bastille, et y laissa de sa réputation.

On crut longtemps qu'il avait perdu toute pensée de dispute avec les
ducs ses anciens. Il y avait encore alors des cérémonies où ils
paraissaient, il s'en absentait toujours\,; et, à la vie, ou occupée de
guerre ou libertine, qu'il mena jusqu'à la fin de sa vie, on n'y prenait
pas garde, lorsqu'à la promotion du Saint-Esprit de 1688 il demanda et
obtint des recevoir l'ordre, sans conséquence, parmi les maréchaux de
France, pour ne pas préjudicier à sa prétention de préséance. Ce fut,
pour le dire en passant, la première fois que les maréchaux de France à
recevoir dans l'ordre y précédèrent les gentilshommes de même promotion,
et à cette démarche de M. de Luxembourg on vit qu'il n'avait pas
abandonné la pensée de sa prétention.

Une grande guerre qui s'ouvrit alors de la France contre toute l'Europe
fit espérer à ce maréchal qu'on aurait besoin de lui, et qu'il y
pourrait trouver de ces moments heureux d'acquérir de la gloire et, avec
elle, le crédit d'emporter sa préséance. En effet, le maréchal
d'Humières, créature de M. de Louvois, ayant mal réussi en Flandre dès
la première campagne, M. de Luxembourg lui fut substitué par ce ministre
tout-puissant, qui, pour son intérêt particulier, avait engagé la guerre
et qui voulait y réussir, et qui fit céder à ce grand intérêt son peu
d'affection pour ce nouveau général, qui ne compta ses campagnes que par
des combats et souvent par des victoires. Ce fut donc après celles de
Leure, qui ne fut qu'un gros combat de cavalerie\,; de Fleurus, qui ne
fut suivie d'aucun fruit\,; de Steinkerque, où l'armée française pensa
être surprise et défaite, trompée par un espion du cabinet du général,
découvert, et à qui, le poignard sous la gorge, on fit écrire ce qu'on
voulut\,; et, enfin, après celle de Neerwinden, qui ne valut que
Charleroi, que M. de Luxembourg se crut assez fort pour entreprendre
tout de bon ce procès de préséance. L'intrigue, l'adresse, et, quand il
le fallait, la bassesse le servait bien. L'éclat de ses campagnes et son
état brillant de général de l'armée la plus proche et la plus nombreuse
lui avaient acquis un grand crédit. La cour était presque devenue la
sienne par tout ce qui s'y rassemblait autour de lui, et la ville,
éblouie du tourbillon et de son accueil ouvert et populaire, lui était
dévouée. Les personnages de tous états croyaient avoir à compter avec
lui, surtout depuis la mort de Louvois, et la bruyante jeunesse le
regardait comme son père, et le protecteur de leur débauche et de leur
conduite, dont la sienne à son âge ne s'éloignait pas. Il avait captivé
les troupes et les officiers généraux. Il était ami intime de M. le Duc,
et surtout de M. le prince de Conti, le Germanicus d'alors. Il s'était
initié dans le plus particulier de Monseigneur, et, enfin, il venait de
faire le mariage de son fils aîné avec la fille aînée du duc de
Chevreuse, qui, avec le duc de Beauvilliers, son beau-frère, et leurs
épouses, avaient alors le premier crédit et toutes les plus intimes
privances avec le roi et avec M\textsuperscript{me} de Maintenon.

Dans le parlement la brigue était faite. Harlay, premier président,
menait ce grand corps à baguette\,; il se l'était dévoué tellement qu'il
crut qu'entreprendre et réussir ne serait que même chose et que cette
grande affaire lui coûterait à peine le courant d'un hiver à emporter.
Le crédit de ce nouveau mariage venait de faire ériger, en faveur du
nouvel époux, la terre de Beaufort en duché, vérifié sous le nom de
Montmorency, et, à cette occasion, il ne manqua pas de persuader à tout
le parlement que le roi était pour lui dans sa prétention contre ses
anciens. Lorsque bientôt après il la recommença tout de bon, le premier
président, extrêmement bien à la cour, l'aida puissamment à cette
fourberie, de sorte que, lorsqu'on s'en fut aperçu, le plus grand remède
y devint inutile. Ce fut une lettre au premier président, de la part du
roi, écrite par Pontchartrain, contrôleur général des finances et
secrétaire d'État, par laquelle il lui mandait que le roi, surpris des
bruits qui s'étaient répandus dans le parlement qu'il favorisait la
cause de M. de Luxembourg, voulait que la compagnie sût, par lui, et
s'assurât entièrement que Sa Majesté était parfaitement neutre et la
demeurerait entre les parties dans tout le cours de l'affaire.

\hypertarget{chapitre-ix.}{%
\chapter{CHAPITRE IX.}\label{chapitre-ix.}}

1694

~

\relsize{-1}

{\textsc{Novion premier président.}} {\textsc{- Harlay premier
président.}} {\textsc{- Harlay, auteur de la légitimation des doubles
adultérins, sans nommer la mère\,; source de sa faveur.}} {\textsc{-
Causes de sa partialité pour M. de Luxembourg.}} {\textsc{- Situation
des deux partis.}} {\textsc{- Ducs de Chevreuse et de Bouillon en
prétentions et à part.}} {\textsc{- Talon président à mortier.}}
{\textsc{- Labriffe procureur général.}} {\textsc{- Mesures de déférence
de moi à M. de Luxembourg.}} {\textsc{- Sommaire de la question formant
le procès.}} {\textsc{- Opposants à M. de Luxembourg.}} {\textsc{-
Conduite inique en faveur de M. de Luxembourg.}} {\textsc{- Mes lettres
d'État.}} {\textsc{- Cavoye.}} {\textsc{- Mes ménagements pour M. de
Luxembourg mal reçus.}} \relsize{1}

~

Lors du mariage de M. de Luxembourg, et qui l'entreprit pour se faire un
chausse-pied à une érection nouvelle, M. le Prince avait obtenu des
lettres patentes de renvoi au parlement. M. Talon, lors avocat général
d'une grande réputation, y parla avec grande éloquence et une grande
capacité, et, après avoir traité la question à fond, avec toutes les
raisons de part et d'autre, avait conclu en plein contre M. de
Luxembourg. Ce fut aussi où il arrêta son affaire, eut son érection
nouvelle et attendit sa belle. Il crut l'avoir trouvée quelques années
après\,: Novion, premier président, était Potier comme le duc de
Gesvres\,; l'intérêt de son cousin, qu'on a vu dans la généalogie
ci-dessus, l'avait mis dans celui de M. de Luxembourg\,; ils crurent
pouvoir profiter de l'état, prêt à être jugé, où le procès en était
demeuré, et résolurent de l'étrangler à l'improviste, et peut-être en
seraient-ils venus à bout sans le plus grand hasard du monde. À une
audience ouvrante de sept heures du matin, destinée à rendre une
sommaire justice au peuple, aux artisans et aux petites affaires qui
n'ont qu'un mot, l'intendant de mon père et celui de M. La
Rochefoucauld, qui se trouvèrent sans penser à rien moins qu'à ce procès
de préséance, en entendirent appeler la cause et tout aussitôt un avocat
parler pour M. de Luxembourg. Ils s'écrièrent, s'opposèrent,
représentèrent l'excès d'une telle surprise et en arrêtèrent si bien le
coup que, manqué par là, et les mesures rompues par ce singulier
contre-temps, M. de Luxembourg demeura court et laissa de nouveau dormir
son affaire jusqu'au temps dont il s'agit ici.

Ce M. de Novion fut surpris en quantité d'iniquités criantes, et souvent
à prononcer à l'audience, à l'étonnement des deux côtés. Chacun croyait
que l'autre avait fait l'arrêt et ne le pouvait comprendre, tant qu'à la
fin ils se parlèrent au sortir de l'audience et découvrirent que ces
arrêts étaient du seul premier président. Il en fit tant que le roi
résolut enfin de le chasser. Novion tint ferme, en homme qui a toute
honte bue et qui se prend à la forme, qui rendait son expulsion
difficile\,; mais on le menaça enfin de tout ce qu'il méritait\,; on lui
montra une charge de président à mortier pour son petit-fils, car son
fils était mort de bonne heure, et il prit enfin son parti de se
retirer. Harlay, procureur général, lui succéda, et Labriffe, simple
maître des requêtes, mais d'une brillante réputation, passa à
l'importante charge de procureur général.

Harlay était fils d'un autre procureur général du parlement et d'une
Bellièvre, duquel le grand-père fut ce fameux Achille d'Harlay, premier
président du parlement après ce célèbre Christophe de Thou son
beau-père, lequel était père de ce fameux historien. Issu de ces grands
magistrats, Harlay en eut toute la gravité qu'il outra en cynique\,; en
affecta le désintéressement et la modestie, qu'il déshonora l'une par sa
conduite, l'autre par un orgueil raffiné, mais extrême, et qui, malgré
lui, sautait aux yeux. Il se piqua surtout de probité et de justice,
dont le masque tomba bientôt. Entre Pierre et Jacques il conservait la
plus exacte droiture\,; mais, dès qu'il apercevait un intérêt ou une
faveur à m nager, tout aussitôt il était vendu. La suite de ces Mémoires
en pourra fournir des exemples\,; en attendant, ce procès-ci le
manifesta à découvert.

Il était savant en droit public, il possédait fort le fond des diverses
jurisprudences, il égalait les plus versés aux belles-lettres, il
connaissait bien l'histoire, et savait surtout gouverner sa compagnie
avec une autorité qui ne souffrait point de réplique, et que nul autre
premier président n'atteignit jamais avant lui. Une austérité
pharisaïque le rendait redoutable par la licence qu'il donnait à ses
répréhensions publiques, et aux parties, et aux avocats, et aux
magistrats, en sorte qu'il n'y avait personne qui ne tremblât d'avoir
affaire à lui. D'ailleurs, soutenu en tout par la cour, dont il était
l'esclave, et le très humble serviteur de ce qui y était en vraie
faveur, fin courtisan, singulièrement rusé politique, tous ces talents,
il les tournait uniquement à son ambition de dominer et de parvenir, et
de se faire une réputation de grand homme. D'ailleurs sans honneur
effectif, sans mœurs dans le secret, sans probité qu'extérieure, sans
humanité même, en un mot, un hypocrite parfait, sans foi, sans loi, sans
Dieu et sans âme, cruel mari, père barbare, frère tyran, ami uniquement
de soi-même, méchant par nature, se plaisant à insulter, à outrager, à
accabler, et n'en ayant de sa vie perdu une occasion. On ferait un
volume de ses traits, et tous d'autant plus perçants qu'il avait
infiniment d'esprit, l'esprit naturellement porté à cela et toujours
maître de soi pour ne rien hasarder dont il pût avoir à se repentir.

Pour l'extérieur, un petit homme vigoureux et maigre, un visage en
losange, un nez grand et aquilin, des yeux beaux, parlants, perçants,
qui ne regardaient qu'à la dérobée, mais qui, fixés sur un client ou sur
un magistrat, étaient pour le faire rentrer en terre\,; un habit peu
ample, un rabat presque d'ecclésiastique et des manchettes plates comme
eux, une perruque fort brune et fort mêlée de blanc, touffue, mais
courte, avec une grande calotte par-dessus. Il se tenait et marchait un
peu courbé, avec un faux air plus humble que modeste, et rasait toujours
les murailles pour se faire faire place avec plus de bruit, et
n'avançait qu'à force de révérences respectueuses et comme honteuses à
droite et à gauche, à Versailles.

Il y tenait au roi et à M\textsuperscript{me} de Maintenon par l'endroit
sensible, et c'était lui qui, consulté sur la légitimation inouïe
d'enfants sans nommer la mère, avait donné la planche du chevalier de
Longueville, qui fut mise en avant, sur le succès duquel ceux du roi
passèrent. Il eut dès lors parole de l'office de chancelier de France,
et toute la confiance du roi, de ses enfants et de leur toute-puissante
gouvernante, qu'il sut bien se conserver et s'en ménager de continuelles
privances.

Il était parent et ami du maréchal de Villeroy, qui s'était attaché à M.
de Luxembourg, et ami intime du maréchal de Noailles. La jalousie des
deux frères de Duras, capitaines des gardes, avait uni les deux autres
capitaines des gardes ensemble, tellement que Noailles, pour cette
raison, et Villeroy, par son intérêt d'être lié à M. de Luxembourg,
disposaient, en sa faveur, du premier président. M. de Chevreuse avait
toujours eu dans la tête l'ancien rang de Chevreuse, et c'était
peut-être pour cela que M. de La Rochefoucauld s'était roidi, à leur
commune promotion dans l'ordre en 1688, à ne vouloir lui céder, comme
duc de Luynes, qu'après sa réception au parlement en cette qualité, pour
avoir un titre public qui n'avait cédé qu'à l'ancienneté de Luynes, et
ne s'était pas voulu contenter de la simple cession du duc de Luynes,
parce que cet acte particulier de famille pouvait aisément ne se pas
représenter dans la suite. Cette idée, que M. de Chevreuse avait lors et
qu'il a toujours sourdement conservée, jointe au mariage de sa fille
aînée avec le fils aîné de M. de Luxembourg, l'égara de son intérêt de
duc de Luynes commun avec le nôtre, et l'unit à celui de M. de
Luxembourg, et, avec lui, M. de Beauvilliers, qui tous deux n'étaient
qu'un même cœur et un même esprit. Dirait-on, de personnages d'une vertu
si pure et toujours si soutenue, que l'humanité, qui se fourre partout,
avait mis, entre eux et M. de La Rochefoucauld, une petite séparation
qui ne contribua pas à leur faire trouver bonne la cause qu'il
soutenait\,? Ce dernier, au plus haut point de faveur, mais destitué de
confiance et naturellement jaloux de tout, ne pouvait souffrir que l'une
et l'autre, de la part du roi, fussent réunies dans les deux
beaux-frères. Leur vie, leur caractère, leurs occupations, leurs
liaisons et les siennes, tout était entièrement ou opposé, ou pour le
moins très différent.

Entre ces deux sortes de faveurs, le premier président ne balança pas à
trouver celle des beaux-frères préférable. Il y joignit celles de
Noailles et de Villeroy, qui étaient grandes aussi, et tout l'éclat dont
brillait M. de Luxembourg. De tous ceux qu'il attaquait, aucun n'était
en faveur que le seul duc de La Rochefoucauld\,; les mieux avec le roi,
ce n'était que distinction à quelques-uns, et considération pour la
plupart\,; ainsi, le choix du premier président ne fut pas difficile.

Talon, devenu président à mortier, flatté de voir M. de Luxembourg
réclamer les parents de sa mère, oublia qu'il avait été avocat
général\,; il ne craignit point le blâme d'être contraire à soi-même, et
après avoir parlé autrefois avec tant de force dans la même affaire
contre M. de Luxembourg, comme avocat général, on le vit devenir le
sien, et travailler à ses factums. Il fouilla les bibliothèques,
rassembla les matériaux, présida tout ce qui se fit, écrivit pour M. de
Luxembourg, à visage découvert, et rien ne s'y fit que par lui.

Le célèbre Racine, si connu par ses pièces de théâtre, et par la
commission où il était employé lors pour écrire l'histoire du roi, prêta
sa belle plume pour polir les factums de M. de Luxembourg, et en réparer
la sécheresse de la matière par un style agréable et orné, pour les
faire lire avec plaisir et avec partialité aux femmes et aux courtisans.
Il avait été attaché à M. de Seignelay, était ami intime de Cavoye, et
tous deux l'avaient été de M. de Luxembourg, et Cavoye l'était encore.
En un mot, les dames, les jeunes gens, tout le bel air de la cour et de
la ville était pour lui, et personne parmi nous à pouvoir
contre-balancer ce grand air du monde, ni même y faire aucun partage.
Que si on ajoute le soin de longue main pris, de captiver les principaux
du parlement, et toute la grand'chambre par parents, amis, maîtresses,
confesseurs, valets, promesses, services, il se trouvera qu'avec un
premier président tel que Harlay à la tête de ce parti, nous avions
affaire à incomparablement plus forts que nous.

Un inconvénient encore, qui n'était pas médiocre, fut la lutte d'une
communauté de gens en même intérêt contre un seul qui conduisait le sien
avec indépendance et qui n'avait besoin d'aucun concert. Le nôtre
subsista pourtant fort au-dessus de ce qui se pouvait attendre d'une si
grande diversité d'esprits et d'humeurs, dans une parité de dignité et
d'intérêt. M. de Bouillon, avec la chimère de l'ancien rang d'ancienneté
d'Albret et de Château-Thierry, imita le duc de Chevreuse, et dès le
premier commencement de l'affaire. Mais celui-ci se contenta de n'y
prendre aucune part. Telle était notre situation, lorsque M. de
Luxembourg l'entama.

Le premier pas fut de faire donner des conclusions au procureur général.
Labriffe, maître des requêtes, si brillant, se trouvait accablé du poids
de cette grande charge, et n'y fut pas longtemps sans perdre la
réputation qui l'y avait placé. Accoutumé à être l'aigle du conseil,
Harlay en prit jalousie, et prit à tâche de le contrecarrer\,; l'autre,
plein de ce qui l'avait si rapidement porté, voulut lutter d'égal, et ne
tarda pas à s'en repentir. Il tomba dans mille panneaux que l'autre lui
tendait tous les jours, et dont il le relevait avec un air de
supériorité qui désarçonna l'autre. Il sentit son faible à l'égard du
premier président en tout genre\,; il se lassa des camouflets que
l'autre ne lui épargnait point, et peu à peu il devint soumis et
rampant. C'était sa situation lorsqu'il fut question de ses conclusions.
Tout abattu qu'il était, il ne manquait point d'esprit\,; mais la
crainte et la défiance avaient pris le dessus. Il sentit où penchait le
premier président, et il n'osa le choquer, de sorte que M. de Luxembourg
eut ses conclusions comme et quand il les voulut. Nos productions
n'étaient pas faites, rien n'était donc en état, et Labriffe avait
promis aux ducs de La Trémoille et de La Rochefoucauld de les attendre,
comme il était de règle et de droit, lorsque M. de Luxembourg, qui les
regardait comme un premier coup de partie, se vanta de les avoir
favorables et en effet les fit voir.

C'était un autre pas de clerc puisqu'elles devaient être remises au
greffe cachetées, et que personne ne devait savoir quoi que ce soit de
ce qu'elles contenaient. M. de Chaulnes voulut au moins s'en venger. Dès
que notre premier factum fut imprimé, il le porta à Labriffe et lui dit
que c'était sans intérêt, puisque tout le monde savait ses conclusions
données, et en faveur de M. de Luxembourg\,; mais que notre procès ne
pouvant être que curieux en soi et célèbre au parlement, il avait voulu
lui apporter notre premier mémoire tout mouillé encore de l'impression,
dans la lecture duquel il croyait qu'il ne serait pas fâché de se
délasser en ses heures perdues, et dans lequel il apprendrait des faits,
et beaucoup de choses très importantes pour l'intelligence et la
décision de l'affaire, très nettement exposés, et dont aucun n'avait
encore paru. La gravité et la réputation de M. de Chaulnes ajouta
beaucoup au poids de cette raillerie, qui embarrassa extrêmement le
procureur général\,; il voulut se jeter dans les excuses\,; mais M. de
Chaulnes, qui souriait de le voir balbutiant, l'assura que ce n'était
pas à lui qu'il les fallait faire, mais à MM. de La Trémoille et de La
Rochefoucauld qui, à ce qu'il s'était laissé dire, n'étaient pas tout à
fait tant ses serviteurs que lui.

J'ai mis le procureur général et ses conclusions ainsi données en
écolier, à la suite de ce que j'ai cru devoir faire connaître du premier
président, de M. Talon, et de tout ce qui se ralliait pour M. de
Luxembourg, afin de montrer une fois pour toutes à qui nous eûmes
affaire, et l'inégalité de la partie en même temps. Avant d'aller plus
loin, il faut dire comment j'y entrai et comment je m'en démêlai.

On peut juger qu'à mon âge, et fils d'un père de la cour du feu roi, et
d'une mère qui n'avait connu que les devoirs domestiques, et sans aucuns
proches, je n'étais en aucun commerce avec pas un de ceux que M. de
Luxembourg attaquait. Eux qui se voulaient réunir le plus en nombre
qu'ils pourraient, comptant peu sur de certains ducs, et désertés par
MM. de Chevreuse et de Bouillon, n'en voulurent négliger aucun, parce
que chacun a ses amis et sa bourse, pour les frais qui se faisaient en
commun. M. de La Trémoille m'aborda donc chez le roi et me dit que lui
et plusieurs autres qu'il me nomma étaient attaqués par M, de Luxembourg
en préséance, par la reprise d'un ancien procès, où mon père avait été
partie avec eux, qu'ils espéraient que je ne les abandonnerais pas dans
cette affaire, quoique M. de Luxembourg fût mon général\,; qu'ils
l'avaient chargé de m'en parler, et ajouta du sien les compliments
convenables. C'était dans tous les premiers commencements de cette
reprise, et assez peu depuis mon retour de l'armée. J'ignorais donc
parfaitement l'affaire, mais mon parti fut bientôt pris. Je remerciai M.
de La Trémoille, tant pour lui que pour ces messieurs, de ce qu'ils
avaient pensé à moi, et je lui dis que je ne craindrais jamais de
m'égarer en si bonne compagnie, en suivant l'exemple de mon père, et que
je le priais d'être persuadé et de les assurer que rien ne me séparerait
d'eux. M. de La Trémoille me parut fort content, et dans la journée M.
de La Rochefoucauld me chercha et plusieurs des autres, et m'en firent
mille compliments.

Enrôlé de la sorte, je crus devoir toutes sortes de ménagements à un
homme tel qu'était lors M. de Luxembourg, sous qui j'avais fait la
campagne, qui m'avait bien traité, quoique sans être connu de lui que
par ce que j'étais, et sous qui je pouvais servir souvent. J'allai donc
le lendemain chez lui, Où il n'était pas, et je le fus trouver chez le
duc de Montmorency\,; le marquis d'Harcourt et Albergotti étaient avec
eux. Je fis là mon compliment à M. de Luxembourg, et lui demandai la
permission de ne me pas séparer de ceux des ducs sur lesquels il
demandait la préséance\,; que, de toute autre affaire, je l'en
laisserais absolument le maître\,; que sur celle-là même si je n'avais
voulu faire aucun pas sans savoir s'il le trouverait bon, et j'ajoutai
tout ce que l'âge et l'état exigeaient d'un jeune homme. Cela fut reçu
avec toute la politesse et la galanterie imaginables\,; la compagnie y
applaudit, et M. de Luxembourg m'assura que je ne pouvais moins faire
que suivre l'exemple de mon père, et qu'il ne m'en marquerait pas moins,
etc., en toutes occasions. Ce devoir rempli, je ne songeai plus qu'à
bien soutenir l'affaire commune conjointement avec les autres, sans rien
faire qui pût raisonnablement déplaire à M. de Luxembourg. Maintenant
voici le sommaire du procès, car d'entrer dans le détail des lois, des
exemples, des défenses de part et d'autre, ce serait la matière de
volumes entiers, et il s'en trouve plusieurs faits de part et d'autre
qui en instruiront suffisamment et à fond les curieux.

M. de Luxembourg prétendait que l'effet des érections femelles allait à
l'infini\,; que M\textsuperscript{me} de Tingry, quoique dans le monde
demeurant sous ses vœux, et son frère ayant cédé sa dignité et ses biens
à sa sœur du second lit, par son contrat de mariage, lui diacre et par
conséquent hors d'état de se pouvoir marier, cette fille du second lit
qu'il avait épousée passait aux droits des enfants du premier lit qui se
trouvaient épuisés, et de plein droit le faisait duc et pair de la date
de la première érection\,; que la clause en \emph{tant que besoin
serait}, apposée aux lettres nouvelles qu'il avait obtenues aussitôt
après son mariage, annulait toute la force que cette nouvelle érection
pouvait donner contre lui, et que ce qui achevait de l'anéantir était ce
qu'il avait plu au roi de déclarer par ses lettres patentes en 1676,
qu'il n'a point entendu ériger de nouveau Piney en duché-pairie en 1661,
mais bien le renouveler en faveur de M. de Luxembourg, d'où il concluait
qu'il était par là manifeste que son ancienneté remontait à la première
érection de 1581.

Les opposants prétendaient au contraire qu'aucune érection femelle
n'était infinie\,; que son effet était borné à la première fille qui le
recueillait, et que si elles avaient quelquefois passé à une seconde
fille, ç'avait été tout, jamais au delà, et encore par grâce et à la
faveur de nouvelles lettres en continuation de pairie, avec rang du jour
de ces nouvelles lettres\,; qu'ainsi l'ancienne érection de Piney était
éteinte dans le sang du premier mari de la duchesse héritière. Ce qui
était si vrai qu'elle avait perdu son rang et ses honneurs de duchesse
en se remariant, bien loin qu'elle les eût communiqués à son second
mari, tant la dignité demeurait fixée et immuable dans son fils du
premier lit\,; que, pour la démission qu'il en avait faite ainsi que de
ses biens à sa sœur du second lit par son contrat de mariage, cette
démission avait deux vices qui la rendaient absurde et nulle, et un
troisième qui la faisait impossible\,: le premier, son état d'interdit
devant et après, qui, n'ayant été levé que pour le moment nécessaire de
cette démission, n'était qu'une dérision de la. Justice qui ne pouvait
avoir d'effet ni être reçue sérieusement\,; 2° que les grandes sommes
données à cet interdit pair le futur époux de sa sœur du second lit,
motivées dans son contrat de mariage, comme cause de cette démission,
l'annulaient par cela même, puisqu'on ne peut devenir duc et pair que
par deux voies, érection en sa faveur, ou succession, et que l'acquéreur
en est formellement exclu\,; 3° que la volonté de l'interdit, quand bien
même il ne l'eût jamais été, et qu'il n'eût rien reçu pour sa démission,
était entièrement insuffisante pour faire un duc et pair en se
démettant, puisqu'une démission ne pouvait opérer cet effet que par deux
choses réunies, un sujet naturellement héritier de la dignité à qui la
démission ne fait qu'en avancer la succession, et la permission du roi
de la faire, qui toutes deux manquaient totalement en celle-ci. Que la
clause en \emph{tant que besoin serait}, glissée dans les nouvelles
lettres d'érection de 1661, accordées à M. de Luxembourg, ne lui donnait
aucun droit\,; ce qui était évident, puisqu'il avait obtenu ces
nouvelles lettres et pris le dernier rang en conséquence, sans quoi il
n'eût point été duc et pair, et que de plus cette clause, n'ayant point
été communiquée, n'avait pu être contredite, ni faire aucun effet entre
les parties. Enfin, sur les lettres de 1676, par lesquelles le roi
déclarait n'avoir point fait d'érection nouvelle en 1661, mais renouvelé
l'érection de Piney en faveur de M. de Luxembourg, deux réponses\,: la
première que c'était pour la première fois qu'on en entendait parler (et
en effet M. de La Rochefoucauld en ayant témoigné au roi sa surprise, il
lui répondit qu'il ne se souvenait pas d'avoir jamais donné ces lettres,
à quoi M. de La Rochefoucauld, en colère, répliqua que c'étaient là des
tours de passe-passe de M. de Louvois, qui en ce temps-là était fort ami
de M. de Luxembourg)\,; que ces lettres, qui n'étaient point
enregistrées, étaient surannées, et partant de nul effet\,; que
d'ailleurs n'ayant jamais été connues jusqu'alors, elles ne pouvaient
passer pour contradictoires et pour juger, sans entendre les parties, un
procès pendant entre elles, et un procès de telle qualité et entre de
telles parties sous la cheminée, et demeurer incognito vingt ans ainsi
dans la poche de M. de Luxembourg. Deuxièmement enfin, qu'à toute
rigueur l'expression de renouveler n'emportait point le rang d'ancienne
érection, puisqu'en effet un ancien duché-pairie, autrefois érigé pour
une maison, et depuis érigé pour une autre, n'était à l'égard de cette
terre qu'un véritable renouvellement. Telles furent les raisons
fondamentales de part et d'autre sur lesquelles on comprend que les
avocats trouvèrent de quoi exercer leur éloquence d'une part, leurs
subtilités de l'autre\,; mais ce qui vient d'être exposé suffit pour
expliquer toute la matière en gros sur laquelle roula tout ce procès.

Disons un mot des opposants, desquels il faut ôter MM. de Chevreuse et
de Bouillon, par les raisons qui en ont été rapportées. M. d'Elbœuf ne
fit que nombre et ne se mêla jamais de rien, sinon de demeurer uni aux
autres. M. de Ventadour parut quelquefois aux assemblées, fit à peu près
ce qu'on désira de lui, mais au payement près, il ne menait pas une vie
à le mettre en œuvre. M. de Vendôme se présenta et fit bien, mais à sa
manière et ne pouvant se contraindre à rien. M. de Lesdiguières était un
enfant, et sa mère une espèce de fée, sur qui son cousin de Villeroy
avait tout crédit\,; ainsi ce fut beaucoup pour elle que de laisser le
nom de son fils, dont elle était tutrice, parmi ceux des opposants. M.
de Brissac obscur, ruiné et d'une vie étrange, ne sortait plus de son
château de Brissac, et ne fit que laisser son nom parmi les autres. M.
de Sully peu assidûment, mais fermement. MM. de Chaumes, de Richelieu,
de La Rochefoucauld et de La Trémoille, étaient ceux sur qui tout
portait, auquel le bonhomme M. de La Force se joignit dignement tant
qu'il put, et M. de Rohan aussi\,; mais M. de Richelieu et lui étaient
gens à boutade qui ne donnèrent pas peu d'affaires aux autres. M. de
Monaco y était ardent, sauf ses parties et sa bourse, encore payait-il
bien en rognonnant\,; mais c'était des farces pour tirer le contingent
du duc de Rohan.

Les intendants de MM. de La Trémoille et de La Rochefoucauld, nommés
Magneux et Aubry, gens d'honneur, capables, laborieux, et infiniment
touchés de cette affaire, en étaient les principaux directeurs, et
Riparfonds, avocat célèbre consultant, était le chef de nos avocats et
de notre conseil, chez qui se tenaient toutes nos assemblées toujours
une après-dînée de chaque semaine et quelquefois plus souvent, où M. de
La Rochefoucauld ne manquait jamais quoiqu'il ne couchât presque jamais
à Paris, et qui y rendit par son exemple les autres très assidus et fort
ponctuels à l'heure\,; les plus ardents et les plus continuellement à
tout étaient MM. de La Trémoille, de Chaulnes, La Rochefoucauld et La
Force, M. de Monaco autant qu'il était en lui, et plus qu'aucuns MM. de
Riche lieu et de Rohan, mais comme il a été dit, pleins de boutades et
de fantaisies.

Je me rendis assidu aux assemblées, je m'instruisis et de l'affaire en
soi, et de ce qui se passait par rapport à elle\,; ce que je hasardai de
dire dans les assemblées n'y déplut point. Riparfonds et les deux
intendants conducteurs me prirent en amitié\,; je plus aux ducs. M. de
La Rochefoucauld, tout farouche qu'il était, et par son nom et le mien
peu disposé pour moi, s'apprivoisa tout à fait avec moi\,; l'intimité de
M. de Chaulnes avec mon père se renouvela avec moi, ainsi que l'amitié
qu'il avait eue avec le bonhomme La Force\,; je fis une amitié intime
avec M. de La Trémoille, et je n'oserais dire que j'acquis une sorte
d'autorité sur M. de Richelieu, qui avait été aussi fort ami de mon
père, et sur le duc de Rohan, qui fut plus d'une fois salutaire et à la
cause que nous soutenions et à eux-mêmes. Chacun opinait là en son
rang\,; on ne s'interrompait point, on n'y perdait pas un instant en
compliments ni en nouvelles, et personne ne s'impatientait de la
longueur des séances, qui étaient souvent fort prolongées, pas même M.
de La Rochefoucauld qui retournait toujours au coucher du roi, à
Versailles, et chacun se piqua d'exactitude et d'assiduité.

Le procès commencé tout de bon, nous fîmes nos sollicitations ensemble,
couplés deux dans un carrosse, et nous ne fûmes pas longtemps sans nous
apercevoir de la mauvaise volonté du premier président qui, dans une
affaire qui, par sa nature et le droit, ne pouvait être jugée que par
l'assemblée de toutes les chambres, et les pairs, non parties, ajournés,
se hâta de nommer de petits commissaires pour être examinée chez lui et
s'en rendre plus aisément le maître\,; ce qui était contre toutes les
règles dans une affaire de cette qualité. Catinat, frère du maréchal,
Bochard de Saron, Maunourry et Portail rapporteur, furent les quatre
petits commissaires. Harlay fit bientôt pis\,: Bochard s'étant récusé
comme parent de la duchesse de Brissac-Verthamont, Joly de Fleury lui
fut substitué. Or Joly était beau-frère du président Talon, qui s'était
récusé comme parent de M. de Luxembourg et s'était, comme on l'a dit,
mis ouvertement à la tête de son conseil\,; et, outre que ces deux
hommes étaient si proches, ils étaient de plus amis intimes. Les choses
ainsi bien arrangées par le premier président, il voulut étrangler le
jugement et passa sur toutes sortes de formes pour exécuter promptement
ce dessein.

Tandis qu'on lui laissait faire ce qu'on ne pouvait empêcher, nous fûmes
avertis d'un nouveau factum de M. de Luxembourg, dont on avait tiré très
secrètement peu d'exemplaires\,; qu'il en avait fait aussitôt après
rompre les planches, et qu'il se distribuait sous le manteau aux petits
commissaires et à peu de conseillers sur lesquels il comptait le plus.
Ce factum contre toutes règles ne nous fut point signifié et par ce
défaut ne pouvait servir de pièce au procès\,; mais l'intérêt de nous le
cacher était capital de peur d'une réponse, et le conseil de M. de
Luxembourg comptait persuader ses juges par ces nouvelles raisons,
quoique non produites. Maunourry, l'un des petits commissaires, eut
horreur d'une supercherie qui n'allait à rien moins qu'à nous faire
perdre notre procès. Il prêta ce factum si secret à Magneux, intendant
du duc de La Trémoille, qui le fit copier en une nuit, et qui le
lendemain, qui était un mardi, fit assembler chez Riparfonds
extraordinairement. Là, ce factum fut lu. On y trouva quantité de faits
faux, plusieurs tronqués et un éblouissant tissu de sophismes. La
science de Talon et l'élégance et les grâces de Racine y étaient toutes
déployées. On jugea qu'il était capital d'y répondre\,; et, comme nous
devions être jugés le vendredi suivant, il fut arrêté de nous assembler
le lendemain mercredi matin chez Riparfonds, et de partir de là tous
ensemble pour aller demander au premier président délai jusqu'au lundi,
lui représenter l'importance dont il nous était de répondre à la
découverte que nous avions faite, et que, du mercredi où nous étions au
lundi suivant, ce n'était pas trop pour répondre, imprimer, et
distribuer notre mémoire\,; et pour faciliter cette justice, il fut
résolu de donner notre parole de ne rien faire qui pût retarder le
jugement au delà du lundi.

Le lendemain matin donc, nous nous trouvâmes chez Riparfonds, rue de la
Harpe\,: MM. de Guéméné ou Montbazon, La Trémoille, Chaulnes, Richelieu,
La Rochefoucauld, La Force, Monaco, Rohan et moi, d'où nous allâmes tous
et avec tous nos carrosses chez le premier président à l'heure de
l'audience, qu'il donnait toujours chez lui en revenant du palais. Nous
entrâmes dans sa cour, le portier dit qu'il y était et ouvrit la porte.
Ce fracas de carrosses fit apparemment regarder des fenêtres ce que
c'était, et comme nous nous attendions les uns les autres à être tous
entrés pour descendre de nos carrosses et monter ensemble le degré,
arriva un valet de chambre du premier président, aussi composé que son
maître, qui nous vint dire qu'il n'était pas chez lui et à qui nous ne
pûmes jamais faire dire où il était ni à quelle heure de la journée il
serait visible. Nous n'eûmes d'autre parti à prendre que de retourner
chez notre avocat et délibérer là de ce qui était à faire. Chacun y
exhala sa bile sur le parti pris de nous étrangler, et sur l'espèce
d'injure, d'une part, et de déni de justice, de l'autre, de nous avoir
renvoyés, comme le premier président, constamment chez lui, venait de
faire.

Dans cette situation on résolut de rompre ouvertement avec un homme qui
ne gardait aucune mesure, et de ne rougir de rien pour traîner en
longueur, tant qu'il nous serait possible, un procès où la partie était
manifestement faite et sûre de nous le faire perdre, et faite, par ce
que nous voyions, tout ouvertement. Pour l'exécuter il fut proposé de
former une demande au conseil, par M. de Richelieu qui avait toutes ses
causes commises au grand conseil, pour y faire renvoyer celle-ci\,; ce
qui formerait un procès de règlement de juges, au moyen duquel nous
aurions le temps de respirer et de trouver d'autres chicanes. Je dis
\emph{chicanes}, car ce procès ne pouvait, de nature et de droit, sortir
du parlement, ni être valablement jugé ailleurs. On applaudit à
l'expédient\,; mais, dès qu'on se mit à en examiner la mécanique, il se
trouva que le temps était trop court, jusqu'au surlendemain que nous
devions être jugés, pour qu'aucune requête de M. de Richelieu, tendante
à cet expédient, pût être introduite.

L'embarras devint grand, et notre affaire se regardait comme déplorée,
lorsqu'un des gens d'affaires, élevant la voix, demanda si personne de
nous n'avait de lettres d'État \footnote{Les \emph{lettres d'État}
  étaient accordées aux ambassadeurs, aux officiers de guerre et à tous
  ceux qui étaient obligés de s'absenter pour un service public. Elles
  suspendaient pour six mois toutes les poursuites dirigées contre eux.
  Ce délai expiré, elles pouvaient être reprises.}, chacun se regarda et
pas un d'eux n'en avait. Celui qui en avait fait la demande dit que
c'était pourtant le seul moyen de sauver l'affaire\,; il en expliqua la
mécanique, et nous fit voir que quand elles seraient cassées au premier
conseil de dépêches, comme on devait bien s'y attendre, la requête de M.
de Richelieu se trouverait cependant introduite, et l'instance liée au
conseil en règlement de juges. Sur cette explication je souris, et je
dis que s'il ne tenait qu'à cela, l'affaire était sauvée, que j'avais
des lettres d'État et que je les donnerais, à condition que je pourrais
compter qu'elles ne seraient cassées qu'au seul regard de M. de
Luxembourg. Là-dessus acclamations de ducs, d'avocats, de gens
d'affaires, compliments, embrassades, louanges, remerciements comme des
gens morts qu'on ressuscite, et MM. de La Trémoille et de La
Rochefoucauld se firent fort devant tous que mes lettres d'État ne
seraient cassées qu'au seul regard de M. de Luxembourg. Aucune dette
criarde n'avait fait quoi que ce soit à la mort de mon père. Pussort,
fameux conseiller d'État, d'Orieu et quelques autres magistrats très
riches, nos créanciers, avaient, voulu mettre le feu à mes affaires, qui
m'avaient fait prendre des lettres d'État pour me donner le temps de les
arranger. J'avais été fort irrité contre leurs procédés, mais je fus si
aise de me trouver par cela même celui qui sauvait notre préséance, que
je pense que je les leur pardonnai.

La chose pressait\,; je dis que ma mère avait ces lettres d'État et que
je m'en allais les chercher. J'éveillai ma mère à qui je dis assez
brusquement le fait. Elle, tout endormie, ne laissa pas de vouloir me
faire des remontrances sur ma situation et celle de M. de Luxembourg. Je
l'interrompis et lui dis que c'était chose d'honneur, indispensable,
promise, attendue sur-le-champ, et, sans attendre de réplique, pris la
clef du cabinet, et puis les lettres d'État, et cours encore. Ces
messieurs de l'assemblée eurent tant de peur que ma mère n'y voulût pas
consentir, que je ne fus pas parti qu'ils envoyèrent après moi MM. de La
Trémoille et de Richelieu pour m'aider à exorciser ma mère. Je tenais
déjà mes lettres d'État, comme on nous les annonça. Je les allai trouver
avec les excuses de ma mère qui n'était pas encore visible. Un
contre-temps qui nous arrêta un moment donna courage à ma mère de se
raviser. Comme nous étions sur le degré, elle me manda que, réflexion
faite, elle ne pouvait consentir que je donnasse mes lettres d'État
contre un homme tel qu'était lors M. de Luxembourg. J'envoyai promener
le messager, et je me hâtai de monter en carrosse avec les deux ducs qui
ne se trouvèrent pas moins soulagés que moi de me voir mes lettres
d'État à la main.

Ce contre-temps, le dirai-je à cause de sa singularité\,? M. de
Richelieu avait pris un lavement le matin, et sans le rendre vint de la
place Royale chez Riparfonds, de là chez le premier président avec nous,
et avec nous revint chez Riparfonds, y demeura avec nous à toutes les
discussions, enfin vint chez moi. Il est vrai qu'en y arrivant il
demanda ma garde-robe, et y monta en grande hâte\,; il y laissa une
opération telle que le bassin ne la put contenir, et ce fut ce temps-là
qui donna à ma mère celui de faire ses réflexions, et de m'envoyer
redemander mes lettres d'État. S'exposer à toutes ces courses et garder
un lavement un si long temps, il faut avoir vu cette confiance et ce
succès pour le croire.

En retournant chez Riparfonds, nous trouvâmes le duc de Rohan en chemin,
que ces messieurs, de plus en plus inquiets, envoyaient à notre secours.
Je lui montrai mon papier à la main, et il rebroussa après nous. Je ne
puis dire avec quelle satisfaction je rentrai à l'assemblée, ni avec
combien de louanges et de caresses j'y fus reçu. La pique était grande,
et n'avait pas moins gagné tout notre conseil que nous-mêmes. Ce fut
donc à qui de tous, ducs et conseil, me recevrait avec plus
d'applaudissements et de joie, et à mon âge j'en fus fort flatté. Il fut
conclu que le lendemain jeudi, veille du jour que nous devions être
jugés, mon intendant et mon procureur iraient à dix heures du soir
signifier mes lettres d'État au procureur de M. de Luxembourg et au
suisse de son hôtel, et que le même jour je m'en irais au village de
Longnes, à huit lieues de Paris, où était ma compagnie, pour colorer au
moins ces lettres d'État de quelque prétexte. Le soir je m'avisai que
j'avais oublié un grand bal que Monsieur donnait à Monseigneur au
Palais-Royal, le lendemain au soir jeudi, qui se devait ouvrir par un
branle, où je devais mener la fille de la duchesse de La Ferté qui ne me
le pardonnerait point si j'y manquais, et qui était une égueulée sans
aucun ménagement. J'allai conter cet embarras au duc de La Trémoille,
qui se chargea de faire trouver bon aux autres que je ne m'attirasse pas
cette colère, de manière que j'étais au bal tandis qu'on signifiait mes
lettres d'État.

Le vendredi matin je fus à l'assemblée où tous m'approuvèrent, excepté
M. de La Rochefoucauld, qui gronda et que j'apaisai par mon départ, et
qui se chargea de le dire au roi et sa cause.

En partant je crus devoir tout faire pour me conserver dans les mesures
où je m'étais mis avec M. de Luxembourg. J'écrivis donc dans cet esprit
une lettre ostensible à Cavoye, où je mis tout ce qui convenait à la
différence d'âge et d'emplois, sur la peine que j'avais de la nécessité
où je m'étais trouvé sur cette signification de lettres d'État. Cavoye
était le seul des amis les plus particuliers de M. de Luxembourg, qui
eût été fort de la connaissance de mon père. Sans esprit, mais avec une
belle figure, un grand usage du monde, et mis à la cour par une
maîtresse intrigante de mère qui y avait dans son médiocre état beaucoup
d'amis, il s'en était fait de considérables, mis très bien auprès du roi
et sur un pied de considération à se faire compter fort au-dessus de son
état de gentilhomme très simple, et de grand maréchal des logis de la
maison du roi. Il est aisé de comprendre avec combien de dépit M. de
Luxembourg vit tous ses projets déconcertés par ces lettres d'État. Il
courut au roi en faire, ses plaintes, et n'épargna aucun de nous dans
celles qu'il fit au public. Les lettres d'État furent cassées au premier
conseil des dépêches, comme nous nous y étions bien attendus\,; mais,
comme ces messieurs me l'avaient promis, elles ne le furent qu'à l'égard
de cette seule affaire. M. de Luxembourg en triompha, et compta qu'avec
ce vernis de plus, son procès allait finir tout court à son avantage. Il
employa tout le lendemain de ce succès à le remettre sur le bureau au
même point d'où il avait été suspendu\,; il remua tous ses amis et vit
tous ses juges. En effet, aussi bien secondé qu'il l'était parmi eux, on
fut en état de le juger le lendemain, lorsque, rentrant chez lui bien
tard et bien las de tant de courses, il y trouva la signification de M.
de Richelieu entre les mains de son suisse, que son intendant à peine
osa lui dire avoir aussi été faite à son procureur.

Ce coup porté, les opposants m'envoyèrent mon congé à Longnes où mon
exil n'avait duré que six jours. Je trouvai tout en feu\,: M. de
Luxembourg avait perdu toute mesure, et les ducs qu'il attaquait n'en
gardaient plus avec lui. La cour et la ville se partialisèrent, et
d'amis en amis personne ne demeura neutre ni prenant médiocrement parti.
J'eus à essuyer force questions sur mes lettres d'État. J'avais pour moi
raison, justice, nécessité et un parti ferme et bien organisé, et des
ducs mieux avec le roi que n'y était M. de Luxembourg. J'avais de plus
eu soin de mettre pour moi les procédés. Je les répandis, et comme je
sus que M. de Luxembourg et les siens s'étoient licenciés sur moi comme
sur la partie la plus faible, et de qui le coup qui les déconcertait
était parti, je ne me contraignis avec aucun d'eux.

Cavoye tout en arrivant me dit qu'il avait montré mon billet à M. de
Luxembourg, qu'il voulait bien pardonner à ma jeunesse une chicane
inouïe entre des gens comme nous, et qui en effet était un procédé fort
étrange. Une réponse si fière à mes honnêtetés si attentives me piqua.
Je répondis à Cavoye que je m'étonnais fort d'une réponse si peu
méritée, et que je n'avoir pas encore appris qu'entre gens comme nous,
il ne fût pas permis d'employer une juste défense contre une attaque
dont les moyens l'étaient si peu\,; que, content pour moi-même d'avoir
donné à tout ce qu'était M. de Luxembourg tout ce que mon âge lui
devait, je ne songerais plus qu'à donner aussi à ma préséance et à mon
union à mes confrères tout ce que je leur devais, sans m'arrêter plus à
des ménagements si mal reçus. J'ajoutai qu'il le pouvait dire à M. de
Luxembourg, et je quittai Cavoye sans lui laisser le loisir de la
repartie.

Le roi soupait alors, et je fis en sorte de m'approcher de sa chaise et
de conter cette courte conversation et ce qui y avait donné lieu à
Livry, parce qu'il était tout auprès du roi\,; ce que je ne fis que pour
en être entendu d'un bout à l'autre, comme je le fus en effet\,; et de
là je la répandis dans le monde. Les ducs opposants, et principalement
MM. de La Trémoille, de Chaulnes et de La Rochefoucauld, me remercièrent
de m'être expliqué de la sorte, et je dois à tous, et à ces trois encore
plus, cette justice, qu'ils me soutinrent en tout et partout et firent
leur affaire de la mienne avec une hauteur et un feu qui fit taire
beaucoup de gens, et qui par M. de La Rochefoucauld surtout me servit
fort bien auprès du roi. Au bout de quelques jours je m'aperçus que M.
de Luxembourg, lorsque je le rencontrais, ne me rendait pas même le
salut. Je le fis remarquer, et je cessai aussi de le saluer, en quoi, à
son âge et en ses places, il perdit plus que moi, et fournit par là aux
salles et aux galeries de Versailles un spectacle assez ridicule.

\hypertarget{chapitre-x.}{%
\chapter{CHAPITRE X.}\label{chapitre-x.}}

1694

~

\relsize{-1}

{\textsc{Éclat entre MM. de Richelieu et de Luxembourg, dont tout
l'avantage demeure au premier.}} {\textsc{- M. de Bouillon, moqué par le
premier président Harlay, et son repentir.}} {\textsc{- Sa chimère
d'ancienneté et celle de M. de Chevreuse.}} {\textsc{- Tentative échouée
de la chimère d'Épernon.}} {\textsc{- Prétention de la première
ancienneté des Vendôme désistée en même temps que formée.}} {\textsc{-
D'où naît le rang intermédiaire des bâtards.}} {\textsc{- Ruse, adresse,
intérêt, succès du premier président Harlay et sa maligne formation de
ce rang intermédiaire.}} {\textsc{- Déclaration du roi pour le rang
intermédiaire.}} {\textsc{- Harlay obtient parole du roi d'être
chancelier.}} {\textsc{- Princes du sang priés de la bouche du roi de se
trouver à l'enregistrement et à l'exécution de sa déclaration, et les
pairs, de sa part par une lettre à chacun de l'archevêque-duc de
Reims.}} {\textsc{- M. le duc et M. le prince de Conti mènent M. du
Maine chez MM. du parlement.}} {\textsc{- M. de Vendôme mené chez tous
les pairs et chez MM. du parlement par M. du Maine, et reçu comme lui au
parlement sans presque aucun pair.}} {\textsc{- MM. du Maine et de
Toulouse visités comme les princes du sang par les ambassadeurs.}}
\relsize{1}

~

L'affaire en règlement de juges se poussa vivement au conseil. Chacun de
nous, excepté M. de Lesdiguières et moi à cause de notre minorité, y
forma une demande à part pour allonger, chose dont nous ne nous cachions
plus. Force factums de part et d'autre, et force sollicitations comme
nous avions fait au parlement. M. de Vendôme et moi fûmes chargés
d'aller ensemble parler au chancelier Boucherat, et nous y fûmes à la
chancellerie à Versailles de chez Livry où M. de Vendôme m'avait donné
rendez-vous. Argouges, Bignon, Ribeyre et Harlay, gendre du chancelier,
tous conseillers d'État, furent nos commissaires, et Creil de Choisy,
maître des requêtes, rapporteur. Quantité de conseillers d'État se
récusèrent\,; Bignon aussi, comme parent de la duchesse de Rohan. Nous
regrettâmes sa vertu et sa capacité\,; on ne le remplaça point. Argouges
s'était ouvert à M. de La Rochefoucauld d'être pour nous, et manqua de
parole, ce que le duc lui reprocha cruellement. Ribeyre, gendre du
premier président de Novion, grand ennemi des pairs, et aussi fort
maltraité par eux, fut soupçonné d'avoir épousé les haines de son
beau-père, quoique homme d'honneur et de probité. Harlay fut entraîné
par sa famille et par le bel air, auquel il n'était pas insensible.
Cette même raison donna à M. de Luxembourg le gros des maîtres des
requêtes, petits-maîtres de robe, et fort peu instruits du droit public
et de ces grandes questions, de manière que nous fûmes renvoyés au
parlement\,; mais notre vue n'en fut pas moins remplie. Nous voulions
gagner temps, et par ce moyen notre procès se trouva hors d'état d'être
jugé de cette année.

Cependant les procédures s'étaient peu à peu tournées en procédés\,: il
y avait toujours eu quelques propos aigres-doux à l'entrée du conseil
entre quelques-uns de nous et M. de Luxembourg\,; et comme c'est une
suite presque immanquable dans ces sortes de procès de rang, l'aigreur
et la pique s'y étaient mises. Je ne fus pas le seul à qui plus
particulièrement qu'aux autres. M. de Luxembourg fit sentir la sienne,
qui pour le dire en passant ne saluait presque plus M. de La
Rochefoucauld et plus du tout MM. de La Trémoille et de Richelieu.

Il était plus personnellement outré contre ce dernier d'avoir vu toutes
ses mesures rompues par le règlement de juges entrepris au conseil sous
son nom\,; aussi n'épargna-t-il ni sa personne, ni sa conduite, ni le
ministère du cardinal de Richelieu dans un de ses factums. M. de
Richelieu, très vivement offensé, fit sur-le-champ une réponse, et tout
de suite imprimer et distribuer, par laquelle il attaqua la fidélité
dont M. de Luxembourg avait vanté sa maison, par les complots du dernier
duc de Montmorency pris en bataille dans son gouvernement contre le feu
roi à Castelnaudary, et pour cela exécuté à Toulouse en 1632\,; et la
personne de M. de Luxembourg, par sa conduite sous M. le Prince, par sa
prison pour les poisons et les diableries, par la sellette sur laquelle
il avait été interrogé et avait répondu, et par la lâcheté qui l'avait
empêché en cette occasion de réclamer les droits de sa dignité et
demander à être jugé en forme de pairie. Outre ces faits, fortement
articulés, le sel le plus âcre y était répandu partout.

M. de Richelieu ne s'en tint pas là\,: il rencontra M. de Luxembourg
dans la salle des gardes à Versailles. Il fut droit à lui. Il lui dit
qu'il était fort surpris de son procédé à son égard, mais qu'il n'était
point ladre (ce furent ses termes)\,; que dans peu il en verrait
paraître une réponse aussi vive que son factum la méritait\,; qu'au
reste, il voulait bien qu'il sût qu'il ne le craignait ni à pied ni à
cheval, ni lui ni sa séquelle, ni à la cour ni à la ville, ni même à
l'armée quand bien même il irait, ni en pas un lieu du monde. Tout cela
fut dit avec tant d'impétuosité, et il lui tourna le dos après avec tant
de brusquerie, que M. de Luxembourg n'eut pas l'instant de lui répondre
un mot, et, quoique fort accompagné à son ordinaire et au milieu des
grandeurs de sa charge, il demeura confondu. L'effet répondit à la
menace. Le lendemain le factum fut signifié et débité partout.

Des pièces aussi fortes, et une telle sortie faite à un capitaine des
gardes du corps au milieu de sa salle, firent le bruit qu'on peut
imaginer. Tous les ducs opposants et tout ce qu'ils eurent d'amis très
disposés à soutenir pleinement le duc de Richelieu, tout ce que la
charge et le commandement des armées donnait de partisans en même
dessein pour lui, était un mouvement fort nouveau qui pouvait avoir de
grandes suites. M. de Luxembourg sentit à travers sa colère qu'il
s'était attiré ce fracas par les injures de son factum\,; il comprit que
solliciter pour lui, ou prendre un parti éclatant contre dix-sept pairs
de France, serait alose fort différente, et la dernière une partie
difficile à lier\,; que les princes du sang, ses amis intimes, se
garderaient bien de s'y laisser aller\,; que le roi, qui au fond ne
l'aimait pas, serait tenu de près par le gros de ses parties, et en
particulier par le duc de La Rochefoucauld\,; et que
M\textsuperscript{me} de Maintenon, amie intime, de tous les temps, du
duc de Richelieu, et toujours depuis dans la liaison la plus étroite
avec lui, qui seul de la cour la voyait à toutes heures, ferait son
affaire propre de la sienne. Le héros en pâlit, et eut recours à ses
amis pour le tirer de ce fâcheux pas. Il s'adressa à M. le Prince et aux
ducs de Chevreuse et de Beauvilliers, à quelques autres encore de
moindre étoffe qu'il crut le pouvoir servir. Il fit offrir par les trois
premiers à M. de Richelieu une excuse verbale avec la suppression
entière de son factum à condition de celle de la réponse.

M. de Richelieu, prié de se trouver chez M. le Prince avec les ducs de
Chevreuse et de Beauvilliers, y fut prêché plus d'une fois sans se
vouloir rendre, tandis que sa réponse courait de plus en plus, et qu'il
la faisait distribuer à pleines mains, et à la fin se rendit. Là fut
réglé comme la chose devait se passer. M. de Luxembourg, à jour et heure
marquée, rencontra M. de Richelieu chez le roi dans un de ces temps de
la journée où il y a le plus de monde. Il s'approcha de lui et lui dit
ces propres termes\,: « Qu'il était très fâché de l'impertinence du
factum publié contre lui, qu'il lui en faisait ses excuses, qu'il le
suppliait d'être persuadé qu'il l'avait toujours fort estimé et honoré
et le faisait encore, ainsi que la mémoire de M. le cardinal de
Richelieu\,; qu'au reste il n'avait point du tout vu cette pièce, qu'il
châtierait ses gens d'affaires auxquels il avait toujours soigneusement
défendu toute sorte d'invectives, qu'enfin il avait donné ordre très
précis pour la faire entièrement supprimer.\,» M. de Richelieu, vif et
bouillant, le laissa dire et lui répondit après quelques honnêtetés
entre ses dents, qu'il finit par une assurance mieux prononcée qu'il
ferait aussi supprimer sa réponse. Elles le furent en effet de part et
d'autre, mais après que M. de Richelieu nous en eut donné à nous tous,
et à notre conseil, à ses amis à pleines mains, et surtout aux
bibliothèques.

En même temps, l'honnêteté et la bienséance furent un peu rétablies
entre M. de Luxembourg et nous. Je fus surpris d'en recevoir le premier
des demi-révérences\,; j'y répondis par d'entières qui l'engagèrent à me
saluer désormais à l'ordinaire, mais sans nous parler ni nous approcher,
comme cela n'arrivait que très rarement et à fort peu d'entre nous.

M. de Bouillon, anciennement en cause avec nous, s'en était désisté,
comme je l'ai dit, dès le commencement de ce renouvellement\,; et, sans
nous en dire un mot à pas un, l'avait fait signifier à quelques-uns de
nous, entre autres à M. de La Rochefoucauld et à moi. Son prétexte était
misérable, parce qu'il n'avait rien de commun avec M. de Luxembourg.
Celui-ci prétendait à titre de son mariage, l'autre par celui de son
échange de Sedan avec le roi. Il fut mal payé de cette désertion en plus
d'une manière. Il en parla au premier président qui, n'ayant pas les
mêmes raisons à son égard qu'à celui de M. de Luxembourg, lui répondit,
avec un sourire moqueur et une gravité insultante, que les duchés
d'Albret et de Château-Thierry ne sont point femelles dans leur première
érection\,; qu'elle avait été faite pour Henri III et pour Henri IV,
avant qu'ils parvinssent à la couronne\,; que, pour obtenir l'ancienneté
de ces érections, il fallait qu'il prouvât sa, descendance masculine de
ces princes\,; qu'il souhaitait pour l'amour de lui qu'il le pût faire,
et le laissa fort étourdi et fort honteux d'une réponse si péremptoire
et telle. M. de Luxembourg, de son côté, n'oublia aucune raison dans un
de ses factums, pour mettre au grand jour la chimère de la prétention de
M. de Bouillon et pour la mettre en poudre\,; de sorte que nous aurions
été pleinement vengés, et par nos parties mêmes, si le crédit et la
considération que nous pouvions espérer de son union avec nous avait pu
nous laisser quelque chose à regretter. Honteux enfin d'être si mal
reconnu de ceux à qui il avait voulu plaire\,; et embarrassé à l'excès
des plaisanteries finies de M. de Chaumes, et des railleries piquantes
de MM. de La Trémoille et de La Rochefoucauld, il fit des excuses au
dernier\,; se rejeta sur ses gens d'affaires et avoua son tort et son
repentir.

Pour M. de Chevreuse, qui se couvrit du prétexte du mariage de sa fille,
comme je l'ai dit plus haut, et qui cachait sous cette apparence sa
prétention de l'ancienne érection de Chevreuse, il ne fut point du tout
ménagé par son oncle de Chaulnes, qui le mettait à bout par ses
railleries qui ne finissaient point, et auxquelles il se lâchait avec
moins de ménagements qu'il n'aurait fait avec un étranger. Nous perdîmes
à celui-là beaucoup, et par sa considération, et par son esprit et sa
capacité, et par un grand nombre de mémoires sur toutes ces matières de
pairies, faits ou recueillis par le duc de Luynes, son père, qui y était
fort savant\,; et qu'il ne voulut jamais nous communiquer.

Ce procès donna occasion à une autre tentative. Le célèbre duc d'Épernon
avait été fait duc et pair, 27 novembre 1581\,; un mois avant la
première érection de Piney, dont M. de Luxembourg prétendait
l'ancienneté sur nous. Son fils aîné, mort à Casal, 11 février 1639, à
quarante-huit ans, n'eût point d'enfants\,; le cardinal de La Valette,
son frère\,; mourut à Rivoli, près de Turin, 28 septembre, même année
1639, à quarante-sept ans\,; général de l'armée française, tous deux
avant leur fameux père, mort, retiré à Loches, 1641, à quatre-vingt-huit
ans\,; le duc d'Épernon, son second fils, qui lui succéda, mourut à
Paris, 25 juillet 1661, à soixante et onze ans. Il avait perdu le duc de
Candale, son fils unique, sans alliance, à Lyon, 28 janvier 1658, à
trente ans, et ne laissa qu'une seule fille qui voulut absolument
quitter un si puissant établissement et se faire carmélite à Paris, au
couvent du faubourg Saint-Jacques, où elle est morte, 22 août 1701, à
soixante-dix-sept ans et cinquante-trois de profession\,; que la reine
faisait toujours asseoir et par ordre du roi quand elle allait aux
carmélites, comme duchesse, d'Épernon, malgré toute l'humilité de cette
sainte et spirituelle religieuse. Ainsi, le duché-pairie d'Épernon était
éteint depuis 1661. Le premier et fameux duc d'Épernon avait un frère
aîné tué, sans enfants, devant Roquebrune de Provence qu'il assiégeait,
11 février 1592, général de l'armée du roi, à quarante ans, homme de la
meilleure réputation et de la plus grande espérance. Ils avaient trois
sœurs, dont les deus cadettes moururent mariées, l'une au frère du duc
de Joyeuse, qui de douleur de sa mort se fit capucin, et c'est ce
célèbre capucin de Joyeuse dont la fille unique épousa le duc de
Montpensier, qui ne laissa qu'une fille unique, que le feu roi fit
épouser à Gaston, son frère, qui n'en eut qu'une fille unique,
M\textsuperscript{lle} de Montpensier, morte fille en 1693, dont j'ai
ci-devant parlé. L'héritière de Joyeuse, fille du capucin et de la sœur
du premier duc d'Épernon, et veuve du dernier Montpensier, se remaria au
duc de Guise, fils de celui qui fut tué aux derniers états de Blois,
dont plusieurs fils morts sans alliance\,: le duc de Guise, dit de
Naples, de l'expédition qu'il y tenta, mort sans enfants\,; le duc de
Joyeuse, père du dernier duc de Guise, qui eut l'honneur d'épouser
M\textsuperscript{lle} d'Alençon, dernière fille de Gaston, en 1667, et
qui mourut à Paris, en 1671, à vingt et un ans, ne laissant qu'un fils
unique, mort en 1675, avant cinq ans\,; M\textsuperscript{lle} de Guise
qui avait tait ce grand mariage de son neveu et qui a vécu fille avec
tant de splendeur et est morte à Paris, la dernière de la branche de
Guise, 3 mars 1688, à soixante-dix-sept ans, et l'abbesse de Montmartre.
De cette sœur de M. d'Épernon aucun descendant n'en a réclamé la pairie.
L'autre sœur cadette épousa le comte de Brienne, depuis duc à brevet,
fils du frère aîné du premier duc de Luxembourg-Piney, et elle mourut
sans enfants, et son mari le dernier de sa branche. Ainsi, nulle
prétention.

Leur sœur aînée avait épousé, 21 avril 1582, Jacques Goth, marquis de
Rouillac, grand sénéchal de Guyenne\,; leur fils, Louis Goth, marquis de
Rouillac, hérita de la terre d'Épernon. Il mourut en 1662, et laissa un
fils né en 1631, qui porta le nom de marquis de Rouillac, mais qui fut
plus connu sous le nom de faux duc d'Épernon, parce qu'il en prit le
titre après la mort de son père, qu'il se faisait donner par ses amis et
par ses valets. C'était un homme violent, extraordinaire, grand
plaideur, et qui eut des aventures de procès fort désagréables. Il se
piqua d'une grande connaissance de l'histoire, et fit imprimer un
ouvrage de la véritable origine de la dernière race de nos rois qui
trouva des critiques et des savants qui le réfutèrent. Il n'eut jamais
aucun honneur, ni ne put obtenir permission de porter ses prétentions en
jugement. Il ne laissa qu'une seule fille et point de fils, et fut le
dernier de sa branche. Cette fille se trouva avoir infiniment d'esprit,
de savoir et de vertu\,; elle se fit beaucoup d'amis et d'amies, et
entre autres Mademoiselle, fille de Gaston, qui obtint du roi de fermer
les yeux à ce qu'elle se fit appeler Madame, comme duchesse d'Épernon,
sans pourtant en avoir, ni rang, ni honneur, ni permission de faire
juger sa prétention.

Ce procès de M. de Luxembourg la réveilla. Le cardinal d'Estrées était
fort bien auprès du roi, et toute sa maison était en splendeur. Elle
s'adressa à lui et au maréchal d'Estrées, son frère, pour obtenir la
permission du roi de faire juger sa prétention en épousant le comte
d'Estrées, vice-amiral, en survivance du maréchal son père. Le roi y
entra, et aussitôt MM. d'Estrées se mirent en grand mouvement\,; ils
sentirent bien que la sœur d'un homme fait duc et pair, et non appelée
par ses lettres d'érection au défaut de sa postérité, n'a nul droit d'y
rien prétendre, mais ils espérèrent de nous épouvanter par leur bruit et
leur crédit, et en même temps de nous séparer et de nous séduire. Ils
briguèrent donc ceux qu'ils purent, et nous firent proposer de se
départir de l'ancienneté et de prendre la queue, mais secrètement à
chacun sa part, pour, à cette condition, obtenir un acquiescement de
ceux qui s'en trouveraient éblouis. Malheureusement pour MM. d'Estrées
le procès de M. de Luxembourg avait uni ceux qu'il attaquait, et les
rassemblait en ce temps-là chez Riparfonds, leur avocat, toutes les
semaines, une fois de règle, et très souvent davantage. Là, chacun
rapporta ce qui lui avait été proposé par MM. d'Estrées sous le spécieux
prétexte d'accélérer leur mariage, et d'éviter les piques et les
brouilleries qui naissent si aisément de ces sortes d'affaires, mais
sans toutefois aucune inquiétude du succès. On y trouva\,: 1° un défaut
de droit radical tel que je le viens d'expliquer\,; 2° la proposition de
céder l'ancienneté illusoire comme ne dépendant point d'un duc d'Épernon
par héritage, puisqu'il ne le pouvait être qu'au titre, et par
conséquent de la date de son érection, et que de plus, quand il la
pourrait céder, ses enfants seraient toujours en état de la reprendre.
Il fut donc résolu de se moquer de ses manèges, et de répondre sur le
même ton que les services et le crédit de MM. d'Estrées devaient plutôt
leur procurer une érection nouvelle en faveur de M. le comte d'Estrées,
qu'un procès dont nous soutiendrions unanimement le poids sans aucune
crainte de l'issue. MM. d'Estrées, voyant ainsi la ruse et la menace
inutiles, sentirent bien qu'ils ne réussiraient pas\,: le mariage fut
rompu, et il ne fut plus question de cette prétention.

Toutes ces affaires différentes ne furent rien en comparaison d'une
autre qu'elles firent naître, et dont l'entreprise donna lieu à la plus
grande plaie que la pairie pût recevoir, et qui en devint la lèpre et le
chancre. L'abbé de Chaulieu, qui gouvernait les affaires de M. de
Vendôme, imagina de lui faire prétendre l'ancienneté de la première
érection de Vendôme en faveur du père du roi de Navarre, père d'Henri
IV, et d'attaquer les ducs d'Uzès, d'Elbœuf, Ventadour, Montbazon ou
Guéméné, et La Trémoille ses anciens. Feu M. d'Elbœuf, père de celui-ci,
s'était toujours montré fort uni aux pairs, et fort jaloux des droits et
du rang de la pairie en ce qui ne touchait point les princes étrangers.
M. de Chaulnes avait attaqué M. d'Elbœuf par de fines railleries sur son
indolence contre M. de Luxembourg, et il était venu à bout de l'exciter
à imiter son père jusqu'à lui faire des remerciements de lui avoir
ouvert les yeux, et il en était là, lorsque M. de Vendôme, persuadé par
l'abbé de Chaulieu, obtint la permission du roi d'attaquer ses anciens,
et leur donna la première assignation. Comme cela ne fut point poussé,
je n'entrerai pas dans le prétendu droit de l'un ni dans celui des
autres. L'affaire se commença à l'ordinaire fort civilement de part et
d'autre, mais à peine y eut-il quelques procédures commencées que
l'humeurs y mit.

Dans ces circonstances, il arriva ce qui n'arrivait presque jamais, et
que depuis ne vit-on peut-être plus, que dés gens sans chargé suivissent
le roi s'allant promener de Versailles à Marly. Le roi allait toujours
seul dans une calèche. Ce jour-là le second carrosse fut du capitaine
des gardes et de M. de La Rochefoucauld, et avec eux, de M. le Grand,
qui ne suivait guère, et par extraordinaire des ducs d'Elbœuf et de
Vendôme. Ces deux derniers parlèrent bientôt de leur procès avec
civilités réciproques\,; mais sûr les significations réciproques, ils
s'aigrirent, se picotèrent, et enfin se querellèrent. M. d'Elbœuf dit à
M. de Vendôme qu'il n'était de naissance ni de dignité à ne rien céder
et qu'il le précéderait partout comme avaient fait ses pères. M. de
Vendôme lui répondit avec feu qu'il ne pouvait pas avoir encore oublié
que son père n'avait pas pris l'ordre parce qu'il l'y aurait précédé.
L'autre à lui répliquer avec encore plus de chaleur qu'une fois n'était
pas coutume, et que lui-même se pouvait souvenir de l'aventure de son
grand-père aux obsèques d'Henri IV qui, aux termes de la déclaration
d'Henri IV d'un mois auparavant et non enregistrée, voulut prendre lé
premier rang et qui fut pris lui-même par le bras par le duc de Guise
qui lui dit que ce qui pouvait être hier n'était plus bon aujourd'hui,
en le mettant derrière lui\,; et lui fit prendre le rang de son
ancienneté de pairie, dont ils n'étaient pas sortis depuis. M. de
Vendôme aurait bien pu répliquer sur la promotion de l'ordre de Louis
XIII\,; mais M. de La Rochefoucauld et M. le Grand mirent le holà, les
firent taire, et finirent cette dispute si vive et si haute le plus
doucement qu'ils purent, comme ils arrivaient à Marly. La promenade et
le retour se passèrent sans plus parler du procès et civilement entre
eux\,; mais dès que M. de Vendôme fut revenu à Versailles, il alla
conter à M. du Maine ce qui lui était arrivé. Celui-ci, qui peu à peu
par un usage dont le roi soutenait l'usurpation, avait pris toutes les
manières des princes du sang et en recevait à peu près tous les
honneurs, sentit le peu d'assurance de son état. Il dit à M. de Vendôme
de parler au roi de ce qui lui venait d'arriver, et de le laisser faire.
En effet, dès le même soir, immédiatement avant le coucher du roi, M. du
Maine lui fit sentir le besoin qu'il avait de titres enregistrés qui
constatassent son rang, et le roi, qui n'y avait pas songé, résolut de
n'y perdre pas un moment.

Le lendemain, il ordonna à M. de Vendôme de se désister juridiquement de
sa prétention du rang de la première érection de Vendôme, et il manda
pour le jour suivant le premier président, le procureur général et le
doyen du parlement, et dès ce même jour qui suivit cet ordre, la
signification du désistement fut faite\,; qui surprit infiniment. Ce ne
fut pas pour longtemps. Le roi ordonna à ces messieurs de dresser une
déclaration en faveur de ses fils naturels\,; revêtus de pairie, pour
précéder au parlement et partout tous autres pairs plus anciens qu'eux,
de l'étendre beaucoup plus que celle d'Henri IV\,; et de les mettre au
niveau des princes du sang. Harlay, qui avait cent mille écus de brevet
de retenue \footnote{On appelait ainsi un brevet par lequel le roi
  donnait une certaine somme sur le prix d'une charge, d'un
  gouvernement, etc., à la femme, aux héritiers ou aux créanciers du
  titulaire. C'était une véritable pension de retraite que le roi
  assurait aux principaux fonctionnaires et à leur famille et qui devait
  être payée par leur successeur.} sur sa charge de premier président,
venait d'en obtenir cinquante mille d'augmentation. Il était trop bon
courtisan pour ne pas saisir une si sensible occasion de plaire, et trop
habile pour n'en pas tirer tous ses avantages, et pour soi, et pour les
usurpations de sa compagnie sur les pairs, en leur donnant les bâtards
pour protecteurs par leur intérêt. Il pria donc le roi de trouver bon
qu'il pensât quelques jours à une solide exécution de ses ordres, et
qu'il pût en conférer avec celui principalement qu'ils regardaient.
C'est ce qu'il avait grand intérêt de lui faire goûter, et par lui au
roi, l'adroit parti qu'il se proposait d'en tirer pour les usurpations
du parlement et de s'en faire à soi-même un protecteur, à tirer sur le
temps pour le conduire à son but personnel.

Il fit donc entendre à M. du Maine qu'il ne ferait jamais rien de solide
qu'en mettant les princes du sang hors d'intérêt et en leur en donnant
un de soutenir ce qui serait fait en sa faveur\,; que pour cela il
fallait toujours laisser une différence entière entre les distinctions
que le parlement faisait aux princes du sang et celles qu'on lui
accorderait au-dessus des pairs, et former ainsi un rang intermédiaire
qui ne blessât point les princes du sang, et qui au contraire les
engageât à les maintenir dans tous les temps, par l'intérêt de se
conserver un entre-deux entre eux et les pairs\,; que pour cela il
fallait lui donner la préséance sur tous les pairs, et les forcer à se
trouver à l'enregistrement de la déclaration projetée et à sa réception
en conséquence qui se devait faire tout de suite, lui donner le bonnet
comme aux princes du sang qui depuis longtemps ne l'est plus aux pairs,
mais lui faire prêter le même serment des pairs sans aucune différence
de la forme et du cérémonial, pour en laisser une entière à l'avantage
des princes du sang qui n'en prêtent point, et pareillement le faire
entrer et sortir de séance tout comme les pairs, au lieu que les princes
du sang traversent le parquet, l'appeler par son nom comme les autres
pairs en lui demandant son avis, mais avec le bonnet à la main un peu
moins baissé que pour les princes du sang qui ne sont que regardés sans
être nommés, enfin le faire recevoir et conduire au carrosse par un seul
huissier à chaque fois qu'il viendra au parlement, à la différence des
princes du sang qui le sont par deux, et des pairs, dont aucun n'est
reçu par un huissier au carrosse que le jour de sa réception, et qui
sortant de séance deux à deux sont conduits par un huissier jusqu'à la
sortie de la grande salle seulement.

M. du Maine fut extrêmement satisfait de tant de distinctions au-dessus
des pairs et d'être si rapproché de celles des princes du sang, sans
courir le risque de les blesser, et fut surtout fort touché de l'adresse
avec laquelle ce rang intermédiaire était imaginé par le premier
président pour lui assurer en tout temps la protection de tous ces
avantages, par celui qu'on y faisait trouver aux princes du sang pour
eux-mêmes. M. du Maine content, le roi le fut aussi. Il ne fut donc plus
question que de dresser la déclaration que le premier président avait
déjà minutée et qu'il ne fit qu'envoyer au net pour être scellée.

Ce fut alors qu'il sut se servir de M. du Maine pour faire proposer au
roi sa récompense. Il avait déjà eu quelque sorte de parole ambiguë,
mais qui n'était pourtant qu'une espérance, d'être fait chancelier,
lorsque le roi, voulant légitimer les enfants qu'il avait de
M\textsuperscript{me} de Montespan, sans nommer la mère, dont il n'y
avait point d'exemple, Harlay consulté, lors procureur général, suggéra
l'expédient d'embarquer le parlement par celle du chevalier de
Longueville qui réussit si bien. En cette occasion-ci, il se fit donner
formellement parole par le roi qu'il succéderait à Boucherat, chose qui
le flatta d'autant plus que ce chancelier était alors fort vieux et ne
pouvait le faire attendre longtemps. Pour l'exécution de la déclaration,
le roi en parla aux princes du sang qui ne crurent avoir que des
remerciements à faire\,: le roi les pria de se trouver au parlement, et
M. le Duc et M. le prince de Conti de lui faire le plaisir de conduire
M. du Maine en ses sollicitations. On peut juger s'ils le refusèrent. De
là le roi fit appeler l'archevêque de Reims\,: il lui fit part de ce
qu'il avait résolu\,; lui dit qu'il croyait que les pairs seraient plus
convenablement invités par lui-même à cette cérémonie que par M. du
Maine\,; qu'ainsi M. du Maine n'irait pas chez eux, mais qu'il priait
l'archevêque de se trouver au parlement, et lui ordonnait d'écrire de sa
part une lettre d'invitation à chaque pair. Un fils de M. Le Tellier
était fait pour tenir tout à honneur venant du roi\,; il lui répondit
dans cet esprit courtisan, et de là s'en fut chez M. du Maine\,: ce fut
le seul de tous les pairs qui commit cette bassesse, pas un ne dit un
mot au roi ni à M. du Maine, pas un né fut chez ce dernier ni devant ni
après la cérémonie.

Voici la lettre circulaire de l'archevêque aux pairs\,:

« Monsieur,

« Le roi m'a ordonné de vous avertir que M. le duc du Maine sera reçu au
parlement le 8 de ce mois de mai, en qualité de comte-pair d'Eu, et
qu'il prendra sa place au-dessous de MM. les princes du sang, et
au-dessus de MM. les pairs. Sa Majesté vous prie de vous y trouver, et
m'a chargé de vous assurer que cela lui fera plaisir et qu'elle vous en
saura bon gré.

« Je suis, etc.\,»

Les présidents à mortier, et les présidents et doyens des conseillers de
chaque chambre furent avertis de se trouver chez eux le 5 mai, et à peu
près de l'heure, pour recevoir la sollicitation de M. du Maine. Ce
jour-là arrivé de Versailles à l'hôtel de Condé, il y monta dans le
carrosse de M. le Duc avec M. le prince de Conti, tous deux au derrière
et lui au devant avec M. le comte de Toulouse qui était compris dans la
même déclaration comme duc de Damville, mais qui ne fut pas reçu en même
temps. Ce carrosse était fort chargé de pages et environné de laquais à
pied. Suivaient les carrosses de M. le Duc et de M. le prince de Conti,
de M. du Maine et de M. le comte de Toulouse, dans lesquels étaient les
principaux de leur maison, avec force livrée, chacun un seul carrosse,
excepté M. le Duc qui, outre celui dans lequel il était, en avait un
autre rempli des principaux de chez lui. Ils firent ainsi leurs
sollicitations deux jours de suite, et allèrent de même au parlement, le
jour de l'enregistrement des lettres patentes de la réception de M. du
Maine, mais sans M. le comte de Toulouse. Elle se fit suivant ce qui a
été dit plus haut de la déclaration, et, au sortir de la cérémonie, ils
furent dîner avec les pairs chez le premier président.

Aucun des pairs n'osa manquer à s'y trouver de ceux qui étaient à Paris.
Le bonhomme La Force s'enfuit à sa maison de la Boulaie, proche
d'Évreux, et le duc de Rohan écrivit au roi que sa prétention, de la
première érection de Rohan, pour son grand-père maternel, l'empêchait
d'obéir, en cette occasion, à ses ordres. L'excuse était mal trouvée\,;
c'était pour la première fois qu'il manifestait cette bizarre
prétention\,; il n'en a jamais parlé depuis, et il était un des plus
ardents opposants avec nous à celle de M. de Luxembourg. MM. d'Elbœuf et
de Vendôme n'étaient pas reçus, ni moi non plus, Dieu merci. M. de
Chevreuse fut celui à qui le roi fit son remerciement pour tous les
pairs, de s'être trouvés à la cérémonie pour lesquels il lui fit force
belles promesses générales, monnaie dont aucun ne se paya ni n'espéra
rien de mieux avec trop de raison.

M. de Vendôme fut tôt, après reçu avec les mêmes distinctions que
l'avait été M. du Maine, qui le mena sans cortège faire ses
sollicitations à tout le parlement, mais sans avertir. Ils furent chez
tous les pairs\,; le roi ne leur fit rien dire\,; trois ou quatre
misérables seulement se trouvèrent à cette réception. Un moment avant
celle de M. du Maine, il y eut une petite vivacité de M. de La
Trémoille, qui, impatienté de l'applaudissement que M. de Reims donnait
à cette étrange nouveauté, lui dit qu'il ne doutait pas de son
approbation, parce qu'il ne se souciait guère du rang des archevêques de
Reims, mais que pour lui, il pensait tout autrement, et qu'il était fort
sensible à celui des ducs de La Trémoille. L'archevêque demeura muet, et
le roi n'en fit pas semblant à M. de La Trémoille, et ne l'en traita paf
moins bien.

Peu de jours après cette réception, l'ambassadeur de Venise, avec la
république duquel cela avoir été négocié, fit, à Versailles, sa visite à
MM. du Maine et de Toulouse, conduit par l'introducteur des ambassadeurs
en cérémonie, et en usa, pour le premier exemple, comme avec les princes
du sang. Cette parité, que le roi avait fort à cœur, fut exprès différée
après la réception de M. du Maine au parlement, pour ne pas donner trop
d'éveil auparavant aux princes du sang, à qui cette visite ne pouvait
pas être agréable. Cet exemple eut peine à être suivi par les autres
ambassadeurs\,; mais, avec le temps et des négociations, il le fut à la
fin, excepté des nonces.

\hypertarget{chapitre-xi.}{%
\chapter{CHAPITRE XI.}\label{chapitre-xi.}}

1694

~

\relsize{-.9}

{\textsc{Situation des opposants avec le premier président Harlay.}}
{\textsc{- Duc de Chaulnes.}} {\textsc{- Il négocie l'assemblée de
toutes les chambres avec le premier président Harlay, qui lui en donne
sa parole et qui lui en manque.}} {\textsc{- Rupture entière des
opposants avec le premier président Harlay.}} {\textsc{- Harlay, premier
président, récusé par les opposants.}} {\textsc{- Mort du dernier des
Longueville.}} {\textsc{- Prince et princesse de Turenne.}} {\textsc{-
Mariage du prince de Rohan.}} {\textsc{- M\textsuperscript{me}
Cornuel.}} {\textsc{- Mariage du duc de Montfort, du duc de Villeroy, de
La Châtre.}} {\textsc{- Distribution des armées.}} {\textsc{- Beuvron et
Matignon refusent le monseigneur au maréchal de Choiseul, et le lui
écrivent par ordre du roi.}} {\textsc{- Le roi me charge de Flandre en
Allemagne.}} {\textsc{- M. de Créqui chassé hors du royaume, et
pourquoi.}} {\textsc{- M\textsuperscript{me} du Roure exilée en
Normandie.}} {\textsc{- Monseigneur préfère la Flandre au Rhin.}}
{\textsc{- La Feuillée lui est donné pour son mentor.}} {\textsc{- Je
vais à l'armée d'Allemagne.}} {\textsc{- Belle marche du maréchal de
Lorges devant le prince Louis de Bade.}} \relsize{1}

~

Le procès avec M. de Luxembourg, renvoyé au parlement, y recommença avec
la même vigueur, la même partialité, la même injustice. Comme nous nous
vîmes exclus d'en sortir, nous ne songeâmes plus qu'à chercher les
moyens d'obtenir l'assemblée de toutes les chambres, selon la forme de
pairie, l'usage et le droit en pareils procès. Pour y parvenir, il n'y
avait que deux voies, la procédure ou la négociation. La dernière était
bien la plus sûre si elle réussissait\,; mais la difficulté était la
situation où nous nous trouvions avec le premier président qui pouvait
seul assembler les chambres à sa volonté, mais avec qui nous ne gardions
plus de mesures. Fort peu de nous le saluaient lorsqu'ils le
rencontraient, pas un n'allait chez lui, quoique nous sollicitassions
tous nos autres juges, et tous parlaient de lui sans ménagement. Il le
sentait d'autant plus vivement que c'était l'homme du monde le plus
glorieux, le plus craint, le plus ménagé, et qui n'avait jamais été mené
de la sorte\,; et, ce qui le touchait le plus, c'étaient les plaintes
prouvées que nous faisions de sa probité et de son injustice, parce
qu'il se piquait là-dessus de la plus austère vertu, dont nous faisions
tomber le masque.

Personne ne se voulait donc charger d'une négociation aussi difficile
avec lui, lorsque M. de Chaulnes, qui s'était acquis une grande
réputation et une grande considération par les siennes au dehors, voulut
bien hasarder celle-ci. C'était sous la corpulence, l'épaisseur, la
pesanteur, la physionomie d'un bœuf, l'esprit le plus délié, le plus
délicat, le plus souple, le plus adroit à prendre et à pousser ses
avantages, avec tout l'agrément et la finesse possible, jointe à une
grande capacité et à une continuelle expérience de toutes sortes
d'affaires, et la réputation de la plus exacte probité, décorée à
l'extérieur d'une libéralité et d'une magnificence également splendide,
placée et bien entendue, et de beaucoup de dignité avec beaucoup de
politesse. Il eut du premier président l'heure qu'il désira.

Il ouvrit son discours par les raisons que nous avions de nous plaindre
de son procédé, et lui fit sentir après avec délicatesse qu'il n'y a
point de places où on ne soit exposé à des ennemis\,; que tout le monde
était convaincu de sa partialité pour M. de Luxembourg\,; que seize
pairs de France, et dont plusieurs fort bien auprès du roi ou grandement
établis, n'étaient pas toujours impuissants à beaucoup nuire\,; que le
seul moyen d'effacer sa partialité de l'idée publique, et de regagner
les pairs qu'il s'était si grandement aliénés, était l'assemblée de
toutes les chambres pour les juger, et de lui en donner sa parole
positive\,; qu'il voulait bien lui avouer que nous l'avions prié de lui
faire cette proposition, bien moins par aucune espérance de succès, que
pour n'avoir rien à reprocher à leur conduite à son égard, pénétrer
définitivement où nous en étions avec lui, et éclater ensuite avec plus
de raisons et moins de mesures.

Le poids avec lequel ce discours fut prononcé étourdit le premier
président qui se mit sur une défense de sa conduite avec nous, confuse
et embarrassée. M. de Chaulnes vit qu'il ne tendait qu'à échapper, le
remit sur l'assemblée des chambres, et le pressa vivement. Serré de si
près, il se retrancha sur la difficulté de la faire, et diminua tant
qu'il put son autorité à cet égard. M. de Chaulnes n'avait garde de s'y
laisser tromper\,: il se servit habilement de sa faiblesse pour les
personnes de crédit à la cour et de sa propre vanité\,; il lui
représenta qu'inutilement il voudrait lui persuader qu'il n'était pas
maître d'assembler les chambrés toutes les fois qu'il le voulait\,;
qu'on savait bien que c'est honnêteté à lui et non pas un devoir d'en
prendre avis de la grand'chambre, et qu'on ne savait pas moins qu'il
était tellement le maître de ses délibérations que, quand même celles de
la grand'chambre y seraient nécessaires, ce n'était pas une difficulté
qu'il pût objecter, ni qui pût être reçue, dès que son intention serait
véritable de nous accorder l'assemblée de toutes les chambres.

Ces raisons ne donnèrent pas, à la vérité, de meilleurs sentiments au
premier président, mais bien un vif repentir de ne s'être pas assez
ménagé avec nous, et un regret cuisant sur l'intérêt de sa réputation,
qui lui arrachèrent enfin la parole positive qu'il donna à M. de
Chaumes, pour nous, qu'il assemblerait toutes les chambres pour la
continuation et le jugement de notre procès, après un long raisonnement
pour mieux faire valoir cet effort.

Le lendemain M. de Chaumes rendit compte à notre assemblée du succès
inespéré de sa négociation, et il reçut de nous tous les remerciements
si dignement mérités. Nous publiâmes ensuite cet engagement si
solennellement pris par le premier président avec tout ce que nous y
pûmes ajouter pour compenser nos plaintes, et pour l'engager de plus en
plus. Mais notre politique et notre confiance en la parole du premier
président furent bientôt confondues. Il ne put tenir contre ses intimes
liaisons prises avec M. de Luxembourg\,; il lui lit l'aveu de la parole
qu'il avait donnée, et ne put résister à s'engager à lui de ne la pas
tenir.

L'intérêt de M. de Luxembourg était grand d'empêcher l'assemblée des
chambres. Il aurait fallu y revoir sommairement tout le procès pour
l'instruction de tant de nouveaux juges. Leur nombre était difficile à
corrompre, et l'autorité du premier président, en laquelle étaient
remises toutes les espérances de M. de Luxembourg, était entière sur la
grand'chambre\,; et faible sur toutes les chambres assemblées. La
frayeur que M. de Luxembourg en avait conçue le trahit par la joie qu'il
ne put dissimuler de l'avoir rompue. Il nous en revint des soupçons. M.
de Chaulnes résolut de s'en éclaircir, et prit prétexte d'une autre
affaire pour voir le premier président. Il le trouva embarrassé avec
lui, et bientôt ce magistrat lui en avoua la cause par un discours
confus qui tendait à éluder sa parole. M. de Chaulnes le pressa avec
surprise, et lui dit\,: qu'il ne pouvait croire ce qu'il entendait, et
qu'il le priait de se souvenir qu'en grande connaissance de cause il lui
avait donné sa parole nette, précise, positive, d'assembler toutes les
chambres pour la continuation et le jugement de notre procès. Le premier
président, avec un air respectueux et ce masque de sévérité qu'il ne
quittait jamais, avoua qu'en effet il la lui avait donnée, forcé par son
éloquence et par son autorité\,; mais qu'il se repentait de s'être
engagé trop légèrement\,; qu'il était nécessité par de sérieuses
réflexions de lui déclarer qu'il se trouverait dans l'impossibilité de
l'effectuer, et tombant tout court en des respects et des compliments
sans fin, se mit à reconduire M. de Chaulnes, qui n'avait point du tout
envie de s'en aller, mais comme il faisait toujours à ceux dont il se
voulait défaire. M. de Chaulnes, indigné de se voir si étrangement
éconduit, le quitta en lui protestant qu'il avait sa parole, qu'il ne
voulait ni ne pouvait la lui rendre, qu'au reste il pouvait en manquer
et à lui et avec lui, à tout ce qu'il y avait de plus distingué dans le
royaume, et en user tout comme bon lui semblerait.

Le duc vint nous en rendre compte dans une assemblée extraordinaire\,;
il y fut résolu non seulement de ne plus garder aucune mesure avec un
homme aussi perfide, mais de chercher encore tous les moyens possibles
de le récuser, et après, tous ceux d'obtenir par la procédure
l'assemblée de toutes les chambres, surtout de ne rien oublier pour
tirer le procès en longueur, suivant nos précédentes résolutions. On
peut juger du bruit, des plaintes et des discours qui, de notre part,
suivit ce manquement de parole, contre un homme sur lequel aucune
considération ne pouvait plus nous retenir, et contre lequel nous ne
pouvions plus employer d'autres armes. Aussi en fut-il d'autant plus
outré, qu'il voyait sa réputation s'en aller en pièces, et qu'il n'avait
quoi que ce soit à opposer aux faits que nous publiions, et qu'il était
bien loin d'être accoutumé à un éclat si soutenu, et qui ne ménageait
pas plus les termes que les choses.

Pour en venir à sa récusation, voici ce dont on s'avisa ce fut de mettre
en procès le duc de Rohan avec l'avocat général, fils unique du premier
président, parce que la maxime reçue est que, \emph{qui est en procès
avec le fils, ne peut être jugé par le père}. Cet avocat général avait
épousé une riche héritière de Bretagne, dont deux belles terres
relevaient du duc de Rohan. Il fut donc prié d'en vouloir bien faire
demander le dénombrement, et d'ordonner à ses baillis de former un
procès bon ou mauvais à l'avocat général, pourvu que c'en fût un, et il
le promit de bonne grâce\,; mais, comme ses réflexions sont plus lentes
que ses décisions, je pense qu'il se repentit bientôt de l'engagement
qu'il avait pris\,; on s'en douta bientôt et on le pressa d'engager
quelques procédures dont il ne se put défendre. Le premier président en
fut bientôt averti, et sentit aussitôt ce que cela voulait dire. Sa
passion de demeurer notre juge l'emportant sur son orgueil, il n'est
soumission qu'il ne fit, et ne fit faire à Paris et en Bretagne à M. de
Rohan, et telles qui ne s'exigent pas même des moindres vassaux.

Ce procédé flatta le duc de Rohan déjà bien ébranlé par son irrésolution
naturelle\,: il voulut donc obliger le premier président en un point si
sensible, et pour y parvenir, nous déclara à une assemblée qu'il s'en
allait à Moret faire pêcher un grand étang qui demandait sa présence. Je
sentis et ne pus souffrir cette défection. Je m'écriai que c'était nous
abandonner dans la plus importante crise, où sa présente seule était
plus nécessaire que celle de tous les autres ensemble\,; qu'il était
inconcevable que la pêche d'un étang l'attirât à deux lieues de
Fontainebleau dans des moments si pressants, où ses gens d'affaires, ou
tout au plus la duchesse sa femme suffiraient de reste, et qu'à l'heure
que je parlais, on en pêchait quatre beaux à la Ferté-Vidame, à
vingt-quatre lieus de Paris, où ma mère ni moi n'avions jamais imaginé
d'aller pour aucune pêche. M. de Chaulnes, M. de La Rochefoucauld, tout
ce qui était à l'assemblée, ducs et conseils, lui firent les prières et
les remontrances les plus pressantes\,: mais le parti était pris\,; il
nous amusa seulement de la promesse de revenir dès que quelques choses
presseraient et qu'on le manderait. Le cas arriva en moins de huit
jours, où, sans le retour de M. de Rohan, toutes ses procédures contre
l'avocat général tombaient. Un laquais de M. de La Trémoille lui fut
dépêché toute la nuit, avec une lettre de son maître, tant pour lui que
comme chargé de tous, et une de Riparfonds, qui lui expliquait la
nécessité pressante et indispensable du retour. Le courrier le fit
éveiller\,: il lut les deux lettres, puis dit au laquais de faire ses
excuses, mais que les affaires qu'il avait à Moret ne lui permettaient
pas de les quitter, et sans autre réponse, fit tirer son rideau, et se
tourna de l'autre côté. À l'arrivée du courrier, Riparfonds fit une
seconde, lettre à M. de Rohan de la dernière force pour l'engager à
revenir\,; elle fut signée de dix ou douze ducs qui se trouvèrent à
l'assemblée et portée tout de suite par un autre courrier.

Je m'étais donné une violente entorse qui m'a voit empêché de me trouver
aux deux assemblées d'où on avait dépêché ces deux courriers, mais
j'étais instruit de ce qui s'y était passé. Je n'avais donc point signé
la lettre commune, ni écrit en particulier. Ma surprise fut donc grande
de voir arriver ce second courrier chez moi avec une lettre de M. de
Rohan, par laquelle il expliquait ses prétendues raisons de demeurer à
Moret, et me priait de faire ses excuses. J'envoyai aussitôt cette
lettre à l'assemblée qui se tenait pour attendre la réponse. À sa
lecture l'indignation fut grande\,; on ne put plus douter de la
défection préméditée, et on admira avec raison qu'un homme d'esprit
comme M. de Rohan nous sacrifiât, et son honneur même, à une
réconciliation personnelle dont il se flattait par là avec le premier
président, duquel l'orgueil ne lui pardonnerait jamais les bassesses
qu'il lui avait fallu faire pour se délivrer de ce procès.

Le coup manqué de la sorte, nous nous tournâmes à d'autres moyens. Ce
fut d'allonger par celui des ducs d'Uzès et de Lesdiguières. Ce dernier
était un enfant sous la tutelle de sa mère, espèce de fée, demeurant
presque toujours seule dans un palais enchanté, et sur qui presque
personne n'avait aucun crédit. M. de Chaulnes qui la voyait quelquefois
s'offrit de lui parler, et il en obtint la reprise de son fils avec
nous, au lieu du feu duc son père, qui n'avait pas encore été faite. De
M. d'Uzès je m'en chargeai, et il voulut bien se joindre à nous sous
prétexte que si ces anciennes pairies renaissaient ainsi de leurs
cendres, il s'en trouverait d'antérieures à son érection, qu'il avait
intérêt d'empêcher d'avance de pouvoir se mettre en prétention.

Cependant nous cherchions avec soin les moyens de récuser le premier
président, lorsque son dépit nous les fournit lui-même. Nous vivions
avec lui en attendant comme s'il l'était déjà. Magneux et Aubry,
intendants de MM. de La Trémoille et de La Rochefoucauld, également
habiles et attachés à leurs martres, et vifs sur notre affaire, étaient
par là devenus odieux au premier président\,; il n'avait pu s'en cacher,
nous le savions, et par cela même jamais il n'entendait parler de nous
que par eux. Ce mépris que nous affections et que nous publiions même le
désolait tellement, qu'un jour qu'ils étoient allés lui parler, il leur
dit qu'il ne pouvait pas douter que nous ne cherchassions toutes sortes
de moyens pour le récuser, que la chose n'était pourtant pas difficile,
puisque nous n'avions qu'à mettre le duc de Gesvres en cause, duquel il
avait l'honneur d'être parent. Il fut servi avec promptitude\,: M. de
Gesvres reçut le surlendemain une assignation de notre part. La raison
s'en voit ci-dessus dans la généalogie\,: il était fils de la fille et
sœur des deux ducs de Piney-Luxembourg. Je ne comprends pas comment
aucun de nous ni de notre conseil ne trouva pas ce moyen. Le premier
président ne tarda pas à se repentir de nous en avoir avisés, mais il
demeura récusé.

L'affaire en resta là pour cette année. La belle saison rappela M. de
Luxembourg et ses trois fils en Flandre\,; pas un de ses gens
d'affaires, ni de ses protecteurs, ne voulurent s'en charger en son
absence, non plus que l'abbé de Luxembourg son fils. La mort du duc de
Sully qui arriva pendant la campagne fit un délai naturel de quatre
mois, et la maladie de Portail, notre rapporteur, dura jusqu'à la fin de
l'année, et gagna la mort de M. de Luxembourg, que je rapporterai en son
temps.

Cet hiver finit enfin la fameuse maison de Longueville, si connue par sa
fortune inouïe et si prodigieusement soutenue jusqu'à son extinction. M.
de Longueville, qui parut tant de divers côtés pendant les troubles de
la minorité de Louis XIV, n'avait laissé que la duchesse de Nemours de
son premier mariage avec la sœur de la princesse de Carignan et du
dernier comte de Soissons, prince du sang, tué à la bataille de Sedan,
le dernier de cette branche. De son second mariage avec la fameuse
duchesse de Longueville, sœur de M. le Prince, le héros, et de M. le
prince de Conti, il n'avait eu que deux fils\,: le cadet, d'une grande
espérance, tué au passage du Rhin, sans alliance\,; l'autre, d'un esprit
faible, qu'on envoya à Rome, que les jésuites empaumèrent et que le pape
fit prêtre. Revenu en France il devint de plus en plus égaré, en sorte
qu'il fut renfermé dans l'abbaye de Saint-Georges près de Rouen pour le
reste de sa vie, où il n'était vu de personne, et M. le Prince prit
l'administration de ses biens. Il mourut les premiers jours de février,
et il se trouva un testament de lui fait à Lyon, allant à Rome, par
lequel il donne tout son bien à son frère, tué depuis au passage du
Rhin, et à son défaut et de sa postérité, à M\textsuperscript{me} sa
mère, et après elle à MM. les princes de Conti l'un après l'autre.
L'aîné de ces princes était mort il y avait déjà longtemps, en sorte que
celui-ci devint le seul appelé à ce grand héritage, que
M\textsuperscript{me} de Nemours résolut bien de lui contester.

M. de Soubise fit presque en même temps le mariage de l'héritière de
Ventadour avec son fils aîné. Elle était veuve du prince de Turenne,
fils aîné de M. de Bouillon, et son survivancier, tué à Steinkerque et
mort le lendemain, de ses blessures, écrivant à sa maîtresse. Il avait
montré par plusieurs pointes qu'il n'était pas indigne
arrière-petit-fils du maréchal de Bouillon, pour ne parler de rien de
plus récent\,; et le cardinal de Bouillon en eut une telle douleur qu'il
força le P. Gaillard, jésuite, fort attaché à eux tous, d'en faire
l'oraison funèbre. Il n'en avait point eu d'enfants dans un assez court
mariage\,; mais elle y avait eu le temps de se faire connaître par tant
de galanterie publique qu'aucune femme ne la voyait, et que les chansons
qui avaient mouché \footnote{Le mot \emph{moucher} se trouve plusieurs
  fois dans Saint-Simon avec le sens particulier de \emph{voler en
  bourdonnant comme une mouche}.} s'étaient chantées en Flandre, dans
l'armée où le prince de Rohan ne l'avait pas épargnée, et souvent et
publiquement chantée. Elle avait voulu épouser le chevalier de Bouillon
qu'elle trouvait fort à son gré, et lui le désirait fort pour les grands
biens qu'elle avait déjà et d'autres immenses qui la regardaient. M. et
M\textsuperscript{me} de Ventadour ne voulaient pas ouïr parler d'un
cadet fort peu accommodé. M. et M\textsuperscript{me} de Bouillon ne s'y
opposaient pas moins, parce qu'ils désiraient la remarier au duc
d'Albret, devenu leur aîné, duquel elle ne voulait en aucune sorte,
tellement que, par concert de famille, le roi fut supplié d'envoyer le
chevalier de Bouillon refroidir ses amours à Turenne, où ils le tinrent
jusqu'à ce qu'il n'en fût plus question\,; mais elle aussi tint bon à
refuser l'aîné. M. de Soubise regarda ce grand mariage comme la plus
solide base de sa branche. Il avait de bonnes raisons pour n'être pas
difficile au choix la beauté de sa femme l'avait fait prince et
gouverneur de province, avec espérance de plus encore. La richesse d'une
belle-fille, de quelque réputation qu'elle fût, lui parut mériter le
mépris du qu'en-dira-t-on. En deux mots, le mariage se fit.

Il y avait une vieille bourgeoise au Marais chez qui son esprit et la
mode avaient toujours attiré la meilleure compagnie de la cour et de la
ville\,; elle s'appelait M\textsuperscript{me} Cornuel, et M. de Soubise
était de ses amis. Il alla donc lui apprendre le mariage qu'il venait de
conclure, tout engoué de la naissance et des grands biens qui s'y
trouvaient joints. « Ho\,! monsieur, lui répondit la bonne femme qui se
mourait, et qui mourut deux jours après, que voilà un grand et bon
mariage pour dans soixante ou quatre-vingts ans d'ici\,!\,»

Le duc de Montfort, fils aîné du duc de Chevreuse, épousa en même temps
la fille unique de Dangeau, chevalier de l'ordre et de sa première
femme, fille de Morin dit le Juif, sœur de la maréchale d'Estrées. Elle
passe pour très riche, mais aussi pour ne pas retenir ses vents, dont on
fit force plaisanteries.

Le duc de Villeroy en même temps épousa la seconde fille de
M\textsuperscript{me} de Louvois, fort riche et charmante, sœur de M. de
Barbezieux, et sœur aussi fort cadette de la duchesse de La Rocheguyon.
L'archevêque de Reims, son oncle, aussi humble sur sa naissance, comme
tous les Tellier, que les Colbert sont extravagants sur la leur, et par
cela même assez dangereux sur celles des autres\,: « Ma nièce, lui
dit-il, vous allez être duchesse comme votre sœur, mais n'allez pas
croire que vous soyez pareilles. Car je vous avertis que votre mari ne
serait pas bon pour être page de votre beau-frère.\,» On peut juger
combien cette franchise qui ne fut pas tue obligea son bon ami pourtant,
le maréchal de Villeroy.

Enfin le marquis de La Châtre épousa la fille unique du premier mariage
du marquis de Lavardin, chevalier de l'ordre, avec une sœur du duc de
Chevreuse.

Il y eut cet hiver force bals et plusieurs beaux au Palais-Royal, au
premier desquels j'eus l'honneur de mener au branle
M\textsuperscript{me} la princesse de Conti, douairière, fille du roi,
et le mardi gras, grande mascarade à Versailles dans le grand
appartement où le roi amena le roi et la reine d'Angleterre, après leur
avoir donné à souper. Les dames y étaient partagées en quatre
quadrilles, conduites par M\textsuperscript{me} la duchesse de Chartres,
Mademoiselle, M\textsuperscript{me} la Duchesse et M\textsuperscript{me}
la princesse de Conti, douairière. Malgré la mascarade on commença par
le branle, et j'y menai la fille unique du duc de La Trémoille qui était
parfaitement bien faite, et qui dansait des mieux. Elle était en moresse
de la première quadrille qui l'emporta par la magnificence, et la
dernière par la galanterie des habits.

Les armées furent distribuées à l'ordinaire, la grande de Flandre à M.
de Luxembourg, une moindre au maréchal de Boufflers, et le marquis
d'Harcourt son camp volant, celle d'Allemagne au maréchal de Larges,
celle de Piémont au maréchal Catinat, et le duc de Noailles chez lui en
Roussillon. Le maréchal de Villeroy doubla sous M. de Luxembourg, et le
maréchal de Joyeuse sous M. de Lorges. Le maréchal de Choiseul alla en
Normandie avec un commandement fort étendu.

\begin{enumerate}
\def\labelenumi{\Roman{enumi}.}
\setcounter{enumi}{1999}
\tightlist
\item
  de Beuvron et de Montignon, chevaliers de l'ordre et lieutenants
  généraux de la province, firent difficulté de lui écrire
  monseigneur\,; ils reçurent ordre du roi de le faire, et il fallut
  obéir. Monseigneur fut, après ces destinations, déclaré commander les
  armées en Flandre et tous les princes avec lui.
\end{enumerate}

Le régiment que j'avais acheté se trouvait en quartier dans la
généralité de Paris, par conséquent destiné pour la Flandre, où je
n'avais pas envie d'aller après tout ce qui s'était passé avec M. de
Luxembourg. Par le conseil de M. de Beauvilliers, j'écrivis au roi mes
raisons fort abrégées, et lui présentai ma lettre comme il entrait de
son lever dans son cabinet le matin qu'il s'en allait à Chantilly et à
Compiègne faire des revues, et revenir incontinent après. Je le suivis à
la messe, et de là à son carrosse pour partir. Il mit le pied dans la
portière, puis le retira, et se tournant à moi\,: « Monsieur, me dit-il,
j'ai lu votre lettre, je m'en souviendrai.\,» En effet, j'appris peu de
temps après qu'on m'avait changé avec le régiment du chevalier de Sully
qui était à Toul, et qui allait en Flandre en ma place, et moi en
Allemagne en la sienne. J'eus d'autant plus de joie d'échapper ainsi à
M. de Luxembourg, et par une attention particulière du roi, pleine de
bonté, que je sus que M. de Luxembourg en eut un dépit véritable.

Il y avait quelques années que Monseigneur avait été fort amoureux d'une
fille du duc de La Force, que, dans la dispersion de sa famille pour la
religion, on avait mise fille d'honneur de M\textsuperscript{me} la
Dauphine pour la première fille de duc qui eût jamais pris ces sortes de
places, et le roi en avait chargé la duchesse d'Arpajon, dame d'honneur,
qui la logea et nourrit dans son appartement de Versailles lorsque la
chambre des filles fut cassée. On l'avait depuis mariée au fils du comte
de Roure avec la survivance de sa charge de lieutenant général de
Languedoc, et quelque argent que le roi donna pour s'en défaire
honorablement, après quoi elle avait reçu défense de venir à la cour par
M. de Seignelay. Monseigneur le souffrit respectueusement et se servit
du marquis de Créqui pour continuer secrètement cette intrigue\,; mais
il arriva que le marquis et M\textsuperscript{me} du Roure se trouvèrent
au gré l'un de l'autre. Monseigneur le sut\,; ils se brouillèrent avec
éclat\,; les présents furent rendus de part et d'autre, chose rare pour
un dauphin, et le marquis de Créqui fut chassé hors du royaume où il
passa quelque temps.

Cet hiver-ci le feu mal éteint se ralluma\,; M\textsuperscript{me} du
Roure ne put voir Monseigneur à Versailles si secrètement que le roi
n'en fût averti. Il en parla à Monseigneur et il n'y gagna rien. Ce
prince ne fit point ses pâques, dont le roi fut fort fâché, tellement
qu'il chassa la dame en Normandie dans les terres de son père jusqu'à
nouvel ordre. Monseigneur n'y sut faire autre chose que lui envoyer
mille louis par Joyeu, son premier valet de chambre, et faire après ses
dévotions. Le roi avait envie qu'il allât en Allemagne, mais il préféra
la Flandre par une intrigue qui se développa pendant la campagne, et le
roi y consentit. Il choisit le bonhomme La Feuillée, lieutenant général
très distingué, de près de quatre-vingts ans, pour être son conseil à
l'armée, et ne rien faire sans son avis. Cela ne devait pas être bien
agréable à M. de Luxembourg\,; mais le roi voulait un mentor particulier
à son fils. Il se souvint peut-être de ce qui s'était passé l'année
précédente à Heilbronn, et il lui en voulut donner un dont il n'eût pas
les mêmes inconvénients à craindre.

La Feuillée eut la distinction de ne prendre point jour à l'armée et d'y
être pourtant reconnu et traité comme lieutenant général, toujours logé
de préférence chez Monseigneur ou le plus près de lui, avec défense
expresse du roi de faire les marches autrement qu'en carrosse, et de
monter à cheval qu'auprès de Monseigneur devant les ennemis. C'était un
très honnête gentilhomme, doux, sage, valeureux, excellent officier
général et qui méritait toute cette confiance. M. de Chaulnes alla en
son gouvernement de Bretagne\,; le duc d'Aumont, bien qu'en année de
premier gentilhomme de la chambre, à Boulogne\,; le maréchal d'Estrées,
au pays d'Aunis, Saintonge et Poitou\,; et le maréchal de Tourville
commanda l'armée navale, le comte d'Estrées une moindre et à ses ordres
en cas de jonction dont Tourville demeura le maître.

J'allai voir à Soissons mon régiment assemblé. Je l'avais dit au roi qui
me parla longtemps dans son cabinet et met recommanda la sévérité, ce
qui fut cause que j'en eus dans cette revue plus que je n'aurais fait
sans cela. J'avais été voir les maréchaux de Lorges et de Joyeuse qui
étaient revenus chez moi. J'étais bien avec le second\,; la probité de
l'autre me plaisait, de sorte que je me trouvai aussi content d'aller,
en cette armée que je me serais trouvé affligé de servir en Flandre. Je
partis enfin pour Strasbourg où je fus surpris de la magnificence de
cette ville et du nombre, de la grandeur et de la beauté de ses
fortifications.

J'eus le plaisir d'y revoir un de mes anciens amis\,: c'était le P. Wolf
que j'envoyai d'avance quêter en cinq ou six maisons de jésuites là
autour, et qu'on trouva à Haguenau, où il était recteur. Il avait été
compagnon du P. Adelman, confesseur de M\textsuperscript{me} la
Dauphine, et comme dès ma jeunesse je savais et parlais parfaitement
l'allemand, on prenait soin de me procurer des connaissances allemandes,
et ces deux-là m'avaient fort plu. À la mort de M\textsuperscript{me} la
Dauphine on les envoya en Alsace\,; mais on leur défendit d'aller plus
loin. Le P. Adelman ne se put tenir d'aller revoir sa patrie. Cela fut
trouvé si mauvais, que, pour conserver sa pension du roi, il fut obligé
de s'en aller à Nîmes, et de se confiner en Languedoc, où il mourut. Le
P. Wolf, plus sage, s'était tenu en Alsace, et y demeura toujours.

Nous fîmes quelques repas à la mode du pays dans la belle maison de M.
Rosen, avec qui j'avais fait amitié la campagne précédente en Flandre,
où il servait de lieutenant général et était mestre de camp général de
la cavalerie, et qui, très obligeamment, me la prêta depuis tous les
ans. Je m'arrêtai six jours à Strasbourg, où je fus conseillé de prendre
le Rhin jusqu'à Philippsbourg. Je pris pour moi et le peu de gens que je
menais, deux redelins attachés ensemble, qui sont de très petits bateaux
longs et étroits, fort légers, et d'autres pour ce qui me suivait. Je
couchai au fort Louis, où j'arrivai de bonne heure, et que j'eus le
loisir de visiter en arrivant. Rouville, qui en était gouverneur, m'y
reçut avec beaucoup de politesse et bonne chère\,; et le lendemain
j'allai coucher à Philippsbourg, où Desbordes, gouverneur, me logea et
me fit bonne chère et force civilités aussi. Là je trouvai grande
compagnie de gens qui allaient joindre l'armée, entre autres le prince
palatin de Birkenfeld, capitaine de cavalerie dans Bissy, extrêmement de
mes amis.

Le lendemain nous partîmes pour aller joindre la cavalerie campée à
Obersheim, sous Mélac, lieutenant général\,; l'infanterie était sous
Landau avec les maréchaux et tous les officiers généraux. Dès que je fus
arrivé, j'allai chez Mélac qui me vint voir le lendemain. Je reçus la
visite de tout ce qu'il y avait de brigadiers et de mestres de camp, et
d'une infinité d'autres officiers, et je leur fis aussi la mienne,
c'est-à-dire aux premiers. Ce camp, si voisin du Rhin, ressemblait par
sa tranquillité à un camp de paix, mais bientôt toute notre cavalerie
alla passer le Rhin sur le pont de Philippsbourg, et joindre de l'autre
côté l'infanterie qui y était déjà avec tous les généraux, et ce fut là
que j'allai pour la première fois d'abord chez les deux maréchaux de
France. J'allai aussi voir Villars, lieutenant général et commissaire
général de la cavalerie, qui la commandait, et à mon loisir les
principaux officiers généraux.

Je me trouvai avec Sonastre dans la brigade d'Harlus qui formait la
gauche de la seconde ligne. C'étaient deux très honnêtes gens et fort
sociables. Sonastre était gendre de Montbron, chevalier de l'ordre, et
seul lieutenant général de Flandre qui avait été fort à la mode, et qui
se tenait presque toujours dans son gouvernement de Cambrai. Harlus
était un vieil officier de distinction, gaillard et pourtant sachant
fort vivre\,: il avait une charge d'écuyer du roi, et il était frère
aîné de Vertilly, major de la gendarmerie, aussi fort galant homme.

La veille de la Saint-Jean, dînant chez moi avec les marquis de Grignan,
d'Arpajon et de Lautrec, et plusieurs autres officiers, nous apprîmes
que les ennemis paraissaient sur les hauteurs en assez grand nombre\,:
nous étions campés le cul dans le Necker, à la petite portée du canon
d'Heidelberg, et nous en apprîmes la confirmation au quartier général,
où nous courûmes. On donna divers ordres, et sur le minuit l'armée se
mit en marche.

Barbezières était devant avec un assez gros détachement pour les
reconnaître au plus près qu'il pourrait, mais avec défense de rien
engager. Les petits détachements qu'il poussa devant lui s'approchèrent
si près des ennemis, qu'ils furent obligés de se reployer sur
Barbezières qui les blâma de s'être indiscrètement avancés. Au jour qui
commençait à se faire grand, il se reconnut fort inférieur à eux qui
venaient à lui, et il envoya demander du secours au maréchal de Lorges.
Ce général, qui ne voulait rien entamer sans savoir bien ce qu'il
faisait, fut fort fâché de cet engagement, envoya soutenir Barbezières,
et lui manda de se retirer. Ce secours trouva les pistolets en l'air,
mais les ennemis qui n'étaient là qu'en détachement, et qui crurent
notre armée tout proche, ne suivirent plus Barbezières que mollement,
qui fit sa retraite aisément.

Cependant l'armée continua sa marche en forme de croissant, en faisant
de longues haltes. Elle arriva vers une heure après midi fort près du
village de Roth, et fort proche aussi des ennemis qui occupaient les
hauteurs de Weisloch fort entrecoupées de haies et de vignes, dont le
revers nous était inconnu. Le village de Weisloch était sur la crête, un
peu en penchant et vers notre droite, et au bas de ces hauteurs il y
avait un ruisseau dont les bords étaient assez mauvais. Il vint un faux
avis, et qui nous fit faire halte en colonnes, que les bagages, qui
marchaient en assez mauvais ordre, étaient abandonnés et au pillage. Le
maréchal de Joyeuse y poussa à toute bride, mais il apprit en chemin que
ce n'était qu'une fausse alarme, et revint promptement sur ses pas.

Les ennemis avaient de petits postes sur ce ruisseau que j'ai dit,
surtout un pour en garder un pont de pierre. Le comte d'Averne,
brigadier de dragons, eut ordre de l'attaquer, et il l'emporta\,; mais
il y fut tué après les avoir chassés de là et poursuivis fort loin.
C'était un Sicilien de condition que le malheur, plus que le choix,
avait jeté dans la révolte de sa patrie et que M. de La Feuillade ramena
avec quelques autres, lorsqu'il retira les troupes françaises de Sicile.
Il fut fort regretté pour son mérite et sa valeur, et surtout de M. le
maréchal de Lorges, à qui il s'était fort attaché et à M. de La
Rochefoucauld.

Le marquis du Châtelet passa le ruisseau avec la brigade de Mérinville,
qu'il commandait en son absence, et chassa les ennemis des hauteurs,
aidé de quelques compagnies de gendarmerie. Il n'y eut que les troupes
qui formaient les deux ailes de la droite, par où on avait marché, qui
eurent part à ce petit combat dont le reste était trop éloigné. Le
maréchal de Lorges, qui voyait tout près des coteaux fourrés dont il ne
connaissait ni les revers ni ce qui y pouvait être de troupes, fit
retirer les siennes, garda le ruisseau et se campa dans la plaine, son
quartier général à Roth. Il y demeura huit jours avec beaucoup de
précaution, jusqu'à ce que les magasins de farine à Philippsbourg se
trouvant épuisés et les fourrages mangés dans tout ce petit pays, il
ramena son armée en deçà du Rhin.

Il fit la plus belle marche du monde. Il décampa de Roth, à onze heures
du matin, à grand bruit de guerre, sur neuf colonnes qui firent la
caracole en partant, en présence des ennemis qui occupaient l'autre côté
du ruisseau, et campaient sur le revers des hauteurs qui étaient
derrière, où le petit combat s'était donné. Toutes ces colonnes
passèrent un bois avec tant de justesse que dans la plaine de
Schweitzingen, où elles se mirent en bataille aussitôt, chaque brigade
s'y trouva dans son ordre et dans sa place. On défila ensuite avec grand
ordre et promptitude, sur un pont et par un gué d'un gros ruisseau, les
troupes en bataille, jusqu'à ce que ce fût à chacune à passer. Le
maréchal de Joyeuse se tint au pont pour maintenir l'ordre et diligenter
tout, et le maréchal de Lorges à son arrière-garde. Tout fut passé en
deux heures, parce que les vivres, l'artillerie et les gros et menus
bagages avaient pris les devants. On crut quelque temps que cette marche
serait inquiétée, mais on sut après que le prince Louis de Bade, qui
commandait l'armée impériale, ne l'avait osé, et avait dit tout haut aux
siens que cette marche était trop bien ordonnée pour qu'il la pût
attaquer avec succès.

Nous campâmes aux Capucins de Philippsbourg, où en allant toute l'armée
s'était jointe, et comme tous les équipages étaient à Obersheim, avec la
réserve et Romainville, qui la commandait, un des plus anciens et des
plus dignes brigadiers de cavalerie, chacun se fourra comme il put dans
Philippsbourg, où le gouverneur me fit donner la chambre du major, et où
La Châtre, qui en eut le vent, me fit demander de s'y venir réfugier
avec moi. Le lendemain, le major nous donna à déjeuner\,; et, tandis que
l'armée défilait sur le pont du Rhin, j'allai faire ma cour aux deux
maréchaux, et de là je la fus joindre à Obersheim, où elle campa.

Nous passâmes à Spire, dont je ne pus m'empêcher de déplorer la
désolation. C'était une des plus belles et des plus florissantes villes
de l'empire\,; elle en conservait les archives\,; elle était le siège de
la chambre impériale, et les diètes de l'empire s'y sont souvent
assemblées. Tout y était renversé par le feu que M. de Louvois y avait
fait mettre, ainsi qu'à tout le Palatinat, au commencement de la
guerre\,; et ce qu'il y avait d'habitants, en très petit nombre, étaient
buttés sous ces ruines ou demeurant dans les caves. La cathédrale avait
été plus épargnée ainsi que ses deux belles tours et la maison des
jésuites, mais pas une autre. Chamilly, premier lieutenant général de
l'armée et gouverneur de Strasbourg, demeura à Obersheim avec
Vaubecourt, maréchal de camp, et toute l'infanterie\,: les maréchaux,
tous les officiers généraux, toute la cavalerie et la seule brigade de
Picardie, allèrent à Osthoven et Westhoven, et, huit jours après, à
Guinsheim, le cul dans le Vieux-Rhin. Ce fut là où se firent les
réjouissances des succès de Catalogne.

\hypertarget{chapitre-xii.}{%
\chapter{CHAPITRE XII.}\label{chapitre-xii.}}

1694

~

\relsize{-1}

{\textsc{Bataille du Ter en Catalogne.}} {\textsc{- Palamos, Girone,
Castel-Follit pris.}} {\textsc{- M. de Noailles fait vice-roi de
Catalogne.}} {\textsc{- Bombardement aux côtes.}} {\textsc{- Dieppe
brûlée.}} {\textsc{- Belle et diligente marche de Monseigneur et de M.
de Luxembourg du camp de Vignamont.}} {\textsc{- Préférence de l'avis de
l'intendant à celui du général, qui coûte une irruption des ennemis en
Alsace.}} {\textsc{- Les ennemis retirés au delà du Rhin.}} {\textsc{-
Procédé entre les maréchaux de Lorges et de Joyeuse raccommodé par les
marquis d'Huxelles et de Vaubecourt.}} {\textsc{- Maréchal d'Humières,
sa fortune et sa famille.}} {\textsc{- Sa mort.}} {\textsc{- Maréchal de
Boufflers gouverneur de Flandre et Lille.}} {\textsc{- Maréchal de
Lorges gouverneur de Lorraine.}} {\textsc{- M. du Maine grand maître de
l'artillerie.}} {\textsc{- Duc de Vendôme général des galères.}}
\relsize{1}

~

M. de Noailles fit passer le Ter à son armée, le 28 mai, devant le
marquis de Villena, ou duc d'Escalone, car c'est le même, vice-roi de
Catalogne, et le défit\,; quinze cents prisonniers, tout le canon et le
bagage, et les ennemis en fuite et poursuivis. Ils y ont perdu cinq
cents hommes, M. de Noailles trois cents. Le vieux Chazeron, chevalier
de l'ordre et premier lieutenant général de cette armée, eut tout
l'honneur du passage et du combat. M. de Noailles ne passa le Ter que
pendant la déroute des ennemis\,; au moins c'est ce qui se débita et qui
a été cru. Nous y avons eu peu d'officiers principaux blessés, et gagné
force drapeaux. Le commandant de la cavalerie espagnole, un
sergent-major et quelques colonels ont été pris. Le marquis de Noailles,
frère du maréchal, qui a apporté cette nouvelle, en fut brigadier avec
huit mille livres de gratification, outre sa course. Palamos fut
emporté, le 7 juin, l'épée à la main on leur y a tué trois cents hommes,
et pris six cents. La citadelle se rendit peu après, c'est-à-dire le 10,
la garnison de quinze cents hommes prisonnière de guerre. La place est
considérable par son port et par elle-même. Cela fit chanter des
\emph{Te Deum} et valut une lettre, de la main du roi, à la vieille
duchesse de Noailles. C'était une sainte fort aimable, qui avait été
longtemps dame d'atours de la reine mère, et bien avec elle et avec le
roi\,; toujours vertueuse à la cour et depuis longtemps retirée à
Châlons-sur-Marne, dans une grande solitude, et se confessant tous les
soirs à l'évêque son fils.

M. de Noailles suivit sa pointe, et prit Girone en six jours de tranchée
ouverte. La place capitula, le 29 juin, et la garnison de trois mille
hommes ne servira point jusqu'au 1er novembre. Une si riante campagne
valut au duc de Noailles des patentes de vice-roi de Catalogne, dont il
prit possession dans la cathédrale de Girone, et n'y oublia rien de
toutes les cérémonies et les distinctions qui pouvaient le flatter.

Il prit encore par la témérité d'un seul homme, le château de
Castel-Follit, sur un pain de sucre de roche, fort haut, qui commande
toute la plaine. Il prit envie à un soldat déterminé d'aller voir si le
premier retranchement était gardé par beaucoup de monde\,; il le trouva
abandonné et y entra l'épée à la main, faisant de grands cris pour être
suivi. Il le fut de cinq ou six autres qui entrèrent avec lui dans le
second. Il était, l'on dit, plein de monde, mais qui s'épouvanta
tellement de se voir attaqué dans un poste cru inaccessible, qu'ils
crurent, aux cris, avoir un assaut à soutenir\,; et, s'enfuyant,
donnèrent une si chaude alarme au château et furent si vivement
poursuivis par ce petit nombre, qui cependant s'était fort accru, qu'ils
entrèrent tous pêle-mêle et que la place fut emportée avec beaucoup de
carnage. Ostalric tomba aussi entre les mains de M. de Noailles et
termina cette heureuse campagne.

L'amiral Russel avait mouillé avec force vaisseaux à Barcelone, où le
marquis de Villena s'était retiré avec le débris de son armée, et nos
forces navales n'étaient pas bastantes \footnote{Suffisantes.} contre
celles de Russel. Celles des ennemis avaient visité nos côtes tout
l'été, bombardé ce qu'elles avaient pu, et brûlé presque toute la ville
de Dieppe. Le chevalier de Lorraine, qui était à Forges, y courut avec
quelques preneurs d'eaux et y aida de son mieux le maréchal de Choiseul
et M. de Beuvron. Le roi écrivit au chevalier de Lorraine, pour le
remercier du zèle qu'il avait témoigné.

Il ne se passa rien en Italie, et tout s'y termina au blocus de Casal.
MM. de Vendôme passèrent presque toute la campagne en Provence, où le
maréchal Catinat les avait détachés avec quelques troupes.

En Flandre on ne fit que s'observer et subsister. Il s'en passa une
grande partie au camp de Vignamont, où à la fin les fourrages devinrent
éloignés et difficiles. Le prince d'Orange fut obligé d'en aller
chercher le premier, et prit son temps de décamper le 17, que presque
toute l'armée de Monseigneur était au fourrage. Néanmoins, le soir même,
la gauche marcha avec les maréchaux de Villeroy et de Boufflers, lequel
avait joint depuis deux jours, et le lendemain 18, Monseigneur et M. de
Luxembourg suivirent avec le reste de l'armée. Les ennemis avaient deux
marches d'avance, et Monseigneur la Sambre et force ruisseaux et défilés
à passer, et avait à gagner le camp d'Espierres avant qu'ils s'en
fussent saisis. Son armée marcha en plusieurs corps séparés.
L'infanterie fut soulagée par un grand nombre de chariots qu'on fit
trouver, et la Sambre convoya l'artillerie et les vivres tant qu'on s'en
put aider. La marche se fit avec un grand ordre et une telle diligence,
le maréchal de Villeroy toujours en avant, que Monseigneur prit le camp
d'Espierres, le 25, en même temps que la tête des ennemis paraissait de
l'autre côté. On se canonna le reste du jour, le ruisseau d'Espierres
entre-deux, et les ennemis, sur le soir, se retirèrent. Cette importante
marche fut très belle et fort admirée. Le reste de cette campagne ne fut
plus que subsistances. Les princes s'en allèrent d'assez bonne heure à
Fontainebleau, et M. de Luxembourg, après leur départ, courut en vain en
personne avec quelques troupes pour enlever le quartier du comte
d'Athlone, qu'il trouva décampé sur l'avis qu'il avait eu.

Notre campagne d'Allemagne s'acheva fort tranquillement. Nous demeurâmes
quarante jours à Gaw-Boecklheim dans le plus beau et le meilleur camp du
monde, et par un temps charmant, quoique tournant un peu sur le froid\,:
ce commencement de froid m'y attira une dispute pour une maison avec
d'Esclainvilliers, mestre de camp de cavalerie\,; cela alla pourtant
jusqu'à M. le maréchal de Lorges, qui sur-le-champ m'envoya dire par
Permillac, maréchal des logis de la cavalerie, que la maison était à
moi, et qui le signifia à d'Esclainvilliers. Peut-être lui en dit-il
davantage, car d'Esclainvilliers vint dès le soir à moi qui causais sur
le pas de ma porte avec le prince de Talmont et cinq ou six autres
brigadiers ou mestres de camp, et me fit force excuses. Il revint encore
chez moi deux jours après\,; je le fus voir ensuite, puis lui donnai à
dîner avec d'autres, comme j'avais toujours du monde à manger, généraux,
mestres de camp et autres officiers. C'était un brave homme, épais, mais
bon homme et galant homme, et qui savait fort bien mener une troupe de
cavalerie.

Après un si long séjour dans un camp abondant, il fallut aller
ailleurs\,; le maréchal de Lorges voulut laisser un gros corps
d'infanterie en Alsace pour empêcher les ennemis d'y entrer par un pont
de bateaux diligemment jeté, quand il s'en serait éloigné pour ses
subsistances, et ne se rendit point aux représentations de La Grange,
intendant de l'armée, qui l'était aussi d'Alsace. Celui-ci en écrivit à
la cour, manda que, si cette infanterie demeurait en Alsace, elle
mettrait la province hors d'état de payer cent mille écus prêts à
toucher, que c'étaient des inquiétudes et des précautions inutiles, et
qu'il répondait sur sa tête que les ennemis ne passeraient pas le Rhin,
et n'étaient pas même en état d'y songer. Barbezieux, qui avec tous ses
grands airs sentait plus l'intendant que le général d'armée, et plus
enclin aussi à croire l'un que l'autre, mit le roi de son côté,
tellement que le maréchal reçut un ordre positif tel que La Grange
l'avait proposé. À cela le maréchal de Lorges ne put qu'obéir\,; et, ne
trouvant plus de subsistances plus proches que les bords de la Nave, il
se mit, avec la première ligne, tout contre Creutznach, et envoya
Tallard avec la seconde au delà de cette petite rivière, guéable
partout, dans le Hondsrück où nous eûmes des fourrages et des vivres en
abondance.

À peine la goûtions-nous, que Tallard reçut ordre de partir aussitôt
avec toutes ses troupes pour aller rejoindre le maréchal de Lorges\,;
c'est que le prince Louis de Bade avait calculé sur notre éloignement
qu'il aurait le loisir de faire une rafle en Alsace avant que nous
pussions être rejoints, et de se retirer avant que nous pussions aller à
lui. Il avait donc jeté un pont de bateaux sur le Rhin à Hagenbach, à la
faveur d'une grande île dans laquelle il avait mis de l'artillerie, et
de là s'était espacé en Alsace par corps séparés. Au premier avis le
maréchal de Lorges s'était porté avec quelque cavalerie jusqu'à Landau,
où le maréchal de Joyeuse lui mena ses troupes, et nous partîmes le
lendemain de l'arrivée de l'ordre pour passer la Nave et camper le
lendemain à Flonheinm. Tallard y eut avis que le prince de Hesse se
préparait, avec vingt mille hommes, à l'attaquer le lendemain dans sa
marche\,; mais ce que nous appréhendions, c'était de trouver le défilé
de Durckheim occupé, où il était aisé d'empêcher le passage et de tenir
ainsi les deux lignes de notre armée séparées et par conséquent fort
embarrassées, et de désoler l'Alsace, tandis que la première ligne seule
ne le pourrait empêcher, et que la seconde demeurerait inutile.

Dans cet embarras, il se trouva une cousine de l'homme chez qui j'étais
logé, qui arrivait de Mayence, d'où elle était partie la veille, et je
le sus de mes gens qui le découvrirent. Elle ne parlait qu'allemand\,;
je la menai à Tallard qui me pria de lui servir d'interprète. Nous sûmes
d'elle que les portes de Mayence étaient fermées, qu'on n'y laissait
entrer personne de ce côté-ci\,; qu'on l'en avait fait sortir\,; qu'elle
avait vu quantité de tentes au delà de Mayence, et que des hussards lui
avaient dit que c'était le prince de Hesse qui allait joindre le prince
Louis. Cela ne nous instruisit guère. Tallard, n'ayant aucun avis des
partis qu'il avait, en envoya encore deux dehors. Nous avions bien fait
quatorze lieues de France et n'étions arrivés qu'à huit heures du soir,
de sorte qu'il fallut bien donner la nuit au repos, et nous avions
encore huit lieues jusqu'aux défilés de Durckheim. Nous marchâmes le
lendemain dans la disposition de trouver les ennemis, dont il ne parut
nul vestige\,; et on sut, après, que ce camp sous Mayence était de huit
mille hommes, plus envieux de butin que de combat. Romainville, avec sa
réserve, avait pris en partant d'Arienthal dans le Hondsrück où nous
étions, un autre chemin par les montagnes avec les bagages, de sorte que
nous marchions légèrement. Nous traversâmes les défilés de Durckheim
sans aucun obstacle, et nous campâmes encore à quatre lieues au delà, à
deux lieues près de la première ligne avec laquelle le maréchal de
Joyeuse nous attendait. Tallard poussa jusqu'à lui pour recevoir ses
ordres, qui furent de marcher le lendemain sur Landau. En chemin nous
joignîmes la première ligne, et ce fut une grande joie pour toutes les
deux que cette réunion.

J'allai tout de suite à Landau voir M. le maréchal de Lorges qui avait
attendu son armée avec impatience. Je le trouvai dans le jardin de
Mélac, gouverneur de la place et un des lieutenants généraux de l'armée,
avec presque tous les officiers généraux, et La Grange, fort embarrassé
de sa contenance et la tête fort basse. Nous y apprîmes que les ennemis,
répandus en plusieurs corps, avaient enlevé un grand butin et quantité
d'otages, et qu'ils se retranchaient fort dans l'île et dans les bois
d'Hagenbach\,; mais le nombre de ce qui avait passé le Rhin on ne le sut
jamais. Ce n'était pas faute de soins\,; Mélac avait battu un gros parti
des ennemis, où Girardin avait été légèrement blessé au ventre. C'était
un très bon officier, brigadier de cavalerie et fils de Vaillac,
chevalier de l'ordre en 1661, qui était à Monsieur, et gens de fort
bonne maison. Il avait servi de lieutenant général en Irlande, et y
avait commandé l'armée après la mort de Saint-Ruth qui y fut tué\,; mais
il avait déplu à M. de Louvois qui l'avait donné au roi pour un
ivrogne\,; il en était bien quelque chose, et il en était demeuré là.

Le lendemain, après une longue marche, on prit un camp fort étendu, d'où
le marquis d'Alègre, maréchal de camp de jour, prit en arrivant les
gardes et les dragons de Bretomelles pour aller voir ce qui était dans
la plaine au delà. Il poussa jusqu'au bois, où il força un grand
retranchement d'où il chassa le général Soyer. On se reposa le
lendemain\,; le jour suivant, les deux maréchaux se mirent en campagne,
M. de Lorges pour aller chasser les ennemis de Weissembourg qu'il en
trouva délogés, M. de Joyeuse pour aller dans les bois, où il trouva un
grand retranchement qu'il n'avait pas assez de troupes pour forcer. Le
lendemain on se reposa encore. Le surlendemain on laissa tout plié dans
le camp, et on marcha aux ennemis en colonne renversée. On n'avait pas
fait beaucoup de chemin à travers de grands abatis d'arbres, qu'on sut
que les ennemis avaient repassé le Rhin, et rompu et retiré leur pont,
de sorte que l'armée s'en retourna au camp aussi triste qu'elle en était
partie gaillarde. Trois jours après, les ordres arrivèrent en ce même
camp pour la séparation des troupes. Ils portaient que Tallard irait aux
Deux-Ponts, le maréchal de Joyeuse dans le Hondsrück, et le maréchal de
Lorges où il le jugerait à propos, avec une destination de troupes et
d'officiers généraux pour chacun des trois.

Le maréchal de Joyeuse sut d'abord la sienne et n'en dit mot, soit oubli
ou autre raison\,; le maréchal de Lorges ne lui en parla point, mais le
jour de la séparation, il lui écrivit le matin par un page qu'il le
priait de partir dans deux heures. Joyeuse, piqué, répondit verbalement
qu'il n'était préparé à rien, et qu'il ne pouvait partir, puis s'alla
promener. Le maréchal de Lorges, inquiet de cette réponse, s'en allait
chez lui, lorsqu'il le rencontra se promenant, et qui ne détourna point
son cheval pour aller à lui. L'autre le joignit. L'abord fut très froid,
les propos furent de même\,: excuses de l'un, plaintes de l'autre, et
fermeté à ne point partir. Ils se quittèrent de la sorte. Le maréchal de
Lorges, inquiet de plus en plus, avait des ordres précis, et la matinée
s'avançait. Il eut recours à la négociation, et il en chargea le marquis
d'Huxelles, chevalier de l'ordre et lieutenant général, et Vaubecourt,
maréchal de camp. Ils allèrent trouver le maréchal de Joyeuse, qu'ils
persuadèrent de venir au moins chez l'autre maréchal, et qu'ils y
amenèrent. Ils y entendirent la messe, puis s'enfermèrent. Au bout d'une
heure ils sortirent, et les ordres furent donnés pour le départ du
maréchal de Joyeuse et de ses troupes dont j'étais, et les deux
maréchaux allèrent dîner ensemble chez le marquis d'Huxelles au quartier
du maréchal de Joyeuse qui était le chemin du départ, qui ne se put
faire qu'après midi.

J'étais fort bien avec le maréchal de Joyeuse, qui me fit loger après le
dernier maréchal de camp et devant Harlus, mon brigadier, qui, comme
l'ancien des brigadiers de notre petite armée, y commandait la
cavalerie. Il n'en fut point fâché\,; mais les autres brigadiers ne le
trouvèrent pas trop bon, et moins qu'eux le prince palatin de
Birkenfeld, fort de mes amis et qui ne m'en dit rien. Il était capitaine
dans Bissy, deuxième brigadier de ce corps. Notre brigade échut à
Naurum, sur le bord de la Nave, fort près d'Eberbourg, noyé dans le
fourrage. J'y demeurai jusqu'au 16 octobre, que le maréchal de Joyeuse
me donna congé de fort bonne grâce, et je m'en allai à Paris par Metz,
où je vis M. de Sève, qui y était premier président du parlement.
C'était un des plus intègres et des plus éclairés magistrats, qui avait
été fort des amis de mon père.

Avant de rentrer dans Paris, il faut réparer un oubli. Lorsque nous
étions au camp de Gaw-Boecklheim, La Bretesche fut chargé d'aller
reconnaître quelque chose vers Rhinfels. C'était un gentilhomme qui
avait perdu une jambe à la guerre, qui avait été partisan distingué, qui
avait acquis une capacité plus étendue, très galant homme d'ailleurs, et
en qui le maréchal de Lorges se fiait fort. Il était un des lieutenants
généraux de son armée, et, nonobstant ce grade, il ne voulut prendre
avec lui que deux cents hommes de pied et cent cinquante dragons. Arrivé
à la nuit après une grande traite à un village à quatre lieues de
Rhinfels, il s'y arrêta, posta son infanterie, tint quelques dragons à
cheval dehors, et le reste attacha ses chevaux à une haie devant la
grange où La Bretesche se mit à manger un morceau avec les officiers.
Comme ils étaient à table, la lune qui était belle s'obscurcit tout d'un
coup, et voilà un orage affreux d'éclairs, de tonnerre et de pluie.
Aussitôt La Bretesche, craignant quelque surprise par ce mauvais temps,
fait monter les dragons à cheval, y monte lui-même, et dans cet instant
entend une grosse décharge qui justifie sa précaution\,; il donne ses
ordres à celui qui commandait les dragons, et s'en va à son infanterie
et la dispose. Il revient tout de suite à ses dragons, n'y en trouve
plus que deux ou trois avec un seul capitaine et nuls autres. Au
désespoir de cet abandon, il retourne à son infanterie, charge les
ennemis, profite de l'obscurité et du désordre ou il les met, les pousse
et les chasse du village, quoique trois fois plus forts que lui, et est
légèrement blessé au bras et à la cuisse, et parce que le jour allait
poindre, se retire en bon ordre à Eberbourg. En chemin il rencontra une
des troupes de dragons qui l'avaient abandonné. Le capitaine qui la
menait eut l'impudence de lui demander s'il voulait qu'il l'escortât, et
s'attira la réponse qu'il méritait, sur quoi les dragons sa mirent à
faire des excuses à La Bretesche, et à rejeter cette infamie sur leurs
officiers qui les avaient emmenés malgré eux. De notre camp à Eberbourg,
il n'y avait que trois lieues. La Bretesche, qui était fort aimé et
estimé, fut fort visité de toute l'armée\,; j'y fus des premiers. Il en
fut quitte pour y demeurer dix ou douze jours. Il eut la générosité de
demander grâce pour ces dragons, et le maréchal de Lorges, naturellement
bon et doux, la facilité de la lui accorder. Il ne faut pas ôter à
Marsal, capitaine des guides, l'honneur qui lui est dû\,: il avait suivi
La Bretesche, ne le quitta jamais d'un pas et fit très bien son devoir.
Il eut depuis une commission de capitaine d'infanterie, et il entendait
fort bien son métier. Il avait commencé, disait-on, par être maître de
la poste d'Hombourg d'où La Bretesche était gouverneur et d'où il
l'avait tiré.

Ce fut dans le loisir de ce long camp de Gaw-Boecklheim que je commençai
ces Mémoires par le plaisir que je pris à la lecture de ceux du maréchal
de Bassompierre qui m'invita à écrire aussi ce que je verrais arriver de
mon temps.

Nous trouvâmes à notre retour le maréchal d'Humières mort. C'était un
homme qui avait tous les talents de la cour et du grand monde et toutes
les manières d'un fort grand seigneur, avec cela homme d'honneur quoique
fort liant avec les ministres et très bon courtisan. Ami particulier de
M. de Louvois qui contribua extrêmement à sa fortune, qui ne le fit pas
attendre\,; il était brave, et se montra meilleur en second qu'en
premier\,; il était magnifique en tout, bien avec le roi qui le
distinguait fort et était familier avec lui. On peut dire que sa
présence ornait la cour et tous les lieux où il se trouvait. Il avait
toujours sa maison pleine de tout ce qu'il y avait de plus grand et de
meilleur. Les princes du sang n'en bougeaient, et il ne se contraignait
en rien pour eux ni pour personne\,; mais avec un air de liberté, de
politesse, de discernement qui lui était naturel, et qui séparait toute
idée d'orgueil d'avec la dignité et la liberté d'un homme qui ne veut ni
se contraindre ni contraindre les autres. Il avait les plus plaisantes
colères du monde, surtout en jouant, et avec cela le meilleur homme du
monde, et que tout le monde aimait.

Il avait le gouvernement général de Flandre et de Lille, où il tenait
comme une cour, et avait fait un beau lieu de Mouchy à deux lieues de
Compiègne dont il était capitaine. Le roi l'avait souvent aidé à
accommoder Mouchy, et y avait été plusieurs fois. M. de Louvois, qui à
la mort du duc du Lude voulut rogner l'office de grand maître de
l'artillerie en faveur de sa charge de secrétaire d'État, fit faire le
maréchal d'Humières grand maître en son absence, comme il revenait
d'Angleterre complimenter, de la part du roi, le roi Jacques II sur son
avènement à la couronne. Ce ministre contribua beaucoup à le faire faire
duc vérifié, et à lui faire accorder la grâce très singulière de faire
appeler dans ses lettres celui qui avec l'agrément du roi épouserait sa
dernière fille, belle comme le jour, et qu'il aimait passionnément. Il
avait perdu son fils unique sans alliance au siège de Luxembourg. Il
avait marié sa fille aînée au prince d'Isenghien en obtenant un tabouret
de grâce, et la seconde à Vassé, vidame du Mans, qui s'était remariée à
Surville, cadet d'Hautefort, dont elle avait été longtemps sans voir son
père.

Le maréchal mourut assez brusquement à Versailles. Il regretta amèrement
de n'avoir jamais pensé à son salut ni à sa santé\,; il pouvait ajouter
à ses affaires, et mourut pourtant fort chrétiennement, et fut
généralement regretté. On put remarquer qu'il fut assisté à la mort par
trois antagonistes, M. de Meaux et l'abbé de Fénelon qui écrivirent
bientôt après l'un contre l'autre, et le P. Caffaro, théatin, son
confesseur, qui, s'étant avisé d'écrire un livre en faveur de la comédie
pour la prouver innocente et permise, fut puissamment réfuté par M. de
Meaux.

Le maréchal de Boufflers eut le gouvernement de Lille et de la Flandre,
en se démettant de celui de Lorraine qui fut donné au maréchal de
Lorges, lequel sentit vivement cette préférence de son cadet, qui valant
beaucoup ne le valait pourtant pas. M. du Maine eut l'artillerie en
quittant les galères, qui furent données à M. de Vendôme en son absence.
Ainsi les bâtards durent être assez contents de cette année.

Le roi donna une pension de vingt mille livres à la maréchale
d'Humières, qui sans cela aurait été réduite à fort peu, et ce fut le
premier exemple d'une si forte pension à une femme. Elle était La
Châtre, avait été fort belle et riche, car elle était unique, et avait
été dame du palais de la reine. C'était une précieuse qui importunait
quelquefois le maréchal et toute sa bonne compagnie, et qui, avec un
livre de compte qu'elle avait toujours devant elle, croyait tout faire
et ne fit rien que se ruiner. Elle se retira dans une maison borgne au
dehors des carmélites du faubourg Saint-Jacques, s'y fit dévote en titre
d'office, et se mêla après de tout ce dont elle n'avait que faire, et
peu d'accord avec ses enfants.

\hypertarget{chapitre-xiii.}{%
\chapter{CHAPITRE XIII.}\label{chapitre-xiii.}}

1694

~

\relsize{-1}

{\textsc{Tracasseries de Monsieur et des princesses.}} {\textsc{-
Aventure de M\textsuperscript{me} la princesse de Conti, fille du roi,
qui chasse de chez elle M\textsuperscript{lle} Choin.}} {\textsc{-
Disgrâce, exil, etc., de Clermont.}} {\textsc{- Cabale et désarroi.}}
{\textsc{- M\textsuperscript{lle} Choin et Monseigneur.}} {\textsc{- M.
de Noyon, de l'Académie française, étrangement moqué par l'abbé de
Caumartin, qui en est perdu.}} {\textsc{- Grande action de M. de Noyon
sur l'abbé de Caumartin.}} {\textsc{- Dauphiné d'Auvergne et comté
d'Auvergne terres tout ordinaires.}} {\textsc{- Folie du cardinal de
Bouillon.}} {\textsc{- Changements chez Monsieur.}} \relsize{1}

~

Il était arrivé pendant la campagne quelques aventures aux princesses.
C'était le nom distinctif par lequel on entendait seulement les trois
filles du roi. Monsieur avait voulu avec raison que la duchesse de
Chartres appelât toujours les deux autres ma sœur\,; et que celles-ci ne
l'appelassent jamais que Madame. Cela était juste, et le roi le leur
avait ordonné, dont elles furent fort piquées. La princesse de Conti
pourtant s'y soumit de bonne grâce\,; mais M\textsuperscript{me} la
Duchesse, comme sœur d'un même amour, se mit à appeler
M\textsuperscript{me} de Chartres mignonne\,; or rien n'était moins
mignon que son visage, que sa taille, que toute sa personne. Elle n'osa
le trouver mauvais\,; mais quand, à la fin, Monsieur le sut, il en
sentit le ridicule, et l'échappatoire de l'appeler Madame, et il éclata.
Le roi défendit très sévèrement à M\textsuperscript{me} la Duchesse
cette familiarité qui en fut encore plus piquée, mais elle fit en sorte
qu'il n'y parût pas.

À un voyage de Trianon, ces princesses qui y couchaient, et qui étaient
jeunes, se mirent à se promener ensemble les nuits, et à se divertir la
nuit à quelques pétarades. Soit malice des deux aînées, soit imprudence,
elles en tirèrent une nuit sous les fenêtres de Monsieur qui
l'éveillèrent, et qui le trouva fort mauvais\,; il en porta ses plaintes
au roi qui lui fit force excuses, gronda fort les princesses, et eut
grand'peine à l'apaiser. Sa colère fut surtout domestique\,:
M\textsuperscript{me} la duchesse de Chartres s'en sentit longtemps, et
je ne sais si les deux autres en furent fort fâchées. On accusa même
M\textsuperscript{me} la Duchesse de quelques chansons sur
M\textsuperscript{me} de Chartres. Enfin tout fut replâtré, et Monsieur
pardonna tout à fait à M\textsuperscript{me} de Chartres par une visite
qu'il reçut à Saint-Cloud de M\textsuperscript{me} de Montespan qu'il
avait toujours, fort aimée, qui raccommoda aussi ses deux filles, et qui
avait conservé de l'autorité sur elles, et en recevait de grands
devoirs.

M\textsuperscript{me} la princesse de Conti eut une autre aventure qui
fit grand bruit et qui eut de grandes suites. La comtesse de Bury avait
été mise auprès d'elle pour être sa dame d'honneur à son mariage.
C'était une femme d'une grande vertu, d'une grande douceur et d'une
grande politesse, avec de l'esprit et de la conduite\,; elle était
d'Aiguebonne et veuve sans enfants, en 1666, d'un cadet de Rostaing,
frère de la vieille Lavardin, mère du chevalier de l'ordre, ambassadeur
à Rome. M\textsuperscript{me} de Bury avait fait venir de Dauphiné
M\textsuperscript{lle} Choin, sa nièce, qu'elle avait mise fille
d'honneur de M\textsuperscript{me} la princesse de Conti. C'était une
grosse fille écrasée, brune, laide, camarde, avec de l'esprit et un
esprit d'intrigue et de manège. Elle voyait sans cesse Monseigneur qui
ne bougeait de chez M\textsuperscript{me} la princesse de Conti. Elle
l'amusa, et sans qu'on s'en aperçût se mit intimement dans sa confiance.
M\textsuperscript{me} de Lislebonne et ses deux filles, qui ne sortaient
pas non plus de chez la princesse de Conti, et qui étaient parvenues à
l'intimité de Monseigneur, s'aperçurent les premières de la confiance
entière que la Choin avait acquise, et devinrent ses meilleures amies.
M. de Luxembourg qui avait le nez bon l'écuma. Le roi ne l'aimait point
et ne se servait de lui que par nécessité\,; il le sentait, et s'était
entièrement tourné vers Monseigneur. M. le prince de Conti l'y avait mis
fort bien, et le duc de Montmorency son fils. Outre l'amitié, ce prince
ménageait fort ce maréchal pour en être instruit et vanté, dans
l'espérance d'arriver au commandement des armées\,; et la débauche avait
achevé de les unir étroitement. La jalousie de M. de Vendôme, en tout
genre contre le prince de Conti, n'osant s'en prendre ouvertement à lui,
l'avait brouillé avec M. de Luxembourg, et fait choisir l'armée de
Catinat, où il n'avait rien au-dessus de lui\,; et M. du Maine, par la
jalousie des préférences, n'était pas mieux avec le général. Tout cela
l'attachait de plus en plus au prince de Conti, et le tournait vers
Monseigneur avec plus d'application, et c'est ce qui fit que Monseigneur
avait préféré la Flandre à l'Allemagne, où le roi le voulait envoyer,
qui commençait à sentir quelque chose des intrigues de M. de Luxembourg
auprès de Monseigneur.

Ce prince avait pris du goût pour Clermont, de la branche de Chattes,
enseigne des gens d'armes de la garde. C'était un grand homme,
parfaitement bien fait, qui n'avait rien que beaucoup d'honneur, de
valeur, avec un esprit assez propre à l'intrigue, et qui s'attacha à M.
de Luxembourg à titre de parenté. Celui-ci se fit honneur de le
ramasser, et bientôt il le trouva propre à ses desseins\,: il s'était
introduit chez M\textsuperscript{me} la princesse de Conti\,; il en
avait fait l'amoureux\,; elle la devint bientôt de lui\,; avec ses
appuis il devint bientôt un favori de Monseigneur, et déjà initié avec
M. de Luxembourg, il entra dans toutes les vues que M. le prince de
Conti et lui s'étaient proposées, de se rendre les maîtres de l'esprit
de Monseigneur et de le gouverner, pour disposer de l'État quand il en
serait devenu le maître.

Dans cet esprit ils avisèrent Clermont de s'attacher à la Choin, d'en
devenir l'amant, et de paraître vouloir l'épouser. Ils lui confièrent ce
qu'ils avaient découvert de Monseigneur à son égard, et que ce chemin
était sûrement pour lui celui de la fortune. Clermont, qui n'avait rien,
les crut bien aisément\,: il fit son personnage, et ne trouva point la
Choin cruelle\,; l'amour qu'il feignait, mais qu'il lui avait donné, y
mit la confiance\,; elle ne se cacha plus à lui de celle de Monseigneur,
ni bientôt Monseigneur ne lui fit plus mystère de son amitié pour la
Choin\,; et bientôt après la princesse de Conti fut leur dupe. Là-dessus
on partit pour l'armée, où Clermont eut toutes les distinctions que M.
de Luxembourg lui put donner.

Le roi, inquiet de ce qu'il entrevoyait de cabale auprès de son fils,
les laissa tous partir, et n'oublia pas d'user du secret de la poste\,;
les courriers lui en dérobaient souvent le fruit, mais à la fin
l'indiscrétion de ne pas tout réserver aux courriers trahit l'intrigue.
Le roi eut de leurs lettres\,; il y vit le dessein de Clermont et de la
Choin de s'épouser, leur amour, leur projet de gouverner Monseigneur et
présentement et après lui\,; combien M. de Luxembourg était l'âme de
toute cette affaire, et les merveilles pour soi qu'il s'en proposait.
L'excès du mépris de la Choin et de Clermont pour la princesse de Conti,
de qui Clermont lui sacrifia les lettres que le roi eut pour ce même
paquet intercepté à la poste, après beaucoup d'autres dont il faisait
rendre les lettres après en avoir pris les extraits, et avec ce paquet
une lettre de Clermont accompagnant le service, où la princesse de Conti
était traitée sans ménagement, où Monseigneur n'était marqué que sous le
nom de leur gros ami, et où tout le cœur semblait se répandre. Alors le
roi crut en avoir assez, et une après-dînée de mauvais temps qu'il ne
sortit point, il manda à la princesse de Conti de lui venir parler dans
son cabinet. Il en avait aussi des lettres à Clermont et des lettres de
Clermont à elle où leur amour était fort exprimé, et dont la Choin et
lui se moquaient ensemble.

La princesse de Conti qui comme ses sœurs n'allait jamais chez le roi
qu'entre son souper et son coucher, hors des étiquettes de sermon ou des
chasses, se trouva bien étonnée du message. Elle s'en alla chez le roi
fort en peine de ce qu'il lui voulait, car il était redouté de son
intime famille, plus s'il se peut encore que de ses autres sujets. Sa
dame d'honneur demeura dans un premier cabinet, et le roi l'emmena plus
loin\,; là, d'un ton sévère, il lui dit qu'il savait tout, et qu'il
n'était pas question de lui dissimuler sa faiblesse pour Clermont, et
tout de suite ajouta qu'il avait leurs lettres, et les lui tira de sa
poche en lui disant\,: « Connaissez-vous cette écriture\,? \,» qui était
la sienne, puis celle de Clermont. À ce début la pauvre princesse se
trouva mal, la pitié en prit au roi qui la remit comme il put, et qui
lui donna les lettres sur lesquelles il la chapitra, mais assez
humainement\,; après il lui dit que ce n'était pas tout, et qu'il en
avait d'autres à lui montrer par lesquelles elle verrait combien elle
avait mal placé ses affections, et à quelle rivale elle était sacrifiée.
Ce nouveau coup de foudre, peut-être plus accablant que le premier,
renversa de nouveau la princesse. Le roi la remit encore, mais ce fut
pour en tirer un cruel châtiment\,: il voulut qu'elle lût en sa présence
ses lettres sacrifiées et celles de Clermont et de la Choin. Voilà où
elle pensa mourir, et elle se jeta aux pieds du roi baignée de ses
larmes, et ne pouvant presque articuler\,; ce ne fut que sanglots,
pardons, désespoirs, rages, et à implorer justice et vengeance\,; elle
fut bientôt faite. La Choin fut chassée le lendemain, et M. de
Luxembourg eut ordre en même temps d'envoyer Clermont dans la place la
plus voisine qui était Tournai, avec celui de se défaire de sa charge,
et de se retirer après en Dauphiné pour ne pas sortir de la province. En
même temps le roi manda à Monseigneur ce qui s'était passé entre lui et
sa fille, et par là le mit hors de mesure d'oser protéger les deux
infortunés. On peut juger de la part que le prince de Conti, mais
surtout M. de Luxembourg et son fils, prirent à cette découverte, et
combien la frayeur saisit les deux derniers.

Cependant comme l'amitié de Monseigneur pour la Choin avait été
découverte par ces mêmes lettres, la princesse de Conti n'osa ne pas
garder quelques mesures. Elle envoya M\textsuperscript{lle} Choin dans
un de ses carrosses à l'abbaye de Port-Royal à Paris, et lui donna une
pension et des voitures pour emporter ses meubles. La comtesse de Bury,
qui ne s'était doutée de rien sur sa nièce, fut inconsolable et voulut
se retirer bientôt après.

M\textsuperscript{me} de Lislebonne et ses filles se hâtèrent d'aller
voir la Choin, mais avec un extrême secret. C'était le moyen sûr de
tenir immédiatement à Monseigneur\,; mais elles ne voulaient pas se
hasarder du côté du roi ni de la princesse de Conti qu'elles avaient
toutes sortes de raisons de ménager avec la plus grande délicatesse.
Elles étaient princesses, mais le plus souvent sans habits et sans pain,
à la lettre, par le désordre de M. de Lislebonne. M. de Louvois leur en
avait donné souvent. M\textsuperscript{me} la princesse de Conti les
avait attirées à la cour, les y nourrissait, leur faisait des présents
continuels, leur y procurait toutes sortes d'agréments, et c'était à
elle qu'elles avaient l'obligation d'avoir été connues de Monseigneur,
puis admises dans sa familiarité, enfin dans son amitié la plus déclarée
et la plus distinguée. Les chansons achevèrent de célébrer cette étrange
aventure de la princesse et de sa confidente.

M. de Noyon en avait fourni une autre à notre retour, qui lui fut
d'autant plus sensible, qu'elle divertit fort tout le monde à ses
dépens. On a vu, dès l'entrée de ces Mémoires, quel était ce prélat. Le
roi s'amusait de sa vanité qui lui faisait prendre tout pour
distinction, et les effets de cette vanité feraient un livre. Il vaqua
une place à l'Académie française, et le roi voulut qu'il en fût. Il
ordonna même à Dangeau qui en était, de s'en expliquer de sa part aux
académiciens. Cela n'était jamais arrivé, et M. de Noyon, qui se piquait
de savoir, en fut comblé, et ne vit pas que le roi se voulait divertir.
On peut croire que le prélat eut toutes les voix sans en avoir brigué
aucune, et le roi témoigna à M. le Prince et à tout ce qu'il y avait de
distingué à la cour qu'il serait bien aise qu'ils se trouvassent à sa
réception. Ainsi M. de Noyon fut le premier du choix du roi dans
l'Académie, sans que lui-même y eût auparavant pensé, et le premier
encore à la réception duquel le roi eût pris le soin de convier.

L'abbé de Caumartin se trouvait alors directeur de l'Académie, et par
conséquent à répondre au discours qu'y ferait le prélat. Il en
connaissait la vanité et le style tout particulier à lui\,; il avait
beaucoup d'esprit et de savoir. Il était jeune et frère de différent lit
de Caumartin, intendant des finances, fort à la mode en ce temps-là, et
qui les faisait presque toutes sous Pontchartrain, contrôleur général,
son parent proche et son ami intime. Cette liaison rendait l'abbé plus
hardi, et, se comptant sûr d'être approuvé du monde et soutenu du
ministre, il se proposa de divertir le public aux dépens de l'évêque
qu'il avait à recevoir. Il composa donc un discours confus et imité au
possible du style de M. de Noyon, qui ne fut qu'un tissu des louanges
les plus outrées et de comparaisons emphatiques dont le pompeux
galimatias fut une satire continuelle de la vanité du prélat, qui le
tournait pleinement en ridicule.

Cependant, après avoir relu son ouvrage, il en eut peur, tant il le
trouva au delà de toute mesure\,; pour se rassurer, il le porta à M. de
Noyon comme un écolier à son maître, et comme un jeune homme à un grand
prélat qui ne voulait rien omettre des louanges qui lui étaient dues, ni
rien dire aussi qui ne fût de son goût, et qui ne méritât son
approbation. Ce respect si attentif combla l'évêque\,; il lut et relut
le discours, il en fut charmé, mais il ne laissa pas d'y faire quelques
corrections pour le style et d'y ajouter quelques traits de sa propre
louange. L'abbé revit son ouvrage de retour entre ses mains avec grand
plaisir\,; mais quand il y trouva les additions de la main de M. de
Noyon et ses ratures, il fut comblé à son tour du succès du piège qu'il
lui avait tendu, et d'avoir en main un témoignage de son approbation qui
le mettait à couvert de toute plainte.

Le jour venu de la réception, le lieu fut plus que rempli de tout ce que
la cour et la ville avaient de plus distingué. On s'y portait dans le
désir d'en faire sa cour au roi, et dans l'espérance de s'y divertir. M.
de Noyon parut avec une nombreuse suite, saluant et remarquant
l'illustre et nombreuse compagnie avec une satisfaction qu'il ne
dissimula pas, et prononça sa harangue avec sa confiance ordinaire, dont
la confusion et le langage remplirent l'attente de l'auditoire. L'abbé
de Caumartin répondit d'un air modeste, d'un ton mesuré, et, par de
légères inflexions de voix aux endroits les plus ridicules ou les plus
marqués au coin du prélat, aurait réveillé l'attention de tout ce qui
l'écoutait, si la malignité publique avait pu être un moment distraite.
Celle de l'abbé, toute brillante d'esprit et d'art, surpassa tout ce
qu'on en aurait pu attendre si on avait prévu la hardiesse de son
dessein, dont la surprise ajouta infiniment au plaisir qu'on y prit.
L'applaudissement fut donc extrême et général, et chacun, comme de
concert, enivrait M. de Noyon de plus en plus, en lui faisant accroire
que son discours méritait tout par lui-même, et que celui de l'abbé
n'était goûté que parce qu'il avait su le louer dignement. Le prélat
s'en retourna charmé de l'abbé et du public, et ne conçut jamais la
moindre défiance.

On peut juger du bruit que fit cette action, et quel put être le
personnage de M. de Noyon se louant dans les maisons et par les
compagnies et de ce qu'il avait dit et de ce qui lui avait été répondu,
et du nombre et de l'espèce des auditeurs, et de leur admiration
unanime, et des bontés du roi à cette occasion. M. de Paris, chez lequel
il voulut aller triompher, ne l'aimait point. Il y avait longtemps qu'il
avait sur le cœur une humiliation qu'il en avait essuyée\,; il n'était
point encore duc, et la cour était à Saint-Germain, où il n'y avait
point de petites cours comme à Versailles. M. de Noyon, y entrant dans
son carrosse, rencontra M. de Paris à pied\,; il s'écrie, M. de Paris va
à lui, et croit qu'il va mettre pied à terre\,; point du tout\,; il le
prend de son carrosse par la main, et le conduit ainsi en laisse
jusqu'aux degrés, toujours parlant, et complimentant l'archevêque qui
rageait de tout son cœur. M. de Noyon, toujours sur le même ton, monta
avec lui et fit si peu semblant de soupçonner d'avoir rien fait de mal à
propos, que M. de Paris n'osa en faire une affaire\,; mais il ne le
sentit pas moins. Cet archevêque, à force d'être bien avec le roi, de
présider aux assemblées du clergé avec toute l'autorité et les grâces
qu'on lui a connues, et d'avoir part à la distribution des bénéfices
qu'il perdit enfin, s'était mis peu à peu au-dessus de faire aucune
visite aux prélats, même les plus distingués, quoique tous allassent
souvent chez lui. M. de Noyon s'en piqua et lui en parla fort
intelligiblement. C'étaient toujours des excuses. Voyant enfin que ces
excuses dureraient toujours, il en parla si bien au roi, qu'il l'engagea
à ordonner à M. de Paris de l'aller voir. Ce dernier en fut d'autant
plus mortifié qu'il n'osa plus y manquer aux occasions et aux arrivées,
et que cette exception l'embarrassa avec d'autres prélats considérables.

On peut donc imaginer quelle farce ce fut pour M. de Paris que cette
réception d'Académie\,; mais qu'il n'en pourrait être pleinement
satisfait tant que M. de Noyon continuerait de s'en applaudir\,; aussi
ne manqua-t-il pas l'occasion de sa visite pour lui ouvrir les yeux et
lui faire entendre, comme son serviteur et son confrère, ce qu'il
n'osait lui dire entièrement. Il tourna longtemps sans pouvoir être
entendu par un homme si rempli de soi-même, et si loin d'imaginer qu'il
fût possible de s'en moquer\,; à la fin pourtant il se fit écouter, et
pour l'honneur de l'épiscopat insulté, disait-il, par un jeune homme, il
le pria de n'en pas augmenter la victoire par une plus longue duperie,
et de consulter ses vrais amis. M. de Noyon jargonna longtemps avant de
se rendre, mais à la fin il ne put se défendre des soupçons, et de
remercier l'archevêque avec qui il convint d'en parler au P. de La
Chaise qui était de ses amis. Il y courut en effet au sortir de
l'archevêché. Il dit au P. de La Chaise l'inquiétude qu'il venait de
prendre, et le pria tant de lui parler de bonne foi, que le confesseur,
qui de soi était bon, et qui balançait entre laisser M. de Noyon dans
cet extrême ridicule, et faire une affaire à l'abbé de Caumartin, ne put
enfin se résoudre à tromper un homme qui se fiait à lui, et lui
confirma, le plus doucement qu'il put, la vérité que l'archevêque de
Paris lui a voit le premier apprise. L'excès de la colère et du dépit
succéda à l'excès du ravissement. Dans cet état il retourna chez lui, et
alla le lendemain à Versailles, où il fit au roi les plaintes les plus
amères de l'abbé de Caumartin, dont il était devenu le jouet, et la
risée de tout le monde.

Le roi, qui avait bien voulu se divertir un peu, mais qui voulait
toujours partout un certain ordre et une certaine bienséance, avait déjà
su ce qui s'était passé, et l'avait trouvé fort mauvais. Ces plaintes
l'irritèrent d'autant plus qu'il se sentit la cause innocente d'une
scène si ridicule et si publique, et que, quoiqu'il aimât à s'amuser des
folies de M. de Noyon, il ne laissait pas d'avoir pour lui de la bonté
et de la considération. Il envoya chercher Pontchartrain, et lui
commanda de laver rudement la tête à son parent, et de lui expédier une
lettre de cachet pour aller se mûrir la cervelle, et apprendre à rire et
à parler dans son abbaye de Busay en Bretagne. Pontchartrain n'osa
presque répliquer\,: il exécuta bien la première partie de son ordre,
pour l'autre il la suspendit au lendemain, demanda grâce, fit valoir la
jeunesse de l'abbé, la tentation de profiter du ridicule du prélat, et
surtout la réponse corrigée et augmentée de la main de M. de Noyon, qui,
puisqu'il l'avait examinée de la sorte, n'avait qu'à se prendre à
lui-même de n'y avoir pas aperçu ce que tout le monde avait cru y voir.
Cette dernière raison, habilement maniée par un ministre agréable et de
beaucoup d'esprit, fit tomber la lettre de cachet, mais non pas
l'indignation. Pontchartrain pour cette fois n'en demandait pas
davantage. Il fit valoir le regret et la douleur de l'abbé, et sa
disposition d'aller demander pardon à M. de Noyon, et lui témoigner
qu'il n'avait jamais eu l'intention de lui manquer de respect et de lui
déplaire. En effet, il lui fit demander la permission d'aller lui faire
cette soumission\,; mais l'évêque outré ne la voulut point recevoir, et
après avoir éclaté sans mesure contre les Caumartin, s'en alla passer sa
honte dans son diocèse, où il demeura longtemps.

Il faut dire tout de suite que, peu après son retour à Paris, il tomba
si malade qu'il reçut ses sacrements. Avant de les recevoir, il envoya
chercher l'abbé de Caumartin, lui pardonna, l'embrassa, tira de son
doigt un beau diamant qu'il le pria de garder et de porter pour l'amour
de lui, et quand il fut guéri il fit auprès du roi tout ce qu'il put
pour le raccommoder\,; il y a travaillé toute sa vie avec chaleur et
persévérance, et n'a rien oublié pour le faire évêque, mais ce trait
l'avait radicalement perdu dans l'esprit du roi, et M. de Noyon n'en eut
que le bien devant Dieu par cette grande action, et l'honneur devant le
monde.

L'orgueil du cardinal de Bouillon donna vers ce même temps une autre
sorte de scène. Pour l'entendre il faut dire qu'il y a dans la province
d'Auvergne deux terres particulières dont l'une s'appelle le comté
d'Auvergne, l'autre le dauphiné d'Auvergne. Le comté a une étendue
ordinaire et des mouvances ordinaires d'une terre ordinaire sans droits
singuliers, et sans rien de distingué de toutes les autres. Comment elle
a retenu ce nom et le dauphiné le sien, mènerait à une dissertation trop
longue. Le dauphiné est encore plus petit en étendue que le comté, et
bien qu'érigé en princerie, n'a ni rang ni distinction par-dessus les
autres terres, ni droits particuliers, et n'a jamais donné aucune
prétention à ceux qui l'ont possédé. Mais la distinction du nom de
prince-dauphin avait plu à la branche de Montpensier qui possédait cette
terre dont quelques-uns ont porté ce titre du vivant de leur père avant
de devenir ducs de Montpensier. Le comté d'Auvergne tel qu'il vient
d'être dépeint était entré et sorti de la maison de La Tour par des
mariages et des successions. Ce nom était friand pour des gens qui
minutaient de changer leur nom de La Tour en celui d'Auvergne, et ils
firent si bien auprès du roi lors et depuis l'échange de Sedan, que
cette terre est rentrée chez eux, et c'est de là que le frère du duc et
du cardinal de Bouillon porte le nom de comte d'Auvergne.

Le dauphiné d'Auvergne était échu à Monsieur par la succession de
Mademoiselle, et aussitôt le cardinal avait conçu une envie démesurée de
l'avoir. Il en parla à Bechameil qui était surintendant de Monsieur, au
chevalier de Lorraine, et fit sa cour à tous ceux qui pouvaient avoir
part à déterminer Monsieur à le lui vendre. À la fin et à force de
donner gros, le marché fut conclu, et Monsieur en parla au roi, qui
s'était chargé de son agrément comme d'une bagatelle\,; mais il fut
surpris de trouver le roi sur la négative. Monsieur insista et ne
pouvait la comprendre\,: « Je parie, mon frère, lui dit le roi, que
c'est une nouvelle extravagance du cardinal de Bouillon qui veut faire
appeler un de ses neveux prince-dauphin\,; dégagez-vous de ce marché.\,»
Monsieur, qui avait promis et qui trouvait le marché bon, insista\,;
mais le roi tint bon, et dit à Monsieur qu'il n'avait qu'à faire mander
au cardinal qu'il ne le voulait pas.

Cette réponse lui fut écrite par le chevalier de Lorraine de la part de
Monsieur, et le pénétra de dépit. Ce nom singulier et propre à éblouir
les sots dont le nombre est toujours le plus grand, et un nom que des
princes du sang avaient porté, avait comblé son orgueil de joie\,; le
refus le combla de douleur. N'osant se prendre au roi, il répondit au
chevalier de Lorraine un fatras de sottises qu'il couronna par ajouter
qu'il était d'autant plus affligé de ce que Monsieur lui manquait de
parole, que cela l'empêcherait d'être désormais autant son serviteur
qu'il l'avait été par le passé. Monsieur eut plus envie de rire de cette
espèce de déclaration de guerre que de s'en offenser. Le roi d'abord la
prit plus sérieusement, mais touché par les prières de M. de Bouillon,
et plus encore par la grandeur du châtiment d'une pareille insolence si
elle était prise comme elle le méritait, il prit le parti de l'ignorer,
et le cardinal de Bouillon en fut quitte pour la honte et pour s'aller
cacher une quinzaine dans sa belle maison de Saint-Martin de Pontoise,
que par un échange il avait depuis peu trouvé moyen de séculariser, et
de faire de ce prieuré un bien héréditaire et patrimonial.

Le marquis d'Arcy était mort à Maubeuge, à l'ouverture de la campagne\,:
de gouverneur de M. le duc de Chartres il était devenu premier
gentilhomme de sa chambre et le directeur discret de sa conduite. Ce
prince, qui eut le bon esprit de sentir tout ce qu'il valait, l'a
regretté toute sa vie et l'a témoigné, par tous les effets qu'il a pu, à
sa famille, et jusqu'à ses domestiques. Il était chevalier de l'ordre de
1688, conseiller d'État d'épée, et avait été ambassadeur en Savoie.
C'était un homme d'une vertu et d'une capacité peu communes, sans nulle
pédanterie et fort rompu au grand monde, et un très vaillant homme sans
nulle ostentation. Un roi à élever et à instruire eût été dignement et
utilement remis entre ses mains. Il n'était point marié ni riche, et
n'avait guère que soixante ans\,; homme bien fait et de fort bonne mine.
Au retour de l'armée on fut surpris de celui que le roi mit auprès de
son neveu pour le remplacer. Ce fut Cayeu, brigadier de cavalerie, brave
et très honnête gentilhomme, qui buvait bien et ne savait rien au delà.
M. de Chartres fut fort aise d'avoir affaire à un tel inspecteur dont il
se moqua, et le fit tomber dans tous les panneaux qu'il lui tendit.

Il y avait eu aussi pendant la campagne quelques changements chez
Monsieur. Il permit à Châtillon, son ancien favori, de vendre à son
frère aîné la moitié de sa charge de premier gentilhomme de sa chambre.
Châtillon avait épousé par amour M\textsuperscript{lle} de Pienne\,;
c'était, sans contredit, le plus beau couple de la cour, et la mieux
fait, et du plus grand air. Ils se brouillèrent et se séparèrent à ne se
jamais revoir. Elle était dame d'atours de Madame, et sœur de la
marquise de Villequier, aussi mariée par amour. M. d'Aumont avait été
des années sans y vouloir consentir. Enfin, M\textsuperscript{me} de
Maintenon s'en mêla, parce que la mère de cette belle était parente et
de même nom que l'évêque de Chartres, directeur de Saint-Cyr et de
M\textsuperscript{me} de Maintenon, laquelle enfin en était venue à
bout. Le comte de Tonnerre, neveu de M. de Noyon, dont je viens de
parler, vendit aussi l'autre charge de premier gentilhomme de la chambre
de Monsieur, qu'il avait depuis longtemps, à Sassenage qui quitta le
service. Tonnerre avait beaucoup d'esprit, mais c'était tout\,; il en
partait souvent des traits extrêmement plaisants et salés, mais qui lui
attiraient des aventures qu'il ne soutenait pas, et qui ne purent le
corriger de ne se rien refuser, et il était parvenu enfin a cet état,
qu'il eût été honteux d'avoir une querelle avec lui\,; aussi ne se
contraignait-on point sur ce qu'on voulait lui répondre ou lui dire. Il
était depuis longtemps fort mal dans sa petite cour par ses bons mots.
Il lui avait échappé de dire qu'il ne savait ce qu'il faisait de
demeurer en cette boutique\,; que Monsieur était la plus sotte femme du
monde, et Madame le plus sot homme qu'il eût jamais vu. L'un et l'autre
le surent, et en furent très offensés. Il n'en fut pourtant autre
chose\,; mais le mélange des brocards sur chacun et du mépris extrême
qu'il avait acquis, le chassèrent à la fin pour mener une vie fort
pitoyable.

\hypertarget{chapitre-xiv.}{%
\chapter{CHAPITRE XIV.}\label{chapitre-xiv.}}

1694

~

\relsize{-1}

{\textsc{Directeurs et inspecteurs en titre.}} {\textsc{- Horrible
trahison qui conserve Barcelone à l'Espagne pour perdre M. de
Noailles.}} {\textsc{- Établissement de la capitation.}} {\textsc{-
Comte de Toulouse reçu au parlement et installé à la table de marbre par
Harlay, premier président.}} {\textsc{- Procès de M. le prince de Conti
contre M\textsuperscript{me} de Nemours pour les biens de Longueville.}}
{\textsc{- Un bâtard obscur du dernier comte de Soissons, prince du
sang, comblé de biens par M\textsuperscript{me} de Nemours.}} {\textsc{-
Il prend le nom de prince de Neuchâtel, et épouse la fille de M. de
Luxembourg.}} \relsize{1}

~

Lors même de ce retour des armées, le roi créa huit directeurs généraux
de ses troupes et deux inspecteurs sous chaque directeur. M. de Louvois,
pour en être plus maître et anéantir l'autorité des colonels, avait
imaginé d'envoyer des officiers de son choix, sous le nom de celui du
roi, voir les troupes par frontière et par district, et de leur donner
tout crédit et toute confiance. Le roi, content que c'était la meilleure
chose du monde pour son service, et encore piqué de n'avoir jamais pu
tirer la charge de colonel général de la cavalerie des mains du comte
d'Auvergne pour M. du Maine, voulut ajouter à ce que M. de Louvois avait
inventé, et s'en servir à des récompenses. Il donna douze mille livres
d'appointements aux directeurs et une autorité fort étendue sur tout le
détail des troupes de leur dépendance. Chacun d'eux devait faire deux
revues par an, en sortant de campagne et à la fin de l'hiver, et entre
ces deux revues les inspecteurs devaient en faire plusieurs. Ils eurent
six mille livres, devaient rendre compte de tout à leur directeur, et
celui-ci au secrétaire d'État de la guerre, et quelquefois au roi,
chaque département de directeur séparé en deux pour les deux
inspecteurs, desquels tous la moitié était fixée à l'infanterie et
l'autre moitié à la cavalerie\,; outre un pouvoir étendu en toute espèce
de détails de troupes\,; les directeurs les pouvaient voir en campagne,
mettre aux arrêts, interdire même les brigadiers de cavalerie et
d'infanterie\,; et les inspecteurs, qui furent tous pris d'entre les
brigadiers, eurent un logement au quartier général, et dispense de leur
service de brigadiers pendant la campagne. Telle fut la fondation de ces
emplois qui blessa extrêmement les officiers généraux de la cavalerie et
des dragons.

Le comte d'Auvergne, nourri de couleuvres sur sa charge depuis
longtemps, avala encore celle-ci en silence. Rosen, étranger et soldat
de fortune jusqu'à avoir tiré un billet pour maraude, quoique de bonne
noblesse de Poméranie, devenu lieutenant général et mestre de camp
général de la cavalerie, était un matois rusé qui n'avait garde de se
blesser, et qui loua au contraire cet établissement. Villars, lieutenant
général et commissaire général de la cavalerie, ébloui de sa fortune et
de celle de son père, se fit moquer des deux autres à qui il proposa de
s'opposer à une nouveauté si préjudiciable à leurs charges, et encore
plus du roi à qui il osa en parler. Huxelles pour l'infanterie et du
Bourg pour la cavalerie eurent la direction du Rhin\,; ils se
retrouveront ailleurs\,: le premier lieutenant général et chevalier de
l'ordre, l'autre maréchal de camp. Chamarande et Vaudray, deux hommes
distingués par leur valeur, par leur application et par leur mérite\,:
Vaudray était d'une naissance fort distinguée, du comté de Bourgogne,
singulièrement bien fait, mais cadet et pauvre. De chanoine de Besançon,
il prit un mousquet, devint capitaine de grenadiers et reçut trente-deux
blessures, dont plusieurs presque mortelles, à l'attaque de la
contrescarpe de Coni sans vouloir quitter prise, et y fut laissé pour
mort. Cette action le fit connaître, et lui valut peu après le régiment
de la Sarre. Chamarande avait été premier valet de chambre du roi en
survivance de son père qui l'avait achetée de Beringhen, et en avait
conservé toutes les entrées. Le père était de ces sages que tout le
monde révérait pour sa probité à toute épreuve et pour sa modestie. Il
avait vendu sa charge, et le roi, qui l'aimait et le considérait fort
au-dessus de son état, l'avait fait premier maître d'hôtel de
M\textsuperscript{me} la Dauphine lors du mariage de Monseigneur. Il fit
cette charge au gré de toute la cour et eut toujours la meilleure
compagnie à sa table. Son fils eut encore sa survivance. Ayant perdu sa
charge avec sa maîtresse, il demeura à la cour et y eut toujours chez
lui la plus illustre compagnie, quoiqu'il n'eût plus de table, qu'il fût
perclus de goutte, et qu'on ne vît jamais de vivres chez lui. Le roi
envoyait quelquefois savoir de ses nouvelles, car il ne pouvait plus
marcher, et lui faire des amitiés\,; et je me souviens qu'il était en
telle estime que, lorsque mon père me présenta au roi et ensuite à ce
qu'il y avait de plus principal à la cour, il me mena voir Chamarande.
Son fils était fort joliment fait, discret, sage, respectueux, et fort
au gré des dames du meilleur air. Il eut par degrés le régiment de la
reine, et se distingua fort à la guerre. M. le Duc, M. le prince de
Conti, M. de La Rocheguyon et de Liancourt, MM. de Luxembourg père et
fils, et quantité d'autres des plus distingués l'aimaient fort, et
vivaient avec lui en confiance et en société. Monseigneur le traitait
fort bien et avec distinction, quoique la difficulté de manger avec lui
l'empêchât d'être de ses parties et de ses voyages. Mais le rare avec
cela est qu'ayant épousé M\textsuperscript{lle} d'Anglure, fille du
comte de Bourlaymont, unique et riche, et femme d'un vrai mérite, sa
naissance aidée de ce mérite et de l'amitié du roi pour le bonhomme
Chamarande la fit entrer enfin dans les carrosses de
M\textsuperscript{me} la Dauphine.

Romainville et Montgomery furent les deux inspecteurs pour la cavalerie.
De Romainville, j'en ai déjà parlé, vieil officier extrêmement aimé et
estimé, et qui méritait de l'être. Le nom de l'autre annonce sa haute
naissance\,; mais sa pauvreté profonde l'avait réduit aux plus étranges
extrémités en ses premières années, d'autant plus cruelles à supporter
qu'il sentait le poids de son nom, et était pétri d'honneur et de
vertus. Parvenu à grand'peine à une compagnie de cavalerie, il se
distingua tellement en un petit combat contre le général Massiette qui
était dehors avec un fort gros parti, que Massiette qui l'avait pris le
renvoya sur sa parole comblé d'éloges. Le roi qui commandait son armée
le loua extrêmement, lui donna une épée et un des plus beaux chevaux de
ceux qu'il montait, et lui fit l'honneur de le faire manger avec lui,
qu'aucun capitaine de cavalerie n'avait eu avant lui. Un mois après il
vaqua un régiment de cavalerie qu'il eut avec grande distinction, et
servit depuis avec application et soutint la réputation qu'il avait
acquise. Il aurait été plus aimé si la capacité lui avait permis d'être
moins inquiet, et si l'humeur n'avait pas été un nuage qu'on ne se
soucie pas toujours de percer pour trouver la vertu qu'il cache. Les
maréchaux de Duras et de Lorges, ses parents, le protégeaient fort, et
encore plus M. de La Feuillade, tant qu'il vécut, attaché au char de
M\textsuperscript{me} de Quintin, chez qui Montgomery logeait à Paris,
tous deux enfants des deux frères. Il s'était de nouveau signalé à la
bataille de Staffarde où il eut une main estropiée. Il ne laissa pas
d'avoir la double douleur de voir du Bourg, son cadet, maréchal de camp
et directeur, et lui d'être brigadier et inspecteur sous lui. On cria
fort et de la préférence et de cette espèce d'affectation, et
Montgomery, bien qu'outré, n'osa refuser, et se conduisit avec beaucoup
de sagesse.

Besons qui n'était que brigadier de cavalerie, et Artagnan, major du
régiment des gardes françaises, eurent les deux directions de Flandre.
Je parlerai d'eux ailleurs. Coigny, beau-frère de MM. de Matignon, et le
vieux Genlis {[}furent{]} directeurs en Catalogne, avec Nanclas et le
marquis du Gambout sous eux\,; et en Italie Larré et Saint-Sylvestre, et
Villepion-Chartaigne et le comte de Chamilly sous eux.

Avant de quitter la guerre de cette année, il la faut finir par un
étrange incident. M. de Noailles et M. de Barbezieux étaient fort mal
ensemble\,: tous deux bien avec le roi, tous deux hauts, tous deux
gâtés. M. de Noailles avait soutenu et obtenu quantité de choses dans
son gouvernement de Roussillon, qui l'y rendaient fort maître et fort
indépendant du secrétaire d'État de la guerre. M\textsuperscript{me} de
Maintenon, ennemie de M. de Louvois, l'y avait aidé, et le fils encore
moins autorisé que le père n'avait pu y rien changer. Il n'aimait point
M. de Luxembourg, très lié à M. de Noailles, et de tout cela naissait un
groupe de chaque côté qui se regardait fort de travers.

Les succès de M. de Noailles, cette année en Catalogne, avaient outré
Barbezieux. Il en craignait de nouveaux comme des avant-coureurs de sa
perte, par le crédit augmenté de ses ennemis. Tout ce qui avait été
exécuté en Catalogne aplanissait les voies du siège de Barcelone, et
cette conquête mettait le sceau à celle de toute cette principauté, et
mettait le roi en état d'attaquer avec succès à la fin de l'hiver le
cœur de l'Espagne. Il avait toujours eu ce but, et M. de Noailles qui
savait par le roi même l'affection qu'il avait à ce projet, et qui en
vit enfin les moyens si avancés, n'en souhaitait pas moins l'exécution,
et avec d'autant plus d'ardeur, qu'elle assurerait solidement la
vice-royauté qu'il avait obtenue, augmenterait son éclat et sa faveur,
et le rendait nécessairement le général de l'armée qui attaquerait
l'année suivante l'Espagne, par les endroits les plus sensibles et les
plus aisés à pénétrer, et à la forcer à demander la paix dont il aurait
toute la gloire. Il pressa donc le roi de donner ses ordres à temps pour
le mettre en état d'entreprendre ce siège avec sûreté, et M. de
Barbezieux qu'il mettait au désespoir n'osait manquer â ce qui lui était
prescrit, et qui était éclairé par le double intérêt de M. de Noailles
de ne manquer de rien à temps, et de ne le pas ménager s'il n'avait
toutes choses à point.

Une flotte de cinquante-deux vaisseaux partit le 3 octobre de Toulon,
chargée de cinq mille deux cents hommes de troupes prises en Provence de
celles de M. de Vendôme\,; et rien ne manquait plus que de mettre la
main à l'œuvre, lorsque M. de Noailles voulut rendre au roi un compte
particulier de tout et recevoir directement ses ordres, et le tout à
l'insu de M. de Barbezieux. Pour une commission si importante pour lui,
il choisit Genlis qui, étant sans bien et sans fortune, s'était donné à
lui, et qu'il ne faut pas confondre avec le vieux Genlis dont j'ai parlé
plus haut et à qui il ne cédait point. Ce Genlis gagna l'amitié de M. de
Noailles jusqu'à faire la jalousie de toute sa petite armée. M. de
Noailles lui procura un régiment et le poussa fort brusquement à la
brigade, puis à être fait maréchal de camp. Il avait de l'esprit et du
manège, et n'avait d'autre connaissance ni d'autre protection que celle
dont il avait tout reçu. M. de Noailles crut donc ne pouvoir mieux faire
que de le charger d'une simple lettre de créance pour le roi, et de le
lui annoncer comme une lettre vivante qui répondrait à tout
sur-le-champ, et qui sans l'importuner d'une longue dépêche lui en
dirait plus en une demi-heure qu'il ne pourrait lui en écrire en
plusieurs jours. Les paroles volent, l'écriture demeure\,; un courrier
peut être volé, peut tomber malade et envoyer ses dépêches\,; cet
expédient obviait à tous ces inconvénients et laissait M. de Barbezieux
dans l'ignorance et dans l'angoisse de tout ce qui se passerait ainsi
par Genlis.

Barbezieux qui avait d'autant plus d'espions, et de meilleurs en
Catalogne, que c'était pour lui l'endroit le plus dangereux, fut averti
de l'envoi de Genlis et du jour de son départ, et sut de plus qu'il
devait arriver droit au roi, et que surtout il avait défense de le voir
en tout. Là-dessus il prit un parti hardi, il fit attendre Genlis aux
approches de Paris, et se le fit amener chez lui à Versailles sans le
perdre un moment de vue. Quand il le tint, il le cajola tant et sut si
bien lui faire sentir la différence pour sa fortune de l'amitié de M. de
Noailles, quelque accrédité qu'il fût, d'avec celle du secrétaire d'État
de la guerre et de sa sorte et de son âge, qu'il le gagna au point de
l'embarquer dans la plus noire perfidie, de ne voir le roi qu'en sa
présence et de lui dire tout le contraire de sa commission. Barbezieux
lui prescrivit donc tout ce qu'il voulut après avoir tiré de lui tout ce
dont il était chargé, et en fut pleinement obéi. Par ce moyen le projet
du siège de Barcelone fut entièrement rompu sur le point de son
exécution, et avec toutes les plus raisonnables apparences d'un succès
certain, et sans crainte d'aucuns secours, dans l'état des forces de
l'Espagne sur cette frontière comme abandonnée depuis leur défaite\,; et
M. de Noailles demeuré chargé auprès du roi de toute l'iniquité et du
manquement d'une telle entreprise, par cette précaution-là même qu'il
avait prise de ne donner qu'une simple lettre de créance, en sorte que
tout ce que dit Genlis, directement opposé à ce dont il était chargé,
n'eut point de contradicteur, et passa en entier pour être de M. de
Noailles et pour son propre fait. On peut croire que Barbezieux ne
perdit pas de temps à expédier les ordres nécessaires pour dissiper
promptement tous les préparatifs, et de procurer à la flotte ceux de
regagner Toulon. On peut juger aussi quel coup de foudre ce fut pour M.
de Noailles, mais l'artifice avoir si bien pris qu'il ne put jamais s'en
laver auprès du roi\,; on en verra les suites qui servirent de base à la
grandeur de M. de Vendôme.

Vers ce temps-ci la capitation \footnote{La capitation était un impôt
  personnel, payé par tête (\emph{caput}), comme l'indique le mot
  \emph{capitation}, sans distinction de rang ni de condition. Les
  pauvres, les ordres mendiants et ceux dont la contribution personnelle
  n'atteignait pas quarante sous, en furent seuls exempts. Tous les
  autres Français furent divisés en vingt-deux classes et soumis à une
  taxe proportionnée à leur fortune. Ce projet ne fut qu'imparfaitement
  exécuté\,: le clergé se racheta de la capitation par un don gratuit\,;
  la noblesse eut des receveurs spéciaux\,; les parlements et autres
  tribunaux obtinrent de faire eux-mêmes la répartition de leur
  capitation\,; enfin les provinces, qui avaient conservé leurs
  assemblées et qu'on appelait pays d'états, parvinrent à se racheter de
  la capitation en payant une certaine somme pour toute la province.}
fut établie. L'invention et la proposition fut de Basville, fameux
intendant de Languedoc. Un secours si aisé à imposer d'une manière
arbitraire, à augmenter de même, et de perception si facile, était bien
tentant pour un contrôleur général embarrassé à fournir à tout.
Pontchartrain néanmoins y résista longtemps et de toutes ses forces, et
ses raisons étaient les mêmes que je viens de rapporter. Il en prévoyait
les terribles conséquences, et que cet impôt était de nature à ne jamais
cesser. À la fin, à force de cris et de besoins, les brigues lui
forcèrent la main.

Le 27 novembre, M. le comte de Toulouse qui avait acheté le duché-pairie
de Damville, et en avait obtenu une érection nouvelle en sa faveur, fut
reçu en cette qualité au parlement, comme l'avait été M. du Maine et
après lui M. de Vendôme. La planche faite par M. du Maine comme il a été
dit, le roi avait cessé de faire inviter les pairs par M. de Reims, pour
M. de Vendôme qui les visita, ce qui n'avait pas été hasardé la première
fois. M. le comte de Toulouse les visita, comme avait fait M. de
Vendôme, et MM. du parlement. Peu de pairs osèrent ne s'y pas trouver.
Il fut peu de jours après installé, comme amiral de France, à la table
de marbre \footnote{Voy., sur la \emph{table de marbre}, les notes à la
  fin du volume.} par le premier président. M. de Vendôme, grand-père de
celui-ci, y avait été installé en la même qualité par un conseiller.

M. de Luxembourg fit en arrivant un étrange mariage pour sa fille. On a
vu ci-dessus la mort du dernier de tous les Longueville, et son
testament en faveur de M. le prince de Conti, son cousin germain.
M\textsuperscript{me} de Nemours était sa sœur du premier lit, fille de
la sœur de la princesse de Carignan et du dernier prince du sang de la
branche de Soissons, tué en 1641 à la bataille de Sedan, sans avoir été
marié. M\textsuperscript{me} de Nemours était veuve sans enfants du
dernier des ducs de Nemours de la maison de Savoie. C'était une femme
fort haute, extraordinaire, de beaucoup d'esprit, qui se tenait fort
chez elle à l'hôtel de Soissons, où elle ne voyait pas trop bonne
compagnie. Riche infiniment et vivant très magnifiquement, avec une
figure tout à fait singulière et son habit de même, quoique sentant fort
sa grande dame. Elle avait hérité de la haine de la branche de sa mère
contre celle de Condé\,; elle s'était fort accrue par l'administration
des grands biens de M. de Longueville, qu'après la mort de sa mère, sœur
de M. le Prince, le même M. le Prince avait emportée sur elle, et M. le
Prince son fils après lui. Le testament fait en faveur de M. le prince
de Conti ne la diminua pas. Il s'en trouva un postérieur fait en faveur
de M\textsuperscript{me} de Nemours\,; elle prétendit le faire valoir et
anéantir le premier. M. le prince de Conti soutint le sien et disputa
l'autre comme fait depuis la démence\,: cela forma un grand procès.

Dans la colère où il mit M\textsuperscript{me} de Nemours et dans le
mépris où elle avait toujours vécu pour ses héritiers, elle déterra un
vieux bâtard obscur du dernier comte de Soissons, frère de sa mère qui
avait l'abbaye de la Couture du Mans, dont il vivait dans les cavernes.
Il n'avait pas le sens commun, n'avait jamais servi, ni fréquenté en
toute sa vie un homme qu'on pût nommer. Elle le fit venir, loger chez
elle, et lui donna tout ce qu'elle pouvait donner et en la meilleure
forme, et ce qu'elle pouvait donner était immense. Dès lors elle le fit
appeler le prince de Neuchâtel, et chercha à l'appuyer d'un grand
mariage. M\textsuperscript{lle} de Luxembourg n'était rien moins que
belle, que jeune, que spirituelle\,; elle ne voulait point être
religieuse et on ne lui voulait rien donner. La duchesse de Meckelbourg
dénicha ce nouveau parti. Son orgueil ne rougit point d'y penser, ni
celui de M. de Luxembourg son frère, à qui elle en écrivit\,; mais il
palpita assez pour oser se proposer un rang en considération de ce
mariage, sous prétexte de la souveraineté de Neuchâtel donnée à ce
bâtard qui en portait déjà le nom. M. de Luxembourg, qui, en partant,
avait obtenu une grande grâce qui était encore secrète et dont je
parlerai bientôt, n'osa proposer celle-ci, et en laissa la conduite à
l'adresse de sa sœur\,; et, pour éviter tout embarras entre le demander
et ne le demander point, il ne parla point au roi de ce mariage par
aucune de ses lettres. Il avait déjà transpiré avec l'idée du rang,
lorsque M\textsuperscript{me} de Meckelbourg alla demander au roi la
permission d'entendre à ce mariage. Au premier mot qu'elle en dit, le
roi l'interrompit et lui dit que M. de Luxembourg ne lui en avait rien
mandé\,; qu'il n'empêcherait point qu'elle ne fit là-dessus ce que son
frère et elle jugeraient à propos, mais qu'au moins il comptait bien
qu'ils n'imagineraient pas de lui demander un rang pour le chevalier de
Soissons sous aucun prétexte, à qui il n'en accorderait jamais, et barra
ainsi cette belle chimère. Le mariage ne s'en fit pas moins, et il fut
célébré au plus petit bruit à l'hôtel de Soissons, dès que M. de
Luxembourg fut arrivé. M\textsuperscript{me} de Nemours logea les mariés
et les combla d'argent, de présents et de revenus, en attendant sa
succession, et se prit de la plus parfaite affection pour le mari et
pour la femme qui se renfermèrent auprès d'elle, et ne virent d'autre
monde que le sien.

\hypertarget{chapitre-xv.}{%
\chapter{CHAPITRE XV.}\label{chapitre-xv.}}

1695

~

\relsize{-1}

{\textsc{Année 1695. Mort de M. de Luxembourg.}} {\textsc{- Maréchal de
Villeroy capitaine des gardes et général de l'armée de Flandre.}}
{\textsc{- Opposition à la réception au parlement du duc de Montmorency,
qui prend le nom de duc de Luxembourg.}} {\textsc{- Qualité de premier
baron de France, fausse et insidieuse, que les opposants ont fait rayer
au maréchal-duc de Luxembourg.}} {\textsc{- M. d'Elbeuf.}} {\textsc{-
Roquelaure insulté par MM. de Vendôme.}} {\textsc{- Mort de la princesse
d'Orange dont le roi défend le deuil aux parents.}} {\textsc{-
Catastrophe de Koenigsmarck et de la duchesse d'Hanovre.}} {\textsc{-
Échange forcé des gouvernements de Guyenne et de Bretagne.}} {\textsc{-
M. d'Elbeuf à l'adoration de la croix après MM. de Vendôme.}} {\textsc{-
Origine de mon amitié particulière avec la duchesse de Bracciano, depuis
dite princesse des Ursins.}} {\textsc{- Phélypeaux fils et survivancier
de Pontchartrain.}} {\textsc{- Origine de ma liaison avec lui.}}
{\textsc{- Maréchal et maréchale de Lorges.}} {\textsc{- Famille du
maréchal de Lorges.}} {\textsc{- Mon mariage.}} {\textsc{- Trahison
inutile de Phélypeaux.}} {\textsc{- Mariage de ma belle-sœur avec le duc
de Lauzun.}} {\textsc{- Mort de la marquise de Saint-Simon et de sa
nièce la duchesse d'Uzès, de La Fontaine, de Mignard, de Barbançon.}}
{\textsc{- Échange de Meudon et de Choisy avec un grand retour.}}
\relsize{1}

~

M. de Luxembourg ne survécut pas longtemps à ce beau mariage. À
soixante-sept ans, il s'en croyait vingt-cinq, et vivait comme un homme
qui n'en a pas davantage. Au défaut de bonnes fortunes dont son âge et
sa figure l'excluaient, il suppléait par de l'argent\,; et l'intimité de
son fils et de lui, de M. le prince de Conti et d'Albergotti, portait
presque toute sur des mœurs communes et des parties secrètes qu'ils
faisaient ensemble avec des filles. Tout le faix des marches, des
ordres, des subsistances portait, toutes les campagnes, sur Puységur,
qui même dégrossissait les projets. Rien de plus juste que le coup d'œil
de M. de Luxembourg, rien de plus brillant, de plus avisé, de plus
prévoyant que lui devant les ennemis, ou un jour de bataille, avec une
audace, une flatterie, et en même temps un sang-froid qui lui laissait
tout voir et tout prévoir au milieu du plus grand feu et du danger du
succès le plus imminent\,; et c'était là où il était grand. Pour le
reste, la paresse même. Peu de promenades sans grande nécessité, du jeu,
de la conversation avec ses familiers, et tous les soirs un souper avec
un très petit nombre, presque toujours le même, et si on était voisin de
quelque ville, on avait soin que le sexe y fût agréablement mêlé. Alors
il était inaccessible à tout, et s'il arrivait quelque chose de pressé,
c'était à Puységur à y donner ordre. Telle était à l'armée la vie de ce
grand général, et telle encore à Paris, où la cour et le grand monde
occupaient ses journées, et les soirs ses plaisirs. À la fin l'âge, le
tempérament, la conformation le trahirent. Il tomba malade à Versailles
d'une péripulmonie dont Fagon eut tout d'abord très mauvaise opinion\,:
sa porte fut assiégée de tout ce qu'il y avait de plus grand\,; les
princes du sang n'en bougeaient, et Monsieur y alla plusieurs fois.
Condamné par Fagon, Caretti, Italien à secrets qui avaient souvent
réussi, l'entreprit et le soulagea, mais ce fut l'espérance de quelques
moments. Le roi y envoya quelquefois par honneur plus que par sentiment.
J'ai déjà fait remarquer qu'il ne l'aimait point, mais le brillant de
ses campagnes et la difficulté de le remplacer faisaient toute
l'inquiétude. Devenu plus mal, le P. Bourdaloue, ce fameux jésuite que
ses admirables sermons doivent immortaliser, s'empara tout à fait de
lui. Il fut question de le raccommoder avec M. de Vendôme, que la
jalousie de son amitié et de ses préférences pour M. le prince de Conti
avait fait éclater en rupture, et se réfugier à l'armée d'Italie, comme
je l'ai déjà dit. Roquelaure, l'ami de tous et le confident de personne,
les amena l'un après l'autre au lit de M. de Luxembourg où tout se passa
de bonne grâce et en peu de paroles. Il reçut ses sacrements, témoigna
de la religion et de la fermeté. Il mourut le matin du 4 janvier 1695,
cinquième jour de sa maladie, et fut regretté de beaucoup de gens,
quoique, comme particulier, estimé de personne, et aimé de fort peu.

Pendant sa maladie il fit faire un dernier effort auprès du roi par le
duc de Chevreuse pour obtenir sa charge pour son fils, gendre de ce duc.
Il en fut refusé, et le roi lui fit dire qu'il devait se souvenir qu'il
ne lui avait donné le gouvernement de Normandie en survivance pour son
fils, qu'à condition qu'il ne lui parlerait jamais de la charge. Tous
ses enfants et M\textsuperscript{me} de Meckelbourg, sa sœur, ne le
quittèrent que lorsqu'on les mit hors de sa chambre comme il allait
passer, où ils laissèrent éclater leur douleur. Le P. Bourdaloue les
reprit de ce qu'ils s'affligeaient de ce qu'un homme payait le tribut à
la nature\,; il ajouta qu'il mourait en chrétien et en grand homme, et
que peut-être aucun d'eux n'aurait le bonheur de mourir de la sorte.
Pour en grands hommes, aucun d'eux n'y était tourné\,; en chrétiens, ce
sera leur affaire\,: mais la prophétie ne tarda pas à s'accomplir en la
personne de la duchesse de Meckelbourg. Elle mourut dans le même mois de
janvier et de la même maladie peu de jours après lui, sans aucun secours
spirituel, ni presque de corporels, laissant tout ce qu'elle avait au
comte de Luxembourg, second fils de son frère.

M. de Luxembourg ne vit à la mort pas un des ducs qu'il avait attaqués,
pas un aussi ne s'empressa pour lui. Je n'y allai ni n'y envoyai pas une
seule fois, quoique je fusse à Versailles, et il faut avouer que je
sentis ma délivrance d'un tel ennemi. On eut la malignité de me vouloir
faire parler sur cette mort. Je me contentai de répondre que je
respectais trop le discernement du roi dans ses choix pour le remplacer,
et avais trop bonne opinion de ses généraux et de ses troupes, pour
m'affliger pour l'État d'une perte dont en mon particulier j'avais tant
de raisons de me consoler. Avec cette réponse je vis tarir les
questions.

Le maréchal de Villeroy eut la charge de capitaine des gardes du corps,
en payant cinq cent mille livres de brevet de retenue dont il eut un
pareil, et lui succéda au commandement de l'armée de Flandre. Tout le
monde s'attendait à cette disposition\,: Villeroy, élevé avec le roi,
avait toujours été fort bien avec lui, et dans la confiance domestique
et de maîtresses la plus intime, fils de son gouverneur, et tous deux
bas et fins courtisans toute leur vie. Quelques nuages étrangers avaient
quelquefois éloigné celui-ci\,; mais le goût du roi, ramené par l'art
des souplesses et des bassesses, l'avait toujours rétabli en sa première
faveur.

Disons tout de suite ce qui se passa entre le duc de Montmorency et nous
dans le cours de cet hiver, qui prit le nom du duc de Luxembourg à la
mort de son père. Nos assemblées se continuèrent. MM. d'Elbœuf,
Montbazon, La Trémoille, Sully qui avait repris le procès depuis la mort
de son père, Chaulnes, La Rochefoucauld, Richelieu, Monaco, Rohan et moi
signâmes deux oppositions à ce que nul hoir mâle, sorti du feu maréchal
de Luxembourg, ne fût reçu au parlement en qualité de pair de France
pour les raisons que nous réservions à dire en temps et lieux, dont
l'une fut signifiée à Dongois qui faisait la charge de greffier en chef
du parlement, l'autre à la personne du procureur général, et nous
résolûmes en même temps de faire rayer au fils la qualité qu'il prenait
de premier baron de France, comme nous y avions obligé le père. Ce jeu
de mots leur a fort servi à abuser le monde et à se faire passer pour
premiers barons du royaume, et se préparer, par là des chimères, tandis
que la terre de Montmorency, mouvante de l'abbaye de Saint-Denis, est
peut-être première baronnie de ce district étroit connu sous le nom de
l'Île-de-France, comme on dit de cette même abbaye Saint-Denis en
France.

Ensuite nous minimes sur le tapis notre résolution précédente de mettre
en cause le duc de Gesvres pour récuser par ce moyen le premier
président. Ce magistrat, depuis la mort de M. de Luxembourg, prenait
toutes sortes de formes pour éviter cet affront. Il employa dés
présidents à mortier, amis de quelques-uns de nous, et d'autres
personnes de leur confiance, qui, sous prétexte d'amitié et d'intérêt à
ce qui les touchait, leur exagérèrent la peine et la douleur du premier
président de s'être brouillé avec nous\,; qu'il sen toit amèrement ses
torts à notre égard, et combien la mort de celui dont il espérait un
grand appui le laissait exposé à notre haine\,; qu'ils étaient sûrs
qu'il donnerait toutes choses pour se rapprocher de nous, et qu'ils ne
doutaient point que sa profonde capacité ne lui fournît des moyens
depuis cette mort d'être autant pour nous qu'il nous avait été
contraire. Ils ajoutèrent même qu'ils lui en avaient ouï échapper des
demi-mots bien significatifs, et qui les assuraient que le cœur
s'expliquait par sa bouche. MM. de Chaulnes, de La Rochefoucauld et de
La Force s'infatuèrent de ce piège, et opinèrent fortement à y donner.
MM. de La Trémoille, de Rohan et moi ne primes point un si dangereux
change. Nous remîmes aux yeux de ces messieurs toutes les injustices et
quelque chose de pis, que nous avions essuyées du premier président\,;
son refus d'audience qui nous força aux lettres d'État\,; son manque de
parole et sans détour au duc de Chaulnes sur l'assemblée de toutes les
chambres\,; le danger de se fier à un homme si autorisé au parlement, et
d'autant plus offensé contre nous, que nous avions publié ses iniquités
et ses perfidies sans plus garder de mesure avec lui. Nous remontrâmes
combien il était apparent que ces attaques nous étaient faites sous sa
direction par l'ardeur de venger son orgueil blessé\,; et quelles
seraient notre honte et notre imprudence d'être ses dupes en nous
remettant volontairement en ses filets. Nous n'étions que nous six ce
jour à l'assemblée, et trois contre trois ne purent se persuader l'un
l'autre. Elle se rompit donc, sans rien conclure, un peu
tumultuairement\,; et M. de La Trémoille déclara en sortant qu'il
protestait et protesterait contre l'opinion des autres trois\,; et que
pour éviter des querelles inutiles et personnelles, il cesserait de se
trouver aux assemblées. Ce commencement de scission nous fit prendre le
parti, au duc de Rohan et à moi, de tenter de convertir M. de La
Rochefoucauld, et cette pensée nous réussit deux jours après fort
heureusement en une heure de temps que nous fûmes enfermés tous trois
ensemble dans sa chambre à Versailles. Nous remîmes donc cette affaire
sur le tapis avec plus de confiance à la première assemblée, où M. de La
Trémoille ne parut point. M. de Chaumes fut étonné et fort fâché de se
voir abandonné de M. de La Rochefoucauld revenu à notre avis. Il avait
de l'amitié pour moi\,; son chagrin tomba sur le duc de Rohan, qui, vif,
aigre et peu considéré, mit le bonhomme Chaumes, toujours si mesuré, en
telle colère, que de part et d'autre les grosses paroles commençaient à
échapper entre les dents. Cela nous hâta, de peur de pis, de rompre
brusquement l'assemblée, où il ne fut encore rien conclu.

M. de La Rochefoucauld et moi raisonnâmes le lendemain ensemble, et
sentîmes que le plus grand mal qui nous pût arriver serait la désunion,
et nous conclûmes qu'avant tout, il fallait se hâter de raccommoder ces
deux ducs et les disposer à opiner plus paisiblement, et mettant tout
autre intérêt à part et toute fantaisie personnelle, n'aller qu'au but
et au bien de notre affaire commune. Après un assez long entretien tête
à tête, M. de La Rochefoucauld s'en chargea\,; il n'y perdit pas un
moment, et heureusement il y réussit avant la première assemblée.
Celle-ci fut tranquille, et M. de La Trémoille y revint. Il fut proposé
de négocier avec le premier président et de le faire sonder\,; mais ce
hameçon fut modestement mais très fermement rejeté, et enfin la
récusation du premier président résolue. On accorda seulement, à la
considération que nous avions tous pour M. de Chaumes, qu'on ne ferait
point assigner M. de Gesvres tant que rien ne péricliterait, et qu'on
attendrait à le faire autant qu'on le pourrait sans hasarder ce qui
venait d'être résolu. Ensuite on proposa de prendre une requête civile
au nom des ducs de Lesdiguières, de Brissac et de Rohan, dont pour
abréger je n'expliquerai ni les raisons ni la procédure\,; mais M. de
Rohan refusa d'y consentir jusqu'à ce que préalablement le duc de
Gesvres eût été mis en cause, et ne se contenta d'aucunes raisons ni
d'aucunes paroles qu'on lui voulut donner. Son consentement enfin ne
s'arracha qu'après tant d'allées et venues que le projet de la requête
civile vint à M. de Luxembourg qui prit aussitôt ses mesures avec le
vieux chancelier Boucherat, gouverné par M\textsuperscript{me} d'Harlay
sa fille, qui ménageait fort le premier président, cousin de son mari,
qui fit en sorte qu'aucun des maîtres des requêtes ne scellât rien
là-dessus du petit sceau sans grande connaissance de cause, c'est-à-dire
sans que M. de Luxembourg fût averti à temps de s'y opposer. Il est
difficile de comprendre comment une aussi bonne tête que M. de Chaulnes,
et un homme aussi digne que lui, se montra si difficile à la récusation
du premier président après qu'il lui avait si indignement manqué de
parole, et avec la connaissance qu'il avait de ses souplesses et tous
les tours et détours de perfidie dont il avait usé jusqu'à découvert
avec nous, et d'autre part, il ne fut pas moins étrange que M. de Rohan
se montrât si roide pour la récusation, après la mollesse et la
variation, pour ne pas dire pis, avec laquelle il avait fait avorter
entre ses mains, après l'avoir entreprise, et avec certitude de succès,
comme je l'ai raconté plus haut. Toutes ces longueurs coulèrent le temps
jusqu'à l'ouverture de la campagne. M. de Luxembourg, maréchal de camp,
servant dans l'armée de Flandre, s'y rendit, et notre procès demeura
accroché jusqu'à l'hiver suivant. Il avait perdu sa femme, et perdit tôt
après le seul enfant qu'il en avait eu, sans que son union intime avec
M. et M\textsuperscript{me} de Chevreuse en ait été en rien diminuée.

Il y avait eu, sur les fins de l'été et dans les commencements de
l'hiver, des tentatives de négociations de paix, je ne sais sur quoi
fondées. Crécy alla en Suisse comme en pays neutre et mitoyen entre
l'empereur et M. de Savoie, et pas fort éloigné de Venise qui se mêlait
de bons offices. Il était frère du P. Verjus, jésuite, ami particulier
du P. de La Chaise, et il avait été résident en plusieurs cours
d'Allemagne dont il connaissait parfaitement le droit public, les
diverses cours des princes et leurs intérêts\,: c'était un homme sage,
mesuré, et qui, sous un extérieur et des manières peu agréables, et qui
sentaient bien plus l'étranger, le nouveau débarqué que le Français à
force d'avoir séjourné dehors, et un langage de même, cachait une
adresse et une finesse peu communes, une prompte connaissance, par le
discernement, des gens avec qui il avait à traiter et de leur but\,; et
qui à force de n'entendre que ce qu'il voulait bien entendre, de
patience et de suite infatigable, et de fécondité à présenter sous
toutes sortes de faces différentes les mêmes choses qui avaient été
rebutées, arrivait souvent à son but.

L'abbé Morel alla vers Aix-la-Chapelle pour négocier dans l'empire.
C'était une excellente tête, pleine de sens et de jugement, produite par
Saint-Pouange, dont il était ami de table et de plaisir, et que M. de
Louvois et le roi ensuite qui s'en était bien trouvé, avait employé en
plusieurs voyages secrets. Il avait un frère conseiller au parlement et
chanoine de Notre-Dame, qui ne lui ressemblait que pour aimer encore
mieux le vin que lui et ne le porter pas si bien, et qu'il fit enfin
aumônier du roi.

Harlay, conseiller d'État et gendre du chancelier, homme d'esprit, mais
c'était à peu près tout, était allé à Maestricht sonder les
Hollandais\,; mais ces démarches ne firent qu'enorgueillir les ennemis
et les éloigner de la paix à proportion qu'ils nous la jugeaient plus
nécessaire, et qu'ils y voyaient un empressement et des recherches si
opposés à l'orgueil avec lequel on s'était piqué de terminer toutes les
guerres précédentes. Ce fut tout le fruit que ces messieurs rapportèrent
dans les premiers mois de cet hiver. Ils eurent même l'impudence de
faire sentir à M. d'Harlay, dont la maigreur et la pâleur étaient
extraordinaires, qu'ils le prenaient pour un échantillon de la réduction
où se trouvait la France. Lui, sans se fâcher, répondit plaisamment que,
s'ils voulaient lui donner le temps de faire venir sa femme, ils
pourraient en concevoir une autre opinion de l'état du royaume. En
effet, elle était extrêmement grosse et était très haute en couleur. Il
fut assez brutalement congédié, et se hâta de regagner notre frontière.

Les hivers ne se passent guère sans aventures et sans tracasseries. M.
d'Elbœuf trouva plaisant de faire l'amoureux de la duchesse de Villeroy,
toute nouvelle mariée, et qui n'y donnait aucun lieu. Il lui en coûta
quelque séjour à Paris pour laisser passer cette fantaisie, qui allait
plus à insulter MM. de Villeroy qu'à toute autre chose. Ce n'était pas
que M. d'Elbœuf eût aucun lieu de se plaindre d'eux, mais c'était un
homme dont l'esprit audacieux se plaisait à des scènes éclatantes, et
que sa figure, sa naissance et les bontés du roi avaient solidement
gâté.

Roquelaure, duc à brevet et plaisant de profession, essuya une triste
aventure. Il avait été toute sa vie extrêmement du grand monde, et ami
intime de M. de Vendôme. Comme il voulait tenir à tout, il s'était
fourré parmi les amis de M. de Luxembourg, de la brillante situation
duquel il espérait tirer parti, et de ce qu'il entrevoyait dans la cour
de Monseigneur, que ce général, intimement uni avec le prince de Conti,
méditait de gouverner et d'avoir une part principale à tout lorsque le
roi n'y serait plus. La difficulté pour Roquelaure était de demeurer
bien avec des gens si opposés, qui devint bien plus fâcheuse lors de la
rupture ouverte de MM. de Vendôme avec M. de Luxembourg dont j'ai parlé
plus haut, et de ses causes. Elle fut si entière qu'il fallut opter, et
Roquelaure, qui ne lisait pas dans l'avenir, ne balança pas à quitter
son ancien ami de tous les temps pour ceux qu'il venait de se faire et
dont il espérait beaucoup. M. de Vendôme en fut piqué au vif, mais il
n'était pas temps de le montrer. L'éloignement de l'Italie, où il
s'était réfugié de Flandre, faisait qu'il ne passait que peu de temps à
la cour, et y vivait assez à l'ordinaire avec Roquelaure lorsqu'ils se
trouvaient en même lieu. C'est ce qui fit qu'à la mort de M. de
Luxembourg, ce fut lui qui mena MM. de Vendôme comme j'ai dit
ci-dessus\,; mais cela même avait renouvelé leur dépit de sa défection
de leur amitié, tellement que le vide que laissait M. de Luxembourg et
l'audace de la nouvelle grandeur et de leur liaison avec M. du Maine qui
les y avait fait monter, rompit les bornes où jusqu'alors il s'était
contenu avec Roquelaure.

À peu de jours de là, celui-ci entra chez M. le Grand, un soir, qui
tenait, soir et matin, une grande table à la cour, et un grand jeu toute
la journée, où la foule de la cour entrait et sortait comme d'une
église, et où celle des joueurs à tous jeux, mais surtout au lansquenet,
ne manquait jamais. M. de Vendôme, qui était un des coupeurs, eut
dispute avec un autre sur un mécompte de sept pistoles. Il était beau
joueur, mais disputeur et opiniâtre au jeu comme partout ailleurs. Les
autres coupeurs le condamnèrent\,; il paya, quitta, et vint grommelant
contre ce jugement à la cheminée, où il trouva Roquelaure debout qui s'y
chauffait. Celui-ci, avec la familiarité qu'il usurpait toujours et cet
air de plaisanterie qu'il mêlait à tout, dit à l'autre qu'il avait tort
et qu'il avait été bien jugé. Vendôme, piqué de la chose, le fut encore
plus de cette indiscrétion, lui répondit en colère et jurant « qu'il
était un f\ldots. décideur, et qu'il se mêlait toujours de ce qu'il
n'avait que faire.\,» Roquelaure, étonné de la sortie, fila doux, et lui
dit qu'il ne croyait pas le fâcher\,; mais Vendôme, s'emportant de plus
en plus, lui répliqua des duretés avec une hauteur qui ne se pouvait
souffrir que par un valet, et dont le ton de voix ne fut pas ménagé.
Roquelaure outré, mais beaucoup trop embarrassé, se contenta de lui
répondre que s'ils étaient ailleurs, il ne lui parlerait pas de la
sorte. Vendôme, se rapprochant plus près et le menaçant, répliqua en
jurant, « qu'il le connaissait bien, et que là ni ailleurs il ne serait
pas plus méchant.\,» Là-dessus le grand prieur qui était assez loin
s'approcha d'eux et prit Roquelaure par le bout de sa cravate, et lui
dit des choses aussi fâcheuses que celles qu'il venait d'essuyer de son
frère, et sans altérer un flegme fort à contre-temps. Aussitôt voilà
toute la chambre en émoi. M\textsuperscript{me} d'Armagnac et le
maréchal de Villeroy coururent à la cheminée. Elle se hâta d'emmener MM.
de Vendôme\,; et le maréchal de Villeroy, Roquelaure, qui n'eut ni le
courage de tirer raison d'un tel affront, ni le supplément de prendre
prétexte du lieu pour en porter sa plainte au roi. Le pis fut que dès le
lendemain d'une scène si publique, il se laissa raccommoder, et en
particulier, avec MM. de Vendôme, par M\textsuperscript{me} d'Armagnac
dans son cabinet. Pour y mettre le comble, la duchesse de Roquelaure
alla partout disant qu'elle était bien fâchée de ce qui était arrivé,
mais que voilà aussi ce que c'était que de s'attaquer à son mari\,: ce
ne pou voit être bêtise, et l'ignorance aurait été bien forte\,; on ne
comprend pas ce qu'elle put espérer d'un si ridicule propos. Quelque
effronté que fût Roquelaure, il parut les premiers jours déconcerté, et
bientôt après il se remit à ses bouffonneries ordinaires et se trouva
partout impudemment avec MM. de Vendôme à Marly, à Choisy, et partout où
cela se rencontrait, et n'évitait pas même de leur parler quand cela se
présentait, à l'étonnement de tout le monde.

Un soir longtemps après, qu'il fit chez le roi plus de bruit et d'éclats
de rire qu'à l'ordinaire et qu'on le remarquait, je répondis froidement
que la cause de tant de gaieté n'était pas difficile à deviner, puisque
ce même soir MM. de Vendôme prenaient congé du roi pour retourner en
Provence. Ce propos fut relevé, et je n'en fus point fâché, parce que je
croyais n'avoir pas lieu d'aimer Roquelaure.

Deux événements étrangers se suivirent fort près à près. Le premier, la
mort de la princesse d'Orange, à la fin de janvier, dans Londres\,; la
cour n'en eut aucune part, et le roi d'Angleterre pria le roi qu'on n'en
prît point le deuil, qui fut même défendu à MM. de Bouillon, de Duras et
à tous ceux qui étaient parents du prince d'Orange. On obéit et on se
tut\,; mais on trouva cette sorte de vengeance petite. On eut des
espérances de changements en Angleterre, mais elles s'évanouirent
incontinent, et le prince d'Orange y parut plus accrédité, plus autorisé
et plus affermi que jamais. Cette princesse, qui avait toujours été fort
attachée à son mari, n'avait pas paru moins ardente que lui pour son
usurpation, ni moins flattée de se voir sur le trône de son pays aux
dépens de sou père et de ses autres enfants. Elle fut fort regrettée, et
le prince d'Orange qui l'aimait et la considérait avec une confiance
entière, et même avec un respect fort marqué, en fut quelques jours
malade de douleur.

L'autre événement fut étrange. Le duc d'Hanovre qui briguait un neuvième
électorat en sa faveur, et qui, par la révolution d'Angleterre, était
appelé à cette couronne après le prince et la princesse d'Orange, et
après la princesse de Danemark, comme le plus proche de ligne
protestante, était fils aîné de la duchesse Sophie, laquelle était fille
de l'électeur palatin, qui se fit couronner roi de Bohème et qui en
perdit sa dignité et ses États, et d'une fille de Jacques Ier, roi
d'Écosse puis d'Angleterre, fils de la fameuse Marie Stuart, et père de
Charles Ier, qui eurent la tête coupée, et {[}aïeul{]} du roi Jacques
II, détrôné par le prince d'Orange \textsuperscript{{[}{]}}{[}{]}. Ce
duc d'Hanovre avait épousé sa cousine germaine, de même maison, fille du
duc de Zell. Elle était belle\,; il vécut bien avec elle pendant quelque
temps. Le comte de Koenigsmarck, jeune et fort bien fait, vint à sa cour
et lui donna de l'ombrage. Il devint jaloux\,; il les épia et se crut
pleinement assuré de ce qu'il eût voulu ignorer toute sa vie\,; mais ce
ne fut qu'après longtemps. La fureur le saisit\,: il fit arrêter le
comte et tout de suite jeter dans un four chaud. Aussitôt après il
renvoya sa femme à son père, qui la mit en un de ses châteaux, gardée
étroitement par des gens du duc d'Hanovre. Il fit assembler le
consistoire pour rompre son mariage. Il y fut décidé fort singulièrement
qu'il l'était à son égard, et qu'il pouvait épouser une autre femme\,;
mais qu'il subsistait à l'égard de la duchesse d'Hanovre\,; qu'elle ne
pouvait se remarier, et que les enfants qu'elle avait eus pendant son
mariage étaient légitimes. Le duc d'Hanovre ne demeura pas persuadé de
ce dernier article.

Le roi, tout occupé de la grandeur solide de ses enfants naturels,
venait de donner au comte de Toulouse toutes les distinctions,
l'autorité et les avantages dont son office d'amiral pouvait être
susceptible entre ses mains. Il lui avait donné depuis longtemps le
gouvernement de Guyenne, à la mort du duc de Roquelaure, père de
celui-ci\,; et pendant sa jeunesse, le maréchal de Lorges en avait eu le
commandement et tous les appointements, qui n'avaient cessé que lorsque,
par la cascade que fit la mort du maréchal d'Humières, il eut le
gouvernement de Lorraine, comme je l'ai dit. M. de Chaulnes avait depuis
très longtemps le gouvernement de Bretagne, et il y était adoré. À ce
gouvernement l'amirauté de la province était unie, qui valait
extrêmement. Rien ne convenait mieux à un amiral de France que de la
réunir à lui, et que le gouvernement de cette vaste péninsule, bordée
par la mer de trois côtés. Le roi y pensa donc avec d'autant plus
d'empressement, qu'il s'était engagé {[}à donner{]} à Monsieur le
premier gouvernement de province qui viendrait à vaquer, pour M. le duc
de Chartres, et c'était une parole donnée à l'occasion du mariage de ce
prince. M. de Chaulnes était vieux et fort gros\,; le roi craignait que
la Bretagne lui échappât pour le Comte de Toulouse par sa vacance, et il
résolut, pour la prévenir, le troc de ces deux gouvernements. Pour
l'adoucir au duc de Chaulnes qui y perdait tout, et pour tirer le duc de
Chevreuse qu'il aimait de l'état fâcheux où il avait mis ses affaires, à
force de s'y croire habile et de vastes projets qui l'avaient ruiné, le
roi voulut lui donner en même temps la survivance de la Guyenne, comme
au neveu, à l'ami et à l'héritier du duc de Chaulnes et de son même nom,
à qui il avait substitué tous ses biens par son contrat de mariage,
c'est-à-dire au second fils qui en naîtrait, en cas que lui-même mourût
sans enfants\,; et il n'en a jamais eu.

Le roi fit entrer un matin le duc de Chaulnes dans son cabinet, lui dora
la pilule au mieux qu'il put, et toutefois conclut en maître. M. de
Chaulnes, surpris et outré au dernier point, n'eut pas la force de rien
répondre. Il dit qu'il n'avait qu'à obéir, et sortit incontinent du
cabinet du roi les larmes aux yeux. Il s'en alla tout de suite à Paris,
et il éclata contre le duc de Chevreuse, qu'il ne douta point avoir eu
toute la part et peut-être fourni au roi une invention à lui si utile.
La vérité était pourtant que le duc ni la duchesse de Chevreuse n'en
avaient rien su qu'en même temps que M. de Chaulnes.

Celui-ci ne voulut pas voir son neveu ni sa nièce, et
M\textsuperscript{me} de Chaulnes, si accoutumée à être la reine de la
Bretagne, et qui y était aussi passionnément aimée, s'emporta plus
encore que son mari. Ni l'un ni l'autre ne cachèrent leur douleur,
tellement que je dis au duc de Chaulnes que je ne lui ferais aucun
compliment, mais que je les porterons tous à M. de Chevreuse. Il
m'embrassa et me témoigna me savoir gré de sentir ainsi pour lui. On fut
longtemps à les apaiser l'un et l'autre. À la fin, M. de Beauvilliers et
d'autres amis communs obtinrent de M. et de M\textsuperscript{me} de
Chaulnes de vouloir bien recevoir M. et M\textsuperscript{me} de
Chevreuse. La visite se fit et fut très sèchement reçue. Jamais on ne
put ôter de leur esprit que M. de Chevreuse n'eût rien contribué à cet
échange forcé, et jamais ni lui ni M\textsuperscript{me} de, Chevreuse
ne purent fondre à leur égard les glaces de M. et de
M\textsuperscript{me} de Chaulnes.

Les Bretons furent au désespoir. Tous le montrèrent par leurs lettres,
leurs larmes et leurs discours\,: tout ce qu'il y en avait à Paris ne
bougea de l'hôtel de Chaulnes, avec plus d'assiduité encore qu'à
l'ordinaire, et M. et M\textsuperscript{me} de Chaulnes, touchés de cet
amour si général et si constant, étaient de plus en plus profondément
affligés. Ils ne s'en consolèrent ni l'un ni l'autre et ne le portèrent
pas loin. Le roi envoya chez lui à Versailles les trois enfants de
France, et sur cet exemple personne ne se dispensa de le visiter. Il
reçut ses compliments avec une triste politesse. Il ne permit pas au
courtisan de cacher l'homme pénétré de douleur, et il s'enfuit à Paris
le soir même.

Cela s'était déclaré à l'issue du lever du roi. Monsieur, qui
s'éveillait beaucoup plus tard, l'apprit en tirant son rideau, et en fut
extrêmement piqué. M. le comte de Toulouse vint peu après le lui dire
lui-même. Il l'interrompit et devant beaucoup de monde qui était à son
lever. « Le roi, lui dit-il, vous a fait là un beau présent. Il témoigne
combien il vous aime\,; mais je ne sais s'il s'accorde bien avec la
bonne politique.\,» Monsieur alla ce même jour chez le roi à son
ordinaire, qui était entre le conseil et le petit couvert, seul dans son
cabinet. Là il ne put contenir ses reproches de le tromper par un troc
forcé qui prévenait une vacance prochaine, et la disposition du
gouvernement de Bretagne pour M. le duc de Chartres. Le roi dont en
effet ç'avait été le motif se laissa gronder, content d'avoir rempli ses
vues. Il essuya la mauvaise humeur de Monsieur tant qu'il voulut\,; il
savait bien le motif de l'apaiser. Le chevalier de Lorraine fit sa
charge accoutumée\,; et quelque argent pour jouer et pour embellir
Saint-Cloud effaça bientôt le chagrin du gouvernement de Bretagne.

M. d'Elbœuf, voyant ce grand vol des bâtards, fit un tour de courtisan
le vendredi saint de cette année. Les Lorrains ni aucun de ceux qui ont
rang de prince étranger ne se trouvaient jamais à l'adoration de la
croix ni à la cène, à cause de la dispute de préséance avec les ducs,
qui étaient aussi exclus de la cène, mais non de l'adoration de la
croix. L'un et l'autre avaient été rendus à MM. de Vendôme, depuis la
préséance au parlement sur tous les pairs\,; ils s'y trouvèrent donc
cette année, et le duc d'Elbœuf aussi, qui comme duc et pair y pouvait
être. Comme le grand prieur en revenait, le roi ne vit personne qui y
allât. Il attendit un moment, puis, se tournant, il vit le duc de
Beauvilliers, et lui dit\,: « Allez donc, monsieur. --- Sire, répondit
le duc, voilà M. le duc d'Elbœuf qui est mon ancien.\,» Et aussitôt M.
d'Elbœuf, comme revenant d'une profonde rêverie, se mit en mouvement et
y alla. Le grand écuyer et le chevalier de Lorraine lui en dirent
fortement leur avis\,; il leur donna pour excuse qu'il n'y avait pas
pensé, mais le roi lui en sut très bon gré.

Tout cet hiver ma mère n'était occupée qu'à me trouver une bon mariage,
bien fâchée de ne l'avoir pu dès le précédent. J'étais fils unique et
j'avais une dignité et des établissements qui faisaient aussi qu'on
pensait fort à moi. Il fut question de M\textsuperscript{lle} d'Armagnac
et de M\textsuperscript{lle} de La Trémoille, mais fort en l'air, et de
plusieurs autres. La duchesse de Bracciano vivait depuis longtemps à
Paris, loin de son mari et de Rome. Elle logeait tout auprès de nous\,;
elle était amie de ma mère qu'elle voyait souvent. Son esprit, ses
grâces, ses manières m'avaient enchanté\,: elle me recevait avec bonté,
et je ne bougeais de chez elle. Elle avait auprès d'elle
M\textsuperscript{lle} de Cosnac sa parente, et M\textsuperscript{lle}
de Royan, fille de sa sœur, et de la maison de La Trémoille comme elle,
toutes deux héritières et sans père ni mère. M\textsuperscript{me} de
Bracciano mourait d'envie de me donner M\textsuperscript{lle} de Royan.
Elle me parlait souvent d'établissements, elle en parlait aussi à ma
mère pour voir si on ne lui jetterait point quelque propos qu'elle pût
ramasser\,: c'eût été un noble et riche mariage, mais j'étais seul, et
je voulais un beau-père et une famille dont je pusse m'appuyer.

Phélypeaux, fils unique de Pontchartrain, avait la survivance de sa
charge de secrétaire d'État. La petite vérole l'avait éborgné, mais la
fortune l'avait aveuglé. Une héritière de la maison de La Trémoille ne
lui avait point paru au-dessus de ce qu'il pouvait prétendre, il y
tournait autour du pot, et sen père ménageait extrêmement la tante dans
cette même vue, qui, en habile femme, profitait de ces ménagements en se
moquant, à part elle, de leur cause. Le père avait toujours été ami du
mien, et avait fort désiré que je le fusse de son fils qui en fit toutes
les avances\,; et nous vivions dans une grande liaison. Il ne craignait
guère que moi pour la préférence de Mile de Royan, et il essayait à
découvrir mes pensées sur elle, en me parlant de divers partis. Je ne me
défiais point de sa curiosité, et moins encore de ses vues, mais je me
contentai de lui répondre vaguement.

Cependant mon mariage s'approchait. Dès l'année précédente il avait été
question de la fille aînée du maréchal de Lorges pour moi. Il s'était
rompu presque aussitôt que traité, et de part et d'autre le désir était
grand de renouer cette affaire. Le maréchal, qui n'avait rien et dont la
première récompense fut le bâton de maréchal de France, avait épousé
incontinent après la fille de Frémont, garde du trésor royal, et qui
sous M. Colbert avait gagné de grands biens, et avait été le financier
le plus habile et le plus consulté. Aussitôt après ce mariage le
maréchal eut la compagnie des gardes du corps, que la mort du maréchal
de Rochefort laissa vacante. Il avait toujours servi avec grande
réputation d'honneur, de valeur et de capacité, et commandé les armées
avec tout le succès que la haine héréditaire de M. de Louvois pour M. de
Turenne et pour tous les siens avait pu se voir forcer à laisser prendre
au neveu favori et à l'élève de ce grand capitaine. La probité, la
droiture, la franchise du maréchal de Lorges me plaisaient infiniment\,;
je les avais vues d'un peu plus près pendant la campagne que j'avais
faite dans son armée. L'estime et l'amour que lui portait toute cette
armée\,; sa considération à la cour\,; la magnificence avec laquelle il
vivait partout\,; sa naissance fort distinguée\,; ses grandes alliances
et proches qui contrebalançaient celle qu'il s'était vu obligé de faire
le premier de sa race\,; un frère aîné très considéré aussi\,; la
singularité unique des mêmes dignités, de la même charge, des mêmes
établissements dans tous les deux\,; surtout, l'union intime des deux
frères et de toute cette grande et nombreuse famille\,; et plus que tout
encore la bonté et la vérité du maréchal de Lorges si rares à trouver et
si effectives en lui, m'avaient donné un désir extrême de ce mariage, où
je croyais avoir trouvé tout ce qui me manquait pour me soutenir,
acheminer, et pour vivre agréablement au milieu de tant de proches
illustres, et dans une maison aimable.

Je trouvais encore dans la vertu sans reproche de la maréchale et dans
le talent qu'elle avait eu\,; enfin de rapprocher M. de Louvois de son
mari, et de le faire duc pour prix de cette réconciliation, tout ce que
je me pouvais proposer pour la conduite d'une jeune femme que je voulais
qui fût à la cour, et où sa mère était considérée et applaudie, par la
manière polie, sage et noble avec laquelle elle savait tenir une maison
ouverte à la meilleure compagnie sans aucun mélange, en se conduisant
avec tant de modestie, sans toutefois rien perdre de ce qui était de son
mari, qu'elle avait fait oublier ce qu'elle était née et à la famille du
maréchal, et à la cour, et au monde où elle s'était acquis une estime
parfaite et une considération personnelle. Elle ne vivait d'ailleurs que
pour son mari et pour les siens, qui avait en elle une confiance
entière, et vivait avec elle et tous ses parents avec une amitié et une
considération qui lui faisaient honneur. Ils n'avaient qu'un fils unique
qu'ils aimaient éperdument et qui n'avait que douze ans, et cinq filles.
Les deux aînées qui avaient passé leur première vie aux Bénédictines de
Conflans, dont la sœur de M\textsuperscript{me} Frémont était prieure,
étaient depuis deux ou trois ans élevées chez M\textsuperscript{me}
Frémont, mère de la maréchale de Lorges, dont les maisons étaient
contiguës et communiquées. L'aînée avait dix-sept ans, l'autre quinze\,;
leur grand'mère ne les perdait jamais de vue\,: c'était une femme de
grand sens, d'une vertu parfaite, qui avait été fort belle et en avait
des restes, d'une grande piété, pleine de bonnes œuvres et d'une
application singulière à l'éducation de ses deux petites filles. Son
mari, depuis longtemps accablé de paralysie et d'autres maux, conservait
toute sa tête et son bon esprit, et gouvernait toutes ses affaires. Le
maréchal vivait avec eux avec toutes sortes d'amitiés et de devoirs\,;
eux aussi le respectaient et l'aimaient tendrement.

Leur préférence secrète à tous trois était pour M\textsuperscript{lle}
de Lorges\,; celle de la maréchale était pour M\textsuperscript{lle} de
Quintin, qui était la cadette\,; et il n'avait pas tenu à ses désirs, à
ses soins, et à quelque chose de plus que l'aînée n'eût pris le parti du
couvent pour mieux marier sa favorite. Celle-ci était une brune avec de
beaux yeux\,; l'autre blonde avec un teint et une taille parfaite, un
visage fort aimable, l'air extrêmement noble et modeste, et je ne sais
quoi de majestueux par un air de vertu et de douceur naturelle\,; ce fut
aussi celle que j'aimai le mieux, dès que je les vis l'une et l'autre,
sans aucune comparaison, et avec qui j'espérai le bonheur de ma vie, qui
depuis l'a fait uniquement et tout entier. Comme elle est devenue ma
femme, je m'abstiendrai ici d'en dire davantage, sinon qu'elle a tenu
infiniment au delà de ce qu'on m'en avait promis, par tout ce qui
m'était revenu d'elle et de tout ce que j'en avais moi-même espéré.

Nous étions, ma mère et moi, informés de tous ces détails par une
M\textsuperscript{me} Damon, femme du frère de M\textsuperscript{me}
Frémont, qui était fort bien faite, fort bien avec eux et qui était plus
du monde que ces sortes de femmes-là n'ont accoutumé d'être, Elle était
amie de ma mère, et je l'aimais fort aussi\,; elle l'avait été de mon
père, et toute sa vie elle avait imaginé et désiré ce mariage, et en
avait parlé une fois à M\textsuperscript{lle} de Lorges. Ce fut elle
aussi qui le traita, et qui avec adresse, mais avec probité, en vint à
bout, à travers les difficultés qui traversent toujours ces affaires si
principales de la vie. M. de Lamoignon, ami intime du maréchal, et
Riparfonds sous lui, cet avocat dont j'ai parlé et qui nous servit si
bien contre M. de Luxembourg, furent ceux dont ils se servirent, et qui
tous deux n'avaient aucune envie de réussir. Lamoignon voulait M. de
Luxembourg, veuf de la fille du duc de Chevreuse, sans enfants, qui le
désirait passionnément, et Riparfonds me voulait pour
M\textsuperscript{lle} de La Trémoille\,; ce que nous découvrîmes après.
Érard, notre avocat, et M. Bignon, conseiller d'État, étaient notre
conseil. Ce dernier avait été assez ami de mon père pour, sans aucune
parenté, avoir bien voulu être mon tuteur, lorsqu'en 1684 j'avais été
légataire universel de M\textsuperscript{me} la duchesse de Brissac,
morte sans enfants, et fille unique du premier lit de mon père. Il avait
été avocat général avec une grande réputation de capacité et
d'intégrité, et il l'avait soutenue tout entière au conseil.
Pontchartrain, contrôleur général et secrétaire d'État, dont il avait
épousé la sœur, l'aimait et le considérait extrêmement, et regarda et
traita toujours ses enfants comme s'ils eussent été les siens. Enfin
toutes les difficultés s'aplanirent, moyennant quatre cent mille livres
comptant, sans renoncer à rien, et des nourritures indéfinies à la cour
et à l'armée.

Les choses à ce point, mais encore secrètes, je crus en pouvoir avancer
la confidence de quelques jours à l'apparente amitié et à la curiosité
de Phélypeaux, d'autant plus même qu'il était neveu de Bignon. À peine
eut-il mon secret qu'il courut à Paris le dire à la duchesse de
Bracciano. J'allai la voir aussi en arrivant à Paris, et je fus surpris
qu'elle me tourna de toutes les façons pour me faire avouer que je me
mariais. La plaisanterie me secourut un temps, mais à la fin elle me
nomma qui, et me montra qu'elle était bien instruite. Alors la trahison
me sauta aux yeux, mais je demeurais ferme dans les termes où je m'étais
mis, sans nier ni avouer rien, et me rabattant à dire qu'elle me mariait
si bien que je ne pouvais que désirer que la chose fût véritable. Elle
me prit en particulier à deux ou trois reprises, espérant de réussir
mieux ainsi, qu'elle n'avait fait par les reproches qu'elle et ses deux
nièces m'avaient faits de mon peu de confiance\,; et je vis que son
dessein allait à essayer de rompre l'affaire par un aveu qui en aurait
éventé le secret, auquel le maréchal était fort attaché, ou, par une
négative formelle, se fonder un sujet de plainte véritable de ce
mensonge. Toutefois elle n'eut pas contentement, et ne put jamais tirer
de moi ni l'un ni l'autre. Je sortis d'un entretien si pénible outré
contre Phélypeaux. Un éclaircissement ou plutôt un reproche de sa
trahison m'aurait mené trop loin avec un homme de sa profession et de
son état. Je pris donc le parti du silence et de ne lui en faire aucun
semblant, mais de vivre désormais avec la réserve que mérite la
trahison. M\textsuperscript{me} de Bracciano me l'avoua dans les suites,
et j'eus le plaisir qu'elle-même me conta sa folle espérance, et s'en
moqua bien avec moi.

Mon mariage convenu et réglé, le maréchal de Lorges en parla au roi,
pour lui et pour moi, pour ne rien éventer. Le roi eut la bonté de lui
répondre qu'il ne pouvait mieux faire, et de lui parler de moi fort
obligeamment\,: il me le conta dans la suite avec plaisir. Je lui avais
plu pendant la campagne que j'avais faite dans son armée, ou, dans la
pensée de renouer avec moi, il m'avait secrètement suivi de l'œil, et
dès lors avait résolu de me préférer à M. de Luxembourg, au duc de
Montfort, fils du duc de Chevreuse, et à bien d'autres. M. de
Beauvilliers, sans qui je ne faisais rien, me porta tant qu'il put à la
préférence de ce mariage sans aucun égard pour les vues de son neveu,
nonobstant la liaison plus qu'intime qui était entre le duc de Chevreuse
et lui, et les deux sœurs leurs femmes.

Le jeudi donc avant les Rameaux, nous signâmes les articles à l'hôtel de
Lorges, nous portâmes le contrat de mariage au roi, etc., deux jours
après, et j'allais tous les soirs à l'hôtel de Lorges, lorsque tout d'un
coup le mariage se rompit entièrement sur quelque chose de mal expliqué
que chacun se roidit à interpréter à sa manière. Heureusement, comme on
en était là butté de part et d'autre, d'Auneuil, maître des requêtes,
seul frère de la maréchale de Lorges, arriva de la campagne, où il était
allé faire un tour, et leva la difficulté à ses dépens. C'est un honneur
que je lui dois rendre et dont la reconnaissance m'est toujours
profondément demeurée. C'est ainsi que Dieu fait réussir ce qui lui
plaît par les moyens les moins attendus. Cette aventure ne transpira
presque point, et le mariage s'accomplit à l'hôtel de Lorges, le 8
avril, que j'ai toujours regardé avec grande raison comme le plus
heureux jour de ma vie. Ma mère m'y traita comme la meilleure mère du
monde. Nous nous rendîmes à l'hôtel de Lorges le jeudi avant la
Quasimodo, sur les sept heures du soir. Le contrat fut signé. On servit
un grand repas à la famille la plus étroite de part et d'autre, et à
minuit le curé de Saint-Roch dit la messe et nous maria dans la chapelle
de la maison. La veille, ma mère avait envoyé pour quarante mille livres
de pierreries à M\textsuperscript{lle} de Lorges, et moi, six cents
louis dans une corbeille remplie de toutes les galanteries qu'on donne
en ces occasions.

Nous couchâmes dans le grand appartement de l'hôtel de Lorges. Le
lendemain M. d'Auneuil, qui logeait vis-à-vis, nous donna un grand
dîner, après lequel la mariée reçut sur son lit toute la France à
l'hôtel de Lorges, où les devoirs de la vie civile et la curiosité
attirèrent la foule, et la première qui vint fut la duchesse de
Bracciano avec ses deux nièces\,; ma mère était encore dans son second
deuil et son appartement noir et gris, ce qui nous fit préférer l'hôtel
de Lorges pour y recevoir le monde. Le lendemain de ces visites,
auxquelles on ne donna qu'un jour, nous allâmes à Versailles. Le soir le
roi voulut bien voir la nouvelle mariée chez M\textsuperscript{me} de
Maintenon où ma mère et la sienne la lui présentèrent. En y allant, le
roi m'en parla en badinant, et il eut la bonté de les recevoir avec
beaucoup de distinction et de louanges. De là elles furent au souper, où
la nouvelle duchesse prit son tabouret. En arrivant à la table le roi
lui dit\,: « Madame, s'il vous plaît de vous asseoir. \,» La serviette
du roi déployée, il vit toutes les duchesses et princesses encore
debout, il se souleva sur sa chaise et dit à M\textsuperscript{me} de
Saint-Simon\,: « Madame, je vous ai déjà priée de vous asseoir\,;\,» et
toutes celles qui le devaient être s'assirent, et M\textsuperscript{me}
de Saint-Simon entre ma mère et la sienne qui était après elle. Le
lendemain elle reçut toute la cour sur son lit dans l'appartement de la
duchesse d'Arpajon comme plus commode parce qu'il était de plain-pied\,;
M. le maréchal de Lorges et moi ne nous y trouvâmes que pour les visites
de la maison royale. Le jour suivant elles allèrent à Saint-Germain,
puis à Paris, où je donnai le soir un grand repas chez moi à toute la
noce, et le lendemain un souper particulier à ce qui restait d'anciens
amis de mon père, à qui j'avais eu soin d'apprendre mon mariage avant
qu'il fût public, et lesquels j'ai tous cultivés avec grand soin jusqu'à
leur mort.

M\textsuperscript{lle} de Quintin ne tarda pas longtemps à avoir son
tour. M. de Lauzun la vit sur le lit de sa sœur avec plusieurs autres
filles à marier\,; elle avait quinze ans et lui plus de soixante-trois
ans. C'était une étrange disproportion d'âge\,; mais sa vie jusqu'alors
avait été un roman, il ne le croyait pas achevé, et il avait encore
l'ambition et les espérances d'un jeune homme. Depuis son retour à la
cour et son rétablissement dans les distinctions qu'il y avait eues\,;
depuis même que le roi et la reine d'Angleterre, qui le lui avaient
valu, lui avaient encore procuré la dignité de duc vérifié, il n'était
rien qu'il n'eût tenté par leurs affaires pour se remettre en quelque
confiance avec le roi, sans avoir pu y réussir. Il se flatta qu'en
épousant une fille d'un général d'armée il pourrait faire en sorte de se
mettre entre le roi et lui, et par les affaires du Rhin s'initier de
nouveau, et se rouvrir un chemin à succéder à son beau-père dans la
charge de capitaine des gardes qu'il ne se consolait point d'avoir
perdue.

Plein de ces pensées, il fit parler à M\textsuperscript{me} la maréchale
de Lorges, qui le connaissait trop de réputation et qui aimait trop sa
fille pour entendre à un mariage qui ne pouvait la rendre heureuse. M.
de Lauzun redoubla ses empressements, proposa d'épouser sans dot, fit
parler sur ce pied-là à M\textsuperscript{me} Frémont et à MM. de Lorges
et de Duras, chez lequel l'affaire fut écoutée, concertée et résolue par
cette grande raison de sans dot, au grand déplaisir de la mère\,; qui à
la fin se rendit, par la difficulté de faire sa fille duchesse comme
l'aînée à qui elle voulait l'égaler. Phélypeaux, qui se croyait à portée
de tout, la voulait aussi pour rien à cause des alliances et des
entours, et la peur qu'en eut M\textsuperscript{lle} de Quintin la fit
consentir avec joie à épouser le duc de Lauzun qui avait un nom, un rang
et des trésors. La distance des tiges et, l'inexpérience du sien lui
firent regarder ce mariage comme la contrainte de deux ou trois ans,
tout au plus, pour être après libre, riche et grande dame, sans quoi
elle n'y eût jamais consenti, à ce qu'elle a bien des fois avoué depuis.

Cette affaire fut conduite et conclue dans le plus grand secret. Lorsque
M. le maréchal de Lorges en parla au roi\,: « Vous êtes hardi, lui
dit-il, de mettre Lauzun dans votre famille\,; je souhaite que vous ne
vous en repentiez pas. De vos affaires vous en êtes le maître\,; mais
pour des miennes, je ne vous permets de faire ce mariage qu'à condition
que vous ne lui en direz jamais le moindre mot.\,»

Le jour qu'il fut rendu public, M. le maréchal de Lorges m'envoya
chercher de fort bonne heure, me le dit et m'expliqua ses raisons\,: la
principale était qu'il ne donnait rien, et que M. de Lauzun se
contentait de quatre cent mille livres, à la mort de M. Frémont, si
autant s'y trouvait outre le partage de ses enfants, et faisait après
lui des avantages prodigieux à sa femme. Nous portâmes le contrat à
signer au roi, qui plaisanta M. de Lauzun et se mit fort à rire, et M.
de Lauzun lui répondit qu'il était trop heureux de se marier, puisque
c'était la première fois, depuis son retour, qu'il l'avait vu rire avec
lui. On pressa la noce tout de suite, en sorte que personne ne put avoir
d'habits. Le présent de M. de Lauzun fut d'étoffes, de pierreries et de
galanteries, mais point d'argent. Il n'y eut que sept ou huit personnes
en tout au mariage, qui se fit à l'hôtel de Lorges à minuit. M. de
Lauzun voulut se déshabiller seul avec ses valets de chambre, et il
n'entra dans celle de sa femme qu'après que tout le monde en fut sorti,
elle couchée et ses rideaux fermés, et lui assuré de ne trouver personne
sur son passage.

Il fit le lendemain trophée de ses prouesses. Sa femme vit le monde sur
son lit à l'hôtel de Lorges où elle et son mari devaient loger, et le
jour suivant nous allâmes à Versailles, où la nouvelle mariée fut
présentée par M\textsuperscript{me} sa mère chez M\textsuperscript{me}
de Maintenon, et de là prit son tabouret au souper. Le lendemain elle
vit toute la cour sur son lit, et tout s'y passa comme à mon mariage.
Celui du duc de Lauzun ne trouva que des censeurs. On ne comprenait ni
le beau-père ni le gendre\,; les raisons de celui-ci ne se pouvaient
imaginer\,; celle de sans dot n'était reçue de personne\,; et il n'y
avait celui qui ne prévit une prochaine rupture de l'humeur si connue de
M. de Lauzun. En revenant à Paris, nous trouvâmes au Cours presque
toutes les filles de qualité à marier, et cette vue consola un peu
M\textsuperscript{me} la maréchale de Lorges, ayant ses filles dans son
carrosse qu'elle venait d'établir en si peu de temps toutes deux.

Peu de jours après, le roi, se promenant dans ses jardins à Versailles,
dans son fauteuil à roues, me demanda fort attentivement l'état et l'âge
de la famille de M. le maréchal de Lorges, et avec un détail qui me
surprit, l'occupation de ses enfants, la figure des filles, si elles
étaient aimées, et si aucune ne penchait à être religieuse. Il se mit
ensuite à plaisanter avec moi sur le mariage de M. de Lauzun, puis sur
le mien\,; il me dit, malgré cette gravité qui ne le quittait jamais,
qu'il avait su du maréchal que je m'en étais bien acquitté, mais qu'il
croyait que la maréchale en savait encore mieux des nouvelles.

À peine mon mariage était-il célébré que la marquise de Saint-Simon
mourut à quatre-vingt-onze ans à Paris. Elle était tante paternelle du
duc d'Uzès, veuve en premières noces de M. de Portes, chevalier de
l'ordre, tué devant Privas, frère de la connétable de Montmorency, mère
de M\textsuperscript{me} la princesse de Condé et du dernier duc de
Montmorency, décapité à Toulouse. Elle en avait eu la première femme de
mon père et M\textsuperscript{lle} de Portes. Elle était veuve du frère
aîné de mon père dont elle avait eu les biens, et nous en avait laissé
les dettes, sans en avoir eu d'enfants. C'était une femme d'esprit,
altière et méchante, qui n'avait jamais pu pardonner à mon père de
s'être remarié, et qui l'avait, tant qu'elle avait pu, séparé de son
frère. Ce fut ainsi un deuil sans douleur. La duchesse d'Uzès, veuve du
fils de son frère et fille unique du feu duc de Montausier, mourut en
même temps.

La perte de deux hommes illustres fit plus de bruit que celle de ces
deux grandes dames\,: {[}de{]} La Fontaine si connu par ses fables et
ses contes, et toutefois si pesant en conversation, et de Mignard si
illustre par son pinceau. Il avait une fille unique parfaitement belle.
C'était sur elle qu'il travaillait le plus volontiers, et elle est
répétée en plusieurs de ces magnifiques tableaux historiques qui ornent
la grande galerie de Versailles et ses deux salons, et qui n'ont pas eu
peu de part à irriter toute l'Europe contre le roi, et à la liguer plus
encore contre sa personne que contre son royaume.

Barbançon, premier maître d'hôtel de Monsieur, mourut aussi, si goûté du
monde par le sel de ses chansons\,; et l'agrément et le naturel de son
esprit.

Le roi, accoutumé à dominer dans sa famille autant pour le moins que sur
ses courtisans et sur son peuple, et qui la voulait toujours rassemblée
sous ses yeux, n'avait pas vu avec plaisir le don de Choisy à
Monseigneur, et les voyages fréquents qu'il y faisait avec le petit
nombre de ceux qu'il nommait à chacun pour l'y suivre. Cela faisait une
séparation de la cour, qui, à l'âge de son fils, ne se pouvait éviter,
dès que le présent de cette maison l'avait fait naître, mais il voulut
au moins le rapprocher de lui. Meudon, bien plus vaste et extrêmement
superbe par les millions que M. de Louvois y avait enfouis, lui parut
propre pour cela. Il en proposa donc l'échange à Barbezieux, pour sa
mère, qui l'avait pris dans les biens pour cinq cent mille livres, et le
chargea de lui en offrir quatre cent mille livres de plus avec Choisy en
retour. M\textsuperscript{me} de Louvois, pour qui Meudon était trop
grand et trop difficile à remplir, fut ravie de recevoir neuf cent mille
livres avec une maison plus à sa portée et d'ailleurs fort agréable\,;
et le même jour que le roi témoigna désirer cet échange, il fut conclu.
Le roi ne l'avait pas fait sans avoir parlé à Monseigneur, pour qui ses
moindres apparences de désir étaient des ordres. M\textsuperscript{me}
de Louvois passa depuis les étés en bonne compagnie à Choisy, et
Monseigneur n'en voltigea que de plus en plus de Versailles à Meudon,
où, à l'imitation du roi, il fit beaucoup de choses dans la maison et
dans les jardins, et combla les merveilles que les cardinaux de Meudon
et de Lorraine et MM. Servien et de Louvois y avaient successivement
ajoutées.

\hypertarget{chapitre-xvi.}{%
\chapter{CHAPITRE XVI.}\label{chapitre-xvi.}}

1695

~

\relsize{-1}

{\textsc{Distribution des armées.}} {\textsc{- Profonde adresse de M. de
Noailles qui le remet mieux que jamais avec le roi, en portant M. de
Vendôme à la tête des armées.}} {\textsc{- Maladie du maréchal de
Lorges, delà le Rhin.}} {\textsc{- Attachement de son armée pour lui.}}
{\textsc{- Maréchal et maréchale de Lorges à Landau, et le maréchal de
Joyeuse fort près des ennemis.}} {\textsc{- Situation des armées.}}
{\textsc{- Maréchal de Joyeuse repasse le Rhin.}} {\textsc{- Traité de
Casal.}} {\textsc{- Bombardement aux côtes.}} {\textsc{- Succès à la
mer.}} {\textsc{- Siège de Namur par le prince d'Orange.}} {\textsc{- Le
maréchal de Boufflers s'y jette.}} {\textsc{- Vaudémont et son armée
échappés au plus grand danger.}} {\textsc{- Maréchal de Villeroy habile
et heureux courtisan.}} {\textsc{- Lavienne, premier valet de chambre.}}
{\textsc{- Sa fortune.}} {\textsc{- Le roi, outré d'ailleurs, rompt sa
canne à Marly sur un bas valet du serdeau.}} {\textsc{- Reddition de la
ville de Namur.}} {\textsc{- Deinse et Dixmude pris.}} {\textsc{-
Bruxelles fort bombardé.}} {\textsc{- Reddition du château de Namur.}}
{\textsc{- Guiscard chevalier de l'ordre.}} {\textsc{- Maréchal de
Boufflers duc vérifié.}} {\textsc{- Maréchal de Lorges de retour à son
armée tombe en apoplexie.}} \relsize{1}

~

Les armées et les corps séparés eurent les mêmes généraux que l'année
précédente, excepté que le maréchal de Villeroy succéda au maréchal de
Luxembourg, et eut M. le duc de Chartres pour général de la cavalerie,
les deux princes du sang et M. du Maine pour lieutenants généraux parmi
les autres, et le comte de Toulouse servant à la tête de son régiment.

M. de Noailles, brouillé avec le roi, jusqu'à être presque perdu par
l'artifice de Barbezieux que j'ai raconté plus haut, sur le projet
manqué du siège de Barcelone, n'avait pu se faire écouter de tout
l'hiver sur la noirceur qui l'avait accablé. Il comprit le danger d'une
situation si forcée à la tête d'une armée, si même il la pouvait
obtenir, et jugea sagement que, parvenu au bâton de maréchal de France,
il n'y avait de bon parti pour lui que de se raccommoder solidement avec
le roi par un sacrifice qui lui serait agréable, et de demeurer à la
cour avec une faveur renouvelée, et à l'abri d'un ennemi avec qui il
n'aurait plus à compter. Trop rusé courtisan, quoique d'ailleurs fort
lourd, pour ne pas sentir l'essor du goût du roi pour les bâtards par
tout ce qu'il venait de faire pour eux, et son peu d'inclination à rien
faire pour M. le Duc et M. le prince de Conti, il avisa à se rétablir
pleinement dans les bonnes grâces du roi, en flattant son goût pour les
uns, et lui ouvrant une porte qui le tirerait d'embarras avec les
autres.

Pour cela, il fit confidence de son projet, sous le dernier secret, à M.
de Vendôme, non pour se servir de lui, mais pour qu'il lui en sût tout
le gré, et par lui M. du Maine\,; puis il témoigna au roi qu'ayant été
assez malheureux de lui déplaire à la tête d'une armée, qui avait réussi
partout, et dont le fruit des succès lui avait été enlevé malgré lui,
sans qu'il eût pu se justifier sur une chose si certaine, il ne pouvait
se résoudre ni à se voir ôter cette même armée ni à la commander\,: que
le premier serait un châtiment qui le déshonorerait, que l'autre
l'exposerait sans cesse aux noirceurs de Barbezieux\,; qu'il aimait
mieux y succomber de bonne grâce, mais en secret, et en faire au roi un
sacrifice\,; que pour cela, il avait imaginé de se rendre à l'ordinaire
en Catalogne, d'y tomber malade en arrivant, de continuer à l'être de
plus en plus, d'envoyer un courrier pour demander son retour\,; qu'en
même temps, il ne voyait personne à portée de ces frontières plus propre
à commander l'armée de Catalogne que M. de Vendôme, qui avait déjà un
corps séparé vers Nice, aux ordres du maréchal Catinat\,; et que si cet
arrangement convenait au roi, il pourrait, pour ne perdre point de temps
à laisser son armée sans général, emporter des patentes de général de
son armée pour M. de Vendôme, et les lui envoyer par un autre courrier
en même temps qu'il demanderait son retour.

Il est impossible d'exprimer le soulagement et la satisfaction avec
laquelle cette proposition fut reçue. La jalousie était extrême entre le
prince de Conti et M. de Vendôme. Le roi, par politique et plus encore
par aversion depuis le voyage de Hongrie, ne voulait point mettre M. le
prince de Conti à la tête de ses armées ni aucun autre prince du sang\,;
cela même le retenait de faire faire ce grand pas à M. de Vendôme. Son
goût pour sa naissance l'en pressait, et plus encore d'en faire en ce
genre le chausse-pied de M. du Maine\,; mais le comment, il n'avait
encore pu le trouver sans mettre les princes du sang au désespoir,
relever le mérite, à lui, déjà si importun, du prince de Conti, l'amour
des armées, de la ville et jusque de la cour, malgré lui, et exciter un
cri public d'autant plus fâcheux qu'il serait plus juste. M. de Noailles
l'affranchissait de tous ces inconvénients\,: c'était un général arrivé
à son armée, mais hors d'état de la commander\,; nécessité donc de lui
en substituer un autre sans délai, et pour cela de le prendre au plus
près qu'il était possible. M. de Vendôme, une fois général d'armée, ne
pouvait plus servir en autre qualité\,; c'était donc une affaire finie,
et finie par un hasard dont les princes du sang pouvaient être fâchés,
mais non offensés\,; et ce chausse-pied de M. du Maine une fois établi,
c'était toujours la moitié de la chose exécutée.

De ce moment, M. de Noailles rentra plus que jamais dans les bonnes
grâces du roi. Ce prince fit la confidence à M. de Vendôme, qui obtint
en même temps pour le grand prieur, son frère, le commandement de ce
corps séparé vers Nice. Le secret demeura impénétrable entre le roi et
les ducs de Vendôme et de Noailles, sans que le grand prieur même en sût
un mot, ni que Barbezieux en eût le moindre vent. Chacun partit pour sa
destination à l'ordinaire, et tout s'exécuta pour la Catalogne comme je
viens de l'expliquer. Mais l'exécution même trahit tout le secret. On
fut surpris d'apprendre M. de Noailles, à peine arrivé à Perpignan,
demander à revenir, et beaucoup plus, qu'il avait envoyé en même temps,
et sans attendre aucune réponse, chercher M. de Vendôme à Nice pour lui
remettre le commandement de son armée\,; et ce qui acheva de lever
toutes les voiles, c'est qu'on sut incontinent après qu'il lui avait
remis des lettres patentes de général de l'armée, qu'il n'avait pu
recevoir d'ici, et que peu après il avait pris le chemin du retour.

Les princes du sang sentirent le coup dans toute sa force\,; mais les
apparences avaient été gardées, en sorte qu'ils furent réduits au
silence. M. de Noailles arriva et fut reçu comme son adresse le
méritait. Il fit l'estropié de rhumatisme, et le joua longtemps, mais il
lui échappait quelquefois de l'oublier et de faire un peu rire la
compagnie. Il se fixa pour toujours à la cour, où il fut en pleine
faveur, et y gagna beaucoup plus qu'il n'eût pu espérer de la guerre, au
grand dépit de Barbezieux qui eut à compter avec M. de Vendôme, lequel,
secouru de M. du Maine, ne le laissa pas broncher à son égard.

Tout le monde partit pour les armées. Celle du Rhin ne tarda pas à le
passer\,; mais à peine étions-nous sur le prince Louis de Bade et en
état d'entreprendre, que M. le maréchal de Lorges tomba extrêmement
malade, le lundi 20 juin, au camp d'Unter-Neishem, sa droite appuyée à
Bornhsall et les ennemie retranchés à Eppingen. On manquait de
fourrages, parce que ce n'était pas un lieu à demeurer et qu'il n'y en
avait guère dans le voisinage. L'armée, qui toujours en est si avide,
pensa moins à elle qu'à son général. Tous les majors de brigade eurent
ordre de demander instamment qu'on ne décampât point, et jamais armée ne
montra tant d'intérêt à la vie de son chef, ni d'amour pour sa personne.
Il fut à la dernière extrémité, tellement que les médecins qu'on avait
fait venir de Strasbourg désespérant entièrement de lui, je pris sur moi
de lui faire prendre des gouttes d'Angleterre\,; on lui en donna cent
trente en trois prises\,: celles qu'on mit dans des bouillons n'eurent
aucun effet\,; les autres dans du vin d'Espagne réussirent. Il est
surprenant qu'un remède aussi spiritueux et qui n'a rien de purgatif ait
mis ceux qui avaient été donnés en si grand mouvement, et qui depuis
plus de vingt-quatre heures qu'on les donnait, n'avaient eu aucun effet.
L'opération fut douce mais prodigieuse, par bas\,; la connaissance
revint et peu à peu le pourpre parut partout. Cette éruption fut son
salut, mais non la fin de la maladie.

Cependant l'armée souffrait beaucoup\,; le maréchal de Joyeuse qui en
avait pris le commandement nous exposa son état, à moi et aux neveux de
M. le maréchal de Lorges. Il nous dit que quoi qu'il pût arriver, il ne
prendrait aucun parti que de notre consentement, et en usa en homme de
sa naissance avec toutes sortes de soins et d'égards. L'armée, informée
qu'il s'agissait de prendre un parti, déclara par la bouche de tous ses
officiers, qui nuit et jour assiégeaient la maison du malade, qu'il n'y
avait point d'extrémité qu'elle ne préférât au moindre danger de son
chef, et ne voulut jamais qu'on fit le moindre mouvement.

Le prince Louis de Bade offrit par des trompettes toutes sortes de
secours, de médecins et de remèdes, et sa parole de toute la sûreté et
de tous les soulagements de vivres et de fourrages pour le général, pour
ce qui demeurerait auprès de lui, et pour l'escorte qui lui serait
laissée si l'armée s'éloignait de lui, avec l'entière sûreté pour la
rejoindre ou aller partout où il voudrait, avec tous ses
accompagnements, sitôt qu'il le voudrait. Il fut remercié comme il le
méritait de ces offres si honnêtes, dont on ne voulut point profiter.

Peu à peu la santé se fit entièrement espérer, et l'armée d'elle-même en
fit éclater sa joie par des feux de joie à la tête de tous les camps,
dés tables qui y furent établies, et des salves qu'on ne put jamais
empêcher. On ne vit jamais un témoignage d'amour si universel ni si
flatteur. Cependant M\textsuperscript{me} la maréchale de Lorges était
arrivée à Strasbourg, puis à Landau, dans une chaise de M. de
Barbezieux, et des gens à lui outre les siens pour la conduire plus
diligemment, et Lacour, capitaine des gardes de M. le maréchal de
Lorges, qui avait été dépêché sur son extrémité. Le roi l'avait
entretenu près d'une heure à Marly, sur l'état de son général et de
l'armée, avait lui-même consulté Fagon son premier médecin, et avait
paru extrêmement sensible à ce grand accident. Toute la cour en fut
infiniment touchée. Il n'y était pas moins aimé et honoré que dans les
troupes. Enfin, dès qu'il fut possible de transporter M. le maréchal de
Lorges à Philippsbourg, M\textsuperscript{me} la maréchale de Lorges y
vint de Landau l'attendre\,: on peut juger de la joie avec laquelle ils
se revirent. J'avais été au-devant d'elle jusqu'à Landau. Toute la fleur
de l'armée avait accompagné son général à Philippsbourg, et la plupart
des officiers généraux. Le lendemain M. le maréchal de Lorges, entre
deux draps en carrosse, et M\textsuperscript{me} la maréchale en chaise,
s'en alla à Landau suivi de tout ce qui était venu de plus distingué à
Philippsbourg. Il s'établit au gouvernement chez Mélac qui lui était
fort attaché, et moi chez Verpel, ingénieur, dans une très jolie maison
tout proche.

Dès le lendemain nous repartîmes tous et allâmes rejoindre l'armée. Nous
couchâmes à Philippsbourg, où Desbordes, gouverneur, nous dit avoir
défense du maréchal de Joyeuse de laisser passer personne pour son
nouveau camp, tellement qu'il nous fallut longer le Rhin en deçà, et le
passer en bateau au village de Ketsch, où on dressait un pont. Comme
l'escorte et la compagnie étaient nombreuses, le passage fut fort
long\,; nous primes les devants pour la plupart, et allâmes à trois
lieues de là, où nous trouvâmes l'armée, sa droite à Roth et sa gauche à
Waldsdorff, où était le quartier général\,; nous y apprîmes que le
maréchal de Joyeuse avait perdu une belle occasion de battre les ennemis
en venant en ce camp, qui s'étaient présentés avec peu de précaution sur
les hauteurs de Malsch. Comme je n'y étais pas je n'en dirai pas
davantage. Le lendemain de notre arrivée, une partie de l'armée monta à
cheval\,; on se mit sous les armes sur les sept heures du matin pour une
légère alarme. Le général Schwartz, avec dix-huit mille hommes de
contingent de Hesse, de Munster et de Lunebourg, parut sur les hauteurs
de Weisloch, et s'y allongea comme pour joindre l'armée du prince Louis
de Bade. On reconnut bientôt qu'il prenait un camp séparé\,; d'un lieu
un peu éminent à notre gauche on découvrit très distinctement les trois
armées.

De notre gauche à la droite de Schwartz, il n'y avait guère que
demi-lieue, et un petit quart de lieue de notre droite à la gauche du
prince Louis, qui était à Kisloch. Tout était séparé par des défilés
qu'on jugeait inaccessibles, mais on ne laissait pas de monter toutes
les nuits un bivouac à chaque aile, avec un lieutenant général à l'un et
un maréchal de camp à l'autre. Celui de la gauche était aux trouées et
au moulin du ruisseau de Weisloch, tout proche du pont où le pauvre
d'Averne avait été tué la dernière campagne. Mon tour de le monter
n'arriva qu'une fois\,; ce fut sous Vaubecourt, pour maréchal de camp,
et Harlus dans la brigade duquel j'étais encore. Schwartz avait un assez
gros poste au Neu-Weisloch et nous au château du Vieux, qu'Argenteuil
lieutenant-colonel d'Harlus, qui était un officier de distinction, alla
relever avec beaucoup d'adresse. Sur les trois heures du matin nous
entendîmes cinq ou six fortes grosses décharges sur la droite\,:
Vaubecourt y voulut courir de sa personne, et Harlus en ce moment-là
n'était pas avec nous. Je représentai à Vaubecourt que ce ne pouvait
être qu'un poste attaqué ou une escarmouche de notre bivouac de la
droite\,; qu'au premier cas tout serait décidé et fini avant qu'il y pût
être, qu'au second c'était un engagement de combat qui ne s'exécuterait
point sans un concert du prince Louis et de Schwartz, lequel attaquerait
bientôt le nôtre, qui étant lé poste du maréchal de camp, il serait
fâché de ne s'y être pas trouvé. Il me crut et envoya au maréchal de
Joyeuse, qui lui manda que les ennemis avaient voulu surprendre un poste
que nous tenions dans l'église de Lehn, à cinq cents pas derrière notre
droite au delà d'un ruisseau\,; qu'ils en avaient été repoussés avec
beaucoup de perte, et qu'il ne nous en avait coûté que quelques soldats,
avec le capitaine, qui était un fort bon officier et qui fut regretté.

Cependant nous manquions tout à fait de fourrage le nez dans les bois,
fort engouffrés entre ces deux camps et acculés au Rhin, tandis que les
ennemis avaient abondance de tout, et se faisaient apporter de loin tout
ce qu'ils voulaient\,: c'était à qui décamperait le dernier. Toute
communication nous était coupée avec Philippsbourg, et tout moyen d'y
aller repasser le Rhin sous la protection de la place. Le prince Louis
avait occupé le défilé des Capucins. Lui et Schwartz étaient postés à
nos deux flancs, et étaient ensemble beaucoup plus nombreux que nous, et
leur situation rendait fort délicat de défiler devant eux dans la plaine
d'Hockenun. Le plus fâcheux inconvénient était l'humeur du maréchal de
Joyeuse qui ne se communiquait à personne, et à qui il échappait des
brusqueries si fréquentes et si fortes même aux officiers généraux les
plus principaux, que personne n'osait lui parler, et que chacun
l'évitait et le laissait faire. Enfin l'excès du besoin lui fit prendre
son parti. Il le communiqua d'abord au comte du Bourg, maréchal de camp,
puis à Tallard, lieutenant général, enfin à Barbezières, un de nos
meilleurs maréchaux de camp, qu'il chargea d'aller reconnaître ce qu'il
voulait savoir, et de l'exécution s'il la trouvait possible. Barbezières
prit ce qu'il voulut d'infanterie et de cavalerie, et en chemin on lui
fit trouver beaucoup d'outils. Je fus de ce détachement.

Barbezières en marchant m'apprit le dessein. Il allait visiter les
ruines de Manheim, que M. de Louvois avait fait brûler en 1688, avec
tout le Palatinat\,; et il allait voir si dans leurs derrières on
pouvait faire un pont de bateaux pour le passage de l'armée. En passant
le ruisseau de Schweitzingen, il y laissa le bonhomme Charmarel,
lieutenant-colonel de Picardie et un des meilleurs et des plus estimés
brigadiers d'infanterie, avec beaucoup d'infanterie, avec ordre d'y
faire une centaine de ponts. Arrivé aux ruines de Manheim, il fit
demeurer toutes les troupes dans la plaine qui est au devant, prit avec
lui cent maîtres et ceux qu'il avait menés pour travailler, et alla tout
reconnaître. Il me permit de le suivre. Nous fîmes le tour de tout ce
qui était la ville et le château de Manheim\,; nous coulâmes ensuite
derrière ces ruines le long du Rhin pour en reconnaître les bords\,; et
après qu'il eut tout fort exactement examiné, il jugea que le pont y
serait construit avec facilité au moyen que je vais expliquer.

On pouvait mettre l'entrée du pont en sûreté avec peu de travail dans
ces ruines et peu d'infanterie à le garder. Il se trouvait en cet
endroit du Rhin une petite île d'abord et une plus grande ensuite, ce
qui donnait la commodité de trois ponts, et celle de rompre le premier
quand tout aurait passé dans la première lie, et le second de même si le
passage se trouvait inquiété ou pressé.

Tout ainsi bien reconnu, nous retournâmes à Charmarel sur le ruisseau de
Schweitzingen, où nous mangeâmes une halte que j'avais, après avoir été
douze heures à cheval. Charmarel demeura avec toute notre infanterie
pour la garde des ponts qu'il faisait, et nous nous retirâmes à l'armée.
En approchant du camp nous trouvâmes tous les vivandiers de l'armée qui
s'en allaient passer le Rhin sur le pont de bateaux que nous avions à
Ketsch, d'où nous comprimes qu'elle marcherait le lendemain. Du Héron,
colonel des dragons, était à une demi-lieue au delà de ce pont avec tous
les gros bagages, il y avait quelques jours, et nous trouvâmes en
arrivant l'ordre donné pour que les menus bagages prissent à minuit le
chemin du même pont et le passer.

L'armée partit en effet le lendemain 20 juillet, et marcha sur quatre
colonnes par les bois, jusque dans la plaine de Hockenun, les deux
lignes rompues par leurs centres qui eurent l'avant-garde, et les
droites et les gauches l'arrière garde. Mélac, lieutenant général de
jour, fit l'arrière-garde de tout à la gauche avec un gros détachement,
et le maréchal de Joyeuse, avec un autre, se chargea de l'arrière-garde
de tout à la droite, le tout sans aucun bruit de trompettes, de timbales
ni de tambours. La Bretesche, lieutenant général, menait notre colonne.
Il entendit quelques bruits de guerre malgré les défenses. Nous étions
vus et entendus des troupes de Schwartz postées sur des hauteurs, qui,
fit ce qu'il put pour nous attirer vers lui. Comme on ne gagnait rien à
cette sourdine imparfaite, La Bretesche permit tout le bruit de guerre.
Le prince Louis ne montra aucune troupe au maréchal de Joyeuse, et nous
arrivâmes tous à la plaine d'Hockenun, sans avoir été suivis de
personne.

Le débouché se fit dans une telle confusion que personne ne se trouva à
sa place, ni à la tête ou à la suite des troupes avec lesquelles on
devait être. À ce désordre il s'en joignit d'autres\,; la cavalerie
était parmi le bois, l'infanterie dans la plaine, nul intervalle entre
les lignes ni entre les bataillons et les escadrons, tout en foule et
pêle-mêle et sans aucun espace à se pouvoir remuer. Une situation si
propre à faire battre toute l'armée par une poignée de gens qui l'aurait
suivie, ou qui s'en serait aperçue à temps, dura plus de quatre heures
qu'on mit à se débrouiller et à se débarrasser les uns des autres, sans
qu'il fût possible aux officiers généraux de replacer les troupes dans
leur ordre. On attendit là que les menus bagages eussent passé le Rhin à
Ketsch, et que notre pont de bateaux y fût rompu et amené et redressé à
Manheim.

Deux raisons avaient empêché de faire traverser l'armée à Ketsch\,: la
difficulté d'y faire d'assez grands et bons retranchements pour bien
assurer le passage de l'arrière-garde, et la hauteur des bords du Rhin,
très supérieure à l'autre côté, qui aurait donné aux batteries que les
ennemis auraient pu établir la facilité de rompre le pont sous l'armée à
demi passée, de fouetter l'autre rivage, et d'y démonter les batteries
que nous y aurions faites.

Après quatre heures de halte assez inutile pour remettre quelque ordre
dans l'armée, elle continua sa marche sur les quatre mêmes colonnes
autant qu'elle le put, jusqu'aux ponts que Charmarel avait faits sur le
ruisseau de Schweitzingen, et bientôt après on entendit sept ou huit
coups de canon\,; des brigades entières firent volte-face et coururent,
sans aucun commandement, vers ce bruit pendant un bon quart d'heure, que
les officiers généraux arrêtèrent tout, et les firent remarcher d'où
elles étaient parties. C'était Schwartz, qui, sorti des bois avec très
peu de monde et quelques petites pièces de campagne, était venu enfin
voir s'il ne pourrait point profiter de notre désordre, et, suivant ce
qu'il trouverait, se faire soutenir de tout son corps\,; mais il s'en
était avisé trop tard. Le maréchal de Joyeuse débanda sur lui Gobert,
excellent brigadier de dragons, avec son régiment et quelques troupes
détachées, qui rechassèrent fort brusquement ce peu de monde dans les
bois. Si le maréchal eût fait soutenir Gobert, comme il en fut fort
pressé, il aurait eu bon marché de cette poignée de gens trop éloignée
de leur gros et leur eût pris leurs pièces de campagne\,; mais il aima
mieux allonger sa marche sans s'amuser à ce petit succès, dans
l'incertitude de ce qui pouvait être dans les bois, où on sut depuis
qu'il n'y avait personne, par Derrondes, major de Gobert, officier très
distingué qui fut pris, et comblé de civilités par le prince Louis, qui
blâma fort cette équipée que Schwartz avait hasardée de lui-même.

On continua donc la marche par une telle chaleur, que plusieurs soldats
moururent de soif et de lassitude. Le tonnerre tomba en plusieurs
endroits et même sur l'artillerie où heureusement il ne causa aucun
accident. Les bois et les défilés qu'on rencontra de nouveau la
retardèrent tellement et avec tant de confusion, que les premières
troupes n'arrivèrent qu'à une heure de nuit, et les dernières fort avant
dans la matinée du lendemain. On campa dans la plaine qui règne le long
du Necker, depuis vis-à-vis d'Heidelberg, jusqu'à son embouchure, le cul
à Manheim, et la gauche appuyée au bord du Necker au village de
Seckenheim, en attendant la queue de l'armée, encore fort éloignée à
cause des défilés.

La Bretesche, lieutenant général, et moi crûmes que le quartier général
était en ce village, et comme la brigade d'Harlus dont j'étais y
touchait, j'y allai avec lui\,: il n'y avait personne\,; nous ne
laissâmes pas d'entrer dans une assez grande maison, de faire jeter
force paille fraîche dans une grande chambre en bas, et d'y faire
décharger ce que malgré les défenses j'avais à manger. Plusieurs
officiers étaient avec nous. Comme nous tâchions à nous refaire des
fatigues de la journée, nous entendîmes grand bruit et bientôt un
vacarme épouvantable\,: c'était un débandement de l'armée qui, à travers
la nuit cherchant de l'eau, avait trouvé ce village, qui par le bout
opposé à celui où nous étions touchait au Necker, et qui après s'être
désaltéré se mit à piller, violer, massacrer et faire toutes les
horreurs que la licence la plus effrénée inspire, couverte par une nuit
fort noire. Incontinent le désordre vint jusqu'à nous, et nous eûmes
peine à nous défendre dans notre maison. Il faut pourtant dire qu'au
milieu de cette fureur, la livrée de M. le maréchal de Lorges, dont
quelques-uns avaient suivi mes gens parce que le gros de ses équipages
était demeuré à l'armée, fut respectée de ces furieux, et mit à couvert
les maisons auprès desquelles elle fut reconnue, tandis qu'en même temps
un garde du maréchal de Joyeuse, et bien reconnu pour tel avec ses
marques et en sauvegarde, fut battu, dépouillé et chassé. La Bretesche
se sut bon gré de ne m'avoir pas cru, qui lui avais conseillé de défaire
sa jambe de bois pour se reposer plus à son aise\,; il m'a souvent dit
qu'il n'avait jamais rien vu de semblable quoiqu'il se fût plusieurs
fois trouvé à des pillages et à des sacs. Nous achevâmes de passer la
nuit du mieux que nous pûmes en ce malheureux endroit, qui ne fut
abandonné que longtemps après qu'il n'y eut plus rien à y trouver. Dès
qu'il fit grand jour La Bretesche et moi allâmes au camp.

Nous trouvâmes l'armée qui commençait à s'ébranler. Elle avait passé la
nuit comme elle avait pu, sans ordre, les troupes arrivant toujours, et
les dernières ne faisant que de joindre. On alla camper sur sept ou huit
lignes à une grande demi-lieue, la droite et le quartier général au
village de Neckerau, la gauche au Necker, le centre et le cul aux ruines
de Manheim\,; on avait réparé comme on avait pu avec des palissades
celles de la citadelle, et on y travaillait encore. On y jeta six
brigades d'infanterie avec Chamilly pour lieutenant général de l'armée,
et Vaubecourt, maréchal de camp. L'embouchure du Necker dans le Rhin
était tout à fait près de notre gauche. On demeura là deux jours, et
tous sans exception réduits à la paille et à la gamelle des cavaliers,
jusqu'à ce que le pont de bateaux fut achevé où Barbezières l'avait
marqué, et cependant on dressa une batterie de canons dans la première
île. Enfin le 24, toute l'armée repassa le Rhin sans que les ennemis
eussent seulement fait mine de nous suivre, en sorte que tout se passa
avec la plus grande tranquillité. Nous n'ouïmes plus parler d'eux de
toute la campagne\,; et le lendemain de ce passage le maréchal de
Joyeuse me permit d'aller à Landau, où je demeurai avec M. et
M\textsuperscript{me} la maréchale de Lorges, jusqu'à ce que ce général
s'alla remettre à la tête de l'armée.

M. de Vendôme promit de grandes choses, prit Ostalric, battit quelques
miquelets, se présenta pour secourir Palamos que les ennemis
assiégeaient, se retira aussitôt sans rien entreprendre, et ce qu'il
n'avait pu, l'arrivée de la flotte du roi sur ces côtes l'opéra, et
Palamos par cela seul fut délivré du siège.

L'Italie ne fournit rien non plus que le siège de Casal, qui fut long\,;
à la fin, Crenan, lieutenant général, qui en était gouverneur, capitula
par ordre du roi. Le traité fut que la place et le château seraient
démolis, qu'il y demeurerait avec sa garnison jusqu'à la démolition
entièrement achevée, qu'il serait après conduit à Pignerol avec toutes
ses troupes, et leurs armes et bagages, qu'il emmènerait avec lui toute
l'artillerie de la place et du château qui se trouverait marquée aux
armes de France, et que Casal serait remise au duc de Mantoue comme à
son seigneur naturel.

Les flottes ennemies bombardèrent nos côtes de Bretagne et de Normandie.
Saint-Malo s'en ressentit peu, Dieppe beaucoup davantage. Nos armateurs
et nos escadres leur prirent force vaisseaux marchands, en battirent les
convois et valurent force millions à notre commerce, au roi et à M. le
comte de Toulouse.

Il se passa en Flandre des choses plus intéressantes. Ce fut d'abord un
beau jeu d'échecs, et plusieurs marches du prince d'Orange et des corps
détachés de son armée sous l'électeur de Bavière et sous le comte
d'Athlone, le maréchal de Villeroy avec la grande armée. Le maréchal de
Boufflers avec la moindre, le marquis d'Harcourt avec son corps vers la
Meuse, et, le vieux Montai vers la mer, réglaient leurs mouvements sur
ceux qu'ils voyaient faire ou qu'ils croyaient deviner. Montal, toujours
le même, malgré son grand âge et la douleur du bâton, sauva la Kenoque
et eut divers avantages l'épée à la main, en prit d'autres par sa
capacité et sa prudence, et eut enfin Dixmude et Deinse avec les
garnisons prisonnières de guerre.

Après diverses montres de différents côtés et avoir menacé plusieurs de
nos places, le prince d'Orange, qui avait bien pris toutes ses mesures
pour couvrir son vrai dessein et n'y manquer de rien, tourna tout à coup
sur Namur, et l'investit les premiers jours de juillet. L'électeur de
Bavière, demeuré au gros de l'armée, l'y fut promptement joindre avec un
grand détachement, et laissa le reste sous M. de Vaudemont. Le maréchal
de Boufflers s'en était toujours douté. Il avait toujours eu soin que la
place fût abondamment fournie\,; il avait sans cesse averti Guiscard,
lieutenant général, qui en était gouverneur et qui était dedans avec
Leaumont qui y commandait sous lui\,; et ce maréchal cependant s'était
mis à portée, et il se jeta dans Namur par la porte du Condros, le 2
juillet, la seule qui était encore libre et qui dès le soir du même jour
ne la fut plus. Il mena avec lui Mesgrigny, gouverneur de la citadelle
de Tournai, maréchal de camp et ingénieur de grande réputation, d'autres
ingénieurs et sept régiments de dragons. Il y en avait un huitième déjà
dans la place et vingt et un bataillons, qui tous ensemble firent plus
de quinze mille hommes effectifs. Harcourt et Bartillat avaient
accompagné le maréchal, et ramenèrent la cavalerie qu'il avait avec lui
et les chevaux de six des sept régiments de dragons entrés avec lui\,;
le comte d'Horn, colonel de cavalerie, et plusieurs autres l'y suivirent
volontaires.

Cette grande entreprise parut d'abord téméraire à notre cour, d'où on
m'écrivit qu'on s'en réjouissait comme d'une expédition qui ruinerait
leurs troupes et ne réussirait pas. J'en eus une autre opinion, et je me
persuadai qu'un homme de la profondeur du prince d'Orange ne se
commettrait pas à un siège si important sans savoir bien comment en
sortir, autant que toute prudence humaine en peut être capable.

Le comte d'Albert, frère du duc de Chevreuse d'un autre lit, était
demeuré à Paris avec congé du roi pour des affaires. Les
dragons-Dauphins, dont il était colonel, étaient dans Namur\,; il y
courut, se déguisa à Dinant en batelier, traversa le camp des
assiégeants et entra dans Namur en passant la Meuse à la nage.

Cependant le maréchal de Villeroy serrait M. de Vaudemont le plus près
qu'il pouvait, et celui-ci, de beaucoup plus faible, mettait toute son
industrie à esquiver. L'un et l'autre sentaient que tout était entre
leurs mains\,: Vaudemont, que de son salut dépendait le succès du siège
de Namur, et Villeroy, qu'à sa victoire était attaché le sort des
Pays-Bas et très vraisemblablement une paix glorieuse et toutes les
suites personnelles d'un pareil événement. Il prit donc si bien
{[}ses{]} mesures qu'il se saisit de trois châteaux occupés sur la
Mundel par cinq cents hommes des ennemis, et qu'il s'approcha tellement
de M. de Vaudemont, le 13 au soir, qu'il était impossible qu'il lui
échappât le 14, et le manda ainsi au roi par un courrier. Le 14 dès le
petit jour tout fut prêt. M. le Duc commandait la droite, M. du Maine la
gauche. M. le prince de Conti toute l'infanterie, M. le duc de Chartres
la cavalerie\,: c'était à la gauche à commencer, parce qu'elle était la
plus proche. Vaudemont, pris à découvert, n'avait osé entreprendre de se
retirer la nuit devant des ennemis si proches, si supérieurs en nombre
et en bonté de troupes, toutes les meilleures étant au siège\,; et un
ennemi dont rien ne le séparait. Il n'osa encore l'attendre sans être
couvert de quoi que ce soit, et il n'eut de parti à prendre que de
marcher au jour avec toutes les précautions d'un général qui compte bien
qu'il sera attaqué dans sa marche, mais qui a un grand intérêt à
s'allonger toujours pour se tirer d'une situation fâcheuse, et gagner
comme il pourra un pays plus couvert et coupé, à trois bonnes lieues
d'où il se trouvait.

Le maréchal de Villeroy manda dès qu'il fut jour à M. du Maine
d'attaquer et d'engager l'action, comptant de le soutenir avec toute son
armée, et qui pour arriver à temps avait besoin que les ennemis fussent
retardés, puis empêchés de marcher par l'engagement dans lequel notre
gauche les aurait mis. Impatient de ne point entendre l'effet de cet
ordre, il dépêche de nouveau M. du Maine, et redouble cinq ou six fois.
M. du Maine voulut d'abord reconnaître, puis se confesser, après mettre
son aile en ordre qui y était depuis longtemps et qui pétillait d'entrer
en action. Pendant tous ces délais, Vaudemont marchait le plus
diligemment que la précaution le lui pouvait permettre. Les officiers
généraux de notre gauche se récriaient. Montrevel, lieutenant général le
plus ancien d'eux, ne pouvant plus souffrir ce qu'il voyait, pressa M.
du Maine, lui remontra l'instance des ordres réitérés qu'il recevait du
maréchal de Villeroy, la victoire facile et sûre, l'importance pour sa
gloire, pour le succès de Namur, pour le grand fruit qui s'en devait
attendre de l'effroi et de la nudité des Pays-Bas après la déroute de la
seule armée qui les pouvait défendre. Il se jeta à ses mains, il ne put
retenir ses larmes, rien ne fut refusé ni réfuté, mais tout fut inutile.
M. du Maine balbutiait, et fit si bien que l'occasion échappa, et que M.
de Vaudemont en fut quitte pour le plus grand péril qu'une armée pût
courir d'être entièrement défaite, si son ennemi qui la voyait et la
comptait homme par homme eût fait le moindre mouvement pour l'attaquer.

Toute notre armée était au désespoir, et personne ne se contraignait de
dire ce que l'ardeur, la colère et l'évidence suggéraient. Jusqu'aux
soldats et aux cavaliers montraient leur rage sans se méprendre\,; en un
mot, officiers et soldats, tous furent plus outrés que surpris. Tout ce
que put faire le maréchal de Villeroy fut de débander trois régiments de
dragons, menés par Artagnan, maréchal de camp, sur leur arrière-garde,
qui prirent quelques drapeaux et mirent quelque désordre dans les
troupes qui faisaient l'arrière-garde de tout.

Le maréchal de Villeroy, plus outré que personne, était trop bon
courtisan pour s'excuser sur autrui. Content du témoignage de toute son
armée et de ce que toute son armée n'avait que trop vu et senti, et des
clameurs dont elle ne s'était pas tenue, il dépêcha un de ses
gentilshommes au roi, à qui il manda que la diligence dont Vaudemont
avait usé dans sa retraite l'avait sauvé de ses espérances qu'il avait
crues certaines, et sans entrer en aucun détail se livra à tout ce qu'il
pourrait lui en arriver. Le roi, qui depuis vingt-quatre heures les
comptait toutes dans l'attente de la nouvelle si décisive d'une
victoire, fut bien surpris quand il ne vit que ce gentilhomme au lieu
d'un homme distingué, et bien touché quand il apprit la tranquillité de
cette journée. La cour en suspens, qui pour son fils, qui pour son mari,
qui pour son frère, demeura dans l'étonnement, et les amis du maréchal
de Villeroy dans le dernier embarras. Un compte si général et si court
rendu d'un événement si considérable et si imminent réduit à rien, tint
le roi en inquiétude\,; il se contint en attendant un éclaircissement du
temps. Il avait soin de se faire lire toutes les gazettes de Hollande.
Dans la première qui parut, il lut une grosse action à la gauche, des
louanges excessives de la valeur de M. du Maine\,; que ses blessures
avaient arrêté le succès et sauvé M. de Vaudemont et que M. du Maine
avait été emporté sur un brancard. Cette raillerie fabuleuse piqua le
roi, mais il le fut bien davantage de la gazette suivante qui se
rétracta du combat qu'elle avait raconté, et ajouta que M. du Maine
n'avait pas même été blessé. Tout cela, joint au silence qui avait régné
depuis cette journée, et au compte si succinct que le maréchal de
Villeroy lui en avait rendu et sans chercher aucune excuse, donna au roi
des soupçons qui l'agitèrent.

Lavienne, baigneur à Paris fort à la mode, était devenu le sien du temps
de ses amours. Il lui avait plu par ses drogues qui l'avaient mis en
état plus d'une fois de se satisfaire davantage, et ce chemin l'avait
conduit à devenir un des quatre premiers valets de chambre. C'était un
fort honnête homme, mais rustre, brutal et franc\,; et cette franchise,
dans un homme d'ailleurs vrai, avait accoutumé le roi à lui demander ce
qu'il n'espérait pas pouvoir tirer d'ailleurs quand c'étaient des choses
qui ne passaient point sa portée. Tout cela conduisit jusqu'à un voyage
à Marly, et ce fut là où il questionna Lavienne. Celui-ci montra son
embarras, parce que, dans la surprise, il n'eut pas la présence d'esprit
de le cacher. Cet embarras redoubla la curiosité du roi et enfin ses
commandements. Lavienne n'osa pousser plus loin la résistance\,; il
apprit au roi ce qu'il eût voulu pouvoir ignorer toute la vie, et qui le
mit au désespoir. Il n'avait eu tant d'embarras, tant d'envie, tant de
joie de mettre M. de Vendôme à la tête d'une armée que pour y porter M.
du Maine, toute son application était d'en abréger les moyens en se
débarrassant des princes du sang par leur concurrence entre eux. Le
comte de Toulouse étant amiral avait sa destination toute faite. C'était
donc pour M. du Maine qu'étaient tous ses soins. En ce moment il les vit
échoués, et la douleur lui en fut insupportable. Il sentit pour ce cher
fils tout le poids du spectacle de son armée, et des railleries que les
gazettes lui apprenaient qu'en faisaient les étrangers, et son dépit en
fut inconcevable.

Ce prince, si égal à l'extérieur et si maître de ses moindres mouvements
dans les événements les plus sensibles, succomba sous cette unique
occasion. Sortant de table à Marly avec toutes les dames et en présence
de tous les courtisans, il aperçut un valet du serdeau \footnote{Lieu ou
  office de la maison du roi où l'on portait ce que l'on desservait de
  sa table.} qui en desservant le fruit mit un biscuit dans sa poche.
Dans l'instant il oublie toute sa dignité, et sa canne à la main qu'on
venait de lui rendre avec son chapeau, court sur ce valet qui ne
s'attendait à rien moins, ni pas un de ceux qu'il sépara sur son
passage, le frappe, l'injurie et lui casse sa canne sur le corps\,: à la
vérité, elle était de roseau et ne résista guère.

De là, le tronçon à la main et de l'air d'un homme qui ne se possédait
plus, et continuant à injurier ce valet qui était déjà bien loin, il
traversa ce petit salon et un antichambre, et entra chez
M\textsuperscript{me} de Maintenon, où il fut près d'une heure, comme il
faisait souvent à Marly après dîner. Sortant de là pour repasser chez
lui, il trouva le P. de La Chaise. Dès qu'il l'aperçut parmi les
courtisans\,: « Mon père, lui dit-il fort haut, j'ai bien battu un
coquin et lui ai cassé ma canne sur le dos\,; mais je ne crois pas avoir
offensé Dieu.\,» Et tout de suite lui raconta le prétendu crime. Tout ce
qui était là tremblait encore de ce qu'il avait vu ou entendu des
spectateurs. La frayeur redoubla à cette reprise\,: les plus familiers
bourdonnèrent contre ce valet\,; et le pauvre père fit semblant
d'approuver entre ses dents pour ne pas irriter davantage, et devant
tout le monde. On peut juger si ce fut la nouvelle, et la terreur
qu'elle imprima, parce que personne n'en put alors deviner la cause, et
que chacun comprenait aisément que celle qui avait paru ne pouvait être
la véritable. Enfin tout vient à se découvrir, et peu à peu et d'un ami
à l'autre, on apprit enfin que Lavienne, forcé par le roi, avait été
cause d'une aventure si singulière et si indécente.

Pour n'en pas faire à deux fois, ajoutons ici le mot de M. d'Elbœuf.
Tout courtisan qu'il était, le vol que les bâtards avaient pris lui
tenait fort au cœur, et le repentir peut-être de son adoration de la
croix après MM. de Vendôme. Comme la campagne était à son déclin et les
princes sur leur départ, il pria M. du Maine, et devant tout le monde,
de lui dire où il comptait de servir la campagne suivante, parce que, où
que ce fût, il y voulait servir aussi. Et après s'être fait presser pour
savoir pourquoi, il répondit que « c'est qu'avec lui on était assuré de
sa vie.\,» Ce trait accablant et sans détour fit un grand bruit. M. du
Maine baissa les yeux et n'osa répondre une parole\,; sans doute qu'il
la lui garda bonne\,; mais M. d'Elbœuf fort bien avec le roi et par lui
et par les siens, était d'ailleurs en situation de ne s'en soucier
guère.

Plus le roi fut outré de cette aventure qui, influa tant sur ses
affaires, mais que le personnel lui rendit infiniment plus sensible,
plus il sut de gré au maréchal de Villeroy, et plus encore
M\textsuperscript{me} de Maintenon augmenta d'amitié pour lui. Sa faveur
devint plus éclatante, la jalousie de tout ce qui était le mieux traité
du roi, et la crainte même des ministres.

Le fruit amer de cet événement en Flandre fut la prise de la ville de
Namur qui capitula le 4 août, après {[}plusieurs{]} jours de tranchée
ouverte.

Le prince d'Orange, pour éviter les difficultés de ce que le roi ne le
reconnaissait point, ne parut en rien, ni par conséquent le maréchal de
Boufflers\,; et tout se passa sous leur direction et à peu près comme ce
dernier le demanda, entre l'électeur de Bavière et Guiscard, qui
signèrent. Maulevrier, fils aîné du lieutenant général, mort chevalier
de l'ordre, Vieuxbourg, gendre d'Harlay conseiller d'État qui l'était de
Boucherat chancelier de France, et Morstein, tous trois colonels
d'infanterie, et de grande espérance, y furent tués. Ce dernier était
fils du grand trésorier de Pologne qui avait autrefois été ambassadeur
ici. Il s'était fort enrichi et avait excité l'envie de ses
compatriotes. La peur qu'il eût d'être poussé le fit retirer en France
avec sa femme, ce fils unique et quantité de richesses. Elles
séduisirent le duc de Chevreuse qui n'avait rien à donner à ses
filles\,; il en donna une au jeune Morstein, dont le monde fut assez
surpris. Par l'événement il avait bien fait\,: ce jeune homme, s'il eût
vécu, eût été un grand sujet en tous genres. Je le regrettai fort et
Maulevrier, qui étaient fort de mes amis. Nous n'avons guère perdu que
douze cents hommes\,; tout ce qui était sain se retira au château.

Montal cependant avait pris Dixmude et Deinse, et, par ordre du roi, on
avait retenu les garnisons\,: c'est-à-dire que, s'étant rendues
prisonnières de guerre, on n'avait pas voulu les échanger. Le maréchal
de Villeroy bombarda aussi Bruxelles qui fut fort maltraité, en
représailles de nos côtes\,; ensuite il eut ordre de tenter tout pour le
secours de Namur\,; mais l'occasion, qui est chauve, ne revint plus. Il
trouva les ennemis si bien retranchés sur la Mehaigne, qu'il ne put les
attaquer. Il la longea, et, chemin faisant, il la fit passer aux
brigades de cavalerie de Praslin et de Sousternon qu'il lâcha sur une
quarantaine d'escadrons des ennemis dont ces brigades se trouvèrent le
plus à portée, et qui les poussèrent fort vivement\,: Praslin s'y
distingua fort, et Villequier y eut une main estropiée\,; cette blessure
lui fit moins d'honneur sur les lieux qu'à la cour, mais tout cela ne
fut qu'une échauffourée. Le secours demeura impossible. L'armée
s'éloigna\,; et le château, après avoir pensé être emporté aux deux
derniers assauts, capitula pour sortir le 5 septembre, n'y ayant pas
trois mille hommes en santé de toute la garnison.

La capitulation fut honorable, traitée et signée corme celle de la
ville. La difficulté fut pour la sortie du maréchal de Boufflers\,: il
en faisait une grande, avec raison, de saluer l'électeur de Bavière de
l'épée, et n'en aurait pu faire au prince d'Orange s'il avait été
reconnu. Enfin il fallut s'y résoudre, parce que ce dernier voulut au
moins rendre le salut équivoque. Pour cela, l'électeur se tint toujours
à sou côté, et n'ôtait son chapeau qu'après que le prince d'Orange avait
ôté le sien, qui, par cette affectation, marquait qu'il recevait le
salut, et que l'électeur ne se découvrait ensuite que parce que lui-même
était découvert. Cela se passa donc de la sorte à l'égard du maréchal,
puis de Guiscard, sans mettre pied à terre, et de tout ce qui les
suivit. Les compliments se passèrent entre l'électeur et eux\,; et le
prince d'Orange ne s'y mêla point, parce qu'il n'aurait point eu de
\emph{Sire} ni de \emph{majesté\,;} mais l'électeur lui rapportait tout,
ne lui parlait jamais que le chapeau à la main\,; le prince d'Orange se
contentait de se découvrir quelquefois seulement et peu, pour lui parler
ou pour lui répondre, et le plus souvent sans se découvrir.

Un quart d'heure après que le maréchal de Boufflers eut passé devant eux
et qu'il suivait son chemin entretenu par des officiers ennemis des plus
principaux, il fut arrêté par Overkerke et L'Estang, lieutenants des
gardes du prince d'Orange. Overkerke était un bâtard de Nassau, général
en chef des troupes de Hollande, grand écuyer du prince d'Orange, et de
tous temps dans sa confiance la plus intime\,: L'Estang y était aussi.
Le maréchal fut fort surpris et se récria que c'était violer la
capitulation\,; mais, pour tout ce qu'il put dire et ce qu'il se trouva
des nôtres auprès de lui, ils n'étaient pas les plus forts, et il fallut
monter dans un carrosse qu'on tenait là tout prêt. Du reste cette
violence se passa avec toute la politesse, les égards et le respect que
les ennemis y purent mettre. Portland, favori dès sa jeunesse du prince
d'Orange, sous le nom de Bentinck, et Hollandais, et qu'il avait fait
comte en Angleterre et chevalier de la Jarretière, avec Dyckweldt, frère
d'Overkerke et général, vinrent trouver le maréchal dans la ville de
Namur où il fut conduit, et lui expliquèrent qu'il était arrêté en
représailles des garnisons de Deinse et de Dixmude, prisonnières de
guerre, que le roi n'avait pas voulu laisser racheter. Guiscard
cependant était retourné à l'électeur de Bavière qui lui dit être très
fâché de cet arrêt, qu'il n'avait su que le matin et auquel il ne
pouvait rien\,; et Guiscard, dépêché par le maréchal, vint tout de suite
rendre compte au roi de cet événement et de tout le siège, qui fut très
étonné et piqué de ce procédé. Le maréchal de Boufflers eut toute sa
maison avec lui, la garde et tous les honneurs partout de général
d'armée, et la liberté de se promener partout. Il aurait bien pu faire
rendre les garnisons de Deinse et de Dixmude pour se tirer de prison,
mais il eut la sagesse de n'user point de ce pouvoir, et d'attendre ce
qu'il plairait au roi. L'électeur lui fit faire force compliments et
excuses de ne l'aller pas voir, sur ce qu'il craignait que cette visite
déplût au prince d'Orange.

Guiscard en arrivant fut déclaré chevalier de l'ordre, pour la première
fête. Mesgriny, qui avait été mandé pour rendre compte du siège avant
qu'on sût l'arrivée de Guiscard, eut six mille livres de pension et un
cordon rouge\,; et le roi manda par un courrier au maréchal de Boufflers
qu'il le faisait duc vérifié au parlement. Ce courrier le trouva à Huy,
gardé par L'Estang, mais avec toute sorte de liberté et tous les
honneurs qu'il aurait sur notre propre frontière. Il lui envoya deux
jours après pouvoir de rendre les garnisons de Deinse et de Dixmude qui
le trouva à Maestricht. Il envoya à milord Portland et l'affaire ne
traîna pas. Le maréchal de Boufflers partit dès que tout fut convenu, et
fut reçu à Fontainebleau avec des applaudissements extraordinaires. Il
fit faire Mesgrigny lieutenant général, et avancer en grade tout ce qui
était avec lui dans Namur, ce qui lui fit beaucoup d'honneur. M. le duc
de Chartres était revenu aussitôt après la capitulation. Le prince
d'Orange, peu de jours après, s'en alla à Breda, laissant l'armée à
l'électeur de Bavière\,; et en même temps M. le Duc, M. le prince de
Conti, M. du Maine et M. le comte de Toulouse revinrent à la cour. Le
prince d'Orange, quelque mesuré qu'il fût, ne put s'empêcher d'insulter
à notre perte lorsqu'il apprit toutes les récompenses données au
maréchal de Boufflers, à Guiscard et à tout ce qui avait défendu
Namur\,; il dit que sa condition était bien malheureuse d'avoir toujours
à envier le sort du roi, qui récompensait plus libéralement la perte
d'une place, que lui ne pouvait faire tant d'amis et de dignes
personnages qui lui en avaient fait la conquête. Les armées ne firent
plus que subsister et se séparèrent à la fin d'octobre, et tous les
généraux d'armée revinrent à la cour.

J'ai laissé le maréchal de Joyeuse séparé par le Rhin du prince Louis de
Bade, et M. et M\textsuperscript{me} la maréchale de Lorges à Landau, où
après que nous eûmes repassé je les vins trouver. Pendant plus de six
semaines que nous y demeurâmes, toute l'armée, qui n'était pas loin, les
vint voir. La santé rétablie, M. le maréchal de Lorges eut impatience de
retourner à la tête de son armée\,; et M\textsuperscript{me} la
maréchale s'en alla à Paris. Il est impossible de décrire la joie et les
acclamations de toute l'armée à ce retour de son général\,; tout ce qui
la put quitter vint deux lieues à sa rencontre. Les décharges
d'artillerie et de mousqueterie furent générales et réitérées malgré
toutes ses défenses. Toute la nuit le camp fut en feu et en bonne chère,
et des tables et des feux de joie devant tous les corps. Le maréchal de
Joyeuse ne s'était pas fait aimer. Il était de plus accusé d'avoir
beaucoup pris, et d'avoir réduit la cavalerie et les équipages à une
maigreur extrême, faute de fourrages dans un pays qui en regorgeait\,;
et ce passage de lui à un général qui se faisait adorer par ses manières
et par son désintéressement causa cet incroyable transport de joie qui
fut universel.

Peu après ce retour, l'armée fut partagée pour la commodité des
subsistances. Les maréchaux demeurèrent vers l'Alsace avec une partie,
et Tallard mena l'autre vers la Navé et le Hondsrück, où j'allai avec
mon régiment. Je n'y demeurai pas longtemps que j'appris que le maréchal
de Lorges était tombé en apoplexie, et sur-le-champ je partis pour
l'aller trouver avec le comte de Roucy et le chevalier de Roye, ses
neveux, et une escorte que Tallard nous donna. Le mal eût été léger si
on y eût pourvu à temps, mais il lui est ordinaire de ne se laisser pas
sentir\,; et il n'y eut pas moyen de persuader le malade de se conduire
et de faire ce qu'il aurait fallu\,; tellement que le mal augmenta au
point qu'il en fallut venir aux remèdes les plus violents qui, avec un
grand péril, réussirent. Cependant arrivèrent les quartiers de
fourrages, et en même temps M\textsuperscript{me} la maréchale de Lorges
à Strasbourg qui n'avait eu guère le temps de se reposer à Paris. Nous
fûmes tous l'y voir et y demeurer jusqu'à son départ avec M. le maréchal
pour Vichy. En même temps arrivèrent les quartiers d'hiver, et je m'en
allai à Paris.

\hypertarget{chapitre-xvii.}{%
\chapter{CHAPITRE XVII.}\label{chapitre-xvii.}}

1695

~

\relsize{-1}

{\textsc{Brias archevêque de Cambrai.}} {\textsc{- Sa mort.}} {\textsc{-
Abbé de Fénelon.}} {\textsc{- M\textsuperscript{me} Guyon.}} {\textsc{-
Fénelon précepteur des enfants de France.}} {\textsc{- Fénelon
archevêque de Cambrai.}} {\textsc{- Boucherat, chancelier, ferme sa
porte aux carrosses mêmes des évêques.}} {\textsc{- Harlay archevêque de
Paris.}} {\textsc{- Dégoût de ses dernières années.}} {\textsc{- Sa
mort.}} {\textsc{- Sa dépouille.}} {\textsc{- Coislin, évêque d'Orléans,
nommé au cardinalat.}} {\textsc{- Noailles, évêque-comte de Châlons,
archevêque de Paris, et son frère, évêque-comte de Châlons.}} {\textsc{-
Régularisation de la Trappe.}} {\textsc{- Évêque-duc de Langres.}}
{\textsc{- Gordes\,; sa mort.}} {\textsc{- Abbé de Tonnerre évêque-duc
de Langres\,; sa modestie.}} {\textsc{- M. le maréchal de Lorges ne sert
plus.}} {\textsc{- Forte picoterie des princesses.}} \relsize{1}

~

Avant de parler de ce qui se passa depuis mon retour de l'armée, il faut
dire ce qui se passa à la cour pendant là campagne. M. de Brias,
archevêque de Cambrai, était mort, et le roi avait donné ce grand
morceau à l'abbé de Fénelon, précepteur des enfants de France. Brias
était archevêque lorsque le roi prit Cambrai. C'était un bon gentilhomme
flamand, qui fit très bien pour l'Espagne pendant le siège, et aussi
bien pour la France aussitôt après. Il le promit au roi avec une
franchise qui lui plut, et qui toujours depuis fut si bien soutenue de
l'effet, qu'il s'acquit une considération très marquée de la part du roi
et de ses ministres, qui tous le regrettèrent et son diocèse infiniment.
Il n'en sortait presque jamais, le visitait en vrai pasteur, et en
faisait toutes les fonctions avec assiduité. Grand aumônier, libéral aux
troupes, et prêt à servir tout le monde, il avait une grande, bonne et
fort longue table tous les jours, il l'aimait fort et en faisait grand
usage et en bonne compagnie, et à la flamande, mais sans excès, et s'en
levait souvent pour le moindre du peuple qui l'envoyait chercher pour se
confesser à lui, ou pour recevoir sa bénédiction et mourir entre ses
bras, dont il s'acquittait en vrai apôtre.

Fénelon était un homme de qualité qui n'avait rien, et qui, se sentant
beaucoup d'esprit, et de cette sorte d'esprit insinuant et enchanteur,
avec beaucoup de talents, de grâces et du savoir, avait aussi beaucoup
d'ambition. Il a voit frappé longtemps à toutes les portes sans se les
pouvoir faire ouvrir. Piqué contre les jésuites, où il s'était adressé
d'abord comme aux maîtres des grâces de son état, et rebuté de ne
pouvoir prendre avec eux, il se tourna aux jansénistes pour se dépiquer,
par l'esprit et par la réputation qu'il se flattait de tirer d'eux, des
dons de la fortune qui l'avait méprisé. Il fut un temps assez
considérable à s'initier, et parvint après à être des repas
particuliers, que quelques importants d'entre eux faisaient alors une ou
deux fois la semaine chez la duchesse de Brancas. Je ne sais s'il leur
parut trop fin, ou s'il espéra mieux ailleurs qu'avec gens avec qui il
n'y avait rien à partager que des plaies, mais peu à peu sa liaison avec
eux se refroidit, et à force de tourner autour de Saint-Sulpice, il
parvint à y en former une dont il espéra mieux. Cette société de prêtres
commençait à percer, et d'un séminaire d'une paroisse de Paris à
s'étendre. L'ignorance, la petitesse des pratiques, le défaut de toutes
protections, et le manque de sujets de quelque distinction en aucun
genre, leur inspira une obéissance aveugle pour Rome et pour toutes ses
maximes, un grand éloignement de tout ce qui passait pour jansénisme, et
une dépendance des évêques qui les fit successivement désirer dans
beaucoup de diocèses. Ils parurent un milieu très utile aux prélats qui
craignaient également la cour sur les soupçons de doctrine, et la
dépendance des jésuites qui les mettaient sous leur joug dès qu'ils
s'étaient insinués chez eux, ou les perdaient sans ressource, de manière
que ces sulpiciens s'étendirent fort promptement. Personne parmi eux qui
pût entrer en comparaison sur rien avec l'abbé de Fénelon\,; de sorte
qu'il trouva là de quoi primer à l'aise et se faire des protecteurs qui
eussent intérêt à l'avancer pour en être protégés à leur tour. Sa piété
qui se faisait toute à tous, et sa doctrine qu'il forma sur la leur en
abjurant tout bas tout ce qu'il avait pu contracter d'impur parmi ceux
qu'il abandonnait, les charmes, les grâces, la douceur, l'insinuation de
son esprit le rendirent un ami cher à cette congrégation nouvelle, et
lui y trouva ce qu'il cherchait depuis longtemps, des gens à qui se
rallier, et qui pussent et voulussent le porter. En attendant les
occasions, il les cultivait avec grand soin sans toutefois être tenté de
quelque chose d'aussi étroit pour ses vues que de se mettre parmi eux,
et cherchait toujours à faire des connaissances et des amis. C'était un
esprit coquet qui, depuis les personnes les plus puissantes jusqu'à
l'ouvrier et au laquais, cherchait à être goûté et voulait plaire, et
ses talents en ce genre secondaient parfaitement ses désirs.

Dans ces temps-là, obscur encore, il entendit parler de
M\textsuperscript{me} Guyon, qui a fait depuis tant de bruit dans le
monde qu'elle y est trop connue pour que je m'arrête sur elle en
particulier. Il la vit, leur esprit se plut l'un à l'autre, leur sublime
s'amalgama. Je ne sais s'ils s'entendirent bien clairement dans ce
système et cette langue nouvelle qu'on vit éclore d'eux dans les suites,
mais ils se le persuadèrent, et la liaison se forma entre eux. Quoique
plus connue que lui alors, elle ne l'était pas néanmoins encore
beaucoup, et leur union ne fut point aperçue, parce que personne ne
prenait garde à eux, et Saint-Sulpice même l'ignora.

Le duc de Beauvilliers devint gouverneur des enfants de France, sans y
avoir pensé, comme malgré lui. Il avait été fait chef du conseil royal
des finances, à la mort du maréchal de Villeroy, par l'estime et la
confiance du roi. Elle fut telle qu'excepté Moreau que, de premier valet
de garde-robe, il fit premier valet de chambre de Mgr le duc de
Bourgogne, il laissa au duc de Beauvilliers la disposition entière des
précepteurs, sous-gouverneurs et de tous les autres domestiques de ce
jeune prince, quelque résistance qu'il y fit. En peine de choisir un
précepteur, il s'adressa à Saint-Sulpice où il se confessait depuis
longtemps et qu'il aimait et protégeait fort. Il avait déjà ouï parler
de l'abbé de Fénelon avec éloge\,; ils lui vantèrent sa piété, son
esprit, son savoir, ses talents, enfin ils le lui proposèrent\,; il le
vit, il en fut charmé, il le fit précepteur. Il le fut à peine qu'il
comprit de quelle importance il était pour sa fortune de gagner
entièrement celui qui venait de le mettre en chemin de la faire et le
duc de Chevreuse, son beau-frère, avec qui il n'était qu'un, et qui tous
deux étaient au plus haut point de la confiance du roi et de
M\textsuperscript{me} de Maintenon. Ce fut là son premier soin, auquel
il réussit tellement au delà de ses espérances qu'il devint très
promptement le maître de leur cœur et de leur esprit et le directeur de
leurs âmes. M\textsuperscript{me} de Maintenon dînait de règle une et
quelquefois deux fois la semaine à l'hôtel de Beauvilliers et de
Chevreuse, en cinquième entre les deux sœurs et les deux maris, avec la
clochette sur la table, pour n'avoir point de valets autour d'eux et
causer sans contrainte. C'était un sanctuaire qui tenait toute la cour à
leurs pieds, et auquel Fénelon fut enfin admis. Il eut auprès de
M\textsuperscript{me} de Maintenon presque autant de succès qu'il en
avait eu auprès des deux ducs. Sa spiritualité l'enchanta\,; la cour
s'aperçut bientôt des pas de géant de l'heureux abbé, et s'empressa
autour de lui. Mais le désir d'être libre et tout entier à ce qu'il
s'était proposé, et la crainte encore de déplaire aux ducs et à
M\textsuperscript{me} de Maintenon, dont le goût allait à une vie
particulière et fort séparée, lui fit faire bouclier de modestie et de
ses fonctions de précepteur, et le rendit encore plus cher aux seules
personnes qu'il avait captivées, et qu'il avait tant d'intérêt de
retenir dans cet attachement.

Parmi ces soins, il n'oubliait pas sa bonne amie M\textsuperscript{me}
Guyon\,; il l'avait déjà vantée aux deux ducs et enfin à
M\textsuperscript{me} de Maintenon. Il la leur avait même produite, mais
comme avec peine et pour des moments, comme une femme tout en Dieu, et
que l'humilité et l'amour de la contemplation et de la solitude
retenaient dans les bornes les plus étroites, et qui craignait surtout
d'être connue. Son esprit plut extrêmement à M\textsuperscript{me} de
Maintenon\,; ses réserves, mêlées de flatteries fines, la gagnèrent.
Elle voulut l'entendre sur des matières de piété, on eut peine à l'y
résoudre. Elle sembla se rendre aux charmes et à la vertu de
M\textsuperscript{me} de Maintenons, et des filets si bien préparés la
prirent. Telle était la situation de Fénelon, lorsqu'il devint
archevêque de Cambrai et qu'il acheva de se faire admirer par n'avoir
pas fait un pas vers ce grand bénéfice\,; et qu'il rendit en même temps
une belle abbaye qu'il avait eue lorsqu'il fut précepteur, et qui,
jusqu'à Cambrai, fut sa seule possession. Il n'avait eu garde de
chercher à se procurer Cambrai\,; la moindre étincelle d'ambition aurait
détruit tout son édifice, et de plus ce n'était pas Cambrai qu'il
souhaitait.

Peu à peu il s'était approprié quelques brebis distinguées du petit
troupeau que M\textsuperscript{me} Guyon s'était fait, et qu'il ne
conduisait pourtant que sous la direction de cette prophétesse. La
duchesse de Mortemart, sœur des duchesses de Chevreuse et de
Beauvilliers, M\textsuperscript{me} de Morstein, fille de la première,
mais surtout la duchesse de Béthune, étaient les principales. Elles
vivaient à Paris, et ne venaient guère à Versailles qu'en cachette et
pour des instants, lorsque, pendant les voyages de Marly, où Mgr le duc
de Bourgogne n'allait point encore, ni par conséquent son gouverneur,
M\textsuperscript{me} de Guyon faisait des échappées de Paris chez ce
dernier et y faisait des instructions à ces dames. La comtesse de
Guiche, fille aînée de M. de Noailles, qui passait sa vie à la cour, se
dérobait tant qu'elle pouvait pour profiter de cette manne. L'Échelle et
Dupuy, gentilshommes de la manche de Mgr le duc de Bourgogne, y étaient
aussi admis, et tout cela se passait avec un secret et un mystère qui
donnaient un nouveau sel à ces faveurs.

Cambrai fut un coup de foudre pour tout ce petit troupeau. Ils voyaient
l'archevêque de Paris menacer ruine\,; c'était Paris qu'ils voulaient
tous, et non Cambrai, qu'ils considérèrent avec mépris comme un diocèse
de campagne dont la résidence, qui ne se pourrait éviter de temps en
temps, les priverait de leur pasteur. Paris l'aurait mis à la tête du
clergé, et dans une place de confiance immédiate et durable qui aurait
fait compter tout le monde avec lui, et qui l'eût porté, dans une
situation à tout oser avec succès pour M\textsuperscript{me} Guyon et sa
doctrine qui se tenait encore dans le secret entre eux. Leur douleur fut
donc profonde de ce que le reste du monde prit pour une fortune
éclatante, et la comtesse de Guiche en fut outrée jusqu'à n'en pouvoir
cacher ses larmes. Le nouveau prélat n'avait pas négligé les prélats qui
faisaient le plus de figure, qui de leur côté regardèrent comme une
distinction d'être approchés de lui. Saint-Cyr, ce lieu si précieux et
si peu accessible, fut le lieu destiné à son sacre, et M. de Meaux, le
dictateur alors de l'épiscopat et de la doctrine, fut celui qui le
sacra. Les enfants de France en furent spectateurs,
M\textsuperscript{me} de Maintenon y assista avec sa petite et étroite
cour intérieure, personne d'invité, et portes fermées à l'empressement
de faire sa cour.

Il y avait eu cet été une assemblée du clergé, et c'était la grande,
comme il y en a une grande et petite de cinq ans en cinq ans,
c'est-à-dire de quatre ou de deux députés par province. Le chancelier
Boucherat, dès qu'il fut dans cette grande place, ferma sa porte aux
carrosses des magistrats, puis des gens de condition sans titre, enfin
des prélats. Jamais chancelier n'avait imaginé cette distinction, et la
nouveauté sembla d'autant plus étrange, que les princes du sang n'ont
jamais fermé la porte de la cour à aucun carrosse. On cria, on se moqua,
mais chacun eut affaire au chancelier, et comme en ce temps-ci rien ne
décide plus que les besoins, on subit\,: cela forma l'exemple, et il ne
s'en parla plus. À la fin de cette assemblée qui se tendit à
Saint-Germain, elle fit une députation au chancelier pour mettre la
dernière main aux affaires, et l'archevêque de Bourges, fils du duc de
Gesvres, était à la tête. Quand leurs carrosses se présentèrent à la
chancellerie à Versailles, la porte ne s'ouvrit point\,; on parlementa,
les députés prétendirent que le chancelier était convenu de les laisser
entrer, non à la vérité comme évêques, mais comme députés du premier
ordre du royaume. Lui maintint qu'ils avaient mal entendu. Conclusion,
qu'ils n'entrèrent point, mais aussi qu'ils ne le voulurent pas voir
chez lui, et que par accommodement tout se finit entre eux dans la pièce
du château où le chancelier tient le conseil des parties \footnote{Voy.
  notes à la fin du volume.}.

Harlay, archevêque de Paris, avait présidé à cette assemblée, et lui qui
avait toujours régné sur le clergé par la faveur déclarée et la
confiance du roi qu'il avait possédée toute sa vie, y avait essuyé
toutes sortes de dégoûts. L'exclusion que peu à peu le P. de La Chaise
était parvenu à lui donner de toute concurrence en la distribution des
bénéfices l'avait déjà éloigné du roi\,; et M\textsuperscript{me} de
Maintenon, à qui il avait déplu d'une manière implacable en s'opposant à
la déclaration du mariage dont il avait été l'un des trois témoins,
l'avait coulé à fond. Le mérite qu'il s'était acquis de tout le royaume,
et qui l'avait de plus en plus ancré dans la faveur du roi, dans
l'assemblée fameuse de 1682, lui fut tourné à poison quand d'autres
maximes prévalurent. Son profond savoir, l'éloquence et la facilité de
ses sermons, l'excellent choix des sujets et l'habile conduite de son
diocèse, jusqu'à sa capacité dans les affaires et l'autorité qu'il y
avait acquise dans le clergé, tout cela fut mis en opposition de sa
conduite particulière, de ses mœurs galantes, de ses manières de
courtisan du grand air. Quoique toutes ces choses eussent été
inséparables de lui depuis son épiscopat et ne lui eussent jamais nui,
elles devinrent des crimes entre les mains de M\textsuperscript{me} de
Maintenon, quand sa haine de puis quelques années lui eut persuadé de le
perdre, et elle ne cessa de lui procurer des déplaisirs. Cet esprit
étendu, juste, solide, et toutefois fleuri, qui pour la partie du
gouvernement en faisait un grand évêque, et pour celle du monde un grand
seigneur fort aimable et un courtisan parfait quoique fort noblement, ne
put s'accoutumer à cette décadence et au discrédit qui l'accompagna. Le
clergé, qui s'en aperçut et à qui l'envie n'est pas étrangère, se plut à
se venger de la domination quoique douce et polie qu'il en avait
éprouvée, et lui résista pour le plaisir de l'oser et de le pouvoir. Le
monde, qui n'eut plus besoin de lui pour des évêchés et des abbayes,
l'abandonna. Toutes les grâces de son corps et de son esprit, qui
étaient infinies et qui lui étaient parfaitement naturelles, se
flétrirent. Il ne se trouva de ressource qu'à se renfermer avec sa bonne
amie la duchesse de Lesdiguières qu'il voyait tous les jours de sa vie,
ou chez elle ou à Conflans, dont il avait fait un jardin délicieux, et
qu'il tenait si propre, qu'à mesure qu'ils s'y promenaient tous deux,
des jardiniers les suivaient à distance pour effacer leurs pas avec des
râteaux.

Les vapeurs gagnèrent l'archevêque\,; elles s'augmentèrent bientôt, et
se tournèrent en légères attaques d'épilepsie. Il le sentit et défendit
si étroitement à ses domestiques d'en parler et d'aller chercher du
secours quand ils le verraient en cet état, qu'il ne fut que trop bien
obéi. Il passa ainsi ses deux ou trois dernières années. Les chagrins de
cette dernière assemblée l'achevèrent. Elle finit avec le mois de
juillet\,; aussitôt après il s'alla reposer à Conflans. La duchesse de
Lesdiguières n'y couchait jamais, mais elle y allait toutes les
après-dînées, et toujours tous deux tout seuls. Le 6 août il passa la
matinée à son ordinaire jusqu'au dîner. Son maître d'hôtel vint
l'avertir qu'il était servi. Il le trouva dans son cabinet, assis sur un
canapé et renversé\,; il était mort. Le P. Gaillard fit son oraison
funèbre à Notre-Dame\,; la matière était plus que délicate, et la fin
terrible. Le célèbre jésuite prit son parti\,; il loua tout ce qui
méritait de l'être, puis tourna court sur la morale. Il fit un
chef-d'œuvre d'éloquence et de piété.

Le roi se trouva fort soulagé, M\textsuperscript{me} de Maintenon encore
davantage. M. de Reims eut sa place de proviseur de Sorbonne, M. de
Meaux celle de supérieur de la maison de Navarre, et M. de Noyon son
cordon bleu. Sa nomination au cardinalat et son archevêché demandent un
peu plus de discussion. M. d'Orléans l'eut, et d'autant plus
agréablement que, ni lui ni pas un des siens, n'avaient eu le temps d'y
penser. M. de Paris était mort à Conflans au milieu du samedi 6 août\,;
le roi ne le sut que le soir. Le lundi matin 8 août, le roi, étant entré
dans son cabinet pour donner l'ordre de sa journée à l'ordinaire, alla
droit à l'évêque d'Orléans, qui se rangea même, croyant que le roi
voulait passer outre\,; mais le roi le prit par le bras sans lui dire un
mot, et le mena en laisse à l'autre bout du cabinet aux cardinaux de
Bouillon et de Furstemberg qui causaient ensemble, et tout de suite leur
dit\,: « Messieurs, je crois que vous me remercierez de vous donner un
confrère comme M. d'Orléans, à qui je donne ma nomination au
cardinalat.\,» À ce mot, l'évêque qui ne s'attendait à rien moins, et
qui ne savait ce que le roi voulait faire de le mener ainsi, se jeta à
ses pieds et lui embrassa les genoux. Grands applaudissements des deux
cardinaux, puis de tout ce qui se trouva dans le cabinet, ensuite de
toute la cour et du public entier où ce prélat était dans une vénération
singulière.

C'était un homme de moyenne taille, gros, court, entassé, le visage
rouge et démêlé, un nez fort aquilin, de beaux yeux avec un air de
candeur, de bénignité, de vertu qui captivait en le voyant, et qui
touchait bien davantage en le connaissant. Il était frère du duc de
Coislin, fils de la fille aînée du chancelier Séguier, qui, d'un second
lit avec M. de Laval, avait eu la maréchale de Rochefort. Le frère du
chancelier était évêque de Meaux et premier aumônier de Louis XIII, puis
de Louis XIV, dont il avait eu la survivance pour son petit-neveu tout
jeune, de manière qu'il avait passé sa vie à la cour. Mais sa jeunesse y
avait été si pure qu'elle était non seulement demeurée sans soupçon,
mais que jeunes et vieux n'osaient dire devant lui une parole trop
libre, et cependant le recherchaient tous, en sorte qu'il a toujours
vécu dans la meilleure compagnie de la cour. Il était riche en abbayes
et en prieurés, dont il faisait de grandes aumônes et dont il vivait. De
son évêché qu'il eut fort jeune, il n'en toucha jamais rien, et en mit
le revenu en entier tous les ans en bonnes œuvres. Il y passait au moins
six mois de l'année, le visitait soigneusement et faisait toutes les
fonctions épiscopales avec un grand soin, et un grand discernement à
choisir d'excellents sujets pour le gouvernement et pour l'instruction
de son diocèse. Son équipage, ses meubles, sa table sentaient la
frugalité et la modestie épiscopales, et, quoiqu'il eût toujours grande
compagnie à dîner et à souper et de la plus distinguée, elle était
servie de bons vivres, mais sans profusion et sans rien de recherché. Le
roi le traita toujours avec une amitié, une distinction, une
considération fort marquées, mais il avait souvent des disputes et
quelquefois fortes sur son départ et son retour d'Orléans. Il louait son
assiduité en son diocèse, mais il était peiné quand il le quittait et
encore quand il demeurait trop longtemps de suite à Orléans. La modestie
et la simplicité avec laquelle M. d'Orléans soutint sa nomination, et
l'uniformité de sa vie, de sa conduite et de tout ce qu'il faisait
auparavant, qu'il continua également depuis, augmentèrent fort encore
l'estime universelle.

L'archevêché de Paris ne fut guère plus long à être déterminé, et devint
le fruit du sage sacrifice du duc de Noailles du commandement de son
armée à M. de Vendôme, et le sceau de son parfait retour dans la faveur.
Son frère avait été sacré évêque de Cahors, en 1680, et avait passé six
mois après à Châlons-sur-Marne. Cette translation lui donna du
scrupule\,; il la refusa et ne s'y soumit que par un ordre exprès
d'Innocent XI. Il y porta son innocence baptismale, et y garda une
résidence exacte, uniquement appliqué aux visites, au gouvernement de
son diocèse et à toutes sortes de bonnes œuvres. Sa mère, qui avait
passé sa vie à la cour, dame d'atours de la reine mère, s'était retirée
auprès de lui depuis bien des années\,; elle y était sous sa conduite et
se confessait à lui tous les soirs, uniquement occupée de son salut dans
la plus parfaite solitude. Ce fut sur ce prélat que le choix du roi
tomba pour Paris. Il le craignit de loin et se hâta de joindre son
approbation à celle de tant d'autres évêques au livre des
\emph{Réflexions morales} du P. Quesnel, pour s'en donner l'exclusion
certaine par les jésuites. Mais il arriva, peut-être pour la première
fois, que le P. de La Chaise ne fut point consulté\,;
M\textsuperscript{me} de Maintenon osa, peut-être aussi pour la première
fois, en faire son affaire. Elle montra au roi des lettres pressantes de
MM. Thiberge et Brisacier, supérieurs des Missions étrangères, que, pour
contrecarrer les jésuites dont le crédit la gênait, elle avait mis à la
mode auprès du roi. Il lui importait que l'archevêque de Paris ne fût
point à eux pour qu'il fût à elle\,; M. de Noailles lui était un bon
garant\,: en un mot elle l'emporta, et M. de Châlons fut nommé à son
insu et à l'insu du P. de La Chaise. Le camouflet était violent, aussi
les jésuites ne l'ont-ils jamais pardonné à ce prélat. Il était pourtant
si éloigné d'y avoir part que malgré les mesures qu'il avait prises pour
s'en éloigner, lorsqu'il se vit nommé il ne put se résoudre à accepter,
et qu'il ne baissa la tête sous ce qu'il jugeait être un joug trop
pesant, qu'à force d'ordres réitérés auxquels enfin il ne put résister.
Il avait été quinze ans à Châlons et il avait la domerie \footnote{Le
  mot \emph{domerie}, dérivé du latin \emph{dominos}, était employé pour
  désigner la dignité abbatiale dans certains monastères. L'abbaye
  d'Aubrac, dont il est ici question, dépendait du diocèse de Rodez.}
d'Aubrac, abbaye sous un titre particulier, mais qui n'est qu'un simple
nom dont il se démit en arrivant à Paris. Le roi si content du duc de
Noailles, et M\textsuperscript{me} de Maintenon tout à lui, voulurent
que la grâce fût entière\,: la domerie fut donnée à l'abbé de Noailles
et l'évêché de Châlons en même temps. C'était le plus jeune des frères
de M. de Noailles et de M. de Châlons qui avait au moins quinze ou
dix-huit ans moins qu'eux.

Peu après mon retour, j'allai me réjouir avec M. de la Trappe de la
solidité que le roi venait de donner à son ouvrage. C'était une abbaye
commendataire de onze mille ou douze mille livres de rente tout au plus
en tout, et la moindre de celles dont il s'était {[}démis{]} en se
retirant, sans penser encore à s'y faire moine, et beaucoup moins à y
rétablir la vie ancienne de saint Bernard dans toute son austérité. Un
commendataire qui lui aurait succédé n'aurait pas laissé de quoi vivre à
ce grand nombre de pénitents qu'il y avait rassemblés, et la régularité
en aurait été fort hasardée. Il le représenta donc au roi par une
lettre, et son désir de se voir un successeur régulier. Le roi non
seulement le lui accorda, mais lui permit de le choisir, et lui promit
qu'il n'y aurait point de commendataire tant que la régularité
subsisterait telle qu'il l'avait établie\,; et le pape y voulut bien
entrer pour que cette grâce ne pût préjudicier à la nomination d'un
commendataire, quand il plairait au roi, même après trois ou un plus
grand nombre de réguliers, parce que sans cette précaution trois abbés
réguliers de suite remettent de droit l'abbaye en règle. M. de la Trappe
nomma le prieur de sa maison qui était un des plus savants et des plus
capables, mais qui ne vécut pas longtemps. Il se démit et parut encore
plus grand en cet état qu'il n'avait fait dans la réforme et le
gouvernement de cet admirable monastère. Avant de quitter les saints, la
mort de M. Nicole, qui arriva à Paris vers la fin de cette année, mérite
de n'être pas oubliée. Cet homme illustre est si connu par toute la
suite de sa vie, par ses talents et sa piété sage et éminente que je ne
m'y arrêterai pas\,; il a laissé des ouvrages d'une instruction infinie,
et qui développent le cœur humain avec une lumière qui apprend aux
hommes à se connaître, et toute tournée à l'édification et à la parfaite
conviction.

M. de Langres mourut presque en même temps. Il était Simiane, fils et
frère de MM. de Gordes, tous deux chevaliers de l'ordre et premiers
capitaines des gardes du corps. Le dernier vendit sa charge à M. de
Chandenier, et fut depuis chevalier d'honneur de la reine. Le père, mort
en 1642, faisait souvent arrêter le carrosse de Louis XIII\,; il lui
disait\,: « Sire, vous ne voulez pas qu'on crève, faites donc arrêter,
s'il vous plaît\,;\,» et il descendait pour pisser. Le roi riait et le
considérait. Mon père qui l'a vu arriver cent fois me l'a conté. L'autre
mourut en 1680\,; c'est le père de M\textsuperscript{me} de Rhodes. M.
de Langres fut donc élevé à la cour, et de très bonne heure premier
aumônier de la reine. C'était un vrai gentilhomme et le meilleur homme
du monde, que tout le monde aimait, répandu dans le plus grand monde et
avec le plus distingué. On l'appelait volontiers le bon Langres. Il
n'avait rien de mauvais, même pour les mœurs, mais il n'était pas fait
pour être évêque\,; il jouait à toutes sortes de jeux et le plus gros
jeu du monde. M. de Vendôme, M. le Grand, et quelques autres de cette
volée, lui attrapèrent gros deux ou trois fois au billard. Il ne dit
mot, et s'en alla à Langres où il se mit à étudier les adresses du
billard, et s'enfermait bien pour cela, de peur qu'on le sût. De retour
à Paris, voilà ces messieurs à le presser de jouer au billard, et lui à
s'en défendre comme un homme déjà battu, et qui, depuis six mois de
séjour à Langres, n'a vu que des chanoines et des curés. Quand il se fut
bien fait importuner il céda enfin. Il joua d'abord médiocrement, puis
mieux, et fit grossir la partie\,; enfin il les gagna tous de suite,
puis se moqua d'eux après avoir regagné beaucoup plus qu'il n'avait
perdu. Il avait un grand désir de l'ordre, et de toutes façons était
fait pour l'avoir, et mourut fort vieux sans y être parvenu.

Langres fut donné à l'abbé de Tonnerre, fils du frère aîné de M. de
Noyon. Il était aumônier du roi, et servait auprès de Monseigneur, qui,
le lendemain au soir, s'en alla à Meudon, où les courtisans qu'il menait
avaient l'honneur de manger tous, et toujours avec lui. Quand son souper
fut servi, et que l'abbé de Tonnerre eut dit \emph{le Benedicite}, il
lui dit de se mettre à table. L'abbé répondit modestement qu'il avait
soupé, car l'aumônier mangeait devant à la table du maître d'hôtel. « Et
pourquoi, monsieur l'abbé\,? lui dit Monseigneur. Vous êtes nommé à
Langres, et dès là vous savez bien que vous devez manger avec moi. Au
moins, ajouta-t-il, n'y manquez plus de tout le voyage.\,» L'abbé de
Tonnerre, après l'avoir remercié, lui dit qu'il n'ignorait pas cet
honneur et cette distinction des évêques-pairs, mais qu'il n'y avait
bontés ni amitiés qu'il ne reçût tous les jours de M. d'Orléans, qui, ne
pouvant avoir cet honneur étant évêque et premier aumônier, il serait
trop peiné de lui donner ce dégoût, lui n'étant encore que nommé et
ayant demandé de continuer à servir dans sa charge d'aumônier jusqu'à
l'arrivée de ses bulles. Il fut extrêmement loué de cette modestie et de
cette considération pour M. d'Orléans, et Monseigneur lui dit qu'il ne
voulait pas le contraindre, mais qu'il serait le maître de se mettre à
table avec lui toutes les fois qu'il le voudrait.

M. et M\textsuperscript{me} la maréchale de Lorges arrivèrent de Vichy,
et se pressèrent trop d'aller à Versailles, où ils furent reçus du roi
avec les plus grandes marques d'amitié et de distinction. Mais M. le
maréchal parut encore en plus mauvais état à la cour qu'il n'avait fait
à Paris, et, presque aussitôt qu'il eut pris le bâton, il fut obligé de
l'envoyer au maréchal de Villeroy. Le roi comprit qu'après deux aussi
fortes maladies et si près à près, il ne serait plus en état de servir,
et ne voulut pas s'exposer, au milieu d'une campagne, aux inconvénients
qui pouvaient naître de la santé du général. Il eut peine à en parler
lui-même au maréchal, et chargea M. de La Rochefoucauld, son ami le plus
intime de tous les temps, de le lui faire entendre, et de tâcher surtout
qu'il ne s'opiniâtrât point là-dessus à vouloir lui parler ni lui
écrire. M. de La Rochefoucauld vint donc dîner chez lui à Paris, et
après le dîner le prit en particulier avec la maréchale. Ce compliment
leur parut amer. M. le maréchal de Lorges se croyait en état de
commander l'armée\,; il voulut une audience du roi, et il l'eut. Tout
s'y passa avec toutes sortes d'égards et d'amitiés du roi, mais il ne
put changer de pensée, et M. de Lorges s'y soumit de bonne grâce,
quoique très peiné de devenir inutile, surtout par rapport à moi et à
ses neveux. Nous en fûmes aussi fort affligés, par la différence infinie
que cela faisait pour nous à l'armée et à la considération même partout
ailleurs.

Peu de jours après nous fûmes d'un voyage de Marly, qui fut pour moi le
premier, où il arriva une singulière scène. Le roi et Monseigneur y
tenaient chacun une table à même heure et en même pièce, soir et
matin\,; les dames s'y partageaient sans affectation, sinon que
M\textsuperscript{me} la princesse de Conti était toujours à celle de
Monseigneur, et ses deux autres sœurs toujours à celle du roi. Il y
avait dans un coin de la même pièce cinq ou six couverts où, sans
affectation aussi, se mettaient tantôt les unes, tantôt les autres, mais
qui n'étaient tenus par personne. Celle du roi était plus proche du
grand salon, l'autre plus voisine des fenêtres et de la porte par où, en
sortant de dîner, le roi allait chez M\textsuperscript{me} de Maintenon,
qui alors dînait souvent à la table du roi, se mettait vis-à-vis de lui
(les tables étaient rondes), ne mangeait jamais qu'à celle-là, et
soupait toujours seule chez elle. Pour expliquer le fait il fallait
mettre ce tableau au net.

Les princesses n'étaient que très légèrement raccommodées, comme on l'a
vu plus haut, et M\textsuperscript{me} la princesse de Conti
intérieurement de fort mauvaise humeur du goût de Monseigneur pour la
Choin, qu'elle ne pouvait ignorer et dont elle n'osait donner aucun
signe. À un dîner pendant lequel Monseigneur était à la chasse, et où sa
table était tenue par M\textsuperscript{me} la princesse de Conti, le
roi s'amusa à badiner avec M\textsuperscript{me} la Duchesse, et sortit
de cette gravité qu'il ne quittait jamais, pour, à la surprise de la
compagnie, jouer avec elle aux olives. Cela fit boire quelques coups à
M\textsuperscript{me} la Duchesse\,; le roi fit semblant d'en boire un
ou deux, et cet amusement dura jusqu'aux fruits et à la sortie de table.
Le roi, passant devant M\textsuperscript{me} la princesse de Conti pour
aller chez M\textsuperscript{me} de Maintenon, choqué peut-être du
sérieux qu'il lui remarqua, lui dit assez sèchement que sa gravité ne
s'accommodait pas de leur ivrognerie. La princesse piquée laissa passer
le roi, puis se tournant à M\textsuperscript{me} de Châtillon, dans ce
moment de chaos où chacun se lavait la bouche, lui dit qu'elle aimait
mieux être grave que sac à vin (entendant quelques repas un peu allongés
que ses sœurs avaient faits depuis peu ensemble). Ce mot fut entendu de
M\textsuperscript{me} la duchesse de Chartres qui répondit assez haut,
de sa voix lente et tremblante, qu'elle aimait mieux être sac à vin que
sac à guenilles\,: par où elle entendait Clermont et des officiers des
gardes du corps qui avaient été, les uns chassés, les autres éloignés à
cause d'elle. Ce mot fut si cruel qu'il ne reçut point de repartie, et
qu'il courut sur-le-champ par Marly, et de là par Paris et partout.
M\textsuperscript{me} la Duchesse qui, avec bien de la grâce et de
l'esprit, a l'art des chansons salées, en fit d'étranges sur ce même
ton. M\textsuperscript{me} la princesse de Conti au désespoir, et qui
n'avait pas les mêmes armes, ne sut que devenir. Monsieur, le roi des
tracasseries, entra dans celle-ci qu'il trouva de part et d'autre trop
forte. Monseigneur s'en mêla aussi\,; il leur donna un dîner à Meudon où
M\textsuperscript{me} la princesse de Conti alla seule et y arriva la
première\,; les deux autres y furent menées par Monsieur. Elles se
parlèrent peu, tout fut aride, et elles revinrent de tout point comme
elles étaient allées.

La fin de cette année fut orageuse à Marly. M\textsuperscript{me} la
duchesse de Chartres et M\textsuperscript{me} la Duchesse, plus ralliées
par l'aversion de M\textsuperscript{me} la princesse de Conti, se mirent
au voyage suivant à un repas rompu, après le coucher du roi, dans la
chambre de M\textsuperscript{me} de Chartres au château\,; Monseigneur
joua tard dans le salon. En se retirant chez lui il monta chez ces
princesses et les trouva qui fumaient avec des pipes qu'elles avaient
envoyé chercher au corps de garde suisse. Monseigneur, qui en vit les
suites si cette odeur gagnait, leur fit quitter cet exercice\,; mais la
fumée les avait trahies. Le roi leur fit le lendemain une rude
correction, dont M\textsuperscript{me} la princesse de Conti triompha.
Cependant ces brouilleries se multiplièrent, et le roi qui avait espéré
qu'elles finiraient d'elles-mêmes, s'en ennuya\,; et un soir à
Versailles qu'elles étaient dans son cabinet après son souper, il leur
en parla très fortement, et conclut par les assurer que s'il en
entendait parler davantage, elles avaient chacune des maisons de
campagne où il les enverrait pour longtemps et où il les trouverait fort
bien. La menace eut son effet, et le calme et la bienséance revinrent et
suppléèrent à l'amitié.

\hypertarget{chapitre-xviii.}{%
\chapter{CHAPITRE XVIII.}\label{chapitre-xviii.}}

1696

~

\relsize{-1}

{\textsc{Année 1696. Banc au lieu de ployant aux cardinaux aux
cérémonies de l'ordre, à la réception de MM. de Noyon et de Guiscard.}}
{\textsc{- Duc Lanti nommé à l'ordre\,; son extraction.}} {\textsc{-
Prince de Conti gagne son procès contre la duchesse de Nemours.}}
{\textsc{- Mariage de Barbezieux avec M\textsuperscript{lle} d'Alègre\,;
de M. de Luxembourg avec M\textsuperscript{lle} de Clérembault\,; de
M\textsuperscript{me} de Seignelay avec M. de Marsan\,; du duc de
Lesdiguières avec M\textsuperscript{lle} de Duras\,; du duc d'Uzès avec
M\textsuperscript{lle} de Monaco.}} {\textsc{- Rang nouveau de prince
étranger de M. de Monaco.}} {\textsc{- Mariage du duc d'Albret et de
M\textsuperscript{lle} de La Trémoille\,; de Villacerf avec
M\textsuperscript{lle} de Brinon\,; de Lassay et d'une bâtarde de M. le
Prince\,; de Feuquières avec la Mignard\,; de Bouzols avec
M\textsuperscript{lle} de Croissy.}} {\textsc{- Comte de Luxe, fait duc
vérifié de Châtillon-sur-Loing, épouse M\textsuperscript{lle} de
Royan.}} {\textsc{- Le prince d'Isenghien obtient un tabouret de grâce
pour toujours.}} {\textsc{- Sourde lutte de l'archevêque de Cambrai et
de l'évêque de Chartres.}} {\textsc{- M\textsuperscript{me} Guyon
chassée de Saint-Cyr, puis à la Bastille.}} \relsize{1}

~

L'année 1696 commença par un petit dégoût à des gens qui n'y étaient pas
accoutumés. Le roi donna l'ordre à M. de Noyon et à Guiscard, et à la
cérémonie, les cardinaux d'Estrées et de Furstemberg n'eurent qu'un banc
comme tous les autres chevaliers. Peu à peu cette dignité, habile en
usurpation, et heureuse à les tourner en droit, avait trouvé moyen
d'avoir chacun un siège ployant à leur place auprès de la crédence de
l'autel, comme Monseigneur et Monsieur et la maison royale en ont auprès
du roi, qui à la fin le trouva mauvais et le leur ôta. Ils l'avalèrent
sans oser dire mot.

Au chapitre qui précéda cette cérémonie le roi nomma à l'ordre le duc
Lanti, dont la sœur était femme de la duchesse de Bracciano, qui l'y
servit fort par elle et par ses amis\,; il était à Rome et l'y reçut au
grand contentement du cardinal d'Estrées, ami intime de la duchesse de
Bracciano, et qui y avait le plus travaillé. Ces Lanti ne sont rien du
tout, ils ont pris le nom della Rovere, parce qu'ils en ont eu une mère,
et ces Rovere eux-mêmes étaient de la lie du peuple avant leur
pontificat. François della Rovere qui fut pape en 1481 \footnote{Sixte
  IV fut pape de 1471 à 1484.} et qui le fut quatorze ans sous le nom de
Sixte IV, toit fils d'un pêcheur des environs de Savone, et ce furieux
Jules II, pape en 1503 et qui le fut dix ans, était fils de son frère.
Ils n'oublièrent rien pour élever leur famille par argent, par
alliances, par troubles et par toutes sortes de voies. Le duché d'Urbin
et d'autres grands fiefs y entrèrent, qui pour la plupart sont retournés
aux papes. Ces la Rovere ont eu trois ducs d'Urbin.

M. le prince de Conti gagna tout d'une voix son procès contre
M\textsuperscript{me} de Nemours à l'audience de la grand'chambre,
c'est-à-dire la permission de prouver que M. de Longueville était en
état de tester lorsqu'il fit son testament en sa faveur, à quoi lui
servit beaucoup son ordination postérieure à l'ordre de prêtrise par les
mains du pape, et ce jugement préliminaire emportait le fond, supposé
les preuves. J'étais dans la lanterne avec M. le prince de Conti, M. le
Duc et M. de La Rocheguyon, assis sur le banc et devant nous le peu des
premiers officiers de ces princes qui y purent tenir. Toute la France en
hommes remplissait la grand'chambre. Le plaidoyer, déjà commencé en une
autre audience, remplit celle-ci. Il fut très éloquent, et tout de suite
suivi d'un jugement. Jamais on n'ouït de tels cris de joie, ni tant
d'applaudissements\,; la grande salle était pleine de monde qui
retentissait\,; à peine pûmes-nous passer. M. le prince de Conti se
contint fort, mais il parut fort sensible et à la chose et à la part
générale qu'on prenait pour lui. On ne laissa pas dans le monde
d'appeler un peu de ce jugement, sans se soucier pourtant de
M\textsuperscript{me} de Nemours, à qui le choix de son héritier ne
laissa pas de faire grand tort. La colère qu'elle conçut de cette
décision est inconcevable, et tout ce qu'elle dit de plaisant et de salé
contre sa partie e contre ses juges. Ce ne fut encore que le
commencement de leurs combats.

Cet hiver fut fertile en mariages, Barbezieux les commença, il épousa la
fille aînée de d'Alègre, qui fit à cette occasion une fête aussi
somptueuse que pour l'alliance d'un prince du sang. Il était maréchal de
camp, il en espérait sa fortune, il eut tout le temps de s'en repentir.

Celui de M. de Luxembourg fut fort avancé avec M\textsuperscript{me} de
Seignelay. C'était une grande femme, très bien faite, avec une grande
mine et de grands restes de beauté. Sa hauteur excessive avait été
soutenue par celle de son mari, par son opulence, sa magnificence, son
autorité dans le conseil et dans sa place, dont il avait bizarrement
tenté de se faire un degré à devenir maréchal de France\,; mais devenue
veuve elle brûlait d'un rang et d'un autre nom quoiqu'elle eût plusieurs
enfants. Le rare fut que M. de Chevreuse, qui avait marié sa fille à M.
de Luxembourg qui en était veuf sans enfants, et Cavoye, le plus grand
favori de M. de Seignelay, furent les entremetteurs de l'affaire, que M.
de Luxembourg rompit fort malhonnêtement parce qu'il la voulut rompre,
les habits achetés et tous les compliments reçus. Il eut lieu de s'en
repentir. Tous deux ne tardèrent pas à trouver ailleurs. M. de
Luxembourg épousa M\textsuperscript{lle} de Clérembault, riche et unique
héritière fort jolie, mais dont la naissance était légère\,; son nom
était Gillier. Elle était fille de Clérembault, qui, étant dans les
basses charges chez Monsieur, donna dans l'œil de la comtesse du
Plessis, dame d'honneur de Madame, en survivance de la maréchale du
Plessis, sa belle-mère, et veuve du comte du Plessis, premier
gentilhomme de la chambre de Monsieur, en survivance du maréchal du
Plessis, son père, qui avait été gouverneur de Monsieur. Le comte du
Plessis fut tué devant Arnheim en Hollande en 1672, à trente-huit ans,
trois ans avant la mort de son père, et laissa un fils unique qui devint
duc et pair par la mort du maréchal son grand-père, et qui fut tué sans
alliance devant Luxembourg, en 1684, ce qui fit duc et pair le chevalier
du Plessis, son oncle, qui prit le nom de duc de Choiseul. Nous avons vu
plus haut l'étrange raison qui l'empêcha d'être maréchal de France. La
comtesse du Plessis s'appelait Le Loup, et était fille de Bellenave, et
riche. Amoureuse de Clérembault, elle l'épousa, et, pour l'approcher un
peu d'elle, eut le crédit de le faire premier écuyer de Madame. L'un et
l'autre la quittèrent, et vécurent dans une grande avarice et fort dans
le néant. Ils voulurent garder leur fille, et M. de Luxembourg se mit
chez eux.

M\textsuperscript{me} de Seignelay, outrée de ce qui venait de lui
arriver, trouva un mari qui lui donnait un rang et de meilleure maison
que M. de Luxembourg. Aussi, ne le manqua-t-elle pas\,; et les Matignon
ses oncles se cotisèrent pour brusquer cette affaire. Ce fut avec M. de
Marsan, frère de M. le Grand. Cavoye, si intime de feu M. de Seignelay
et de feu M. de Luxembourg, piqué du procédé avec M\textsuperscript{me}
de Seignelay, en fit la noce chez lui à Paris où il y eut fort peu de
monde.

M. de Duras fit un grand mariage pour sa seconde fille. L'aînée avait
épousé, il y avait quelques années, le duc de La Meilleraye, fils unique
du duc de Mazarin, mais qui n'avait que dés richesses avec sa dignité.
Il trouva pour l'autre, avec les grands biens, tout ce qu'il pouvait
désirer d'ailleurs\,: ce fut le jeune duc de Lesdiguières, ardemment
désiré des plus grands partis, parce qu'il était lui-même le plus grand
parti de France. Sa mère, héritière des Gondi, était une fée solitaire
qui ne laissait entrer presque personne dans son palais enchanté, et que
la maréchale de Duras sut pourtant pénétrer. Tout convenu dans un grand
secret avec elle, qui était aussi la tutrice, il fut question des
parents\,; le maréchal de Villeroy et M. le Grand, qui étaient les plus
proches du côté paternel, et la maréchale de Villeroy du maternel,
firent grand bruit. Le maréchal et le père du jeune duc étaient enfants
du frère et de la sœur, et la duchesse de Lesdiguières et la maréchale
étaient filles aussi du frère et de la sœur. M\textsuperscript{me}
d'Armagnac était sœur du maréchal\,; lui et M. le Grand étaient intimes.
Il ménageait depuis longtemps M\textsuperscript{me} de Lesdiguières qui
se servait de son crédit à son gré. Plusieurs partis avaient manqué à
M\textsuperscript{lle} d'Armagnac\,; ils voulaient celui-ci, bien que
plus jeune qu'elle, et c'est ce qui les mit en si grand émoi. Pendant ce
vacarme, tout fut signé, et par M. de La Trémoille, tuteur paternel,
gendre du feu duc de Créqui, ami des maréchaux de Duras et de Lorges, et
fils de leur cousin germain. Cela fit taire tout à coup les autres, et
le mariage se fit à petit bruit à l'hôtel de Duras, parce que
M\textsuperscript{me} de Lesdiguières ne voulut point de monde, encore
moins les parents de mauvaise humeur. Il n'en coûta que cent mille écus
de dot à M. de Duras, encore en retint-il onze mille livres de rente
pour loger et nourrir sa fille et son gendre. Il avait marié l'aînée à
aussi bon marché. La mariée était grande, bien faite, belle, avec le
plus grand air du monde, et d'ailleurs très aimable, et l'âge convenait
entièrement.

Il s'en fit un autre d'âges bien disproportionnés, du duc d'Uzès, qui
avait dix-huit ans, et de la fille unique du prince de Monaco, sœur du
duc de Valentinois, gendre de M. le Grand\,: elle avait trente-quatre ou
trente-cinq ans, et les paraissait. Elle était riche\,; sa mère était
sœur du duc de Grammont. Il était lors dans les horreurs de la taille.
M. de Valentinois n'avait ni feu ni lieu que chez son beau-père, et il
n'avait pas lieu d'être bien avec sa femme ni avec les siens\,; M. de
Monaco était à Monaco. La noce se fit donc chez la duchesse du Lude,
veuve en premières noces de ce galant comte de Guiche, frère aîné du duc
de Grammont, et elle était toujours demeurée fort unie avec eux tous.
M\textsuperscript{lle} de Monaco avait le tabouret, parce qu'au mariage
de M. de Valentinois, en 1688, M. le Grand avait obtenu le rang de
prince étranger pour M. de Monaco et pour ses enfants, à quoi ils
n'avaient jamais osé songer jusque-là. La mère de M. de Monaco vint à
Paris pour le faire tenir sur les fonts de baptême par le roi et par la
reine sa mère. Son mari était mort sans que son père, qui vivait encore,
se fût démis. Elle s'appelait la princesse de Mourgues. C'était M.
d'Angoulême qui, étant dans son gouvernement de Provence, avait fait
avec ce même beau-père le traité de se donner à la France. Ce fut donc à
la duchesse d'Angoulême, sa veuve, qu'elle s'adressa pour la présenter
et la mener à la cour. Elle y fut debout, sans prétention ni
équivoque\,; et, après un court séjour, elle s'en retourna avec son
fils, comblée des bontés du roi et de la reine. M\textsuperscript{me}
d'Angoulême, chez qui ma mère a logé longtemps fille et y a été mariée,
le lui a conté cent fois\,; et c'est le père de ce prince de Monaco du
traité, qui le premier s'est fait appeler et intituler prince de
Monaco\,; le père de celui-là et tous ses devanciers ne se sont jamais
dits ni fait appeler que seigneurs de Monaco. C'est, au demeurant, la
souveraineté d'une roche, du milieu de laquelle on peut pour ainsi dire
cracher hors de ses étroites limites.

Le duc d'Albret, fils aîné de M. de Bouillon, épousa la fille du duc de
La Trémoille\,; il y eut d'autres mariages plus tard dont il vaut autant
finir la matière tout de suite. M\textsuperscript{me} la maréchale de
Lorges maria une cousine germaine, qu'elle avait auprès d'elle, au
marquis de Saint-Herem, du nom de Montmorin, qui était fort de mes amis.
Il avait la survivance du gouvernement de Fontainebleau de son père, que
le roi prit en 1688 pour un homme de peu, quoique de très bonne et
ancienne maison et très bien alliée, dont les pères avaient eu le
gouvernement d'Auvergne, et qui ne le fit point chevalier de l'ordre. M.
de La Rochefoucauld, ami du bonhomme Saint-Herem, le détrompa\,; mais il
n'était plus temps.

Villacerf épousa M\textsuperscript{lle} de Brinon, sans bien\,; elle
était Saint-Nectaire et lui Colberte\,: les noms ne se ressemblaient
pas. Son père et Saint-Pouange, son frère, étaient fils d'une sœur du
chancelier Le Tellier. Saint-Pouange faisait tout sous M. de Louvois, et
après sous Barbezieux. Ils avaient répudié les Colbert pour les Tellier,
dont ils avaient pris les livrées et suivi la fortune\,; tous deux
étaient bien avec le roi, surtout Villacerf, avec confiance de longue
main. C'était aussi un très bon homme et fort homme d'honneur. Il eut
les bâtiments à la mort de Louvois, et fut aussi un temps premier maître
d'hôtel de la reine. Son fils strié avait été tué à la tête d'un
régiment qu'on avait fait royal pour lui\,; celui-ci avait servi à la
mer quelque temps.

Lassay épousa à l'hôtel de Condé la bâtarde de M. le Prince et de
M\textsuperscript{lle} de Montalais qu'il avait fait légitimer
\footnote{Julie de Bourbon, fille naturelle d'Henri-Jules de Bourbon,
  prince de Condé, et de Françoise de Montalais, épousa, le 5 mars 1696,
  François de Lesparre de Madaillan, marquis de Lassay. Saint-Simon
  appelle d'abord le père de Lassay \emph{Montalaire}, et à la ligne
  suivante \emph{Madaillan}. Nous avons reproduit exactement le
  manuscrit\,; mais le nom de Madaillan est le seul exact.}. Elle était
fort jolie et avait beaucoup d'esprit. Il en eut du bien et la
lieutenance générale de Bresse. Il était fils de Montalaire, grand
menteur de son métier, et d'une Vipart, très petite demoiselle de
Normandie. Ce nom de Madaillan est étrangement connu par la vie de M.
d'Épernon, et n'a pas brillé depuis Lassay avait déjà été marié deux
fois. D'une Sibour, qu'il perdit tout au commencement de 1675, il eut
une fille unique qui n'eut point d'enfants du marquis de Coligny,
dernier de cette grande et illustre maison. Il devint après amoureux de
la fille d'un apothicaire qui s'appelait Pajot, si belle, si modeste, si
sage, si spirituelle, que Charles IV, duc de Lorraine, éperdu d'elle, la
voulut épouser malgré elle, et n'en fut empêché que parce que le roi la
fit enlever. Lassay, qui n'était pas de si bonne maison, l'épousa, et en
eut un fils unique\,; puis la perdit, et en pensa perdre l'esprit. Il se
crut dévot, se fit une retraite charmante joignant les Incurables, et y
mena quelques années une vie fort édifiante. À la fin il s'en ennuya\,;
il s'aperçut qu'il n'était qu'affligé, et que la dévotion passait avec
la douleur. Il avait beaucoup d'esprit, mais c'était tout. Il chercha à
rentrer dans le monde, et bientôt il se trouva tout au milieu. Il
s'attacha à M. le Duc et à MM. les princes de Conti, avec qui il fit le
voyage de Hongrie. Il n'avait jamais servi et avait été quelque temps à
faire l'important en basse Normandie\,; il plut à M. le Duc par lui être
commode à ses plaisirs\,; et il espéra de ce troisième mariage s'initier
à la cour sous sa protection et celle de M\textsuperscript{me} la
Duchesse\,; il n'y fut jamais que des faubourgs. Il en eut une fille
unique.

Un mariage d'amour fort étrange suivit celui-ci, d'un frère de
Feuquières, qui n'avait jamais fait grand'chose, avec la fille du
célèbre Mignard, le premier peintre de son temps, qui était mort, et
dont j'ai parlé ci-devant\,; elle était encore si belle, que Bloin,
premier valet de chambre du roi, l'entretenait depuis longtemps au vu et
au su de tout le monde, et fût cause que le roi en signa le contrat de
mariage.

Enfin Bouzols, gentilhomme d'Auvergne, tout simple et peu connu, sinon
pour avoir acheté le régiment Royal-Piémont, épousa la fille aînée de
Croissy, déjà fort montée en graine et très laide. Ce n'était pas faute
d'ambition d'être duchesse comme ses cousines, mais à force d'attendre
et d'espérer, il fallut faire une fin et se contenter du possible, fort
éloigné du titre. Elle avait infiniment d'esprit, de grâces, et
d'amusement dans l'esprit, et passait sa vie avec M\textsuperscript{me}
la Duchesse\,; elle ne faisait pas moins de chansons bien assenées
qu'elle, niais elle et son cher ami Lassay ne furent pas à l'épreuve des
siennes, et si parlantes et si plaisantes qu'on s'en souvient toujours.

Le roi fit presque en même temps deux grâces. Il avait fait passer la
Normandie du maréchal de Luxembourg à son fils aîné, à condition qu'il
ne lui parlât jamais pour lui de sa charge de capitaine des gardes du
corps. Le père, hardi de ses lauriers, et qui, avec raison, ne se
croyait pas inférieur en naissance aux Bouillon, aux Rohan, aux Monaco,
auxquels tous le roi avait donné dés rangs de princes étrangers, s'était
mis à le prétendre et à l'en presser\,; et comme il fait toujours bon se
mettre en prétention, comme disait M. Le Tellier, le roi s'en crut
quitte à bon marché de promettre à M. de Luxembourg de faire son second
fils duc, lorsqu'il trouverait quelque mariage. M. de Luxembourg mourut
avant que le comte de Luce fût marié\,; la famille crut ne devoir pas
laisser refroidir trop longtemps la promesse. Le maréchal ne fut pas
plutôt mort, que le roi s'en repentit\,; néanmoins il ne put reculer,
mais il le fit de mauvaise grâce. Il fit donc expédier une érection sur
Châtillon-sur-Loing, que le comte de Luce avait eu du legs universel de
sa tante M\textsuperscript{me} de Meckelbourg, pour être vérifiée au
parlement sans pairie lorsqu'il se marierait, et n'en pas jouir
auparavant. Il épousa enfin M\textsuperscript{lle} de Royan, celle-là
même que la duchesse de Bracciano, sa tante, avait eu tant d'envie de me
donner, et à laquelle Phélypeaux avait osé prétendre. Ce nouveau duc ne
put jamais plaire au roi depuis qu'il le fut, et en essuya tous les
dégoûts qu'il lui put donner toute sa vie, pour se dépiquer de l'avoir
fait duc malgré lui.

L'autre grâce fut fort extraordinaire, et j'avoue franchement que je ne
sais d'où elle vint. Le roi, qui aimait le feu maréchal d'Humières,
avait fait le mariage de sa fille aînée, en lui accordant un tabouret de
grâce, en épousant le prince d'Isenghien\,; ce qui a le même effet que
ce qu'on connaît sous le nom d'un brevet de duc. Il était mort et avait
laissé deux fils\,; le roi, sans aucune occasion, ni de mariage, non
seulement accorda la même grâce à l'aîné, mais, ce qui était sans
exemple, il l'accorda de mâle en mâle à sa postérité\,: c'est-à-dire
que, sans aucun renouvellement, le fils aîné y succéderait à son père,
n'ayant toutefois que des honneurs sans aucun rang, comme les ducs à
brevet.

Le nouvel archevêque de Cambrai s'applaudissait cependant de ses succès
auprès de M\textsuperscript{me} de Maintenon\,; les espérances qu'il en
concevait, avec de si bons appuis, étaient grandes, mais il crut ne les
pouvoir conduire avec sûreté jusqu'où il se les proposait, qu'en
achevant de se rendre maître de son esprit sans partage. Godet, évêque
de Chartres, tenait à elle par les liens les plus intimes\,; il était
diocésain de Saint-Cyr\,: il en était le directeur unique\,; il était de
plus celui de M\textsuperscript{me} de Maintenon\,: ses mœurs, sa
doctrine, sa piété, ses devoirs épiscopaux, tout était irrépréhensible.
Il ne faisait à Paris que des voyages courts et rares, logé au séminaire
de Saint-Sulpice, se montrait encore plus rarement à la cour et toujours
comme un éclair, et voyait M\textsuperscript{me} de Maintenon longtemps
et souvent à Saint-Cyr, et faisait d'ailleurs par lettres tout ce qu'il
voulait. C'était donc là un étrange rival à abattre\,; mais quelque
ancré qu'il fût, son extérieur de cuistre le rassura. Il le crut tel à
sa longue figure malpropre, décharnée, toute sulpicienne\,; un air cru
simple, aspect niais et sans liaisons qu'avec de plats prêtres, en un
mot il le prit pour un homme sans monde, sans talents, de peu d'esprit
et court de savoir, que le hasard de Saint-Cyr, établi dans son diocèse,
avait porté où il était, noyé dans ses fonctions, et sans autre appui,
ni autre connaissance\,: dans cette idée, il ne douta pas de lui faire
bientôt perdre terre par la nouvelle spiritualité de
M\textsuperscript{me} Guyon, déjà si goûtée de M\textsuperscript{me} de
Maintenon\,; il n'ignorait pas qu'elle n'était pas insensible aux
nouveautés de toute espèce, et il se flatta de culbuter par là M. de
Chartres, dont M\textsuperscript{me} de Maintenon sentirait et
mépriserait l'ignorance pour ne plus rien voir que par lui.

Pour arriver à ce but, il travailla à persuader M\textsuperscript{me} de
Maintenon de faire entrer M\textsuperscript{me} Guyon à Saint-Cyr, où
elle aurait le temps de la voir et de l'approfondir tout autrement que
dans de courtes et rares après-dînées, à l'hôtel de Chevreuse ou de
Beauvilliers. Il y réussit. M\textsuperscript{me} Guyon alla à Saint-Cyr
deux ou trois fois. Ensuite M\textsuperscript{me} de Maintenon, qui la
goûtait de plus en plus, l'y fit coucher, et de l'un à l'autre, mais
près à près, les séjours s'y allongèrent, et par son aveu elle s'y
chercha des personnes propres à devenir ses disciples, et elle s'en fit.
Bientôt il s'éleva dans Saint-Cyr un petit troupeau tout à part, dont
les maximes et même le langage de spiritualité parurent fort étrangers à
tout le reste de la maison, et bientôt fort étranges à M. de Chartres.
Ce prélat n'était rien moins que ce que M. de Cambrai s'en était figuré.
Il était fort savant et surtout profond théologien\,; il y joignait
beaucoup d'esprit\,; il y avait de la douceur, de la fermeté, même des
grâces\,; et ce qui était le plus surprenant dans un homme qui avait été
élevé et n'était jamais sorti de la profondeur de son métier, il était
tel pour la cour et pour le monde que les plus fins courtisans, auraient
eu peine à le suivre et auraient eu à profiter de ses leçons. Mais
c'était en lui un talent enfoui pour les autres, parce qu'il ne s'en
servait jamais sans de vrais besoins. Son désintéressement, sa piété, sa
rare probité les retranchaient presque tous, et M\textsuperscript{me} de
Maintenon, au point où il était avec elle, suppléait à tout.

Dès qu'il eut le vent de cette doctrine étrangère, il fit en sorte d'y
faire admettre deux dames de Saint-Cyr sur l'esprit et le discernement
desquelles il pouvait compter, et qui pourraient faire impression sur
M\textsuperscript{me} de Maintenon. Il les choisit surtout parfaitement
à lui et les instruisit bien. Ces nouvelles prosélytes parurent d'abord
ravies et peu à peu enchantées. Elles s'attachèrent plus que pas une à
leur nouvelle directrice, qui, sentant leur esprit et leur réputation
dans la maison, s'applaudit d'une conquête qui lui aplanirait celle
qu'elle se proposait. Elle s'attacha donc aussi à gagner entièrement ces
filles\,; elle en fit ses plus chères disciples\,; elle s'ouvrit à elles
comme aux plus capables de profiter de sa doctrine et de la faire goûter
dans la maison. Elle et M. de Cambrai, qu'elle instruisait de tous ses
progrès, triomphaient, et le petit troupeau exultait. M. de Chartres,
par le consentement duquel M\textsuperscript{me} Guyon était entrée à
Saint-Cyr et y était devenue maîtresse extérieure, la laissa faire. Il
la suivait de l'œil\,; ses fidèles lui rendaient un compte exact de tout
ce qu'elles apprenaient en dogmes et en pratique. Il se mit bien au fait
de tout, il l'examina avec exactitude, et quand il crut qu'il était
temps, il éclata.

M\textsuperscript{me} de Maintenon fut étrangement surprise de tout ce
qu'il lui apprit de sa nouvelle école, et plus encore de ce qu'il lui en
prouva par la bouche de ses deux affidées, et par ce qu'elles en avaient
mis par écrit. M\textsuperscript{me} de Maintenon interrogea d'autres
écolières\,; elle vit par leurs réponses que, plus ou moins instruites
et plus ou moins admises dans la confiance de leur nouvelle maîtresse,
tout allait au même but, et que ce but et le chemin étaient fort
extraordinaires. La voilà bien en peine, puis en grand scrupule\,; elle
se résolut à parler à M. de Cambrai\,; celui-ci, qui ne soupçonnait pas
qu'elle fût si instruite, s'embarrassa et augmenta les soupçons. Tout à
coup M\textsuperscript{me} Guyon fut chassée de Saint-Cyr, et on ne s'y
appliqua plus qu'à effacer jusqu'aux moindres traces de ce qu'elle y
avait enseigné. On y eut beaucoup de peine\,; elle en avait charmé
plusieurs qui s'étaient véritablement attachées à elle et à sa doctrine,
et M. de Chartres en profita pour faire sentir tout le danger de ce
poison et pour rendre M. de Cambrai fort suspect. Un tel revers et si
peu attendu l'étourdit, mais il ne l'abattit pas. Il paya d'esprit,
d'autorités mystiques, de fermeté sur ses étriers. Ses amis principaux
le soutinrent.

M. de Chartres, content de s'être solidement raffermi dans l'esprit et
la confiance de M\textsuperscript{me} de Maintenon, ne voulut pas
pousser si fort de suite un homme si soutenu\,; mais sa pénitente,
piquée d'avoir été conduite sur le bord du précipice, se refroidit de
plus en plus pour M. de Cambrai, et s'irrita de plus en plus contre
M\textsuperscript{me} Guyon. On sut qu'elle continuait à voir sourdement
du monde à Paris\,; on le lui défendit sous de si grandes peines qu'elle
se cacha davantage, mais sans pouvoir se passer de dogmatiser bien en
cachette, ni son petit troupeau de se rassembler par parties autour
d'elle en différents lieux. Cette conduite, qui fut éclairée, lui fit
donner ordre de sortir de Paris. Elle obéit, mais incontinent elle se
vint cacher dans une petite maison obscure du faubourg Saint-Antoine.
L'extrême attention avec laquelle elle était suivie fit que ne la
dépistant de nulle part, on ne douta pas qu'elle ne fût rentrée dans
Paris, et à force de recherches on la soupçonna où elle était, sur le
rapport qu'on eut des voisins des mystères sans lesquels cette porte ne
s'ouvrait point. On voulut être éclairci\,; une servante qui portait le
pain et les herbes fut suivie de si près et si adroitement qu'on entra
avec elle. M\textsuperscript{me} Guyon fut trouvée et conduite
sur-le-champ à la Bastille avec ordre de l'y bien traiter, mais avec les
plus rigoureuses défenses de la laisser voir, écrire, ni recevoir de
nouvelles de personne. Ce fut un coup de foudre pour M. de Cambrai et
pour ses amis, et pour le petit troupeau qui ne s'en réunit que
davantage. Les suites dépasseraient l'année. Il vaut mieux en demeurer
où nous en sommes pour celle-ci et remettre aux événements de la
suivante tout ce qui les amena.

\hypertarget{chapitre-xix.}{%
\chapter{CHAPITRE XIX.}\label{chapitre-xix.}}

1696

~

\relsize{-1}

{\textsc{Cavoye et sa fortune.}} {\textsc{- Projet avorté sur
l'Angleterre.}} {\textsc{- Le roi d'Angleterre à Calais.}} {\textsc{-
Mort de M\textsuperscript{me} de Guise\,; du marquis de Blanchefort\,;
de M. de Saint-Géran.}} {\textsc{- M\textsuperscript{me} de
Saint-Géran.}} {\textsc{- Mort de M\textsuperscript{me} de Miramion.}}
{\textsc{- M\textsuperscript{me} de Nesmond\,; son orgueil.}} {\textsc{-
Mort de M\textsuperscript{me} de Sévigné.}} {\textsc{- Éclat de l'évêque
d'Orléans contre le duc de La Rochefoucauld, sur une place derrière le
roi donnée au dernier.}} {\textsc{- Mort de La Bruyère\,; de Daquin,
ci-devant premier médecin\,; de la reine mère d'Espagne.}} \relsize{1}

~

Il y a dans les cours des personnages singuliers, qui sans esprit, sans
naissance distinguée, et sans entours ni services, percent dans la
familiarité de ce qui y est le plus brillant, et font enfin, on ne sait
pourquoi, compter le monde avec eux. Tel y fut Cavoye toute sa vie, très
petit gentilhomme tout au plus, dont le nom était Oger. Il était grand
maréchal des logis de la maison du roi\,; et le roman qui lui valut
cette charge mérite de n'être pas oublié, après avoir dit ce qui le
regarde en ce temps-ci. J'ai parlé de lui plus d'une fois, et fait
mention de son amitié intime avec M. de Seignelay chez qui la fleur de
la cour était travée. Cette grande liaison, qui devait lui aider à tout
par le crédit où était ce ministre, causa pourtant le ver rongeur de sa
vie. Avec sa charge, ses amis considérables à la cour qui l'y faisaient
figurer, et les bontés du roi toujours distinguées, il se flatta d'être
chevalier de l'ordre en la promotion de 1688. Le roi la fit avec M. de
Louvois qui était chancelier de l'ordre. Ce ministre qui minutait une
grande guerre qu'il avait déjà fait déclarer, et qu'il rendit plus
générale que le roi ne s'y attendait, ne songea qu'à profiter de
l'occasion de se faire des créatures. Il la rendit donc toute militaire
pour la première qui ait jamais été faite de la sorte, et eut grande
attention d'en exclure tous ceux qu'il n'aimait pas tant qu'il put.
L'amitié de Seignelay, son ennemi, pour Cavoye l'avait mis dans ce
nombre\,: il ne fut point de la promotion, et il en pensa mourir de
douleur. Le roi, à qui il parla et fit parler par Seignelay et par
d'autres amis, lui adoucit sa peine par des propos de bonté et
d'espérance pour une autre occasion. Il se fit depuis diverses petites
promotions et toujours Cavoye laissé, parce qu'en effet ces promotions
avaient des causes particulières pour chacun de ceux qui en furent. À la
fin, Cavoye, lassé et outré, écrivit au roi une rapsodie sur sa santé et
ses affaires, et demanda la permission de se défaire de sa charge. Le
roi ne lui dit ni ne lui fit rien dire là-dessus, et cependant Cavoye
prenait publiquement tous ses arrangements pour se retirer de la cour
dont je pense qu'il se fût cruellement repenti. Dix ou douze jours après
avoir remis sa lettre au roi, vint un voyage de Marly\,; et Cavoye, sans
demander, y fut à l'ordinaire. Deux jours après, le roi, entrant dans
son cabinet, l'appela, lui dit avec bonté qu'il y avait trop longtemps
qu'ils étaient ensemble pour se séparer, qu'il ne voulait point qu'il le
quittât, et qu'il aurait soin de ses affaires. Il y ajouta des
espérances sur l'ordre. Cavoye prétendit en avoir eu parole, et le voilà
enrôlé à la cour plus que jamais.

Sa mère était une femme de beaucoup d'esprit, venue je ne sais par quel
hasard de sa province, ni par quel autre connue de la reine mère, dans
des temps où elle avait besoin de toute aorte de gens. Elle lui plut,
elle la distingua en bonté sans la sortir de son petit état.
M\textsuperscript{me} de Cavoye en profita pour mettre son fils à la
cour et se faire à tous deux des amis. Cavoye était un des hommes de
France le mieux faits et de la meilleure mine, et qui se mettait le
mieux. Il en profita auprès des dames. C'était un temps où on se battait
fort malgré les édits\,; Cavoye, brave et adroit, s'y acquit tant de
réputation, que le nom de brave Cavoye lui en demeura.
M\textsuperscript{lle} de Coetlogon, une des filles de la reine
Marie-Thérèse, s'éprit de Cavoye, et s'en éprit jusqu'à la folie. Elle
était laide, sage, naïve, aimée et très bonne créature. Personne ne
s'avisa de trouver son amour étrange\,; et, ce qui est un prodige, tout
le monde en eut pitié. Elle en faisait toutes les avances. Cavoye était
cruel et quelquefois brutal\,; il en était importuné à mourir. Tant fut
procédé, que le roi et même la reine le lui reprochèrent, et qu'ils
exigèrent de lui qu'il serait plus humain. Il fallut aller à l'armée, où
pourtant il ne passa pas les petits emplois. Voilà Coetlogon aux larmes,
aux cris, et qui quitte toutes parures tout du long de la campagne, et
qui ne les reprend qu'au retour de Cavoye. Jamais on ne fit qu'en rire.
Vint l'hiver un combat où Cavoye servit de second et fut mis à la
Bastille\,: autres douleurs, chacun alla lui faire compliment. Elle
quitta toute parure, et se vêtit le plus mal qu'elle put. Elle parla au
roi pour Cavoye, et n'en pouvant obtenir la délivrance, elle, le
querella jusqu'aux injures. Le roi riait de tout son cour\,; elle en fut
si outrée, qu'elle, lui présenta ses ongles, auxquels le roi comprit
qu'il était plus sage de ne se pas exposer. Il dînait et soupait tous
les jours en public avec la reine. Au dîner, la duchesse de Richelieu et
les filles de la reine servaient. Tant que Cavoye fut à la Bastille,
jamais Coetlogon ne voulut servir quoi que ce fût au roi, ou elle
l'évitait, ou elle le refusait tout net, et disait qu'il ne méritait pas
qu'elle le servît\,; la jaunisse la prit, les vapeurs, les désespoirs\,;
enfin tant fut procédé, que le roi et la reine exigèrent bien
sérieusement de la duchesse de Richelieu de mener Coetlogon voir Cavoye
à la Bastille, et cela fut répété deux ou trois fois. Il sortit enfin,
et Coetlogon, ravie, se para tout de nouveau, mais ce fut avec peine
qu'elle consentit à se raccommoder avec le roi. La pitié et la mort de
M. de Froulay, grand maréchal des logis, vinrent à son secours. Le roi
envoya quérir Cavoye qu'il avait déjà tenté inutilement sur ce mariage.
À cette fois il lui dit qu'il le voulait\,; qu'à cette condition il
prendrait soin de sa fortune, et que, pour lui tenir lieu de dot avec
une fille qui n'avait rien, il lui ferait présent de la charge de grand
maréchal des logis de sa maison. Cavoye renifla encore, mais il y fallut
passer. Il a depuis bien vécu avec elle, et elle toujours dans la même
adoration jusqu'à aujourd'hui, et c'est quelquefois une farce de voir
les caresses qu'elle lui fait devant le monde, et la gravité importunée
avec laquelle il les reçoit. Des autres histoires de Cavoye il y aurait
un petit livre à faire\,: il suffit ici d'avoir rapporté cette histoire
pour sa singularité qui est sûrement sans exemple, car jamais la vertu
de M\textsuperscript{me} de Cavoye, ni devant ni depuis son mariage, n'a
reçu le plus léger soupçon. Son mari, lié toute sa vie avec le plus
brillant de la cour, s'était érigé chez lui une espèce de tribunal
auquel il ne fallait pas déplaire, compté et ménagé jusque des
ministres, mais d'ailleurs bon homme, et un fort honnête homme, à qui on
se pouvait fier de tout.

Le duc de Berwick, bâtard du roi d'Angleterre, parti sous prétexte
d'aller faire la revue des troupes que Jacques II avait en France, alla
secrètement en Angleterre où il fut découvert, et au moment d'être
arrêté et peut-être pis. Le but de ce voyage était de voir par lui-même
ce qu'il y avait de réel dans un parti formé pour le rétablissement du
roi Jacques, qui le sollicitait puissamment de passer en Angleterre avec
des troupes. Le retour de Berwick donna de telles espérances, que le roi
d'Angleterre s'en alla le lendemain à Calais où, à tous hasards, dès les
premières notions on s'était préparé à tout ce qui fui était nécessaire.
Les troupes destinées au trajet et qu'on tenait à portée y marchèrent en
même temps, et une escadre s'y rendit pour le transport. Le marquis
d'Harcourt commanda tout sous lui avec Pracomtal, maréchal de camp, et
le duc d'Humières, Biron et Mornay pour brigadiers. Ces messieurs s'y
morfondirent tout le reste de l'hiver et tout le printemps, longtemps
contrariés des vents, puis bloqués par les vaisseaux anglais qui
empêchèrent qu'on ne pût entrer ni sortir. Tout échoua de la sorte comme
il arriva toujours aux projets de ce malheureux prince qui revint enfin
à Saint-Germain, et les troupes retournèrent se rafraîchir, puis joindre
les armées de Flandre.

M\textsuperscript{me} de Guise mourut en ce temps-ci. Bossue et
contrefaite à l'excès, elle avait mieux aimé épouser le dernier duc de
Guise, en mai 1667, que de ne se point marier. Monsieur, son père, frère
de Louis XIII, était mort en 1660. Madame, sa mère, qui était sœur de
Charles IV, duc de Lorraine, et que Monsieur avait clandestinement
épousée à Nancy en 1632, dont Louis XIII voulut si longtemps faire
casser le mariage, et qui ne put venir en France qu'après sa mort, était
morte en 1662. M\textsuperscript{me} de Savoie, sœur du même lit, et
cadette de M\textsuperscript{me} de Guise, était morte sans enfants en
1664, et son autre sœur du même lit et l'aînée était revenue dans un
couvent de France, sans aucune considération, après avoir quitté ses
enfants et son mari, le grand-duc de Toscane, qui ne put jamais
l'apprivoiser. M\textsuperscript{lle} d'Alençon, c'est ainsi qu'on
appelait M\textsuperscript{me} de Guise avant son mariage, avait plus de
vingt ans, étant née 26 septembre 1646. Elle était fort maltraitée par
Mademoiselle, sa sœur, unique du premier lit, puissamment riche, et qui
n'avait jamais pu digérer le second mariage de Monsieur, son père, ni
souffrir sa seconde femme, ni ses filles. Dans cet état d'abandon,
comptée pour rien par le roi et par Monsieur, ses seuls parents
paternels, car la branche de Condé était déjà fort éloignée, elle se
laissa gouverner par M\textsuperscript{lle} de Guise, qui tenait par ses
biens et son rang un grand état dans le monde, et qui s'était soumis
toute la maison de Lorraine. C'était de plus une personne de beaucoup
d'esprit et de desseins, et fort digne des Guises ses pères. Elle avait
perdu tous ses frères, desquels tous il ne restait d'enfants que le seul
duc de Guise né en août 1650. Il y avait un grand inconvénient\,; sa
mère était à peu près folle dès lors, et ne tarda pas à le devenir tout
à fait. Elle était fille unique et héritière du dernier duc d'Angoulême,
fils du bâtard de Charles. IX, et d'une La Guiche, de laquelle j'ai déjà
parlé, chez qui ma mère fut mariée.

M\textsuperscript{lle} de Guise, malgré ce grand contredit, entreprit
cette grande affaire, et elle en vint à bout. Tous les respects dus à
une petite-fille de France furent conservés. M. de Guise n'eut qu'un
ployant devant M\textsuperscript{me} sa femme. Tous les jours à dîner il
lui donnait la serviette, et quand elle était dans son fauteuil, et
qu'elle avait déployé sa serviette, M. de Guise debout, elle ordonnait
qu'on lui apportât un couvert qui était toujours prêt au buffet. Ce
couvert se mettait en retour au bout de la table, puis elle disait à M.
de Guise de s'y mettre, et il s'y mettait. Tout le reste était observé
avec la même exactitude, et cela se recommençait tous les jours sans que
le rang de la femme baissât en rien, ni que, par ce grand mariage, celui
de M. de Guise en ait augmenté de quoi que ce soit. Il mourut de la
petite vérole à Paris, en juillet 1671, et ne laissa qu'un seul fils qui
ne vécut pas cinq ans, et qui mourut à Paris en août 1675.
M\textsuperscript{me} de Guise en fut affligée jusqu'à en avoir oublié
son \emph{Pater}.

Elle fut toujours mal avec Mademoiselle, quoiqu'elles logeassent toutes
deux au Luxembourg, qu'elles partageaient par moitié. C'était une
princesse très pieuse et tout occupée de la prière et de bonnes
œuvres\,; elle passait six mois d'hiver à la cour, fort bien traitée du
roi, et soupant tous les soirs au grand couvert, mais passant les Marlys
à Paris. Les autres six mois elle les passait à Alençon, où elle
régentait l'intendant comme un petit compagnon, et l'évêque de Séez, son
diocésain, à peu près de même, qu'elle tenait debout des heures
entières, elle dans son fauteuil, sans jamais l'avoir laissé asseoir
même derrière elle en un coin. Elle était fort sur son rang, mais du
reste, savait fort ce qu'elle devait, le rendait, et était extrêmement
bonne. En allant et revenant d'Alençon, elle passait toujours quelques
jours à la Trappe et coupait son séjour d'Alençon par y faire un petit
voyage exprès. Elle y logeait au dehors dans une maison que M. de la
Trappe avait bâtie pour les abbés commendataires, afin qu'ils ne
troublassent point la régularité de la maison. Il était le directeur de
M\textsuperscript{me} de Guise, et on a, entre ses ouvrages,
quelques-uns qu'il a faits pour elle. Il venait de perdre l'abbé qu'il
avait choisi et qui était à souhait. Il n'avait pas cinquante ans et il
était d'une bonne santé. Une fièvre maligne l'emporta.
M\textsuperscript{me} de Guise contribua à faire agréer au roi celui que
M. de la Trappe désira mettre en sa place.

Ce fut la dernière bonne œuvre de cette princesse. Elle tomba
incontinent après malade, d'un mal assez semblable à celui dont M. de
Luxembourg était mort, et qui l'emporta de même le 17 mars. Elle avait
reçu ses sacrements, et elle mourut avec une piété semblable à sa vie.
Le roi l'aimait et l'alla voir deux fois, la dernière le matin du jour
qu'elle mourut, et le soir il alla coucher et passer quelques jours à
Marly pour laisser faire les cérémonies. Mais elle les avait toutes
défendues, et voulut être enterrée non à Saint-Denis suivant sa
naissance, mais aux Carmélites du faubourg Saint-Jacques, et en tout
comme une simple religieuse\,: elle fut obéie. On ne sut qu'à sa mort
qu'elle portait un cancer depuis longtemps, qui paraissait prêt à
s'ouvrir. Dieu lui en épargna les douleurs. Elle avait fait et jeûné
tous les carêmes, et toute sa vie n'en était pas moins pénitente. Le roi
donna mille écus de pension à M\textsuperscript{me} de Vibraye sa dame
d'honneur, et cinq cents écus à chacune de ses filles d'honneur. La
duchesse de Joyeuse, sa belle-mère, ne la survécut pas de deux mois,
dans l'abbaye d'Essey, où elle faisait prendre soin d'elle, depuis la
mort de M\textsuperscript{me} d'Angoulême.

Le marquis de Blanchefort, second fils du feu maréchal de Créqui, beau,
bien fait, galant, avancé et fort appliqué à la guerre, mourut en même
temps à Tournai, sans alliance\,; et M. de Saint-Géran tomba mort dans
Saint-Paul à Paris. On dit qu'il venait de faire ses dévotions. C'est ce
comte de Saint-Géran si connu par ce procès célèbre sur son état, qui
est entre les mains de tout le monde. Il portait une calotte d'une
furieuse blessure, qu'il avait reçue devant Besançon, du crâne, du frère
aîné de Beringhen, premier écuyer, à qui un coup de canon emporta la
tête. M. de Saint-Géran était gros, court et entassé, avec de gros yeux
et de gros traits qui ne promettaient rien moins que l'esprit qu'il
avait. Il avait été auprès de quelques princes d'Allemagne lieutenant
général, chevalier de l'ordre en 1688, fort pauvre, presque toujours à
la cour, mais peu de la cour quoique dans les meilleures compagnies. Sa
femme, charmante d'esprit et de corps, l'avait été pour d'autres que
pour lui\,; leur union était moindre que médiocre. M. de Seignelay entre
autres l'avait fort aimée. Elle avait toujours été recherchée dans ce
qui l'était le plus à la cour, et dame du palais de la reine, recherchée
elle-même dans tout ce qu'elle avait et mangeait avec un goût exquis et
la délicatesse et la propreté la plus poussée. Elle était fille du frère
cadet de M. de Blainville, premier gentilhomme de la chambre de Louis
XIII, à la mort duquel sans enfants mon père eut sa charge. Sa viduité
ne l'affligea pas\,; elle ne sortait point de la cour et n'avait pas
d'autre demeure. C'était en tout une femme d'excellente compagnie et
extrêmement aimable, et qui fourmillait d'amis et d'amies.

On perdit en même temps M\textsuperscript{me} de Miramion à soixante-six
ans, dans le mois de mars, et ce fut une véritable {[}perte{]}. Elle
s'appelait Bonneau et son père le sieur de Rubelle, de fort riches
bourgeois de Paris. Elle avait épousé un autre {[}bourgeois{]} d'Orléans
fort riche aussi, dont le père avait obtenu des lettres patentes pour
changer son sale et ridicule nom de Beauvit en celui de Beauharnais.
Elle fut mariée et veuve la même année, en 1645, et demeura grosse d'une
fille qu'elle maria à M. de Nesmond, qu'elle vit longtemps président à
mortier à Paris, et qui n'eut point d'enfants. M\textsuperscript{me} de
Miramion veuve, jeune, belle et riche, fut extrêmement recherchée de se
marier sans y vouloir entendre. Bussy-Rabutin, si connu par son
\emph{Histoire amoureuse des Gaules} et par la profonde disgrâce qu'elle
lui attira, et encore plus par la vanité de son esprit et la bassesse de
son cœur quoique très brave à la guerre, la voulait épouser absolument,
et, protégé par M. le Prince qui n'eut pas, dans les suites, lieu de se
louer de lui, l'enleva et la conduisit dans un château. Tout en y
arrivant elle prononça devant ce qu'il s'y trouva de gens un vœu de
chasteté, puis dit à Bussy que c'était à lui à voir ce qu'il voulait
faire. Il se trouva étrangement déconcerté de cette action si forte et
si publique, et ne songea plus qu'à mettre sa proie en liberté et à
tâcher d'accommoder son affaire. De ce moment M\textsuperscript{me} de
Miramion se consacra entièrement à la piété et à toutes sortes de bonnes
œuvres. C'était une femme d'un grand sens et d'une grande douceur, qui
de sa tête et de sa bourse eut part à plusieurs établissements très
utiles dans Paris\,; et elle donna la perfection à celui de la
communauté de Sainte-Geneviève, sur le quai de la Tournelle, où elle se
retira, et qu'elle conduisit avec grande édification, et qui est si
utile à l'éducation de tant de jeunes filles et à la retraite de tant
d'autres filles et veuves. Le roi eut toujours une grande considération
pour elle, dont son humilité ne se servait qu'avec grande réserve et
pour le bien des autres, ainsi que de celle que lui témoignèrent toute
sa vie les ministres, les supérieurs ecclésiastiques et les magistrats
publics. Sa fille, dont la maison était contiguë à la sienne, se fit un
titre d'en prendre soin après sa mort, et devenue veuve se fit dévote en
titre d'office et d'orgueil, sans quitter le monde qu'autant qu'il
fallut pour se relever sans s'ennuyer. Elle s'était ménagé les accès de
sa mère de son vivant, et les sut bien cultiver après, surtout
M\textsuperscript{me} de Maintenon dont elle se vantait modestement. Ce
fut la première femme de son état qui ait fait écrire sur sa porte «
Hôtel de Nesmond.\,»On en rit, on s'en scandalisa, mais l'écriteau
demeura et est devenu l'exemple et le père de ceux qui de toute espèce
ont peu à peu inondé Paris. C'était une créature suffisante, aigre,
altière, en un mot une franche dévote, et dont le maintien la découvrait
pleinement.

M\textsuperscript{me} de Sévigné, si aimable et de si excellente
compagnie, mourut quelque temps après à Grignan chez sa fille, qui était
son idole et qui le méritait médiocrement. J'étais fort des amis du
jeune marquis de Grignan, son petit-fils. Cette femme, par son aisance,
ses grâces naturelles, la douceur de son esprit, en donnait par sa
conversation à qui n'en avait pas, extrêmement bonne d'ailleurs, et
savait extrêmement de toutes choses sans vouloir jamais paraître savoir
rien.

Le P. Séraphin, capucin, prêcha cette année le carême à la cour. Ses
sermons, dont il répétait souvent deux fois de suite les mêmes phrases,
et qui étaient fort à la capucine, plurent fort au roi, et il devint à
la mode de s'y empresser et de l'admirer\,; et c'est de lui, pour le
dire en passant, qu'est venu ce mot si répété depuis, \emph{sans Dieu
point de cervelle}. Il ne laissa pas d'être hardi devant un prince qui
croyait donner les talents avec les emplois. Le maréchal de Villeroy
était à ce sermon\,; chacun comme entraîné le regarda. Le roi fit des
reproches à M. de Vendôme, puis à M. de La Rochefoucauld de ce qu'il
n'allait jamais au sermon, pas même à ceux du P. Séraphin. M. de Vendôme
lui répondit librement qu'il ne pouvait aller entendre un homme qui
disait tout ce qu'il lui plaisait sans que personne eût la liberté de
lui répondre, et fit rire le roi par cette saillie.

M. de La Rochefoucauld le prit sur un autre ton, en courtisan avisé. Il
lui dit qu'il ne pouvait s'accommoder d'aller, comme les derniers de la
cour, demander une place à l'officier qui les distribuait, s'y prendre
de bonne heure pour en avoir une bonne, et attendre et se mettre où il
plaisait à cet officier de le placer. Là-dessus et tout de suite, le roi
lui donna pour sa charge une quatrième place derrière lui, auprès du
grand chambellan, en sorte que partout il est ainsi placé\,: le
capitaine des gardes derrière le roi, qui a le grand chambellan à sa
droite, et le premier gentilhomme de la chambre à sa gauche, et jamais
que ces trois-là jusqu'à cette quatrième que M. de La Rochefoucauld sut
tirer sur le temps pour sa charge qui n'en avait point, qui est nouvelle
et que le roi fit pour Guitri, tué au passage du Rhin, auquel M. de La
Rochefoucauld succéda. M. d'Orléans, premier aumônier, qui a sa place au
prie-Dieu, mais point ailleurs, s'était peu à peu accoutumé à se mettre
auprès du grand chambellan, et, comme il était fort aimé et honoré, on
l'avait laissé faire sans lui dire mot. C'était celle que le roi donna à
M. de La Rochefoucauld. M. d'Orléans, qui â force de s'y mettre la
voulait croire sienne, fit les hauts cris comme si elle l'eût été, et
n'osant se prendre au roi, qui venait de le nommer si gracieusement au
cardinalat, se brouilla ouvertement avec M. de La Rochefoucauld,
jusqu'alors et de tout temps son ami particulier. Les envieux de sa
faveur, qui ne manquent point dans les cours, firent grand bruit, M. le
Grand surtout et ses frères. Ils étaient eux et le duc de Coislin, et M.
d'Orléans, et le chevalier de Coislin, enfants du frère et de la sœur\,;
ils avaient toujours vécu sur ce pied-là avec eux, et s'étaient surtout
piqués d'une grande amitié pour M. d'Orléans. M. le Grand était l'émule
de la faveur de M. de La Rochefoucauld, et fort jaloux l'un de l'autre.
N'osant aller au roi, ils excitèrent Monsieur dont le chevalier de
Lorraine disposait\,; bref toute la cour se partialisa, et M. d'Orléans
l'emporta par le nombre et par la considération des personnes qui se
déclarèrent pour lui. Le roi tâcha inutilement de lui faire entendre
raison. M. de La Rochefoucauld, vraiment affligé de perdre son amitié,
fit fort au delà de ce dont il était ordinairement capable\,; des amis
communs s'entremirent, M. d'Orléans fut inflexible, et quand il vit que
tout cet éclat n'aboutissait qu'à du bruit, il s'en alla bouder dans son
diocèse.

Le public perdit bientôt après un homme illustre par son esprit, par son
style et par la connaissance des hommes, je veux dire La Bruyère, qui
mourut d'apoplexie, à Versailles, après avoir surpassé Théophraste en
travaillant d'après lui, et avoir peint les hommes de notre temps dans
ses \emph{Nouveaux Caractères} d'une manière inimitable. C'était
d'ailleurs un fort honnête homme, de très bonne compagnie, simple, sans
rien de pédant et fort désintéressé\,; je l'avoir assez connu pour le
regretter, et les ouvrages que son âge et sa santé pouvaient faire
espérer de lui.

Daquin, ci-devant premier médecin du roi, ne put survivre longtemps à sa
disgrâce\,; il alla chercher à prolonger ses jours à Vichy, et y mourut
en arrivant, et avec lui sa famille qui retomba dans le néant.

L'Espagne perdit la reine mère, d'un cancer\,; c'était une méchante et
malhabile femme, toujours gouvernée par quelqu'un, qui remplit de
troubles la minorité du roi son fils. Don Juan d'Autriche lui arracha le
fameux Vasconcellos, puis le jésuite Nitard son confesseur, qu'elle
consola par l'ambassade d'Espagne à Rome, n'étant que simple jésuite, et
le fit cardinal après, mais sans avoir pu le rapprocher d'elle. Elle
régna avec plus de tranquillité sous le nom de son fils, devenu majeur,
et rendit fort malheureuse la fille de Monsieur que ce prince avait
épousée. À la fin son mauvais gouvernement et plus encore son humeur
altière, qui lui avait aliéné toute la cour, refroidit le roi pour elle,
sur qui elle l'exerçait avec peu de ménagement, et elle alla passer ses
dernières années dans un palais particulier dans Madrid, peu comptée et
peu considérée. Elle haïssait extrêmement la France et les Français.
Elle était sœur de l'empereur et seconde femme de Philippe IV, qui de sa
première femme, fille d'Henri IV, avait eu notre reine Marie-Thérèse, en
sorte que le roi en drapa pour un an sans regret.

\hypertarget{chapitre-xx.}{%
\chapter{CHAPITRE XX.}\label{chapitre-xx.}}

1696

~

\relsize{-1}

{\textsc{Reprise du procès de M. de Luxembourg.}} {\textsc{- Récusation
du premier président Harlay.}} {\textsc{- Option hardie de M. de
Luxembourg.}} {\textsc{- Renvoi au parlement de la cause par la bouche
du roi.}} {\textsc{- Pairs postérieurs en cause.}} {\textsc{- Partialité
de Maisons contre nous.}} {\textsc{- Insolence de l'avocat de M. de
Luxembourg, sans suite.}} {\textsc{- Misère des ducs opposants.}}
{\textsc{- D'Aguesseau, avocat général, conclut pour nous.}} {\textsc{-
M. de Luxembourg appointé sur sa prétention et sans qu'il en eût fait
demande\,; mis en attendant au rang de 1662.}} {\textsc{- Pitoyable
conduite des ducs opposants.}} {\textsc{- Projet d'écrit que je fis pour
le roi inutilement.}} {\textsc{- Prévarication solennelle du premier
président Harlay.}} {\textsc{- Honte des juges de leur jugement.}}
{\textsc{- Réception de M. de Luxembourg au parlement.}} \relsize{1}

~

Maintenant il est temps de reprendre la suite du procès de M. de
Luxembourg dont je n'ai pas voulu interrompre {[}mon récit{]}. Le départ
pour les armées avait interrompu le cours de cette affaire que M. de
Luxembourg avait reprise à la mort du maréchal son père. Nous avions
fait notre opposition à sa réception au parlement\,; nous avions résolu
de mettre en cause le duc de Gesvres pour entraîner par là la récusation
du premier président, dont les ruses, les détours et les manèges, dans
la soif de demeurer notre juge, avaient causé une division entre nous,
dont le danger avait été promptement arrêté par notre réunion pour la
récusation, avec ce ménagement, pour ceux qui l'avaient combattue, de
n'y venir point tant que rien ne péricliterait. C'est ce qui se trouve
expliqué page 183, où on voit aussi qu'il fut résolu de commencer par
une requête civile de MM. de Lesdiguières, de Brissac et de Rohan. Ce
fut aussi par où nous voulûmes recommencer cette année. La requête
civile toute scellée et toute prête était entre les mains du procureur
du duc de Rohan, tandis que, dès notre première assemblée, les
agitations se renouvelaient parmi nous, sur la récusation actuelle du
premier président, par toutes les bassesses et les artifices qu'il
prodiguait de nouveau pour se conserver le plaisir de demeurer notre
juge et parer la honte de la récusation. Nous sûmes de ce procureur du
duc de Rohan qu'il avait défense expresse de lui, qui était lors en
Bretagne, de laisser faire aucun usage de la requête civile que
préalablement le duc de Gesvres ne fût en cause. Cette déclaration finit
toutes les diversités d'avis. Le duc de Gesvres fut assigné et mis en
cause sans donner le moindre signe de vie au premier président, non plus
que lors de sa récusation que nous fîmes tout de suite. La rage qu'il en
conçut ne se peut exprimer, et, quelque grand comédien qu'il fit, il ne
la put cacher. Toute son application depuis ne fut plus que de faire
tout ce qu'il pourrait contre nous\,; le reste de masque tomba, et la
difformité du juge parut dans l'homme à découvert.

Aussitôt après, nous fîmes signifier à M. de Luxembourg qu'il eût à
opter, des lettres d'érection de Piney de 1581 ou de celles de 1662. En
abandonnant les premières, le procès tombait\,; en répudiant les
dernières, il renonçait à l'état certain de duc et pair après nous, pour
s'attacher à l'espérance de nous précéder, et courir le risque, s'il
perdait, de n'être plus que duc vérifié de l'érection qui avait été
faite en sa faveur de la terre de Beaufort sous le nom de Montmorency,
lorsqu'il épousa la fille du duc de Chevreuse. Le parti était bien
délicat\,; aussi en fut-il effrayé\,; mais, après avoir bien consulté,
il ne put se résoudre d'abandonner ses prétentions, et choisit d'en
courir tout le danger. Il compta sur son crédit et sûr la compassion des
juges dans une si grande extrémité, et il espéra contre toute raison et
prudence. M. de Gesvres, mis en cause, exclut tous les présidents à
mortier, excepté Maisons seul, et des trois avocats généraux, ne nous
laissa que d'Aguesseau, qui était alors l'aigle du parlement.

Cette reprise pouvait demander des lettres patentes de renvoi au
parlement pour lui donner pouvoir de juger, les pairs, non parties,
convoqués, et par l'option forcée de M. de Luxembourg, par laquelle en
perdant son procès, il tombait entièrement de la dignité de pair de
France\,; mais comme il était pourtant vrai que cette option n'était
qu'une suite en conséquence de la reprise du même procès, le roi aima
mieux y suppléer de bouche. Il manda donc le président de Maisons et les
gens du roi, et leur dit qu'encore que notre affaire ne fût pas
naturellement de la compétence du parlement, il voulait que pour cette
fois il la jugeât selon les lois et définitivement, sans tirer à
conséquence pour de pareilles matières\,; parce qu'il ne se voulait
point mêler de celle-ci, ni la retenir à son conseil. Ce fut le 27 mars,
et le dernier du même mois. Le premier président, le président de
Maisons et plusieurs conseillers de la grand'chambre vinrent faire leur
remerciement au toi de l'honneur qu'il lui plaisait faire au parlement
de lui renvoyer notre affaire, et de ce qu'il avait fait la grâce de
dire là-dessus au président de Maisons et aux gens du roi.

Il ne fut plus question que de se bien défendre de part et d'autre. Nous
persuadâmes à quelques ducs, postérieurs aux lettres d'érection nouvelle
de Piney en 1662, de se joindre à nous par la juste crainte que d'autres
prétentions d'ancienneté les vinssent troubler si celle-ci réussissait,
et les ducs d'Estrées, La Meilleraye, Villeroy, Aumont, La Ferté et
Charost entrèrent avec nous en cause. Harreau plaida pour eux, Fretteau
pour nous, Magneux pour M. de Richelieu à cause de ses pairies femelles,
en expliquer les différences et les écoulements\,; Chardon fut chargé de
la réplique, et Dumont plaida pour M. de Luxembourg. Nous nous mimes à
solliciter tous ensemble, et à les instruire, et nous nous rendîmes
assidus aux audiences qui étaient tous les mardis et samedis matin aux
bas sièges. M. de La Trémoille, en année de premier gentilhomme de la
chambre, et M. de, La Rochefoucauld\,; dont la charge est un service
continuel, s'y rendirent au moins une fois la semaine, très
ordinairement toutes les deux fois\,; je n'en manquai aucune, et presque
tous s'y rendirent aussi assidus. Notre nombre nous détourna d'y mener
nos amis, et M. de Luxembourg n'y fut accompagné que de MM. de Saillant
et de Clérembault, son beau-père, dont le maintien, le vêtement et la
perruque, fort semblable à celle des quais \footnote{Le mot est ainsi
  écrit dans le manuscrit, sans doute pour \emph{laquais}. On avait dit
  d'abord \emph{naquets}, de l'allemand \emph{knecht} (serviteur), et
  probablement par abréviation \emph{quets} ou \emph{quais}.} et qui lui
en avait mérité le surnom, paraissait un vieux valet que l'attachement
conduit à la suite de son maître. Ses écus nous firent plus de mal que
son crédit\,; il ne les épargna pas à une dame Bailly que le président
de Maisons entretenait depuis longues années, qui logeait avec lui, et
pour qui il avait chassé sa femme, saur de Fieubet, conseiller d'État
fort distingué, et qui était elle-même une femme d'esprit et de mérite.
Il avait eu depuis peu la survivance de sa charge pour son fils par le
crédit du premier président\,; aussi ne fit-il aucun pas dans notre
affaire que par ses ordres, et se fit un fidèle canal de sa partialité.
D'Aguesseau s'instruisit avec grande application, et en montra une
extrême à écouter les avocats en toutes les audiences.

Nous nous mettions dans la lanterne du côté de la cheminée, qui était
celui de nos avocats, et sur le banc des gens du roi avec eux\,; et M.
de Luxembourg, avec sa petite suite et son avocat, auprès de la lanterne
du côté de la buvette, avantage de droit qui ne nous fut pas disputé. La
réception du duc de Villeroy, qui se fit un des jours ne nos audiences,
y amena les princes du sang et légitimés et beaucoup d'autres pairs. M.
le prince de Conti, M. de Reims, M. de Vendôme et plusieurs autres y
demeurèrent, et furent si satisfaits d'avoir ouï plaider Harreau, qu'ils
ne doutèrent pas que nous ne gagnassions notre cause.

Nos avocats ayant fini, ce fut à Dumont à parler. Il tint trois
audiences en beaucoup de fatras, et, faute de raisons, battit fort la
campagne\,; à la quatrième, il se licencia fort sur nos avocats\,; la
cinquième fut fertile en subtilités, où hors d'espérance de rien
emporter par raisons, il hasarda tout pour réussir par une impression de
crainte qui persuadât à des gens éloignés du monde et de la cour que le
roi était intéressé dans l'affaire pour M. de Luxembourg, comme le
premier président avait taché sans cesse de le leur persuader. Ce Dumont
était un homme fort audacieux, et qui en lit là ses preuves. Il assimila
tant qu'il put le droit infini des pairies femelles, qu'il s'efforçait
d'établir, au nouveau rang des bâtards, et nous appliqua en propres
termes ce passage de l'Écriture\,: \emph{Populus hic labiis me honorat,
cor autem eorum longe est a une\,;} tandis que nous contestions si
vivement le rang à sa partie, sans cesser de faire assidûment notre cour
au roi. Les ducs de Montbazon (Guéméné), La Trémoille, Sully,
Lesdiguières, Chaulnes et La Force étaient sur le banc des gens du roi,
et moi, assis dans la lanterne entre les ducs de La Rochefoucauld et
d'Estrées. Je m'élançai dehors criant à l'imposture et justice de ce
coquin. M. de La Rochefoucauld me retint à mi-corps et me fit taire. Je
m'enfonçai de dépit plus encore contre lui que contre l'avocat. Mon
mouvement avait excité une rumeur, et il n'y avait qu'à interpeller M.
de Luxembourg s'il avouait son avocat ou non, et sur le champ on aurait
eu justice du parlement contre l'avocat, ou dans la journée, du roi de
M. de Luxembourg\,; mais nous n'étions plus pour la demander, et moins
encore pour nous la faire\,; on laissa achever Dumont, et le président
de Maisons fit une légère excuse.

L'après-dînée nous nous assemblâmes\,: M. de Guéméné y rêva à la suisse
à son ordinaire\,; M. de La Trémoille parut plus fâché que le matin\,;
M. de Lesdiguières tout neuf encore écoutait fort étonné\,; M. de
Chaulnes raisonnait en ambassadeur, avec le froid et l'accablement d'un
courage étouffé par la douleur de son échange dont il ne put jamais
revenir\,; M. de La Rochefoucauld pétillait de colère et d'impatience,
et au fond ne savait que proposer ni que conclure\,; le duc d'Estrées
grommelait en grimaçant sans qu'il en sortit rien\,; et le duc de
Béthune bavardait des misères. Après une longue pétaudière, il fut
résolu que le roi serait informé de cette insolence par MM. de La
Trémoille et de La Rochefoucauld, chez lequel nous nous assemblerions
avec chacun un projet de réponse pour en pouvoir choisir. Ces messieurs
s'en acquittèrent auprès du roi mieux qu'il n'y avait eu lieu de
l'espérer. Le roi témoigna sa surprise que Maisons n'eût pas imposé
silence, et ajouta, sur ce beau passage de l'Écriture, qu'il était à
présumer que ceux qui accusaient les autres de manquements à son égard
en étaient plus coupables, et que pour nous, nous pouvions être
pleinement en repos sur ce qu'il en pensait\,; que M. de Luxembourg ne
lui avait point parlé, qu'il verrait ce qu'il lui dirait, mais qu'il ne
nous disait rien sur notre réponse, sinon qu'il voulait n'en rien savoir
qu'après qu'elle serait faite. Nous portâmes donc chacun la nôtre chez
M. de La Rochefoucauld, où je crus voir des pensionnaires qui ont
composé pour les places. Il s'en fit une assez mauvaise compilation. M.
de Chaulnes se chargea d'aller travailler avec Chardon pour là réplique,
et de lui porter notre réponse\,; il l'affaiblit encore, et elle ne
valut pas la peine d'être prononcée, au moins c'est ce qu'il me partit
quand Chardon la débita.

Tout fini de part et d'autre, ce fut à d'Aguesseau à parler\,: il s'en
acquitta avec une si exacte fidélité à mettre dans le plus grand jour
jusqu'aux moindres raisons alléguées de part et d'autre, et tant de
justesse à les balancer toutes, et à laisser une incertitude entière sur
son avis, que le barreau et les parties mêmes auraient donné les mains à
en passer par son avis. Il se reposa le lendemain\,; et le vendredi, 13
avril, il reparut pour achever. Il tint encore l'auditoire assez
longtemps en suspens, puis commença à se montrer\,; ce fut avec une
érudition, une force, une précision et une éloquence incomparables, et
conclut entièrement pour nous. Il se déroba aussitôt aux acclamations
publiques, et nous fûmes priés de sortir pour laisser opiner les juges
avec liberté. C'est ce qu'ils appellent délibérer sur le registre. Tout
le monde sortit donc en même temps, et ils demeurèrent seuls dans la
grand'chambre. M\textsuperscript{me} de La Trémoille, qui était dans une
lanterne haute, nous vint trouver. Le délibéré ne fut pas long, mais
notre impatience nous fit entrer dans le parquet des huissiers, d'où, un
moment après, nous vîmes sortir de la grand'chambre qui était fermée, et
où il ne devait y avoir que les juges, Poupart, secrétaire du premier
président. Bientôt après on nous fit entrer pour entendre la
prononciation de l'arrêt qui donna gain de cause à M. de Luxembourg sur
l'érection de 1662 et l'appointa sur celle de 1581 \footnote{On
  appointait un procès lorsqu'on renvoyait les parties à une décision
  qui devait être prise ultérieurement sur le vu des pièces, parce que
  la question ne paraissait pas suffisamment éclaircie. C'était
  quelquefois un moyen d'ajourner indéfiniment la solution d'un procès.},
tellement qu'il se trouva par là au même état qu'était son père. Nous
eûmes peine à entendre un arrêt si injuste et si nouveau, et qui
statuait ce qui ne pendait point en question.

Quelque outré que je fusse, je proposai là même de nous aller assembler,
mais je parlais à des gens à qui le dépit avait bouché les oreilles.
Rentré chez moi, ce même dépit qui me faisait tout une autre impression,
m'en fit sortir pour aller tâcher de persuader M. de La Rochefoucauld de
porter ses plaintes au roi, mais je ne trouvai qu'un homme furieux,
incapable de rien entendre ni de rien faire, et qui s'exhalait
inutilement. Je revins donc chez moi plus piqué contre les nôtres que
contre nos juges. Je n'y fus pas longtemps que la duchesse de La
Trémoille me manda d'aller chez Riparfonds. Je fus surpris d'y trouver
M. de La Rochefoucauld avec elle qui l'exhortait avec force comme
j'avais fait le matin. Je me joignis à elle, mais nous y perdîmes notre
temps. Il ne répondit qu'en furie, et au fond qu'en mollesse, et las
enfin d'être serré de si près, il nous laissa. M\textsuperscript{me} de
La Trémoille, outrée, ne se contraignit pas sur son chapitre, et puis
nous nous séparâmes. Rentrant chez moi, il me vint dans la pensée de
faire un mémoire pour le roi. Comme il explique bien l'arrêt et nos
sujets de plaintes, je l'insérerai ici.

« Sire.

« L'arrêt qui a été rendu ce matin sur notre affaire porte des
caractères si singuliers, que nous croyons pouvoir oser supplier la
bonté et la patience de Votre Majesté de trouver bon que nous ayons
l'honneur de lui en rendre compte. Nous commencerons par nous dépouiller
des premiers mouvements qui peuvent échapper à ceux qui sont vivement
persuadés d'un tort considérable qui leur a été fait, et nous
demanderons à Votre Majesté la grâce de lire cet écrit, non comme une
plainte, mais comme un soulagement que nous nous donnons en
l'instruisant de ce qui nous touche si sensiblement, moins encore comme
une censure aigre contre des personnes dont nous ne croyons pas nous
devoir louer, mais comme un récit exactement conforme à la vérité la
plus scrupuleuse.

« Ce matin, Sire, les juges sont entrés un peu avant neuf heures,
apparemment instruits des désirs qu'il y a si longtemps que M. le
premier président ne se donne pas même la peine de cacher contre nos
intérêts, et ce magistrat, seul dès cinq heures et demie dans la
grand'chambre, a eu tout le loisir de leur en rafraîchir la mémoire les
ayant tous attendus et vus entrer un à un.

« M. l'avocat général d'Aguesseau a continué, avec une force et une
éloquence que tous les auditeurs en nombre prodigieux ont unanimement
admirées, le beau plaidoyer qu'il avait commencé avant-hier\,; il avait
ce jour-là rapporté avec une mémoire et une exactitude infinie toutes
les raisons de part et d'autre, et avait si bien réussi à les mettre
dans un jour égal, qu'on ne put pénétrer du tout ce qu'il pensait.
Aujourd'hui, Sire, il s'est expliqué, et pour nous\,; il a si fortement
combattu, et, nous osons vous l'avancer avec la voix du public, terrassé
les raisons de notre partie par les nôtres, par notre droit, par le
droit commun, par le droit public, que chacun nous a donné gain de
cause. Il a fait plus, Sire, il a été tellement convaincu que Votre
Majesté y était intéressée, qu'il a non seulement conclu, mais requis et
demandé en termes exprès et formels, que M. de Luxembourg ne fût point
reçu, et, comme par commisération pour son état et pour son nom, qu'il
fût sursis au jugement de sa réception jusqu'à ce que Votre Majesté se
fût expliquée plus clairement sur ses intentions et ses ordres sur la
diversité qui semble se trouver dans les lettres d'érection de Piney de
1662, et la déclaration de 1676, émanées de Votre Majesté, et, quant à
l'ancienne érection de Piney de 1581, il a conclu à son extinction à
cause des monstrueuses conséquences du contraire, également
préjudiciables à Votre Majesté et à l'État, qu'il a parfaitement
déduites.

« Il a été ordonné un délibéré sur le registre sur-le-champ,
c'est-à-dire que tout le monde s'est retiré pour laisser la liberté aux
juges d'opiner tout haut et plus à leur aise. Durant ce délibéré, où il
ne se doit trouver personne que les juges, M. l'avocat général Harlay et
Poupart, secrétaire de M. le premier président, sont demeurés dans la
grand'chambre. Au bout d'une grosse heure les parties ont été rappelées
pour entendre leur arrêt que voici.

« Nous l'avouerons, Sire, ç'a été pour nous un coup de foudre, et nous
ne croyions pas le parlement assez hardi pour faire tant de choses à la
fois sans exemple\,: accorder à M. de Luxembourg ce qu'il ne demandait
pas, puisque, par l'option qu'il a faite, il a renoncé à l'érection de
1662, dont il lui donné la dignité et le rang\,; et pour prononcer la
réception d'un pair de France, non seulement contre les conclusions
formelles de l'homme de Votre Majesté, et de l'organe de ses volontés,
surtout en telles matières, mais encore contre sa réquisition expresse,
et sans user du tempérament qu'il a dit \emph{ne proposer à la cour que
par une espèce de commisération pour l'état violent mais juste de M. de
Luxembourg}, où il s'est mis par l'option qu'il a faite.

« Oserions-nous, Sire, prendre la liberté de demander en grâce à Votre
Majesté de se faire rendre compte du plaidoyer de M. d'Aguesseau, et
oserions-nous l'assurer qu'il mérite cet honneur\,? Mais, Sire,
oserions-nous davantage, et notre confiance aux bontés et en l'équité de
Votre Majesté nous en donnerait-elle assez pour lui demander comme la
plus grande grâce de se faire rapporter l'affaire pour la juger de
nouveau, si le plaidoyer de votre avocat général et les deux nullités
expliquées de l'arrêt vous paraissent mériter une révision\,? Oui, Sire,
nous l'espérons de votre justice accoutumée et de votre bonté, et à qui
est-ce enfin à décider des dignités et de leur effet, sinon à celui qui
en est le seul maître, dispensateur et arbitre suprême, et à la source
incorruptible de la justice\,? Nous demandons cette grâce à Votre
Majesté avec toute la soumission et toute l'instance dont nous sommes
capables, et aucun de nous ne la désire avec une ardeur moins vive que
la restitution de ses biens et de son honneur, également contents et
soumis au succès, tel qu'il puisse être, pourvu que sa décision sorte de
la bouche de l'oracle de la justice.\,»

Dès que j'eus achevé ce projet de mémoire, j'allais le porter au duc de
La Trémoille à qui j'avais mandé de ne s'en aller pas à Marly que je ne
l'eusse vu. M\textsuperscript{me} de La Trémoille et la duchesse de
Créqui sa mère, qui en entendirent la lecture avec lui, auraient bien
voulu qu'il l'eût porté au roi. Il en avait aussi grande envie, mais la
scène de M. de La Rochefoucauld et sa faiblesse les en détourna. Je ne
trouvai pas mieux mon compte avec le duc de Chaumes, à qui je le portai.
De là je m'en revins chez moi plus fâché, s'il se pouvait encore, que je
n'en étais sorti. Il était pourtant vrai que le roi trouva le jugement
contre toutes les formes et très extraordinaire, et qu'il s'attendait
aux plaintes qui lui en seraient portées. Il s'en expliqua même à son
dîner d'une manière peu avantageuse au parlement, et toute sa promenade
le soir dans ses jardins se passa à ouïr M. de Chevreuse qui revenait de
Paris, et à lui faire des questions peu obligeantes pour les juges. Mais
l'obstination de M. de La Rochefoucauld, qui tourna en dépit contre
soi-même, rendit tout inutile, et me combla de déplaisir que j'allai
chercher à émousser à la Trappe, pour y profiter du temps de la semaine
sainte. En revenant, j'appris que le roi, à son retour à Versailles,
avait fort parlé de ce jugement au premier président\,; que ce magistrat
l'avait fort blâmé, et dit au roi que notre cause était indubitable pour
nous, et qu'il l'avait toujours et dans tous les temps estimée telle.
C'était se jeter à lui-même la dernière pierre. Pensant ainsi, quel
juge, après tout ce qu'il fit contre nous jusqu'à nous forcer à le
récuser, et après en faire plus ouvertement contre nous sa propre
chose\,! S'il ne le pensait pas, quel juge encore et quel prévaricateur
de répondre au roi avec cette flatterie sur ce qu'il voyait quel était
son sentiment\,!

Les juges eux-mêmes, honteux de leur jugement, s'excusèrent sur la
compassion de l'état de M. de Luxembourg, tombé de toute pairie sans cet
expédient, et sur l'impossibilité qu'il gagnât jamais la préséance de
l'ancienne érection de 1581 dont ils lui avaient laissé la chimère,
c'est-à-dire qu'après s'être déshonorés par le jugement, ils montrèrent
par là la honte qu'ils en ressentaient. M. de Luxembourg fut reçu au
parlement au rang de 1662, le vendredi 4 mai suivant\,; le duc de La
Ferté et deux autres de la queue seulement s'y trouvèrent. Il vint chez
nous tous, mais aucun ne voulut d'aucun commerce ni avec lui ni avec ses
juges. Nous portâmes tous nos remerciements à l'avocat général
d'Aguesseau, qui pour la première fois de sa vie fut tondu, et dans la
seule cause qu'il eût peut-être plaidée, où cela était de droit
impossible par son seul caractère d'avocat général.

\hypertarget{chapitre-xxi.}{%
\chapter{CHAPITRE XXI.}\label{chapitre-xxi.}}

1696

~

\relsize{-1}

{\textsc{Destination des armées.}} {\textsc{- Maréchal de Choiseul sur
le Rhin.}} {\textsc{- M. de Lauzun se brouille et se sépare de M. et de
M\textsuperscript{me} la maréchale de Lorges.}} {\textsc{- Le duc de La
Feuillade vole son oncle en passant à Metz.}} {\textsc{- Prévenances du
maréchal de Choiseul en l'armée duquel j'arrive.}} {\textsc{- Mort de
Montal\,; du marquis de Noailles\,; de Varillas\,; du Plessis\,; du roi
de Pologne Jean Sobieski.}} {\textsc{- Cavalerie battue par M. de
Vendôme.}} {\textsc{- Négociation.}} {\textsc{- Armée de Savoie.}}
{\textsc{- Tessé.}} {\textsc{- Conditions de la paix de Savoie.}}
{\textsc{- Succès à la mer.}} \relsize{1}

~

La destination des armées était réglée comme l'année précédente, excepté
que le maréchal de Choiseul eut l'armée du Rhin à la place de M. le
maréchal de Lorges, le maréchal de Joyeuse alla en la sienne sur les
côtés\,; les princes du sang furent de l'armée du maréchal de Villeroy,
où M. de Chartres commanda la cavalerie, et les bâtards en celle de M.
de Boufflers, pour les séparer et mettre M. du gaine moins au grand
jour. Le roi, avant de déclarer le maréchal de Choiseul, le prit en
particulier dans son cabinet, et se fit expliquer par lui, pendant un
assez longtemps, les objets qu'il voyait de ses fenêtres. Il s'assura
par ce moyen de sa vue qui était fort basse de près, mais qui
distinguait bien de loin. Nous demeurâmes persuadés que le roi se sentit
plus à son aise de ce changement.

M. le maréchal de Lorges qui voulait faire, qui en sentait les moyens,
et qui voyait de plus, comme tout le monde, que les succès de Flandre
n'amèneraient point la paix dans un pays tout hérissé de places, à moins
de conjectures uniques, comme avaient été celles de Parc, lorsque le roi
revint, et la dernière qui sauva M. de Vaudemont, ne cessait tous les
hivers de proposer le siège de Mayence et d'emporter les lignes
d'Heilbronn, et d'en presser le roi à temps d'y donner les ordres
nécessaires à une heureuse et sûre exécution, et le roi, demeuré
persuadé qu'il ne fallait rien faire d'important en Allemagne et mesurer
tous ses efforts ailleurs, éconduisait tous les ans le maréchal de
Lorges avec ennui, parce que les répliques lui manquaient hors celles de
sa volonté. M. de Louvois, qui avait procuré cette guerre, et qui ne la
voulait finir de longtemps, avait, par cette raison-là même que je viens
de dire, persuadé au roi l'avis où il était demeuré, et que sa pique
personnelle contre le prince d'Orange lui faisait goûter, lequel
commandait toutes les années l'armée de Flandre, et sa colère aussi
contre les Hollandais. Les sources de toutes ces choses feraient ici une
trop longue parenthèse\,; peut-être se placeront-elles d'elles-mêmes
plus naturellement ailleurs.

Ce changement de situation de M. le maréchal de Lorges en apporta
bientôt un autre dans sa famille. M. de Lauzun, qui n'avait si
opiniâtrement voulu épouser sa seconde fille que par l'espérance de
rentrer dans quelque chose avec le roi, à l'occasion d'un beau-père
général d'armée, ne lui pardonnait pas d'avoir résisté à tous ses
contours, et de ne l'avoir mis à portée de rien. Il ignorait les
précautions et les défenses expresses du roi là-dessus, lors de son
mariage\,; et quand il les aurait sues, il n'aurait pas trouvé moins
mauvais que le maréchal ne les eût pas su vaincre. C'était d'ailleurs un
homme peu suivi et peu d'accord avec soi-même, et dont l'humeur et les
fantaisies lui avaient plus d'une fois coûté la plus haute et la plus
solide fortune. Dépité donc de n'avoir eu part à rien, et hors
d'espérance d'y revenir par un beau-père qui ne commandait plus d'armée,
il ne compta plus assez sur sa charge pour se contraindre plus
longtemps. Ce n'était pas un homme à durer longtemps au pot et au logis
d'autrui, et la jalousie, qui toute sa vie avait été sa passion
dominante, ne se pouvait accommoder d'une maison soir et matin ouverte à
Paris et à la cour, et qui fourmillait à toute heure de ce qu'il y avait
de plus brillant en l'une et en l'autre, sans que la cessation du
commandement eût rien diminué de cette nombreuse et continuelle
compagnie.

Il avait surtout en butte les neveux qui étaient sur le pied d'enfants
de la maison, et il était extrêmement choqué de leur âge et de leur
figure avec une femme de l'âge et de la figure de la sienne\,: elle ne
sortait pourtant jamais des côtés de sa mère\,; et ni le monde ni
lui-même n'avaient pu trouver rien à reprendre en elle\,; mais il
trouvait le danger continuel\,; et comme les vue d'ambition ne le
retenaient plus\,; il ne résista plus à ses fantaisies. Plaintes vagues,
caprices, scènes pour rien, lettres ou d'avis ou de menaces, humeurs
continuelles. Enfin il prit son temps que M. le maréchal de Lorges avait
le bâton à Marly pour M. le maréchal de Duras, il sortit le matin de
l'hôtel de Lorges, manda à sa femme de le venir trouver dans la maison
qu'il avait gardée\,; joignant l'Assomption, rue Saint-Honoré, et
qu'elle aurait un carrosse\,; sur les six heures, pour y aller désormais
demeurer avec lui. Quoique tout eût dû préparer à cette dernière scène,
ce furent des cris et des larmes de la mère et de la fille qui criaient
fort inutilement\,: il fallut obéir. Elle fut reçue chez M. de Lauzun
par les duchesses de Foix et du Lude, parentes et amies de M. de Lauzun,
qui lui donna toute une maison nouvelle, renvoya le soir même tous ses
domestiques, et lui présenta deux filles dont il connaissait la vertu,
et qu'il avait connues à M\textsuperscript{me} de Guise, pour ne la
jamais perdre de vue. Il lui défendit tout commerce avec père et mère et
tous ses parents, excepté M\textsuperscript{me} de Saint-Simon, avec qui
même il fut rare dans les premiers temps, et l'amusa de ce qu'il put de
compagnies qui ne lui étaient point suspectes. Après les premiers jours
d'affliction et d'étonnement, l'âge et la gaieté naturelle prirent le
dessus et servirent bien dans les suites à supporter des caprices
continuels et peu éloignés de la folie. M. le maréchal de Lorges prit
mieux patience que M\textsuperscript{me} sa femme\,; c'était son cœur
qui lui était arraché, une fille pour qui elle n'avait pu cacher ses
continuelles préférences. Le roi fut instruit de cet éclat assez
modérément par M. le maréchal de Lorges, beaucoup plus fortement appuyé
par M. de Duras\,; mais le roi, qui n'avait jamais approuvé ce mariage,
non plus que le public, et qui n'entrait jamais dans les affaires de
famille, ne voulut point se mêler de celle-ci. Le monde tomba fort sur
M. de Lauzun, et plaignit fort sa femme et le père et la mère, mais
personne n'en fut surpris.

Chacun partit pour se rendre aux différentes armées. Le duc de La
Feuillade passa par Metz pour aller à celle d'Allemagne, et s'y arrêta
chez l'évêque, frère de feu son père, qui était tombé en enfance et qui
était fort riche. Il jugea à propos de se nantir, et demanda la clef de
son cabinet et de ses coffres, et, sur le refus que les domestiques lui
en firent, il les enfonça bravement, et prit trente mille écus en or,
beaucoup de pierreries, et laissa l'argent blanc. Le roi d'ailleurs de
longue main fort mal content des débauches et de la négligence de La
Feuillade dans le service, s'expliqua fort durement et fort publiquement
de cet étrange avancement d'hoirie, et fut si près de le casser, que
Pontchartrain eut toutes les peines du monde à l'empêcher. Ce n'est pas
que La Feuillade ne vécut très mal avec Châteauneuf, secrétaire d'État
et avec sa fille qu'il avait épousée dès 1692\,; mais un coup de cet
éclat leur parut à tous mériter tous les efforts de leur crédit pour le
parer.

J'avais vu le maréchal de Choiseul avant partir, chez lui et chez moi,
et j'en avais reçu toutes sortes d'offres et de civilités. Il était
assez de la connaissance de mon père, et comme il était plein d'honneur
et de sentiments, il se piqua de faire merveilles à tout ce qui dans son
armée tenait à M. le maréchal de Lorges. Je trouvai à Philippsbourg
Villiers, mestre de camp de cavalerie, qui y était venu avec un assez
gros détachement, et qui s'en retournait le lendemain à l'armée,
laquelle venait, d'entrée de campagne, de passer le Rhin. En traversant
les bois de Bruchsall, nous trouvâmes les débris de l'escorte qui avait
conduit Montgon la veille, et qui avait été bien battue, assez de gens
tués et pris\,; et Montgon gagna le camp seul et de vitesse comme il
put. J'avais fait tout ce que j'avais pu pour le joindre en arrivant un
jour plus tôt à Philippsbourg, et je ne me repentis pas de n'avoir pu y
réussir. J'allai mettre pied à terre chez le maréchal de Choiseul. Il me
pressa extrêmement de loger au quartier général, mais je le suppliai de
me permettre de camper à la queue de mon régiment, et je l'obtins avec
peine. Il demanda au marquis d'Huxelles comment M. le maréchal de Lorges
en usait avec moi et avec ses neveux, pour que nous ne nous aperçussions
de la différence que le moins qu'il lui serait possible, et en effet, il
ne se lassa point de nous prévenir en tout, tant que la campagne dura,
et de nous combler d'attentions et de toutes les distinctions qu'il put.
De juin, qui commençait, jusqu'en septembre, le maréchal et le prince
Louis de Bade la plupart du temps dans ses lignes d'Eppingen, ne firent
que s'observer et subsister, après quoi nous repassâmes le Rhin à
Philippsbourg, où l'arrière-garde fut tâtée plutôt qu'inquiétée sans le
plus léger inconvénient. La campagne mérita depuis plus d'attention. Je
me servirai de ce loisir jusqu'en septembre, pour faire des courses
ailleurs.

La Flandre ne fournit rien du tout cette année\,; il ne fut question de
part et d'autre que de subsistances et que de s'épier. Le prince
d'Orange laissa de fort bonne heure l'armée à l'électeur de Bavière,
avec lequel il ne se passa rien non plus. Pendant la campagne, le
bonhomme du Montal mourut à Dunkerque. Il avait un corps séparé vers la
mer. C'était un très galant homme, et qui se montra tel jusqu'au bout, à
plus de quatre-vingts ans. Il vaqua par sa mort le gouvernement de
Mont-Royal et un collier de l'ordre, et le public et les troupes qui lui
rendirent justice trouvèrent honteux qu'il n'eût pas été fait maréchal
de France. J'ai parlé de lui lorsqu'on les fit. Le marquis de Noailles,
qui servait en Flandre, y mourut de la petite vérole, et ne laissa que
deux filles. Le duc son frère eut pour un de ses fils enfant la
lieutenance générale d'Auvergne, qu'il avait.

Il ne faut pas omettre la mort de deux hommes célèbres en genre fort
différent, qui arriva en ce même temps\,: de Varillas, si connu par les
histoires qu'il a écrites ou traduites, et du Plessis, écuyer de la
grande écurie, et le premier homme de cheval de son siècle, quoique déjà
fort vieux.

Une autre mort fit plus de bruit dans le monde, et y eut de grandes
suites. C'est celle du fameux roi de Pologne Jean Sobieski, qui arriva
subitement. Ce grand homme est si connu que je ne m'y étendrai pas.

En Catalogne, M. de Vendôme battit la cavalerie d'Espagne\,; elle était
de quatre mille hommes, à la tête desquels était le prince de Darmstadt.
Ils en ont eu le quart tué ou pris, et le comte de Tilly, commissaire
général, neveu de Serelves, est des derniers\,; et il n'en a coûté
qu'une centaine de carabiniers et autant de dragons. Longueval,
lieutenant général, fut reconnaître, après l'action, leur infanterie qui
était dans un camp retranché, et fut emporté d'un coup de canon.

L'Italie fut plus fertile. Le roi, résolu de ne rien oublier pour donner
la paix à son royaume, qui en avait un grand besoin, jugea bien qu'il
n'y parviendrait qu'en détachant quelqu'un des alliés contre lui, dont
l'exemple affaiblirait les autres, et lui donnerait plus de moyens de
leur résister et de les amener à son but, et il pensa au duc de Savoie
comme à celui dont les difficiles accès lui causaient plus de peine et
de dépenses, et qui d'ailleurs se trouvait fort molesté par les hauteurs
de l'empereur, et très mal content de l'Espagne, qui lui tenaient tous
très peu de tout ce qu'ils lui avaient promis et de ce qu'ils lui
promettaient sans cesse. Le roi donc, pour parvenir à réussir dans son
dessein, donna au maréchal Catinat une armée formidable et en même temps
des instructions secrètes fort amples, avec des pleins pouvoirs pour
négocier et, s'il se pouvait, conclure avec M. de Savoie.

Catinat passa les monts de bonne heure, et, gardant une exacte
discipline, menaçait de dévaster tout, et de couper sans miséricorde
tous les mûriers de la plaine, qui faisaient le plus riche commerce du
pays, par l'abondance des soies, et dont la perte l'eût ruiné pour un
siècle, avant de pouvoir être remis. M. de Savoie avait vu brûler ses
plus belles maisons de campagne les années précédentes, et les lieux de
plaisance qu'il avait le plus ornés\,: il avait éprouvé ce que peut une
armée supérieure que rien n'arrête\,: il voulait la paix, et Catinat
crut voir distinctement que c'était tout de bon. Le maréchal avait
contribué à se faire associer le comte de Tessé pour la négociation\,:
il fallait un homme intelligent et de poids, qui, s'il était nécessaire,
pût parler et répondre, ce que le maréchal n'était pas en situation de
faire à la tête d'une armée qui avait les yeux sur lui, et dont il n'y
avait pas moyen qu'il disparût un moment. C'est ce que put Tessé en
faisant le malade, comme il en usa plusieurs fois, et tant, qu'enfin les
temps où on ne le voyait point joints à l'inaction des troupes, on s'en
aperçut dans l'armée, où il était le plus ancien des lieutenants
généraux et chevalier de l'ordre en 1688.

C'était un homme fort bien et fort noblement fait, d'un visage agréable,
doux, poli, obligeant, d'un esprit raconteur et quelquefois point mal,
au-dessous du médiocre, si on en excepte le génie courtisan et tous les
replis qui servent à la fortune, pour laquelle il sacrifia tout. Il
s'était fait un protecteur déclaré de M. de Louvois par ses bassesses,
son dévouement et son attention à lui rendre compte de tout, ce qui ne
servit pas à sa réputation, mais à un avancement rapide, et à en donner
bonne opinion au roi. Son nom est Froulay\,; il était Manceau, et ne
démentait en rien sa patrie. D'une charge caponne de général des
carabins qui n'existaient plus, il s'en fit une réelle de mestre de camp
général des dragons, qui le porta à celle de leur `colonel général,
quand M. de Boufflers la quitta pour le régiment des gardes\,; et on
regarda avec raison comme une signalée faveur, qu'à son âge et n'étant
que maréchal de camp, il fût fait chevalier de l'ordre. Il sut se
maintenir avec Barbezieux comme il avait été auprès de son père, et tant
qu'il pouvait, dans son éloignement de la cour, il ne négligea de
cultiver aucun homme dont il pût espérer près ou loin. Il avait aussi le
riche gouvernement d'Ypres, et quantité de subsistances\,; son lien
d'ailleurs était fort court, et sa femme, qu'il tint toujours au Maine,
ne lui servit de rien, n'étant pas propre à en sortir. Il était cousin
germain de M. de Lavardin, chevalier de l'ordre en même promotion
pendant son ambassade de Rome, par sa mère, petite-fille du maréchal de
Lavardin. Sa femme s'appelait Auber, fille d'un baron d'Aunay du même
pays du Maine. Par sa mère Beaumanoir, il devint héritier de beaucoup de
choses de cette illustre maison.

Pendant la négociation, Catinat se préparait au siège de Turin, et M. de
Savoie qui voyait ses États dans ce danger, et qui d'ailleurs s'y sen
toit moins le maître que ses propres alliés, convint enfin de la plus
avantageuse paix pour lui, et que le roi trouva telle aussi pour
soi-même par le démembrement qu'elle rait parmi ses alliés. Les
principaux articles furent\,: le mariage de Mgr le duc de Bourgogne avec
sa fille aînée, dès qu'elle aurait douze ans, et en attendant envoyée à
la cour de France\,; que le comté de Nice serait sa dot, qui lui
demeurerait et lui serait livré jusqu'à la célébration du mariage\,; la
restitution de tout ce qui lui avait été pris, et même de Pignerol rasé
et deux ducs et pairs en otage à sa cour jusqu'à leur accomplissement\,;
enfin une grande somme d'argent en dédommagement de ses pertes, et
d'autres moindres articles, entre lesquels il obtint pour ses
ambassadeurs en France le traitement entier de ceux des rois, dont
jusqu'alors ils n'avaient qu'une partie, et les offices du roi à Rome
pour leur faire obtenir la salle royale qui est la même chose\,; toutes
les autres cours lui avaient déjà accordé les mêmes honneurs. Il voulut
aussi être un des médiateurs de la paix générale lorsqu'elle se
traiterait. Le roi l'accorda, mais l'empereur n'y voulut jamais
consentir quand il fut question de la faire.

Tout cela signé avec le dernier secret, il songea à se délivrer de ses
alliés qui l'obsédaient, qui le soupçonnaient, qui étaient plus forts
que lui, et qui, selon toute apparence, allaient devenir ses ennemis.
Pour y parvenir, il fit semblant de prêter l'oreille aux nouvelles
propositions qu'ils lui firent, et au renouement de celles de mariage de
sa fille aînée avec le roi des Romains, dont le refus qu'en avait fait
l'empereur l'avait sensiblement piqué\,; en même temps il proposa une
revue des troupes étrangères, à distance éloignée de Turin, où il mit
ses troupes dans les postes qu'elles occupaient. Il avait eu, sous
d'autres prétextes, la même précaution pour Coni et pour ses autres
places, et quand il fallut aller à la revue, il demeura à Turin et s'en
excusa. Après ces précautions, il se déclara. Il leur manda qu'il était
contraint d'accepter la neutralité d'Italie que le roi lui faisait
offrir, et qu'il les priait aussi de l'accepter de même. Le marquis de
Leganez, le prince Eugène et milord Galloway avaient ordre de lui obéir,
et n'osèrent se porter à une violence ouverte, ils se continrent et
attendirent de nouveaux ordres. En même temps M. de Savoie masqua sa
paix d'une trêve de trente jours avec le maréchal Catinat, à qui il
envoya le comte Jana, chevalier de l'Annonciade, et le marquis d'Aix,
pour otages, et reçut en même temps le comte de Tessé et Bouzols en la
même qualité. Ces choses se passèrent les premiers jours de juillet, et
ensuite la trêve fut prolongée.

Cependant le célèbre Jean Bart brûla cinquante-cinq vaisseaux marchands
aux Hollandais, parce qu'il ne put les amener, après avoir battu leur
convoi, et leur coûta une perte de six ou sept millions. Notre île de Ré
fut un peu bombardée\,; ils allèrent après devant Belle-Ile, et se
retirèrent sans rien faire.

\hypertarget{chapitre-xxii.}{%
\chapter{CHAPITRE XXII.}\label{chapitre-xxii.}}

1696

~

\relsize{-1}

{\textsc{Filles d'honneur de la princesse de Conti mangent avec le
roi.}} {\textsc{- Elle conserve sa signature, que les deux autres filles
du roi changent.}} {\textsc{- Mort de Croissy, ministre et secrétaire
des affaires étrangères.}} {\textsc{- Torcy épouse la fille de Pomponne
et fait sous lui la charge de son père.}} {\textsc{- Mort de
M\textsuperscript{me} de Bouteville\,; du marquis de Chandenier\,; sa
disgrâce.}} {\textsc{- Fortune de M. de Noailles.}} {\textsc{- Anthrax
du roi au cou.}} {\textsc{- Ducs de Foix et de Choiseul otages à
Turin.}} {\textsc{- Maison de la future duchesse de Bourgogne.}}
{\textsc{- Duchesse du Lude dame d'honneur.}} {\textsc{- Comtesse de
Mailly dame d'atours.}} {\textsc{- La comtesse de Blansac chassée.}}
{\textsc{- Duchesse d'Arpajon.}} {\textsc{- Comtesse de Roucy, sa
fille.}} {\textsc{- M. de Rochefort menin de Monseigneur.}} {\textsc{-
Dangeau chevalier d'honneur.}} {\textsc{- M\textsuperscript{me} de
Dangeau dame du palais.}} {\textsc{- M\textsuperscript{me} de Roucy dame
du palais.}} {\textsc{- Comte de Roucy.}} {\textsc{-
M\textsuperscript{me} de Nogaret dame du palais.}} {\textsc{- D'O, et
M\textsuperscript{me} D'O dame du palais.}} {\textsc{- Différence des
principaux domestiques des petits-fils de France et de ceux des princes
du sang.}} {\textsc{- Avantages nouveaux de ceux des bâtards sur ceux
des princes du sang.}} {\textsc{- Marquise du Châtelet dame du palais.}}
{\textsc{- M\textsuperscript{me} de Montgon dame du palais.}} {\textsc{-
M\textsuperscript{me} d'Heudicourt.}} \relsize{1}

~

Les princesses firent deux nouveautés\,: le roi à Trianon mangeait avec
les dames, et donnait assez souvent aux princesses l'agrément d'en
nommer deux chacune\,; il leur avait donné l'étrange distinction de
faire manger leurs dames d'honneur\,; ce qui continua toujours d'être
refusé à celles des princesses du sang, c'est-à-dire de
M\textsuperscript{me} la Princesse, et de M\textsuperscript{me} la
princesse de Conti, sa fille. À Trianon, M\textsuperscript{me} la
princesse de Conti, fille du roi, lui fit trouver bon qu'elle nommât ses
deux filles d'honneur pour manger, et elles furent admises\,: elle était
la seule qui en eût. L'autre nouveauté fut dans leurs signatures. Toutes
trois ajoutaient à leur nom \emph{légitimée de France}.
M\textsuperscript{me} la duchesse de Chartres et M\textsuperscript{me}
la Duchesse supprimèrent cette addition, et par là signèrent en plein
comme les princesses du sang légitimes. Cet appât ne tenta point
M\textsuperscript{me} la princesse de Conti. Elle ne perdait point
d'occasion de faire sentir aux deux autres princesses qu'elle avait une
mère connue et nommée, et qu'elles n'en avaient point\,; elle crut que
cette addition la distinguait en cela d'autant plus que les deux autres
la supprimaient, et elle voulut la conserver.

M. de Croissy, ministre et secrétaire d'État des affaires étrangères, et
frère de feu M. Colbert, mourut à Versailles le 28 juillet. C'était un
homme d'un esprit sage, mais médiocre, qu'il réparait par beaucoup
d'application et de sens, et qu'il gâtait par l'humeur et la brutalité
naturelles de sa famille. Il avait été longtemps président à mortier,
dont il avait peu exercé la charge, et avait été ambassadeur à la paix
d'Aix-la-Chapelle et en Angleterre. Enfin, il eut la place de M. de
Pomponne à sa disgrâce, et la survivance de cette place pour M. de
Torcy, son fils, qui avait celle de président à mortier.

Lorsque le roi, enfin indigné de l'abus continuel que le premier
président de Novion faisait de sa place et de la justice, voulut
absolument qu'il se retirât, et fit vendre à son petit-fils de Novion la
charge de président à mortier de MM. de Croissy et Torcy, M. de
Pomponne, qui avait également porté sa faveur et sa disgrâce, et à qui
on n'avait pu ôter l'estime du roi, en avait été mandé à Pomponne, le
jour même de la mort de M. de Louvois, et rentra au conseil en qualité
de ministre d'État sans charge, et eut la piété et la modestie de voir
M. de Croissy sans rancune et sans éloignement. Les histoires de tout
cela, qui sont très curieuses, ne sont pas matière de ces Mémoires. Ce
peu suffit pour entendre ce qui va suivre.

Le roi, qui s'était rattaché à M. de Pomponne, et qui à la retraite de
M. Pelletier, ministre d'État, lui donna la commission de la
surintendance, et par conséquent le secret de la poste, avait imaginé le
mariage de sa fille avec Torcy, pour réunir ces deux familles, et pour
donner un bon maître à ce jeune survivancier des affaires étrangères,
dans la décadence de santé où Croissy, perdu de goutte, était tombé, et
qui était encore plus nécessaire si Croissy venait à manquer. Dès qu'il
fut mort, le roi s'en expliqua à Pomponne et à Torcy d'une manière à
trancher toute espèce de difficultés possibles, et il régla que ce
mariage se ferait sans délai\,; que Torcy conserverait la charge de son
père\,; qu'il ne serait point encore ministre, mais que, sous
l'inspection et la direction de Pomponne, il ferait toutes les
dépêches\,; que Pomponne les rapporterait au conseil, et dirait après à
Torcy les réponses qui y auraient été résolues pour les dresser en
conséquence\,; que les ambassadeurs iraient désormais chez Pomponne qui
leur donnerait audience en présence de Torcy\,; qu'enfin celui-ci aurait
la charge de grand trésorier de l'ordre, que son père avait eue à la
mort de M. de Seignelay\,; et à Versailles, le beau-père et le gendre
partagèrent le logement de la charge de secrétaire d'État des affaires
étrangères, pour être ensemble et travailler en commun plus facilement.
De part et d'autre beaucoup de vertu dans les mariés, mais peu de bien,
auquel le roi pourvut peu à peu par ses grâces, et d'abord par de gros
brevets de retenue. Le mariage se fit à Paris le 13 août suivant chez M.
de Pomponne, et ils vécurent tous dans une grande et estimable union.

En même temps moururent deux personnes fort âgées et depuis bien
longtemps hors du monde\,: M\textsuperscript{me} de Bouteville, mère du
maréchal de Luxembourg, à quatre-vingt-onze ans, qui avait passé toute
sa vie retirée à la campagne, d'où elle avait vu de loin la brillante
fortune de son fils et des siens, avec qui elle n'avait jamais eu grand
commerce, et le marquis de Chandenier, aîné de la maison de
Rochechouart, si célèbre par sa disgrâce et par la magnanimité dont il
la soutint plus de quarante ans jusqu'à sa mort. Il était premier
capitaine des gardes du corps et singulièrement considéré pour sa
valeur, son esprit et son extrême probité. Il perdit sa charge avec les
autres capitaines des gardes du corps, à l'affaire des Feuillants en
{[}1648{]}, qui n'est pas du sujet de ces Mémoires et qui se trouve dans
tous ceux de ces temps-là, et il fut le seul des quatre à qui elle ne
fut point rendue, quoiqu'il ne se fût distingué en rien d'avec eux. Un
homme haut, plein d'honneur, d'esprit et de courage, et d'une grande
naissance avec cela, était un homme importun au cardinal Mazarin,
quoiqu'il ne l'eût jamais trouvé en la moindre faute ni ardent à
demander. Le cardinal tint à grand honneur de faire son capitaine des
gardes premier capitaine des gardes du corps, et il ne manqua pas cette
occasion d'y placer un domestique aussi affidé que lui était M. de
Noailles. M. de Chandenier refusa sa démission\,; le cardinal fit
consigner le prix qu'il avait réglé de la charge chez un notaire, puis
prêter serment à Noailles qui, sans démission de Chandenier, fût
pleinement pourvu et en fonction. Chandenier était pauvre\,: on espéra
que la nécessité vaincrait l'opiniâtreté. Elle lassa enfin la cour, qui
envoya Chandenier prisonnier au château de Loches, au pain du roi comme
un criminel, et arrêta tout son petit revenu pour le forcer à recevoir
l'argent de M. de Noailles et par conséquent à lui donner sa démission.
Elle se trompa\,; M. de Chandenier vécut du pain du roi et de ce que, à
tour de rôle, les bourgeois de Loches lui envoyaient à dîner et à souper
dans une petite écuelle qui faisait le tour de la ville. Jamais il ne se
plaignit, jamais il ne demanda ni son bien ni sa liberté\,; près de deux
ans se passèrent ainsi. À la fin, la cour honteuse d'une violence
tellement sans exemple et si peu méritée, plus encore d'être vaincue par
ce courage qui ne se pouvait dompter, relâcha ses revenus et changea sa
prison en exil, où il a été bien des années, et toujours sans daigner
rien demander. Il en arriva comme de sa prison, la honte fit révoquer
l'exil.

Il revint à Paris, où il ne voulut voir que peu d'amis. Il l'était fort
de mon père, qui m'a mené le voir et qui lui donnait assez souvent à
dîner. Il le menait même quelquefois à la Ferté, et ce fut lui qui fit
percer une étoile régulière à mon père qui voulait bâtir, et qui en tira
son bois, et c'est une grande beauté fort près de la maison, au lieu que
mon père ne songeait qu'à abattre, sans considérer où ni comment. Depuis
sa mort, j'ai vu plusieurs fois M. de Chandenier avec un vrai respect à
Sainte-Geneviève, dans la plus simple mais la plus jolie retraite qu'il
s'y était faite et où il mourut. C'était un homme de beaucoup de goût et
d'excellente compagnie, et qui avait beaucoup vu et lu\,; il fut
longtemps avant sa mort dans une grande piété. On s'en servit dans la
dernière année de sa vie pour lui faire un juste scrupule sur ses
créanciers qu'il ne tenait qu'à lui de payer de l'argent de M. de
Noailles en donnant sa démission, et quand on l'eut enfin vaincu sur cet
article avec une extrême peine, les mêmes gens de bien entreprirent de
lui faire voir M. de Noailles qui avait sa charge après son père.
L'effort de la religion le soumit encore à recevoir cette visite qui de
sa part se passa froidement, mais honnêtement\,; il avait perdu sa femme
et son fils depuis un grand nombre d'années, qui était un jeune homme, à
ce que j'ai ouï dire, d'une grande espérance.

Le roi eut un anthrax au cou qui ne parut d'abord qu'un clou et qui
bientôt après donna beaucoup d'inquiétudes. Il eut la fièvre et il
fallut en venir à plusieurs incisions par reprises. Il affecta de se
laisser voir tous les jours et de travailler dans son lit presqu'à son
ordinaire. Toute l'Europe ne laissa pas d'être fort attentive à un mal
qui ne fût pas sans danger\,: il dépêcha un courrier au duc de La
Rochefoucauld en Angoumois, où il était allé passer un mois dans sa
belle maison de Verteuil, et lui manda sa maladie et son désir de le
revoir, avec beaucoup d'amitié. Il partit aussitôt, et sa faveur parut
plus que jamais. Comme il ne se passait rien en Flandre, et qu'il n'y
avait plus lieu de s'y attendre à rien, le roi manda aux maréchaux de
Villeroy et de Boufflers de renvoyer les princes dès que le prince
d'Orange aurait quitté l'armée, ce qui arriva peu de jours après.

Ce fut pendant le cours de cette maladie que la paix de Savoie devint
publique et que le roi régla tout ce qui regardait la princesse de
Savoie et les deux otages jusqu'aux restitutions accomplies. M. de
Savoie, qui n'ignorait rien jusque des moindres choses des principales
cours de l'Europe, compta que les ducs de Foix et de Choiseul ne
l'embarrasseraient pas. Le premier n'avait jamais songé qu'à son plaisir
et à se divertir en bonne compagnie\,; l'autre était accablé sous le
poids de sa pauvreté et de sa mauvaise fortune, tous deux d'un esprit
au-dessous du médiocre, et parfaitement ignorants de ce qui leur était
dû, très aisés à mener, à contenter, à amuser, tous deux sans rien qui
tint à la cour et sans considération particulière, tous deux enfin de la
plus haute naissance et tous deux chevaliers de l'ordre. C'était
précisément tout l'assemblage que M. de Savoie cherchait. Il voyait
qu'on voulait ici lui plaire dans cette crise d'alliance\,; il fit
proposer au roi ces deux ducs, et le roi les nomma et leur donna à
chacun douze mille livres pour leur équipage et mille écus par mois. Le
comte de Brionne, chevalier de l'ordre et grand écuyer, en survivance de
son père, fut nommé pour aller de la part du roi recevoir la princesse
au Pont Beauvoisin, et Desgranges, un des premiers commis de
Pontchartrain et maître des cérémonies, pour y aller aussi, et faire là
sa charge et pendant le voyage de la princesse.

Sa maison fut plus longtemps à être déterminée. La cour était depuis
longtemps sans reine et sans Dauphine. Toutes les dames d'une certaine
portée d'état ou de faveur s'empressèrent et briguèrent, et beaucoup aux
dépens les unes des autres\,; les lettres anonymes mouchèrent
\footnote{On a déjà vu ce mot plus haut (p. 185) pris ici dans le même
  sens.}, les délations, les faux rapports. Tout se passa uniquement
là-dessus entre le roi et M\textsuperscript{me} de Maintenon qui ne
bougeait du chevet de son lit pendant toute sa maladie, excepté
lorsqu'il se laissait voir et qui y était la plupart du temps seule.
Elle avait résolu d'être la véritable gouvernante de la princesse, de
l'élever à son gré et à son point, de se l'attacher en même temps assez
pour en pouvoir amuser le roi, sans crainte, qu'après le temps de poupée
passé, elle lui pût devenir dangereuse. Lille songeait encore à tenir
par elle Mgr le duc de Bourgogne un jour, et cette pensée l'occupait
d'autant plus que nous verrons bientôt que ses liaisons étaient déjà
bien refroidies avec les ducs et duchesses de Chevreuse et de
Beauvilliers, auxquelles pour cette raison l'exclusion fut donnée de la
place de dame d'honneur que l'une ou l'autre auraient si dignement et si
utilement remplie, M\textsuperscript{me} de Maintenon chercha donc, pour
environner la princesse, des personnes ou entièrement et sûrement à
elle, ou dont l'esprit fût assez court pour n'avoir rien à
appréhender\,; ainsi le dimanche, 2 septembre, la maison fut nommée et
déclarée\,:

Dangeau, chevalier d'honneur\,;

La duchesse du Lude, dame d'honneur\,;

La comtesse de Mailly, dame d'atours\,;

Tessé, premier écuyer.

DAMES DU PALAIS EN CET ORDRE

M\textsuperscript{me} de Dangeau\,;

La comtesse de Roucy\,;

M\textsuperscript{me} de Nogaret\,;

M\textsuperscript{me} d'O\,;

La marquise du Châtelet\,;

M\textsuperscript{me} de Montgon\,;

Et pour première femme de chambre, M\textsuperscript{me} Camoin\,;

Peu après, le P. Lecomte, jésuite, pour confesseur, et dans la suite,

L'évêque de Meaux, premier aumônier ci-devant de M\textsuperscript{me}
la Dauphine, et auparavant précepteur de Monseigneur\,;

Et Villacerf acheta du roi la charge de premier maître d'hôtel.

Il faut voir maintenant ce qu'on sut des raisons de chacun de ces choix
et de celui de M\textsuperscript{me} de Castries pour dame d'atours de
M\textsuperscript{me} la duchesse de Chartres, au lieu de la comtesse de
Mailly, qui se trouvera en son temps.

Pour celui du comte de Tessé, les raisons en sont visibles et j'ai
suffisamment parlé de sa personne.

J'en dis autant de celui de la comtesse de Mailly\,;

Et pour le P. Lecomte, ce fut une affaire intérieure de jésuites, dont
le P. de La Chaise fut le maître.

La duchesse du Lude était sœur du duc de Sully, qui fut chevalier de
l'ordre en 1688, fille de la duchesse de Verneuil et petite-fille du
chancelier Séguier. Elle avait épousé en premières noces ce galant comte
de Guiche, fils aîné du maréchal de Grammont qui a fait en son temps
tant de bruit dans le monde, et qui fit fort peu de cas d'elle et n'en
eut point d'enfants. Elle était encore fort belle et toujours sage, sans
aucun esprit que celui que donne l'usage du grand monde et le désir de
plaire à tout le monde, d'avoir des amis, des places, de la
considération, et avoir été dame du palais de la reine\,: elle eut de
tout cela, parce que c'était la meilleure femme du monde, riche, et qui,
dans tous les temps de sa vie, tint une bonne table et une bonne maison
partout, et basse et rampante sous la moindre faveur, et faveurs de
toutes les sortes. Elle se remaria au duc du Lude par inclination
réciproque, qui était grand maître de l'artillerie, extrêmement bien
avec le roi, et d'ailleurs fort à la mode et qui tenait un grand état.
Ils vécurent très bien ensemble, et elle le perdit sans en avoir eu
d'enfants. Elle demeura toujours attachée à la cour, où sa bonne maison,
sa politesse et sa bonté lui acquirent beaucoup d'amis, et où, sans
aucun besoin, elle faisait par nature sa cour aux ministres, et tout ce
qui était en crédit, jusqu'aux valets. Le roi n'avait aucun goût pour
elle, ni M\textsuperscript{me} de Maintenon\,; elle n'était presque
jamais des Marlys et ne participait à aucune des distinctions que le roi
donnait souvent à un petit nombre de dames. Telle était sa situation à
la cour lorsqu'il fut question d'une dame d'honneur, sur qui roulât
toute la confiance de l'éducation et de la conduite de la princesse que
M\textsuperscript{me} de Maintenon avait résolu de tenir immédiatement
sous sa main pour en faire l'amusement intérieur du roi.

Le samedi matin, veille de la déclaration de la maison, le roi, qui
gardait le lit pour son anthrax, causait, entre midi et une heure, avec
Monsieur qui était seul avec lui. Monsieur, toujours curieux, tâchait de
faire parler le roi sur le choix d'une dame d'honneur que tout le monde
voyait qui ne pouvait plus être différé\,; et comme ils en parlaient,
Monsieur vit à travers la chambre, par la fenêtre, la duchesse du Lude
dans sa chaise avec sa livrée qui traversait le bas de la grande cour,
qui revenait de la messe\,: « En voilà une qui passe, dit-il au roi, qui
en a bonne envie, et qui n'en donne pas sa part, \,» et lui nomme la
duchesse du Lude. « Bon, dit le roi, voilà le meilleur choix du monde
pour apprendre à la princesse à bien mettre du rouge et des mouches, \,»
et ajouta des propos d'aigreur et d'éloignement. C'est qu'il était alors
plus dévot qu'il ne l'a été depuis, et que ces choses le choquaient
davantage. Monsieur, qui ne se souciait point de la duchesse du Lude, et
qui n'en avait parlé que par ce hasard et par curiosité, laissa dire le
roi et s'en alla dîner, bien persuadé que la duchesse du Lude était hors
de toute portée, et n'en dit mot. Le lendemain presque à pareille heure,
Monsieur était seul dans son cabinet\,; il vit entrer l'huissier qui
toit en dehors, et qui lui dix que la duchesse du Lude était nommée.
Monsieur se mit à rire, et répondit qu'il lui en contait de belles\,;
l'autre insiste, croyant que Monsieur se moquait de lui, sortit et ferma
la porte. Peu de moments après entre M. de Châtillon, le chevalier de
l'ordre, avec la même nouvelle, et Monsieur encore à s'en moquer.
Châtillon lui demande pourquoi il n'en veut rien croire, en louant le
choix et protestant qu'il n'y a rien de si vrai. Comme ils en étaient
sur cette dispute, vinrent d'autres gens qui le confirmèrent, de façon
qu'il n'y eut pas moyen d'en douter. Alors Monsieur parut dans une telle
surprise, qu'elle étonna la compagnie qui le pressa d'en dire la raison.
Le secret n'était pas le fort de Monsieur\,; il leur conta ce que le roi
lui avait dit vingt-quatre heures auparavant, et à son tour les combla
de surprise. L'aventure se sut et donna tant de curiosité, qu'on apprit
enfin la cause d'un changement si subit.

La duchesse du Lude n'ignorait pas qu'outre le nombre des prétendantes,
il y en avait une entre autres sur qui elle ne pouvait espérer la
préférence\,; elle eut recours à un souterrain. M\textsuperscript{me} de
Maintenon avait conservé auprès d'elle une vieille servante qui, du
temps de sa misère et qu'elle était veuve de Scarron, à la charité de sa
paroisse de Saint-Eustache, était son unique domestique\,; et cette
servante, qu'elle appelait encore fanon comme autrefois, était pour les
autres M\textsuperscript{lle} Balbien, et fort considérée par l'amitié
et la confiance de M\textsuperscript{me} de Maintenon pour elle. Nanon
se rendait aussi rare que sa maîtresse, se coiffait et s'habillait comme
elle, imitait son précieux, son langage, sa dévotion, ses manières.
C'était une demi-fée à qui les princesses se trouvaient heureuses quand
elles avaient occasion de parler et de l'embrasser, toutes filles du roi
qu'elles fussent, et à qui les ministres qui travaillaient chez
M\textsuperscript{me} de Maintenon faisaient la révérence bien bas. Tout
inaccessible qu'elle fût, il lui restait pourtant quelques anciennes
amies de l'ancien temps, avec qui elle s'humanisait, quoique rarement\,;
et heureusement pour la duchesse du Lude, elle avait une vieille mie qui
l'avait élevée, qu'elle avait toujours gardée, et qui l'aimait
passionnément, qui était de l'ancienne connaissance de Nanon, et qu'elle
voyait quelquefois en privance. La duchesse du Lude la lui détacha, et
finalement vingt mille écus comptant firent son affaire, le soir même du
samedi que le roi a voit parlé à Monsieur le matin avec tant
d'éloignement pour elle\,; et voilà les cours\,! Une Nanon qui en vend
les plus importants et les plus brillants emplois, et une femme riche,
duchesse, de grande naissance par soi et par ses maris, sans enfants,
sans liens, sans affaires, libre, indépendante, a la folie d'acheter
chèrement sa servitude\,! Sa joie fut extrême, mais elle sut la
contenir, et sa façon de vivre et le nombre d'amis et de connaissances
particulières qu'elle avait su toute sa vie se faire et s'entretenir à
la ville et à la cour entraînèrent le gros du monde à l'applaudissement
de ce choix.

La duchesse d'Arpajon et la maréchale de Rochefort furent outrées\,;
celle-ci fit les hauts cris, et se plaignit sans nul ménagement qu'on
manquait à la parole qu'on lui avait donnée, sur laquelle seule elle
avait consenti à être dame d'honneur de M\textsuperscript{me} la
duchesse de Chartres. Elle confondait adroitement les deux places de
dame d'honneur et de dame d'atours pour se relever et crier plus fort.
C'était la dernière qu'elle avait chez M\textsuperscript{me} la
Dauphine, et qui lui avait été promise. M\textsuperscript{me} de
Maintenon, qui la méprisait, en fut piquée, parce qu'elle l'avait fait
donner à M\textsuperscript{me} de Mailly. Elle prit le tour d'accuser la
maréchale d'être elle-même cause de ce dégoût qu'on ne lui voulait pas
donner, par avoir tellement soutenu sa fille, que par considération pour
elle on ne l'avait pas chassée. La maréchale en fut la dupe, et bien
qu'en conservant tout son dépit et que la place fût donnée, elle
abandonna sa fille, de rage, qui fut renvoyée à Paris avec défense de
paraître à la cour. Cette fille était mère de Nangis en premières noces,
qui avait plus que mal vécu avec ce premier mari, et qui ruina son fils
sans paraître, qui était très riche, qui devint grosse de Blansac qu'on
fit revenir de l'armée pour l'épouser, et elle accoucha de
M\textsuperscript{me} de Tonnerre la nuit même qu'elle fut mariée.

On ne pouvait avoir plus d'esprit, plus d'intrigue, plus de douceur,
d'insinuation, de tour et de grâce dans l'esprit, une plaisanterie plus
fine et plus salée, ni être plus maîtresse de son langage pour le
mesurer à ceux avec qui elle était. C'était en même temps de tous les
esprits le plus méchant, le plus noir, le plus dangereux, le plus
artificieux, d'une fausseté parfaite, à qui les histoires entières
coulaient de source avec un air de vérité, de simplicité qui était prêt
à persuader ceux même qui savaient, à n'en pouvoir douter, qu'il n'y
avait pas un mot de vrai\,; avec tout cela une sirène enchanteresse dont
on ne se pouvait défendre qu'en la fuyant, quoiqu'on la connût
parfaitement. Sa conversation était charmante, et personne n'assénait si
plaisamment ni si cruellement les ridicules, même où il n'y en avait
point, et comme n'y touchant pas\,; au demeurant plus que très galante
tant que sa figure lui avait fait trouver avec qui, fort commode
ensuite, et depuis se ruina pour les plus bas valets. Malgré de tels
vices, et dont la plupart étaient si destructifs de la société, c'était
la fleur des pois à la cour et à la ville\,; sa chambre ne désemplissait
pas de ce qui y était de plus brillant et de la meilleure compagnie ou
par crainte ou par enchantement, et avait en outre des amis et des amies
considérables\,; elle était fort recherchée des trois filles du roi.
C'était à qui l'aurait, mais la convenance de sa mère l'avait attachée à
M\textsuperscript{me} la duchesse de Chartres plus qu'aux autres. Elle
la gouvernait absolument. Les jalousies et les tracasseries qui en
naquirent l'éloignèrent de Monsieur et de M. le duc de Chartres jusqu'à
l'aversion\,: elle en fut chassée. À force de temps, de pleurs et de
souplesses de M\textsuperscript{me} la duchesse de Chartres, elle fut
rappelée. Elle retourna à Marly\,; elle fut admise à quelques parties
particulières avec, le roi. Elle le divertit avec tant d'esprit qu'il ne
parla d'autre chose à M\textsuperscript{me} de Maintenon\,; elle en eut
peur, et ne chercha plus qu'à l'éloigner du roi (elle le fit avec soin
et adresse), puis à la chasser de nouveau pour plus grande sûreté, et
elle saisit l'occasion d'en venir à bout. On se moqua bien de la mère,
d'y avoir consenti si inutilement pour la place qu'elle ne pouvait plus
avoir, et par une sotte et folle colère d'honneur et de duperie\,; mais
la fille demeura à Paris pour longtemps.

La duchesse d'Arpajon, mariée belle et jeune à un vieillard qui ne
sortait plus de Rouergue et de son château de Séverac, s'était vue noyée
d'affaires et de procès, depuis qu'elle fut veuve, au parlement de
Toulouse, pour ses reprises et pour sa fille unique, dont des incidents
importants l'amenèrent à Paris pour y plaider au conseil. C'était une
personne d'une grande vertu, d'une excellente conduite, qui avait grande
mine et des restes de beauté. On ne l'avait presque jamais vue à la cour
ni à Paris, et on l'y appelait la duchesse des bruyères. Elle ne l'était
qu'à brevet. M\textsuperscript{me} de Richelieu mourut fort tôt après
son arrivée, et la surprise fut extrême de voir la duchesse d'Arpajon
tout à coup nommée dame d'honneur de M\textsuperscript{me} la Dauphine
en sa place. Elle-même le fut plus que personne\,; jamais elle n'y avait
pensé, ni M. de Beuvron son frère\,; ce fut pourtant lui qui la fit sans
le savoir. Il avait autrefois été plus que bien avec
M\textsuperscript{me} Scarron\,; celle-ci n'oublia point ses anciens
amis de ce genre, elle compta sur l'attachement de sa sœur par lui, par
reconnaissance et par se trouver parfaitement isolée au milieu de la
cour. On ne pouvait avoir moins d'esprit, mais ce qu'elle en avait était
fort sage, et elle avait beaucoup de sens, de conduite et de dignité\,;
et il est impossible de faire mieux sa charge qu'elle la fit, avec plus
de considération et plus au gré de tout le monde. Elle espéra donc être
choisie\,; elle le demanda\,; le monde le crut et le souhaita, mais les
vingt mille écus que M\textsuperscript{me} Barbisi, la vieille mie de la
duchesse du Lude, fit accepter à la vieille servante de
M\textsuperscript{me} de Maintenon, décidèrent contre
M\textsuperscript{me} d'Arpajon. Le roi voulut la consoler, et
M\textsuperscript{me} de Maintenon aussi, et firent la comtesse de
Roucy, sa fille, dame du palais. La mère ne prit point le change, elle
demeura outrée\,; le transport de joie de sa fille l'affligea encore
plus, et leur séparation entière qu'elle envisageait, l'accabla\,; elle
aimait fort sa fille, que cette place attachait en un lieu où la mère ne
pouvait plus paraître que fort rarement avec bienséance, et elle se
voyait tombée en solitude. Elle ne la put porter\,: peu de mois après
elle eut une apoplexie dont elle mourût quelque temps après.

Cette consolation prétendue donnée à M\textsuperscript{me} d'Arpajon, et
cette différence des deux belles-sœurs, la comtesse de Roucy, faite dame
du palais, et M\textsuperscript{me} de Blansac, chassée, combla la
douleur de la maréchale de Rochefort. Elle était cousine germaine de la
duchesse du Lude, filles des deux sœurs, et vivait fort avec elle, autre
crève-cœur. À peine la voulut elle voir, et ne reçut qu'avec aigreur
toutes ses avances. Enfin, après avoir longtemps gémi, elle fut apaisée
par une place nouvelle de menin de Monseigneur donnée au marquis de
Rochefort son fils, sans qu'elle l'eût demandée.

Dangeau était un gentilhomme de Beauce, tout uni, et huguenot dans sa
première jeunesse\,; toute sa famille l'était qui ne tenait à personne.
Il ne manquait pas d'un certain esprit, surtout de celui du monde, et de
conduite. Il avait beaucoup d'honneur et de probité. Le jeu, par lequel
il se fourra à la cour, qui était alors toute d'amour et de fêtes,
incontinent après la mort de la reine mère, le mit dans les meilleures
compagnies. Il y gagna tout son bien\,; il eut le bonheur de n'être
jamais soupçonné\,; il prêta obligeamment\,; il se fit des amis, et la
sûreté de son commerce lui en acquit d'utiles et de véritables. Il fit
sa cour aux maîtresses du roi\,; le jeu le mit de leurs parties avec
lui\,; elles le traitèrent avec familiarité, et lui procurèrent celle du
roi. Il faisait des vers, était bien fait, de bonne mine et galant\,; le
voilà de tout à la cour, mais toujours subalterne. Jouant un jour avec
le roi et M\textsuperscript{me} de Montespan dans les commencements des
grandes augmentations de Versailles, le roi, qui avait été importuné
d'un logement pour lui et qui avait bien d'autres gens qui en
demandaient, se mit à le plaisanter sur sa facilité à faire des vers,
qui, à la vérité, étaient rarement bons, et tout d'un coup lui proposa
des rimes fort sauvages, et lui promit un logement s'il les remplissait
sur-le-champ. Dangeau accepta, n'y pensa qu'un moment, les remplit
toutes, et eut ainsi un logement.

De là il acheta une chargé de lecteur du roi qui n'avait point de
fonctions, mais qui donnait les entrées du petit coucher, etc. Son
assiduité lui mérita le régiment du roi infanterie, qu'il ne garda pas
longtemps, puis fut envoyé en Angleterre où il demeura peu, et à son
retour acheta le gouvernement de Touraine. Son bonheur voulut que M. de
Richelieu fit de si grosses pertes au jeu qu'il en vendit sa charge de
chevalier d'honneur de M\textsuperscript{me} la Dauphine, au mariage de
laquelle il l'avait eue pour rien, et que son ancienne amie,
M\textsuperscript{me} de Maintenon, lui fit permettre de la vendre tant
qu'il pourrait et à qui il voudrait. Dangeau ne manqua pas une si bonne
affaire\,; il en donna cinq cent mille livres, et se revêtit d'une
charge qui faisait de lui une espèce de seigneur, et qui lui assura
l'ordre, qu'il eut bientôt après en 1688. Il perdit sa charge à la mort
de M\textsuperscript{me} la Dauphine, mais il avait eu une place de
menin de Monseigneur, et tenait ainsi partout.

M\textsuperscript{me} la Dauphine avait une fille d'honneur d'un
chapitre d'Allemagne, jolie comme le jour, et faite comme une nymphe,
avec toutes les grâces de l'esprit et du corps. L'esprit était fort
médiocre, mais fort juste, sage et sensée, et avec cela une vertu sans
soupçon. Elle était fille d'un comte de Lovestein et d'une sœur du
cardinal de Furstemberg qui a tant fait de bruit dans le monde, et qui
était dans la plus haute considération à la cour. Ces Lovestein étaient
de la maison palatine, mais d'une branche mésalliée par un mariage
qu'ils appellent de la main gauche, mais qui n'en est pas moins
légitime. L'inégalité de la mère fait que ce qui en sort n'hérite point,
mais a un gros partage, et tombe du rang de prince à celui de comte. Le
cardinal de Furstemberg, qui aimait fort cette nièce, cherchait à la
marier. Elle plaisait fort au roi et à M\textsuperscript{me} de
Maintenon qui se prenaient fort aux figures. Elle n'avait rien vaillant,
comme toutes les Allemandes. Dangeau, veuf depuis longtemps d'une sœur
de la maréchale d'Estrées, fille de Morin le Juif, et qui n'en avait
qu'une fille dont le grand bien qu'on lui croyait l'avait mariée au duc
de Montfort, se présenta pour une si brande alliance pour lui, et aussi
agréable. M\textsuperscript{lle} de Lovestein, avec la hauteur de son
pays, vit le tuf à travers tous les ornements qui le couvraient, et dit
qu'elle n'en voulait point. Le roi s'en mêla, M\textsuperscript{me} de
Maintenon, M\textsuperscript{me} la Dauphine\,; le cardinal son oncle le
voulut et la fit consentir. Le maréchal et la maréchale de Villeroy en
firent la noce, et Dangeau se crut électeur palatin.

C'était le meilleur homme du monde, mais à qui la tête avait tourné
d'être seigneur\,; cela l'avait chamarré de ridicules, et
M\textsuperscript{me} de Montespan avait fort plaisamment, mais très
véritablement dit de lui\,: qu'on ne pouvait s'empêcher de l'aimer ni de
s'en moquer. Ce fut bien pis après sa charge et ce mariage. Sa fadeur
naturelle, entée sur la bassesse du courtisan et recrépie de l'orgueil
du seigneur postiche, fit un composé que combla la grande maîtrise de
l'ordre de Saint-Lazare que le roi lui donna comme l'avait Nerestang,
mais dont il tira tout le parti qu'il put, et se lit le singe du roi,
dans les promotions qu'il fit de cet ordre où toute la cour accourait
pour rire avec scandale, tandis qu'il s'en croyait admiré. Il fut de
l'Académie française et conseiller d'État d'épée, et sa femme, la
première des dames du palais, comme femme du chevalier d'honneur, et,
n'y en ayant point de titrées. M\textsuperscript{me} de Maintenon
l'avait goûtée\,; sa naissance, sa vertu, sa figure, un mariage du goût
du roi et peu du sien, dans lequel elle vécut comme un ange, la
considération de son oncle et de la charge de son mari, tout cela la
porta, et ce choix fut approuvé de tout le monde.

La comtesse de Roucy, j'en ai rapporté la raison en parlant de la
duchesse d'Arpajon sa mère. C'était une personne extrêmement laide, qui
avait de l'esprit, fort glorieuse, pleine d'ambition, folle des moindres
distinctions, engouée à l'excès de la cour, basse à proportion de la
faveur et des besoins, qui cherchait à faire des affaires à toutes
mains, aigre à l'oreille jusqu'aux injures et fréquemment en querelle
avec quelqu'un, toujours occupée de ses affaires que son opiniâtreté,
son humeur et sa malhabileté perdaient, et qui vivait noyée de biens,
d'affaires et de créanciers, envieuse, haineuse, par conséquent peu
aimée, et qui, pour couronner tout cela, ne manquait point de
grand'messes à la paroisse et rarement à communier tous les huit jours.
Son mari n'avait qu'une belle mais forte figure\,; glorieux et bas plus
qu'elle, panier percé qui jouait tout et perdait tout, toujours en
course et à la chasse, dont la sottise lui avait tourné à mérite, parce
qu'il ne faisait jalousie à personne, et dont la familiarité avec les
valets le faisait aimer. Il avait aussi les dames pour lui, parce qu'il
était leur fait, et avec toute sa bêtise un entregent de cour que
l'usage du grand monde lui avait donné. Il était de tout avec
Monseigneur, et le roi le traitait bien à cause de M. de La
Rochefoucauld et des maréchaux de Duras et de Lorges, frères de sa mère,
qui tous trois avaient fait de lui et de ses frères comme de leurs
enfants, depuis que la révocation de l'édit de Nantes avait fait sortir
du royaume le comte et la comtesse de Roye ses père et mère. Son grand
mérite était ses inepties qu'on répétait et qui néanmoins se trouvaient
quelquefois exprimer quelque chose.

M\textsuperscript{me} de Nogaret, veuve d'un Cauvisson à qui le roi
l'avait mariée lorsqu'il cassa la chambre des filles de
M\textsuperscript{me} la Dauphine dont elle était, avec sa sieur
M\textsuperscript{me} d'Urfé, dame d'honneur de M\textsuperscript{me} la
princesse de Conti fille du roi, avait perdu son mari tué à Fleurus, qui
n'était connu que sous le nom de \emph{son impertinence}. Il avait assez
mal vécu avec elle et l'avait laissée pauvre et sans enfants. Elle était
sœur de Biron, et la maréchale de Villeroy et elle étaient enfants du
frère et de la sœur, et en grande liaison. C'était une femme de beaucoup
d'esprit, de finesse et de délicatesse, sous un air simple et naturel,
de la meilleure compagnie du monde, et qui, n'aimant rien, ne laissait
pas d'avoir des amis. Elle n'avait ni feu, ni lieu, ni autre être que la
cour, et presque point de subsistance. Laide, grosse, avec une
physionomie qui réparait tout, d'anciennes raisons de commodité
l'avaient fort bien mise avec Monseigneur qui aimait sa sœur et elle
particulièrement\,; et tout cela ensemble la fit dame du palais. Elle
n'était point méchante, et avait tout ce qu'il fallait pour l'être et
pour se faire fort craindre. Mais, avec un très bon esprit, elle aima
mieux se faire aimer.

M\textsuperscript{me} d'O était une autre espèce. Guilleragues, son
père, n'était rien qu'un Gascon, gourmand, plaisant, de beaucoup
d'esprit, d'excellente compagnie, qui avait des amis, et qui vivait à
leurs dépens parce qu'il avait tout fricassé, et encore était-ce à qui
l'aurait. Il avait été ami intime de M\textsuperscript{me} Scarron, qui
ne l'oublia pas dans sa fortune et qui lui procura l'ambassade de
Constantinople pour se remplumer\,; mais il y trouva, comme ailleurs,
moyen de tout manger. Il y mourut et ne laissa que cette fille unique
qui avait de la beauté. Villers, lieutenant de vaisseau et fort bien
fait, fut de ceux qui portèrent le successeur à Constantinople, et qui
en ramenèrent la veuve et la fille du prédécesseur. Avant partir de
Turquie et chemin faisant, Villers fit l'amour à M\textsuperscript{lle}
de Guilleragues et lui plut, et tant fut procédé, que sans biens de part
ni d'autre, la mère consentit à leur mariage. Les vaisseaux relâchèrent
quelques jours sur les bords de l'Asie Mineure, vers les ruines de
Troie. Le lieu était trop romanesque pour y résister\,; ils mirent pied
à terre et s'épousèrent. Arrivés avec les vaisseaux en Provence,
M\textsuperscript{me} de Guilleragues amena sa fille et son gendre à
Paris et à Versailles et les présenta à M\textsuperscript{me} de
Maintenon. Ses aventures lui donnèrent compassion des leurs.

Villers se prétendit bientôt de la maison d'O et en prit le nom et les
armes. Rien n'était si intrigant que le mari et la femme, ni rien aussi
de plus gueux. Ils firent si bien auprès de M\textsuperscript{me} de
Maintenon, que M. d'O fut mis auprès de M. le comte de Toulouse avec le
titre de gouverneur et d'administrateur de sa maison. Cela lui donna un
être, une grosse subsistance, un rapport continuel avec le roi, et des
privances et des entrées à toutes heures, qui n'avaient aucun usage par
devant, c'est-à-dire comme celles des premiers gentilshommes de la
chambre, mais qui étaient bien plus grandes et plus libres, pouvant
entrer par les derrières dans les cabinets du roi presqu'à toutes
heures, ce que n'avaient pas les premiers gentilshommes de la chambre,
ni pas une autre sorte d'entrée, outre qu'il suivait son pupille chez le
roi et y demeurait avec lui à toutes sortes d'heures et de temps, tant
qu'il y était. Sa femme fut logée avec lui dans l'appartement de M. le
comte de Toulouse, qui lui entretint soir et matin une table fort
considérable. Ils n'avaient pas négligé M\textsuperscript{me} de
Montespan, et l'eurent favorable pour cette place et tant qu'elle
demeura à la cour. Ils la, cultivèrent toujours depuis, parce que M. le
comte de Toulouse l'aimait fort.

D'O peu à peu avait changé de forme, et lui et sa femme tendaient à leur
fortune par des voies entièrement opposées, mais entre eux parfaitement
de concert. Le mari était un grand homme, froid, sans autre esprit que
du manège, et d'imposer aux sots par un silence dédaigneux, une mine et
une contenance grave et austère, tout le maintien important, dévot de
profession ouverte, assidu aux offices de la chapelle, où dans d'autres
temps on le voyait encore en prières, et de commerce qu'avec des gens en
faveur ou en place dont il espérait tirer parti, et qui, de leur côté,
le ménageaient à cause de ses accès. Il sut peu à peu gagner l'amitié de
son pupille, pour demeurer dans sa confiance quand il n'aurait plus la
ressource de son titre et de ses fonctions auprès de lui. Sa femme lui
aida fort en cela, et ils y réussirent si bien que, leur temps fini par
l'âge de M. le comte de Toulouse, ils demeurèrent tous deux chez lui
comme ils y avaient été, avec toute sa confiance et l'autorité entière
sur toute administration chez lui. M\textsuperscript{me} d'O vivait
d'une autre sorte. Elle avait beaucoup d'esprit, plaisante,
complaisante, toute à tous et amusante\,; son esprit était tout tourné
au romanesque et à la galanterie, tant pour elle que pour autrui. Sa
table rassemblait du monde chez elle, et cette humeur y était commode à
beaucoup de gens, mais avec choix et dont elle pouvait faire usage pour
sa fortune et dans les premiers temps où M. le comte de Toulouse
commença à être hors de page et à se sentir, elle lui plut fort par ses
facilités. Elle devint ainsi amie intime de vieilles et de jeunes par
des intrigues et par des vues de différentes espèces, et comme elle
faisait mieux ses affaires de chez elle que de dehors, elle sortait peu,
et toujours avec des vues. Cet alliage de dévotion et de retraite d'une
part, de tout l'opposé de l'autre, mais avec jugement et prudence, était
quelque chose de fort étrange dans ce couple si uni et si concerté.
M\textsuperscript{me} d'O se donnait pour aimer le monde, le plaisir, la
bonne chère\,; et pour le mari on l'aurait si bien pris pour un
pharisien, il en avait tant l'air, l'austérité, les manières, que
j'étais toujours tenté de lui couper son habit en franges par
derrière\,; bref, tous ces manèges firent M\textsuperscript{me} d'O dame
du palais. Si son mari, qui était demeuré avec le titre de gentilhomme
de la chambre de M. le comte de Toulouse et toutes ses entrées par
derrière, l'eût été d'un prince du sang, c'eût été une exclusion sûre\,;
mais le roi avait donné à ses enfants naturels cet avantage sur eux, de
faire manger, entrer dans les carrosses, aller à Marly, et sans
demander, leurs principaux domestiques, sans que M. le Duc, quoique
gendre du roi, eût pu y atteindre pour les siens.

Il arriva, depuis son mariage, que Monseigneur revenant de courre le
loup, qui l'avait mené fort loin, manqua son carrosse et s'en revenait
avec Sainte-Maure et d'Urfé. En chemin il trouva un carrosse de M. le
Duc, dans lequel étaient Saintrailles qui était à lui, et le chevalier
de Sillery qui était à M. le prince de Conti, et frère de Puysieux qui
fut depuis chevalier de l'ordre. Ils s'étaient mis dans ce carrosse
qu'ils avaient rencontré, et y attendaient si M. le Duc ou M. le prince
de Conti ne viendrait point. Monseigneur monta dans ce carrosse pour
achever la retraite, qui était encore longue jusqu'à Versailles, y fit
monter avec lui Sainte-Maure et d'Urfé, laissa Saintrailles et Sillery à
terre, quoiqu'il y eût place de reste encore pour eux, et ne leur offrit
point de monter. Cela ne laissa pas de faire quelque peine à
Monseigneur, par bonté\,; et le soir, pour sonder ce que le roi
penserait, il lui conta son aventure et ajouta qu'il n'avait osé faire
monter ces messieurs avec lui. « Je le crois bien, lui répondit le roi
en prenant un ton élevé, un carrosse où vous êtes devient le vôtre, et
ce n'est pas à des domestiques de prince du sang à y entrer.\,»
M\textsuperscript{me} de Langeron en a été un exemple singulier. Elle
fut d'abord à M\textsuperscript{me} la Princesse, et tant qu'elle y fut,
elle n'entra point dans les carrosses, ni ne mangea à table. Elle passa
à M\textsuperscript{me} de Guise, petite-fille de France, et, de ce
moment, elle mangea avec le roi, M\textsuperscript{me} la Dauphine et
Madame, car la reine était morte avec qui elle aurait mangé aussi, et
entra dans les carrosses sans aucune difficulté. La même
M\textsuperscript{me} de Langeron quitta M\textsuperscript{me} de Guise
et rentra à M\textsuperscript{me} la Princesse, et dès lors il ne fut
plus question pour elle de plus entrer dans les carrosses ni de manger.
Cette exclusion dura le reste de sa longue vie, et elle est morte chez
M\textsuperscript{me} la Princesse.

La marquise du Châtelet était fille du feu maréchal de Bellefonds, et,
comme M\textsuperscript{me} de Nogaret, avait été fille de
M\textsuperscript{me} la Dauphine. Elle avait épousé le marquis du
Châtelet, c'est-à-dire un seigneur de la première qualité, de l'ancienne
chevalerie de Lorraine. Cette maison prétend être de la maison de
Lorraine, et l'antiquité de l'une et de l'autre ôte les preuves du pour
et du contre. Elle y a eu toujours les emplois les plus distingués et
porte les armes pleines de Lorraine, avec trois fleurs de lis d'argent
sur la bande, au lieu des trois alérions de Lorraine, et, depuis quelque
temps, ont pris le manteau ducal, de ces manteaux qui ne donnent rien,
et que M. le prince de Conti appelait plaisamment des robes de chambre.
De rang ni d'honneur ils n'en ont jamais eu ni prétendu. M. du Châtelet
était un homme de fort peu d'esprit et difficile, mais plein d'honneur,
de bonté, de valeur, avec très peu de bien et de santé, et fort bon
officier et distingué. Sa femme était la vertu même et la piété même,
dans tous les temps de sa vie, bonne, douce, gaie, sans jamais ni
contraindre ni trouver à redire à rien, aimée et désirée partout. Elle
vivait retirée avec son mari et sa mère à Vincennes, dont le petit
Bellefonds son neveu était gouverneur. Ils venaient peu à la cour,
n'avaient pas de quoi être à Paris, et cependant M. du Châtelet vivait
fort noblement à l'armée. Ils ne pensaient à rien moins. Le roi avait
toujours aimé le maréchal de Bellefonds et l'avait pourtant laissé à peu
près mourir de faim. Sa considération, quoique mort, la vertu et la
douceur de sa fille la firent dame du palais dans Vincennes, où elle n'y
avait seulement pas songé, et ce choix fut fort applaudi.

M\textsuperscript{me} de Montgon n'y pensait pas davantage, et se
trouvait alors chez son mari en Auvergne, et lui à l'armée\,; mais elle
avait une mère qui y songeait pour elle, et qui ne bougeait de la cour
et d'avec M\textsuperscript{me} de Maintenon\,: c'était
M\textsuperscript{me} d'Heudicourt, qu'il faut reprendre de plus loin.
Le maréchal d'Albret, des bâtards de cette grande maison dès lors
éteinte, avait une grand'mère Pons, mère de son père, fille du chevalier
du Saint-Esprit de la première promotion, sœur de la fameuse
M\textsuperscript{me} de Guiercheville, dame d'honneur de Marie de
Médicis qui introduisit la première le cardinal de Richelieu auprès
d'elle, et qui fut mère en secondes noces du duc de Liancourt. Le
maréchal d'Albret, qui eut son bâton pour avoir conduit M. le Prince, M.
le prince de Conti et M. de Longueville à Vincennes avec les
chevau-légers, fut toute sa vie dans une grande considération, et tenait
un grand état partout. Il était chevalier de l'ordre et gouverneur de
Guyenne. Il avait chez lui, à Paris, la meilleure compagnie, et
M\textsuperscript{lle}s de Pons n'en bougeaient, qui n'avaient rien, et
qu'il regardait comme ses nièces. Il fit épouser l'aînée à son frère,
qui n'eut point d'enfants, et est morte en 1614 \footnote{La date de
  1614 est dans le manuscrit, mais il faut lire 1714. En effet,
  Élisabeth de Pons, veuve du comte de Miossens (François-Amanieu),
  mourut le 23 février 1714.}\,; elle s'appelait M\textsuperscript{me}
de Miossens et faisait peur par la longueur de sa personne. La cadette,
belle comme le jour, plaisait extrêmement au maréchal et à bien
d'autres.

M\textsuperscript{me} Scarron, belle, jeune, galante, veuve et dans la
misère, fut introduite par ses amis à l'hôtel d'Albret, où elle plut
infiniment au maréchal et à tous ses commensaux par ses grâces, son
esprit, ses manières douces et respectueuses, et son attention à plaire
à tout le monde et surtout à faire sa cour à tout ce qui tenait au
maréchal\,; ce fut là où elle fut connue de la duchesse de Richelieu,
veuve en premières noces du frère aîné du maréchal d'Albret, qui de plus
avaient marié ensemble leur fils et leur fille unique\,; et la duchesse,
quoique remariée, était demeurée dans la plus intime liaison avec le
maréchal. Lui et M\textsuperscript{me} de Montespan étaient enfants du
frère et de la sœur. M. et M\textsuperscript{me} de Montespan ne
bougeaient de chez lui, et ce fut où elle connut M\textsuperscript{me}
Scarron et qu'elle prit amitié pour elle. Devenue maîtresse du roi, le
maréchal n'eut garde de se brouiller avec elle pour son cousin\,: en bon
courtisan il prit son parti et devint son meilleur ami et son conseil.
C'est ce qui fit la fortune de M\textsuperscript{me} Scarron, qui fut
mise gouvernante des enfants qu'elle eut du roi, dès leur naissance. Le
maréchal, qui ne savait que faire de M\textsuperscript{lle} de Pons,
trouva un Sublet, de la même famille du secrétaire d'État des Noyers,
qui avait du bien et qui, ébloui de la beauté et de la grande naissance
de cette fille, l'épousa pour l'alliance et la protection du maréchal
d'Albret, qui, pour lui donner un état, lui obtint, en considération de
ce mariage, l'agrément de la charge de grand louvetier dont le marquis
de Saint-Herem se défaisait pour acheter le gouvernement de
Fontainebleau. Ce nouveau grand louvetier s'appelait M. d'Heudicourt et
eut une fille à peu près de l'âge de M. du Maine, quelques années de
plus. M\textsuperscript{me} Scarron fit trouver bon à
M\textsuperscript{me} de Montespan qu'elle prit cette enfant pour faire
jouer les siens, et l'éleva avec eux dans les ténèbres et le secret qui
les couvraient alors. Quand ils parurent chez M\textsuperscript{me} de
Montespan, la petite Heudicourt était toujours à leur suite, et après
qu'ils furent manifestés à la cour, elle y demeura de même.
M\textsuperscript{me} Scarron, devenue M\textsuperscript{me} de
Maintenon, n'oublia jamais le berceau de sa fortune et ses anciens amis
de l'hôtel d'Albret.

C'est ce qui fit si longtemps après M\textsuperscript{me} de Richelieu
dame d'honneur de la reine, puis, par confiance, de
M\textsuperscript{me} la Dauphine, à son mariage\,; M. de Richelieu,
chevalier d'honneur pour rien, et ce qui fit toute la fortune de Dangeau
par la permission qu'eut le duc de vendre sa charge à qui et si cher
qu'il voudrait. Par même cause, M\textsuperscript{me} de Maintenon aima
et protégea toujours ouvertement M\textsuperscript{me} d'Heudicourt et
sa fille qu'elle avait élevée et qu'elle aima particulièrement. Elle
entra dans son mariage avec Montgon que cette faveur lui fit faire.
C'était un très médiocre gentilhomme d'Auvergne, du nom de Cordebœuf,
dont l'esprit réparait tant qu'il pouvait la valeur, et qui toutefois
s'était attaché au service. Il était à l'armée du Rhin brigadier de
cavalerie et inspecteur, et sa femme dans ses terres en Auvergne
lorsqu'elle fut nommée dame du palais, au scandale extrême de toute la
cour. C'était une femme laide, qui brillait d'esprit, de grâce, de
gentillesse\,; plaisante et amusante au possible, méchante à l'avenant,
et qui, sur l'exemple de sa mère, divertit M\textsuperscript{me} de
Maintenon et le roi dans les suites, aux dépens de chacun, avec beaucoup
de sel et d'enjouement. Toute cette petite troupe partit au-devant de la
princesse. M\textsuperscript{me} de Mailly était grosse et ne fut point
du voyage. M\textsuperscript{me} du Châtelet s'en dispensa pour donner à
la maréchale de Bellefonds tout ce temps-là encore à demeurer auprès
d'elle. On ne le trouva pas trop bon, et, au lieu de la troisième place
qui lui était destinée avec grande raison, elle n'eut que la cinquième.
L'éloignement de M\textsuperscript{me} de Montgon en Auvergne ne lui
permit pas d'être du voyage. Laissons-les aller et publier la paix à
Paris et à Turin, où le maréchal Catinat fut reçu avec de grands
honneurs, et y donna chez lui à dîner à M. de Savoie, et retournons sur
le Rhin.

\hypertarget{chapitre-xxiii.}{%
\chapter{CHAPITRE XXIII.}\label{chapitre-xxiii.}}

1696

~

\relsize{-1}

{\textsc{Projet des Impériaux sur le Rhin.}} {\textsc{- Maréchal de
Choiseul dans le Spirebach.}} {\textsc{- Raisons de ce camp.}}
{\textsc{- Dispositions du maréchal de Choiseul.}} {\textsc{- Mouvements
et dispositions du prince Louis de Bade.}} {\textsc{- Retraite des
Impériaux.}} {\textsc{- Précautions du maréchal de Choiseul à la cour,
qui met en quartiers de fourrages et me donne congé.}} {\textsc{- Mort
de M. Frémont, beau-père de M. le maréchal de Lorges.}} {\textsc{-
Naissance de ma fille.}} \relsize{1}

~

Après une longue oisiveté en ces armées et en Flandre, les vingt mille
hommes de Hesse et d'autres contingents furent renvoyés au prince Louis
de Bade, qui, avec ce qu'il avait d'ailleurs, se trouva le double plus
fort que le maréchal de Choiseul, et en état et en volonté
d'entreprendre le siège de Philippsbourg, dont tous les amas étaient
depuis l'hiver dans Mayence, et toutes les précautions prises depuis
pour que rien n'y pût manquer. L'empereur pressait l'exécution de ce
dessein avec toute l'ardeur que lui inspirait son dépit de la paix de
Savoie, et son extrême désir de reculer la générale, à laquelle celle-là
commençait à donner un grand branle. Sur les avis que le maréchal de
Choiseul en donna à la cour, il en reçut deux lettres fort singulières
et en même temps contradictoires. Par la première, Barbezieux lui
faisait écrire par le roi de jeter huit de ses meilleurs bataillons dans
Philippsbourg et quatre dans Landau, et de se retirer après en pays de
sûreté contre l'invasion du prince Louis. Il faut remarquer que le
maréchal n'avait dans son armée que douze bons bataillons et que tout le
reste de son infanterie était de nouvelles levées, ou des bataillons de
salade ramassés des garnisons. En suivant cet ordre il n'avait plus à
compter sur ce qui lui serait resté d'infanterie, et en abandonnant ces
places au renfort qu'il y aurait jeté, l'exemple récent de Namur devait
persuader qu'elles n'en seraient pas moins perdues. Par l'autre lettre
en réponse au maréchal, le roi lui marquait qu'il n'était pas persuadé
que le prince Louis pensât à passer le Rhin à Mayence, mais que, s'il
songeait à l'entreprendre, il se persuadait que la maréchal
l'empêcherait bien d'y déboucher, c'est-à-dire empêcher un ennemi de
passer sur un pont à lui, dans une place à lui, et de déboucher sur une
contrescarpe à lui, dans une plaine.

Le maréchal haussa les épaules, et proposa au moins d'envoyer le marquis
d'Harcourt le renforcer, qui demeurait oisif où il était dans la
situation présente. Harcourt, accoutumé à commander en chef, ami de
Barbezieux et grand maître en souterrains à la cour, ne voulait point
tâter de cette jonction. Il proposa à la cour et au maréchal des partis
téméraires, bien sûr qu'ils ne les adopteraient pas, et que l'honneur de
les avoir imaginés lui en {[}reviendrait{]}. Le maréchal, aux ordres
duquel il n'était point comme de ceux qui étaient en Flandre, ne pouvait
se soumettre à lui en donner\,; et Harcourt, qui le sentait, et qui le
savait mal de tout temps avec son ami Barbezieux, allait à son fait de
ne point joindre et se moquait de lui. Cette conduite ouvrit les yeux au
maréchal sur ses artifices\,; il ne compta plus que sur soi-même, et
résolut de laisser dire Harcourt et ordonner à la cour, {[}et{]} de ne
suivre, à tous risques pour lui même, que le parti unique par lequel il
crut sauver Philippsbourg et Landau. Il se retira donc sur son
infanterie que, pour la commodité des fourrages, il avait laissée
derrière, entra dans le Spirebach, et fit une des plus belles choses
qu'on eût vues depuis bien longtemps à la guerre.

Le prince Louis passa le Rhin avec sa cavalerie à Mayence, après avoir
conféré avec le landgrave de Hesse, qui vint passer la Nave auprès de
Mayence, qui vint après le long des montagnes et se saisit sans peine,
chemin faisant, de Neu-Linange, de Kirken et d'autres postes que nous y
avions, tandis que le prince Louis vint à Oppenheim, où son infanterie,
son artillerie et ses bagages le joignirent par un pont de bateaux\,; il
en fit descendre un à Worms, tant pour tirer ce qu'ils voudraient de
l'autre côté du Rhin, que pour communiquer avec le baron de Thungen,
commandant de Mayence, qui avait environ quinze mille hommes aux vallées
de Ketsch et vers Fribourg, avec des bateaux pour nous donner par nos
derrières l'inquiétude du siège de cette place, ou d'un passage du Rhin.
La cour alors avait changé d'avis, et aurait voulu que le maréchal de
Choiseul eût combattu le prince Louis aux plaines d'Alney. Il était plus
fort que nous du double\,; et s'il avait battu notre armée, il eût
aisément pris Landau, fort méchante place alors, et eût été le maître
d'emporter le fort qui couvrait le bout d'en deçà du pont de
Philippsbourg, de brûler ce pont, de ravager l'Alsace, de s'établir pour
l'hiver à Spire, d'empêcher M. d'Harcourt de déboucher les montagnes,
puis de faire tout à son aise le siège de Philippsbourg.

Ces mêmes raisons détournèrent le maréchal de croire ceux qui lui
proposaient de se mettre à Durckheim\,: cette petite ville ruinée et non
tenable était bien au pied des montagnes, mais entre elles et l'endroit
où les montagnes s'escarpent et se couvrent, il y avait un grand espace
de terrain à passer plusieurs colonnes de front\,; d'ailleurs, le marais
qui aurait couvert l'armée était en figure de T dont la queue la
séparait. Il aurait donc fallu force ponts de communication sur cette
queue, et on laisse à penser de quelles ressources sont de telles
communications à une armée attaquée par le double d'elle. Le marquis
d'Huxelles proposa de se mettre le cul au Rhin et le nez à la montagne.
Ce parti conservait Spire et nous en appuyait, mais il abandonnait
Neustadt, le livrait au prince Louis pour un entrepôt très commode pour
ses vivres, et un passage assuré derrière Landau pour passer en Alsace
et la ruiner, sans crainte que nous osassions nous déplacer\,; il nous
ôtait en même temps en fort peu de jours toute subsistance, parce que
nous ne pouvions tirer de fourrages que de l'Alsace, et bientôt les
vivres, que Thungen ne nous aurait pas même laissé descendre aisément
par le Rhin. D'autres proposèrent la position contraire, le cul à Landau
et la tête au Rhin. Celui-là tenait Landau et Neustadt, mais il laissait
tout le chemin de l'Alsace libre aux ennemis, l'important poste de
Spire, d'où, une fois établis, ils pouvaient brûler le pont de
Philippsbourg, s'en épargner la circonvallation de ce côté-ci, et en
faire le siège de l'autre côté tout à leur aise. D'ailleurs bien établis
à Spire, ils mettaient l'Alsace en contribution, minaient Landau et
renvoyaient nos armées s'assembler bien loin. Se mettre derrière la
petite rivière de Landau, laissait tout en proie Neustadt, Spire, le
pont de Philippsbourg, le passage en Alsace, Landau mémé. Tous ces
partis, quelque mauvais qu'ils fussent, avaient leurs partisans
considérables.

Le maréchal de Choiseul, bien résolu de n'aller qu'au meilleur, dans une
conjoncture si importante, laissa écrire la cour et discourir qui
voulut, et prit de soi tout seul l'unique parti qui sauvait tous ces
inconvénients. Il les avait de longue main pourpensés, et s'y était
préparé, autant qu'il l'avait pu, dans la prévoyance de ce que les
ennemis pourraient entreprendre. C'était de barrer la plaine derrière le
Spirebach, de la montagne au Rhin, et de mettre par là Neustadt, Spire,
Landau, Philippsbourg et l'Alsace à couvert. Lorsqu'il s'était avancé
avec sa cavalerie, pour la commodité des fourrages, dans les plaines de
Mayence, tandis qu'il n'était encore question de rien, et qu'il avait
laissé son infanterie en arrière, il avait chargé le marquis d'Huxelles
avec sa seconde ligne d'infanterie d'accommoder le Spirebach, et, quand
il s'y vint mettre, il trouva cette besogne achevée et parfaitement bien
faite, avec des redoutes d'espace en espace et tous les bords
retranchés. Il avait cependant obtenu la jonction du marquis d'Harcourt
qui se lit fort attendre, et qui manda à la cour qu'il avait joint deux
jours plus tôt qu'il n'avait fait. Comme il arrivait par la montagne, il
fut chargé de Neustadt et de tous ces postes-là. De la montagne aux bois
il y avait une bonne demi-lieue. Cet espace était fermé par les deux
branches du Spirebach réduites en une par une retenue de distance en
distance au dedans et au-dessus de Neustadt, qui formait une inondation
et un marais qui ne se pouvaient passer.

Là se trouvait une commanderie ruinée, qui fut très bien accommodée, où
on jeta quatre bataillons avec Condrieu, très bon brigadier
d'infanterie. De lui jusqu'au bois, des demi-lunes bien ajustées, toutes
flanquées de deux pièces de canon de chaque côté, avec chacune un
bataillon derrière pour s'y jeter à propos, et un espace entre chacune
pour y recevoir un escadron\,; avec cela Neustadt remparé et fortifié au
mieux avec de l'artillerie, et Saint-Frémont, maréchal de camp, pour y
commander sous Harcourt, et la plaine de Musbach, par où seulement les
ennemis pouvaient venir, entièrement découverte et de toutes parts
fouettée des batteries disposées pour cela. Le petit château de Hart, à
mi-côte de la montagne, fut occupé et bien retranché, bien muni avec ce
qu'il put tenir de monde choisi. C'était un petit castel blanc qui se
voyait de partout, un peu à côté et plus avancé au delà de Neustadt. Les
bois devinrent bientôt un fonds de marais artificiel, par les retenues
d'espace en espace du Spirebach qui y coulait. On y fit de grands abatis
d'arbres, et tout du long semés de petits postes pour avertir seulement.
En un endroit plus clair et au bord d'une petite plaine où il y avait en
deçà du ruisseau un moulin appelé Freymülh, dont on se servit
avantageusement pour s'aider de l'eau à retenir et à inonder, on fit
camper quatre bataillons appuyés de la cavalerie de notre droite, parce
que la ligne s'étendait jusque-là, et le quartier du marquis de Renti,
lieutenant général fort bon et beau-frère du maréchal, n'en était pas
éloigné. On mit un peu plus loin au village ruiné de Spirebach la
brigade de cavalerie de Bissy avec de l'infanterie divisée par pelotons
jusqu'à Spire où finissaient les bois. À Spire force de canons et
beaucoup d'infanterie dans les retranchements, avec, pour cavalerie, la
brigade du colonel général\,; le marquis d'Huxelles et le duc de La
Ferté, lieutenants généraux, y commandaient, et sous eux Hautefort et
Lalande, maréchaux de camp. De Spire au Rhin il n'y avait pas l'espace
pour un escadron. Le maréchal de Choiseul prit son quartier général au
village de Lackheim, vis-à-vis du commencement des bois, vers le centre
de la cavalerie. Notre gauche de cavalerie joignait la droite de celle
du marquis d'Harcourt, mais un peu plus reculée\,; et lui se mit dans un
petit village tout à fait dans la montagne, près de Neustadt, en deçà.

Les choses disposées de la sorte, on continua à perfectionner les
retranchements partout où on crut qu'il en était besoin\,; et on
attendit avec une tranquillité très vigilante ce que les ennemis
pourraient ou voudraient entreprendre. On montait tous les soirs un gros
bivouac à la tête des camps, avec le maréchal de camp de jour à la
droite et le brigadier de piquet à la gauche. Le mestre de camp de
piquet se promenait toute la nuit d'un bout à l'autre pour voir si tout
était bien en état. J'étais encore cette campagne de la brigade qui
formait la seconde ligne de la gauche avec le bonhomme Lugny pour
brigadier, très galant homme, de qui je reçus mille honnêtetés, mais qui
n'avait ni l'esprit ni le monde qu'avait Harlus, qui servait, cette
année, sur les côtes, avec le maréchal de Joyeuse. Le chevalier de
Conflans était l'autre mestre de camp avec nous. C'était un très bon
officier, gaillard et de bonne compagnie, plaisant en liberté, avec de
l'esprit, qui savait fort, vivre et dont je m'accommodai fort. Il était
cadet du marquis de Conflans, mestre de camp général en Catalogne pour
le roi d'Espagne, qui lui avait donné la Toison d'or, et qui le fit,
l'année suivante, vice-roi de Navarre et grand d'Espagne de la troisième
classe, dont la grandesse périt avec eux comme nos ducs à brevet. Ils
étaient ou petits-fils ou fort proches, et de même nom, de ce baron de
Batteville ou Vatteville, qui, étant ambassadeur d'Espagne en
Angleterre, fit cette insulte pour la préséance au maréchal d'Estrades,
ambassadeur de France, qui fit tant de fracas et qui fut suivie de la
déclaration solennelle que l'ambassadeur d'Espagne en France eut ordre
de faire au roi, de ne plus prétendre en nul lieu de compétence avec
lui.

Le prince Louis, supérieur au maréchal de Choiseul et au marquis
d'Harcourt joints, de plus de vingt-deux mille hommes, campa deux jours
après notre arrivée à une demi-lieue de nous derrière le village de
Musbach, à la vue de nos montagnes, et se mit à ouvrir des chemins dans
les leurs. On les vit se donner de grands mouvements pendant plusieurs
jours sans qu'on en pût deviner la cause, lorsque après avoir longé
notre front bien des fois, et s'en être approchés tant qu'ils purent
pour reconnaître, et avoir cherché inutilement par où pouvoir attaquer,
on s'aperçut qu'ils avaient établi des batteries sur des montagnes qui
semblaient inaccessibles, d'où ils firent grand bruit de canon.
C'étaient trois batteries de gros canon à diverses hauteurs, dont une
sur la crête tout au haut, et on distinguait très clairement les tentes
de trois bataillons qui campaient auprès. Ils occupèrent diverses
maisons éparses le long de la montagne, auprès de ce petit château de
Hart, le canonnèrent, et firent remuer quelque cavalerie du marquis
d'Harcourt incommodée de cette artillerie. Ce petit castel les mit en
colère, dont ils ne touchaient que le haut des toits. Ils baissèrent
donc une batterie avec laquelle ils y firent une grande brèche\,; ils y
donnèrent quelques assauts sans succès, jusqu'à ce que le brave officier
qui y commandait, se voyant ouvert de toutes parts et sans nulle
espérance de pouvoir être secouru, prit le temps d'un assaut plus grand
que les précédents pour faire retirer sa garnison par un trou qu'il
avait pratiqué, et sortit le dernier de sa place qu'il avait bravement
défendue six jours durant à la vue des deux armées, et se retira avec
ses gens à Neustadt, avec une jambe qu'il se cassa en sortant. Il fut
loué et caressé de toute l'armée\,; le maréchal lui donna le peu qu'il
avait d'argent, et lui procura une gratification. Il avait laissé une
traînée de poudre où il mit le feu, qui fut fatale aux premiers qui se
jetèrent dans leur conquête. Cet exploit achevé, les Impériaux
changèrent et augmentèrent leurs batteries et en battirent la porte de
Neustadt de notre côté par-dessus la ville, ce qui n'eut d'autre effet
que de faire hâter le pas à ceux qui entraient ou sortaient.

Au bout d'un mois ils s'aperçurent si bien de l'inutilité de leur
canonnade et de l'impossibilité d'attaquer nos retranchements avec le
moindre succès, qu'ils se tournèrent à d'autres moyens pour nous obliger
à les abandonner\,; ils envoyèrent donc faire des courses sur la Sarre
jusque vers Metz, et ils ordonnèrent à Thungen de ne rien oublier pour
passer diligemment en Alsace. Sur les avis qu'on en eut, Gobert,
excellent brigadier de dragons, fut envoyé avec un gros détachement sur
la Sarre, et le marquis d'Huxelles sur le haut Rhin joindre Puysieux
avec un régiment de cavalerie, des dragons et de l'infanterie, et
Chamilly fut mis à Spire à la place d'Huxelles. Puysieux, lieutenant
général et gouverneur d'Huningue, n'avait presque point d'autres troupes
pour la garde du haut Rhin que des compagnies franches du Rhin, un ramas
de garnisons et des paysans. Thungen, outre ses ordres, mourait d'envie
de passer et de faire du pis qu'il pourrait, de dépit d'avoir été enlevé
tout au commencement de la campagne par un parti d'infanterie qui
s'était glissé tout contre Mayence, d'où il l'avait mené à
Philippsbourg. Il avait fallu payer pour en sortir libre, et cela joint
à l'affront l'avait mis fort en colère\,; mais il fut observé de si près
qu'il ne put jamais tenter le passage. Sur la Sarre, Gobert ne leur
donna pas loisir de courir ni de piller, tellement que les Impériaux,
sentant enfin qu'une plus longue opiniâtreté ne ferait qu'augmenter leur
honte, résolurent de se retirer. Je m'aperçus étant de piquet, et me
promenant la nuit le long de nos bivouacs, d'une diminution dans leurs
feux ordinaires, qui avec les nôtres faisaient dans ces montagnes et au
bas un effet singulier et tout à fait beau, et le matin nous
n'entendîmes point leur canon. Dès qu'il fit un peu clair, j'allai vers
nos demi-lunes trouver le maréchal de Choiseul qui s'y promenait déjà,
et nous vîmes qu'ils n'avaient plus ni canon, ni camp, ni personne sur
leurs montagnes. Un gros brouillard, qui nous en ôta incontinent la vue,
tomba sur les neuf ou dix heures du matin, et nous laissa apercevoir à
découvert leur retraite. Ils marchaient en bataille derrière la plaine
de Musbach, où ils avaient laissé divers petits pelotons de cavalerie
épars, pour nous observer et escarmoucher s'ils étaient suivis. Harcourt
vint trouver le maréchal à une batterie élevée où nous étions, et chacun
fut fort aise d'être délivré d'un ennemi si peu à craindre dans le poste
où nous étions, mais d'ailleurs si importun par la vigilance que
demandait un si proche voisinage. Saint-Frémont qui se trouvait de jour,
était sorti avec quelques gardes ordinaires à la tête du village de
Weintzingen sous Neustadt\,; il eut envie de se faire valoir à bon
marché, et envoya à plusieurs reprises demander quelques troupes au
maréchal pour pousser ce qui était dans la plaine, dont à la fin, ce
dernier s'impatienta.

Comme son projet avait été d'arrêter les ennemis et non d'aller à eux
pour les combattre, mais de rompre tous leurs desseins en barrant de la
montagne au Rhin, nos inondations étaient faites en sorte qu'il n'y
avait que deux ouvertures par lesquelles on ne pouvait sortir qu'un à
un. La raison du maréchal fut donc que s'il n'y avait dans la plaine que
ces petits pelotons que nous voyions, ce n'était pas la peine d'aller à
eux pour leur faire doubler le pas\,; que si, au contraire, il y avait
des troupes derrière les haies et ce qui bornait notre vue, il ne
fallait pas exposer Saint-Frémont à être battu sous nos yeux sans
pouvoir être secouru et faire ainsi, sans raison, une mauvaise affaire
et honteuse, d'une bonne, puisque les ennemis se retiraient sans avoir
pu exécuter quoi que ce soit. Saint-Frémont, qui avait aussi ses
souterrains et qui était ami du marquis d'Harcourt, ne laissa pas d'être
accusé d'avoir écrit\,: qu'il n'avait tenu qu'au maréchal de Choiseul de
battre l'arrière-garde des ennemis, sans qu'il eût pu le lui persuader.
Les ennemis avaient retiré leurs postes le long du ruisseau et des
inondations qui n'étaient qu'à une portée de carabine des nôtres, toute
la nuit précédente en grand silence, et y avaient laissé leurs feux tant
qu'ils avaient pu durer, et en même temps retiré tout ce qu'ils avaient
de canons en batteries\,; et l'artillerie qui n'y était pas et leurs
bagages, {[}ils{]} les avaient fait passer à Worms avec quelque peu de
troupes sur leur pont de bateaux qu'ils défirent aussitôt après. Leur
armée marcha fort vite à Mayence où elle repassa le Rhin, dédaigna de
prendre Eberbourg et Kirn, deux bons châteaux qu'il ne tenait qu'à eux
de prendre, et se mit aussitôt après en quartiers de fourrage, non sans
force querelles entre les généraux, enragés d'avoir tant éclaté en
menaces et en grands projets et de n'avoir pu rien exécuter. Cela fut
uniquement dû à la capacité et à la fermeté tout ensemble du maréchal de
Choiseul, qui laissa tonner la cour, crier ses premiers officiers
généraux, intriguer M. d'Harcourt, sans s'ébranler en aucune sorte.

Le lendemain de cette retraite, nous fûmes voir leurs camps et leurs
travaux, et nous admirâmes les peines qu'ils eurent sans doute à guinder
leur canon si haut, et le reste de leurs ouvrages qui nous parurent
prodigieux. Les fourrages leur manquaient, ils tiraient de fort loin
leurs vivres, tout enfin les avait obligés à la retraite. Le maréchal
avait gardé toutes les lettres du marquis d'Harcourt et la copie de ses
réponses. Il avait mis un petit commentaire concis et fort, en marge,
vis-à-vis des endroits qui le demandaient et avait envoyé tout cela au
roi dans un grand cahier.

N'y ayant plus rien à faire et les troupes allant dans leurs quartiers
de fourrage, je voulus m'en aller à Paris. Le mois d'octobre était fort
avancé, M\textsuperscript{me} de Saint-Simon avait perdu M. Frémont,
père de M\textsuperscript{me} la maréchale de Lorges, et elle était en
même temps heureusement accouchée de ma fille le 8 septembre. Le
maréchal me le permit\,; il m'avait traité avec tant de politesse et
d'attention que je m'attachai à lui, et qu'il me donna enfin sa
confiance, dont à mon âge je me sentis fort honoré. Je savais tout ce
qui s'était passé entre le marquis d'Harcourt et lui, et il m'avait
montré ce cahier qu'il avait envoyé au roi. Il me pria de conter tous
ces détails au duc de Beauvilliers en arrivant, et de l'engager à le
servir, ce que j'exécutai tout à fait à la satisfaction du maréchal.

\hypertarget{chapitre-xxiv.}{%
\chapter{CHAPITRE XXIV.}\label{chapitre-xxiv.}}

1696

~

\relsize{-1}

{\textsc{Noire invention à mon retour.}} {\textsc{- M. de la Trappe
peint de mémoire.}} {\textsc{- M. de Savoie avec l'armée du roi assiège
Valence.}} {\textsc{- Il lève le siège par la neutralité d'Italie.}}
{\textsc{- Tout accompli avec lui et son ministre mené pour le premier
des ministres étrangers à Marly.}} {\textsc{- La princesse au
Pont-Beauvoisin a le rang de duchesse de Bourgogne.}} {\textsc{-
Prétention étrange du comte de Brionne à l'égard de M. de Savoie.}}
{\textsc{- Le roi à Montargis au-devant de la princesse.}} {\textsc{-
Arrivée à Fontainebleau\,; présentation.}} {\textsc{- Retour à
Versailles.}} {\textsc{- Des présentations.}} {\textsc{- Grâces de la
princesse qui charment le roi et M\textsuperscript{me} de Maintenon.}}
{\textsc{- M\textsuperscript{lle}s de Soissons ont défense de voir la
princesse.}} \relsize{1}

~

En arrivant à Paris, je trouvai la cour à Fontainebleau. Comme j'étais
arrivé un peu devant les autres, je ne voulus pas que le roi le sût sans
me voir, et me crût de retour en cachette. Je voulais de plus voir M. de
Beauvilliers, sur le maréchal de Choiseul. Je me hâtai donc d'aller à
Fontainebleau où je fus très bien reçu, et le roi, à son ordinaire de
mes retours, me parla avec bonté, en me disant toutefois que j'étais
revenu un peu tôt, mais ajoutant qu'il n'y avait point de mal.

J'avais un voyage en tête à brusquer, dont je parlerai tout à l'heure,
qui me pressait de m'en retourner à Paris après mes premiers devoirs
rendus, lorsqu'au sortir du lever du roi, comptant monter en chaise tout
de suite, Louville me mena dans la salle de la comédie, ouverte alors et
où il n'y avait jamais personne les matins, qui était au bout de la
salle des gardes. Là il m'avertit qu'il s'était répandu que lorsqu'en
faisant ma révérence au roi, il m'avait dit qu'il se réjouissait de me
voir de retour en bonne santé, quoique un peu tôt, je lui avais répondu
que j'avais mieux aimé le venir voir tout en arrivant comme ma seule
maîtresse, que de demeurer quelques jours relaissé à Paris, comme
faisaient les jeunes gens avec les leurs. À ce récit le feu me monta au
visage. Je rentrai chez le roi, où il y avait encore beaucoup de monde,
devant qui je m'exhalai sur ce qui me venait d'être rapporté, et
j'ajoutai que je donnerais volontiers bien de l'argent pour savoir qui
avait inventé et semé cette noire friponnerie, afin, quel qu'il fût, de
lui en donner le démenti et force coups de bâton au bout, pour lui
apprendre à calomnier d'honnêtes gens, à lui et aux faquins ses
semblables. Je demeurai tout le jour à Fontainebleau cherchant le monde
pour répéter ces propos, et que, si un grand coquin demeurait assez
caché pour échapper au châtiment, j'espérais du moins qu'il en
apprendrait la menace, et qu'il l'entendrait peut-être lui-même assez
pour en faire son profit, et laisser les gens d'honneur en repos.

Ma colère et mes discours firent la nouvelle. M. le maréchal de Lorges,
qui avait le bâton, et m'avait coupé la parole sur mon arrivée un peu
tôt, en sorte que je n'y pus rien du tout répondre au roi, quand je
l'aurais voulu, bien loin d'ailleurs d'une si indigne flatterie, et
beaucoup de vieux seigneurs avec lui, me blâmèrent d'avoir parlé si
haut, en tels termes, dans la maison du roi et jusque dans son
appartement. Je les laissai dire parce qu'ils ne m'apprenaient rien que
je n'eusse bien prévu\,; mais de deux maux j'avais choisi le moindre,
qui était une réprimande du roi, ou peut-être quelques jours de
Bastille, et j'avais évité le plus grand, qui était de laisser croire la
chose vraie à mon âge, et encore peu connu de la plupart du monde, et me
laisser passer pour un infâme délateur de toute la jeunesse, pour faire
bassement et misérablement ma cour. Le roi n'en sut rien, ou voulut bien
l'ignorer, le bruit que je fis étouffa sa cause et me fit honneur\,; et
je m'en allai faire mon petit voyage, dont je parlerai ici tout de
suite.

Il y avait longtemps que l'attachement que j'avais pour M. de la Trappe,
et mon admiration pour lui me faisaient désirer extrêmement de pouvoir
conserver sa ressemblance après lui, comme ses ouvrages en
perpétueraient l'esprit et les merveilles. Son humilité sincère ne
permettait pas qu'on pût lui demander la complaisance de se laisser
peindre. On en avait attrapé quelque chose au chœur, qui produisit
quelques médailles assez ressemblantes, mais cela ne me contentait pas.
D'ailleurs, devenu extrêmement infirme, il ne sortait presque plus de
l'infirmerie, et ne se trouvait plus en lieu où on le pût attraper.
Rigault était alors le premier peintre de l'Europe pour la ressemblance
des hommes et pour une peinture forte et durable\,; mais il fallait
persuader à un homme aussi surchargé d'ouvrages de quitter Paris pour
quelques jours, et voir encore avec lui si sa tête serait assez forte
pour rendre une ressemblance de mémoire. Cette dernière proposition, qui
l'effraya d'abord, fut peut-être le véhicule de lui faire accepter
l'autre. Un homme qui excelle sur tous ceux de son art est touché d'y
exceller d'une manière unique\,; il en voulut bien faire l'essai, et
donner pour cela le temps nécessaire. L'argent, peut-être, lui plut
aussi. Je me cachais fort, à mon âge, de mes voyages de la Trappe\,; je
voulais donc entièrement cacher aussi le voyage de Rigault, et je mis
pour condition de ma part qu'il ne travaillerait que pour moi, qu'il me
garderait un secret entier, et que, s'il en faisait une copie pour lui,
comme il le voulait absolument, il la garderait dans une obscurité
entière, jusqu'à ce qu'avec les années, je lui permisse de la laisser
voir. Du mien, il voulut mille écus comptant à son retour, être défrayé
de tout, aller en poste en chaise en un jour, et revenir de même. Je ne
disputai rien et le pris au mot de tout. C'était au printemps, et je
convins avec lui que ce serait à mon retour de l'armée, et qu'il
quitterait tout pour cela. En même temps je m'étais arrangé avec le
nouvel abbé, M. Maisne, secrétaire de M. de la Trappe, et retiré là
depuis bien des années, et M. de Saint-Louis, ancien brigadier de
cavalerie, fort estimé du roi, retiré là aussi depuis longtemps,
desquels j'aurai ailleurs occasion de parler, et qui ne désiraient pas
moins que moi ce portrait de M. de la Trappe.

Revenant donc de Fontainebleau, je ne couchai qu'une nuit à Paris, où en
arrivant j'avais pris mes mesures avec Rigault, qui partit le lendemain
de moi. J'avertis en arrivant mes complices, et je dis à M. de la Trappe
qu'un officier de ma connaissance avait une telle passion de le voir,
que je le suppliais d'y vouloir bien consentir (car il ne voyait plus
presque personne)\,; j'ajoutai que, sur l'espérance que je lui en avais
donnée, il allait arriver, qu'il était fort bègue, et ne l'importunerait
pas de discours, mais qu'il comptait s'en dédommager par ses regards. M.
de la Trappe sourit avec bonté, trouva cet officier curieux de bien peu
de chose, et me promit de le voir. Rigault arrivé, le nouvel abbé, M.
Maisne et moi le menâmes dès le matin dans une espèce de cabinet qui
servait le jour à l'abbé pour travailler, et où j'avais accoutumé de
voir M, de la Trappe, qui y venait de son infirmerie. Ce cabinet était
éclairé des deux côtés, et n'avait que des murailles blanches, avec
quelques estampes de dévotion, et des sièges de paille, avec le bureau
sur lequel M. de la Trappe avait écrit tous ses ouvrages, et qui n'était
encore changé en rien. Rigault trouva le lieu à souhait pour la
lumière\,; le père abbé se mit au lieu où M. de la Trappe avait
accoutumé de s'asseoir avec moi à un coin du cabinet, et heureusement
Rigault le trouva tout propre à le bien regarder à son point. De là,
nous le conduisîmes en un autre endroit où nous étions bien sûrs qu'il
ne serait vu ni interrompu de personne. Rigault le trouva fort à propos
pour le jour et la lumière, et il y porta aussitôt tout ce qu'il lui
fallait pour l'exécution.

L'après-dînée, je présentai mon officier à M. de la Trappe\,; il s'assit
avec nous dans la situation qu'il avait remarquée le matin, et demeura
environ trois quarts d'heure avec nous. Sa difficulté de parler lui fut
une excuse de n'entrer guère dans la conversation, d'où il s'en alla
jeter sur sa toile toute préparée les images et les idées dont il
s'était bien rempli. M. de la Trappe, avec qui je demeurai encore
longtemps, et que j'avais moins entretenu que songé à l'amuser, ne
s'aperçut de rien, et plaignit seulement l'embarras de la langue de cet
officier. Le lendemain, la même chose fut répétée. M. de la Trappe
trouva d'abord qu'un homme qu'il ne connaissait point, et qui pouvait si
difficilement mettre dans la conversation, l'avait suffisamment vu, et
ce ne fut que par complaisance qu'il ne voulut pas me refuser de le
laisser venir. J'espérais qu'il n'en faudrait pas davantage, et ce que
je vis du portrait me le confirma, tant il me parut bien pris et
ressemblant\,; mais Rigault voulut absolument encore une séance pour le
perfectionner à son gré\,: il fallut donc l'obtenir de M. de la Trappe,
qui s'en montra fatigué, et qui me refusa d'abord, mais je fis tant, que
j'arrachai plutôt que je n'obtins de lui cette troisième visite. Il me
dit que, pour vair un homme qui ne méritait et qui ne désirait que
d'être caché, et qui ne voyait plus personne, tant de visites étaient du
temps perdu et ridicules\,; que pour cette fois il cédait à mon
importunité, et à la fantaisie que je protégeais d'un homme qu'il ne
pouvait comprendre, et qui ne se connaissaient ni n'avaient rien à se
dire\,; mais que c'était au moins à condition que ce serait la dernière
fois et que je ne lui en parlerais plus. Je dis à Rigault de faire en
sorte de n'avoir plus à y revenir, parce qu'il n'y avait plus moyen de
l'espérer. Il m'assura qu'en une demi-heure il aurait tout ce qu'il
s'était proposé, et qu'il n'aurait pas besoin de le voir davantage. En
effet, il me tint parole, et ne fut pas la demi-heure entière.

Quand il fut sorti, M. de la Trappe me témoigna sa surprise d'avoir été
tant et si longtemps regardé, et par une espèce de muet. Je lui dis que
c'était l'homme du monde le plus curieux, et qui avait toujours eu le
plus grand désir de le voir, qu'il en avait été si aise qu'il m'avait
avoué qu'il n'avait pu ôter les yeux de dessus lui, et que de plus,
étant aussi bègue qu'il l'était, la conversation où il ne pouvait entrer
de suite ne l'ayant point détourné, il n'avait songé qu'à se satisfaire
en le regardant tout à son aise. Je changeai de discours le plus
promptement que je pus, et sous prétexte de le mettre sur des choses qui
ne s'étaient pu dire devant Rigault, je cherchai à le détourner des
réflexions sur des regards qui, n'étant que pour ce que je les donnai,
étaient en effet si peu ordinaires, que je mourais toujours de peur que
leur raison véritable ne lui vînt dans l'esprit, ou qu'au moins il n'en
eût des soupçons qui eussent rendu notre dessein ou inutile ou fort
embarrassant à achever. Le bonheur fut tel qu'il ne s'en douta jamais.

Rigault travailla le reste du jour et le lendemain encore sans plus voir
M. de la Trappe, duquel il avait pris congé, en se retirant d'auprès de
lui la troisième fois, et fit un chef-d'œuvre aussi parfait qu'il eût pu
réussir en le peignant à découvert sur lui-même. La ressemblance dans la
dernière exactitude, la douceur, la sérénité, la majesté de son visage,
le feu noble, vif, perçant de ses yeux si difficile à rendre, la finesse
et tout l'esprit et le grand qu'exprimait sa physionomie, cette candeur,
cette sagesse, paix intérieure d'un homme qui possède son âme, tout
était rendu, jusqu'aux grâces qui n'avaient point quitté ce visage
exténué par la pénitence, l'âge et les souffrances. Le matin je lui fis
prendre en crayon le père abbé assis au bureau de M. de la Trappe pour
l'attitude, les habits et le bureau même tel qu'il était, et il partit
le lendemain avec la précieuse tête qu'il avait si bien attrapée et si
parfaitement rendue, pour l'adapter à Paris sur une toile en grand, et y
joindre le corps, le bureau et tout le reste. Il fut touché jusqu'aux
larmes du grand spectacle du chœur et de la communion générale de la
grand'messe le jour de la Toussaint, et il ne put refuser au père abbé
une copie en grand pareille à mon original. Il fut transporté de
contentement d'avoir si parfaitement réussi d'une manière si nouvelle et
sans exemple, et dès qu'il fut à Paris, il se mit à la copie pour lui et
à celle pour la Trappe, travaillant par intervalles aux habits et au
reste de ce qui devait être dans mon original. Cela fut long, et il m'a
avoué que de l'effort qu'il s'était fait à la Trappe, et de la
répétition des mêmes images qu'il se rappelait pour mieux exécuter les
copies, il en avait pensé perdre la tête, et s'était trouvé depuis dans
l'impuissance pendant plusieurs mois de travailler du tout à ses
portraits. La vanité l'empêcha de me tenir parole malgré les mille écus
que je lui fis porter le lendemain de son arrivée à Paris. Il ne put se
tenir avec le temps, c'est-à-dire trois mois après, de montrer son
chef-d'œuvre avant de me le rendre, et par là de rendre mon secret
publie. Après la vanité vint le profit qui acheva de le séduire, et par
la suite, il a gagné plus de vingt-cinq mille livres en copies, de son
propre aveu, et c'est ce qui fit la publicité. Comme je vis que c'en
était fait, je lui en commandai moi-même après lui avoir reproché son
infidélité, et j'en donnai quantité.

Je fus très fâché du bruit que cela fit dans le monde, mais je me
consolai par m'être conservé pour toujours une ressemblance si chère et
si illustre, et avoir fait passer à la postérité le portrait d'un homme
si grand, si accompli et si célèbre. Je n'osai jamais lui avouer mon
larcin\,; mais, en partant de la Trappe, je lui en laissai tout le récit
dans une lettre par laquelle je lui en demandais pardon. Il en fut peiné
à l'excès, touché et affligé\,; toutefois il ne put me garder de colère.
Il me récrivit que je n'ignorais pas qu'un empereur romain disait\,:
qu'il aimait la trahison, mais qu'il n'aimait pas les traîtres\,; que
pour lui il pensait tout autrement, qu'il aimait le traître, mais qu'il
ne pouvait que haïr sa trahison. Je fis présent à la Trappe de la copie
en grand, d'une en petit, et de deux en petit, c'est-à-dire en buste, à
M. de Saint-Louis et à M. Maisne, que j'envoyai tout à la fois. M. de la
Trappe avait depuis quelques années la main droite ouverte, et ne s'en
pouvait servir. Dès que j'eus mon original où il est peint, la plume à
la main, assis à son bureau, je fis écrire cette circonstance derrière
la toile, pour qu'à l'avenir elle ne fît point erreur, et surtout la
manière dont il fut peint de mémoire, pour qu'il ne fût pas soupçonné de
la complaisance de s'y être prêté. J'arrivai à Paris la veille que le
roi devait arriver de Montargis à Fontainebleau avec la princesse, et je
m'y trouvai à la descente de son carrosse. J'avais espéré de cacher
ainsi parfaitement mon petit voyage.

Avant de parler de la princesse de Savoie, il faut dire un mot de ce qui
se passait en Italie. M. de Savoie, tout à fait déclaré et enhardi en
même temps par une manière de défaite assez considérable des Impériaux
en Hongrie par le Grand Seigneur en personne, parla plus haut sur la
neutralité. Leganez, gouverneur du Milanais, se laissait entendre qu'il
avait les pleins pouvoirs d'Espagne\,; Mansfeld, commissaire général de
l'empereur en Italie, s'y opposait toujours de sa part. On comprit ce
manège, et pour le mettre au net, M. de Savoie s'alla mettre le 15
septembre à la tête de l'armée du maréchal Catinat, pour entrer dans le
Milanais, et fit le siège de Valence. Sur quoi les alliés, qui n'avaient
rien voulu conclure avec le marquis de Saint-Thomas que M. de Savoie
leur avait envoyé à Milan, lui déclarèrent la guerre dans toutes les
formes\,; et, pour la faire compter comme bien certaine, envoyèrent en
même temps le cartel pour l'échange des prisonniers qui se feraient de
part et d'autre. Ce n'était qu'une dernière tentative. Ils se rendirent
bientôt traitables, et dans le 10 octobre la neutralité d'Italie fut
signée de part et d'autre, telle que M. de Savoie l'avait proposée, qui
en même temps leva le siège de Valence\,; et le maréchal Catinat ne
songea plus qu'à faire repasser les monts à son armée. Les restitutions
stipulées avec M. de Savoie lui furent faites\,; les ducs de Foix et de
Choiseul eurent liberté de revenir, et Gouvon, envoyé extraordinaire de
M. de Savoie, vint en remercier le roi, et, en attendant un ambassadeur,
se trouver à l'arrivée de la princesse. C'était un homme habile, de
beaucoup d'esprit et de politesse, fort fait aux cours, et qui plut
extrêmement à tout le monde. Le roi prit du goût {[}pour lui{]} et le
distingua jusqu'à le mener à Marly, familiarité que jusqu'à lui aucun
ministre étranger n'avait obtenue, et qui ne fut communiquée à aucun.

La maison de la princesse s'était arrêtée près de trois semaines à Lyon,
en attendant qu'elle fût à portée du Pont Beauvoisin, où elle la fut
recevoir. Elle y arriva de bonne heure, le mardi 16 octobre, accompagnée
de la princesse de La Cisterne et de M\textsuperscript{me} de Noyers. Le
marquis de Dronero était chargé de toute la conduite, auquel, ainsi
qu'aux officiers et aux femmes de sa suite, il fut distribué beaucoup de
beaux présents de la part du roi. Elle se reposa dans une maison qui lui
avait été préparée du côté de Savoie et s'y para. Elle vint ensuite au
pont, qui tout entier est de France, à l'entrée duquel elle fut reçue
par sa nouvelle maison et conduite au logis du côté de France qui lui
avait été préparé. Elle y coucha, et le surlendemain elle se sépara de
toute sa maison italienne sans verser une larme, et ne fut suivie
d'aucun que d'une seule femme de chambre et d'un médecin qui ne devait
pas demeurer en France, et qui en effet furent bientôt renvoyés.

Avant de passer outre, il ne faut pas oublier deux choses qui arrivèrent
en ce lieu, dont l'une fut cause du séjour que la princesse y fit. Le
comte de Brionne, chargé au nom du roi de recevoir la princesse du
marquis de Dronero qui la livrait au nom de M. de Savoie, prétendit être
traité d'Altesse dans l'instrument de la remise où le duc de Savoie
était traité d'Altesse royale\,; et il s'y opiniâtra si bien, quoi qu'on
pût lui dire des deux côtés, que le marquis de Dronero, pour ne point
arrêter plus longtemps la princesse, ôta l'Altesse des deux côtés en
évitant de faire mention expresse de M. le duc de Savoie. Ce prince fut
extrêmement offensé quand il apprit la difficulté du comte de Brionne,
et le roi le trouva aussi fort mauvais, mais la chose était faite et
terminée, et il ne s'en parla plus.

L'autre chose qui y arriva fut par un courrier du roi par lequel il
arriva un ordre de traiter la princesse en tout comme fille de France,
et comme ayant déjà épousé Mgr le duc de Bourgogne. L'embarras de son
rang avec tout le monde engagea Monsieur à en prier le roi, les princes
et princesses du sang à le désirer, et le roi à le faire. Ce courrier
arriva sur le point de l'arrivée de la princesse, de manière qu'elle ne
baisa que la duchesse du Lude et le comte de Brionne, et qu'il n'y eut
que la duchesse du Lude assise devant elle. Par toutes les villes où
elle passa, elle fut reçue comme duchesse de Bourgogne, et aux jours de
séjour aux grandes villes, elle dîna en public servie par la duchesse du
Lude\,; excepté les repas de séjour, ses dames mangèrent toujours avec
elle. Elle marcha à petites journées.

Le dimanche 4 novembre, le roi, Monseigneur et Monsieur allèrent
séparément à Montargis au-devant de la princesse, qui y arriva à six
heures du soir, et fut reçue par le roi à la portière de son carrosse.
Il la mena dans l'appartement qui lui était destiné dans la même maison
de la ville où le roi était logé, puis lui présenta Monseigneur,
Monsieur et M. le duc de Chartres. Tout ce qui fut rapporté des
gentillesses et des flatteries pleines d'esprit, et du peu d'embarras,
et avec cela de l'air mesuré et des manières respectueuses de la
princesse, surprit infiniment tout le monde et charma le roi dès
l'abord. Il la loua sans cesse et la caressa continuellement. Il se hâta
d'envoyer un courrier à M\textsuperscript{me} de Maintenon \footnote{Voy.,
  à la fin du volume, la lettre que Louis XIV écrivit de Montargis à
  M\textsuperscript{me} de Maintenon.}, pour lui mander sa joie et les
louanges de la princesse. Il soupa ensuite avec les dames du voyage, et
fit mettre la princesse entre lui et Monseigneur.

Le lendemain le roi l'alla prendre, la mena à la messe, et dîna ensuite
comme il avait soupé la veille, et aussitôt après montèrent en carrosse,
le roi et Monsieur au derrière, Monseigneur et la princesse au devant,
de son côté à la portière la duchesse du Lude. Mgr le duc de Bourgogne
les rencontra à Nemours, le roi le fit monter à l'autre portière, et sur
les cinq heures du soir arrivèrent à Fontainebleau, dans la cour du
Cheval-Blanc. Toute la cour était sur le fer à cheval, qui faisait un
très beau spectacle avec la foule qui était en bas. Le roi menait la
princesse qui semblait sortir de sa poche, et la conduisit fort
lentement à la tribune un moment, puis au grand appartement de la reine
mère qui lui était destiné, où Madame avec toutes les dames de la cour
l'attendaient. Le roi lui nomma les premiers d'entre les princes et
princesses du sang, puis dit à Monsieur de lui nommer tout le monde, et
de prendre garde à lui faire saluer toutes les personnes qui le devaient
faire, et qu'il allait se reposer. Monseigneur s'en alla aussi, l'un
chez M\textsuperscript{me} de Maintenon, l'autre chez
M\textsuperscript{me} la princesse de Conti, qui ne s'habillait pas
encore, d'une loupe qu'elle s'était fait ôter de dessus un œil et
qu'elle en avait pensé perdre. Monsieur demeura donc à côté de la
princesse tous deux debout, lui nommant tout ce qui, hommes et dames,
lui venaient baiser le bas de la robe, et lui disait de baiser les
personnes qu'elle devait, c'est-à-dire princes et princesses du sang,
ducs et duchesses et autres tabourets, les maréchaux de France et leurs
femmes. Cela dura deux bonnes heures, puis la princesse soupa seule dans
son appartement, où M\textsuperscript{me} de Maintenon et
M\textsuperscript{me} la princesse de Conti ensuite, la virent en
particulier. Le lendemain elle fut voir Monsieur et Madame chez eux, et
Monseigneur chez M\textsuperscript{me} la princesse de Conti, et reçut
force bijoux et pierrerie\,; et le roi envoya toutes les pierreries de
la couronne à M\textsuperscript{me} de Mailly pour en parer la princesse
tant qu'elle voudrait.

Le roi régla qu'on la nommerait tout court \emph{la princesse}, qu'elle
mangerait seule, servie par la duchesse du Lude, qu'elle ne verrait que
ses dames et celles à qui le roi en donnerait expressément la
permission, qu'elle ne tiendrait point encore de cour, que Mgr le duc de
Bourgogne n'irait chez elle qu'une fois tous les quinze jours, et MM.
ses frères une fois le mois. Toute la cour retourna le 8 novembre à
Versailles, où la princesse eut l'appartement de la reine, et de
M\textsuperscript{me} la Dauphine ensuite, et où, en arrivant, tout ce
qui était demeuré à Paris de considérable se trouva et lui fut présenté
tout de suite comme à Fontainebleau. Le roi et M\textsuperscript{me} de
Maintenon firent leur poupée de la princesse, dont l'esprit flatteur,
insinuant, attentif leur plut infiniment, et qui peu à peu usurpa avec
eux une liberté que n'avait jamais osé tenter pas un des enfants du roi,
et qui les charma. Il parut que M. de Savoie était bien informé à fond
de notre cour, et qu'il avait bien instruit sa fille\,; mais ce qui fut
vraiment étonnant, c'est combien elle en sut profiter, et avec quelle
grâce elle sut tout faire. Rien n'est pareil aux cajoleries dont elle
sut bientôt ensorceler M\textsuperscript{me} de Maintenon, qu'elle
n'appela jamais que \emph{ma tante}, et avec qui elle en usa avec plus
de dépendance et de respect qu'elle n'eût pu faire pour une mère et pour
une reine, et avec cela une familiarité et une liberté apparentes qui la
ravissaient et le roi avec elle.

M\textsuperscript{lle}s de Soissons, qui tenaient dans Paris une
conduite fort étrange et qui ne venaient point à la cour, eurent défense
de voir la princesse. Elles étaient sœurs du comte de Soissons et du
prince Eugène de Savoie\,: celui-ci au service de l'empereur et parvenu
aux premiers grades militaires, l'autre sorti de France depuis un an ou
deux, où il avait toujours demeuré, et rôdant l'Europe sans obtenir
d'emploi nulle part.

\hypertarget{chapitre-xxv.}{%
\chapter{CHAPITRE XXV.}\label{chapitre-xxv.}}

1696

~

\relsize{-1}

{\textsc{Plénipotentiaires nommés pour la paix.}} {\textsc{- Harlay
conseiller d'État.}} {\textsc{- Courtin conseiller d'État.}} {\textsc{-
Courtin, Harlay et le duc de Chaulnes.}} {\textsc{- Callières.}}
{\textsc{- Candidats pour la Pologne.}} {\textsc{- Prince de Conti.}}
{\textsc{- Princes Constantin et Alexandre Sobieski, bien qu'incognito,
baisent la princesse.}} {\textsc{- Vaine entreprise de
M\textsuperscript{me} de Béthune de baiser la princesse.}} {\textsc{-
Mariage de Coetquen avec une fille du duc de Noailles.}} {\textsc{- Mort
de l'abbé Pelletier, conseiller d'État\,; du duc de Roannais.}}
{\textsc{- M\textsuperscript{me} de Saint-Géran exilée.}} {\textsc{-
Disgrâce de Rubantel.}} {\textsc{- M\textsuperscript{me} de Castries
dame d'atours de M\textsuperscript{me} la duchesse de Chartres.}}
{\textsc{- M\textsuperscript{me} de Jussac auprès de
M\textsuperscript{me} la duchesse de Chartres.}} \relsize{1}

~

Le roi, qui tenait depuis quelque temps Caillières secrètement en
Hollande, l'y fit paraître comme son envoyé public après la neutralité
d'Italie, et ne différa guère à nommer ses plénipotentiaires en
Hollande, pour travailler à la paix, Courtin et Harlay, conseillers
d'État, ce dernier gendre du conseiller, et Crécy en troisième. J'ai
déjà fait connaître ce dernier. Harlay avait déjà été inutilement sur
les frontières de Hollande. C'était un homme d'esprit et fort du monde,
qui avait été longtemps intendant en Bourgogne et qui aimait le faste.
Le jugement ne répondait pas à l'esprit, et il était glorieux comme tous
les Harlay, mais il ne tenait pas tant de leurs humeurs et de leurs
caprices. En général son ambition le rendait poli et cherchant à plaire
et à se faire aimer. Il demeura, tôt après et avant même de partir,
premier plénipotentiaire, parce que Courtin qui perdait les yeux
s'excusa. C'était un très petit homme, bellot, d'une figure assez
ridicule, mais plein d'esprit, de sens, de jugement, de maturité et de
grâces, qui avait vieilli dans les négociations, longtemps ambassadeur
en Angleterre, et qui avait plu et réussi partout. Il avait été ami
intime de M. de Louvois. Le roi lui parlait toutes les fois qu'il le
voyait, et le menait même quelquefois à Marly, et c'était le seul homme
de robe qui eût cette privance, et la distinction encore de paraître
devant le roi et partout sans manteau comme les ministres. Pelletier de
Sousi, frère du ministre, l'usurpa à son exemple depuis que le roi lui
eut donné les fortifications, à la mort de M. de Louvois, qui le
faisaient aller à Marly, mais seulement coucher deux nuits pour ses
jours d'y travailler avec le roi.

Pour mieux faire connaître ces deux hommes qui ont tant influé au
dehors, surtout Courtin, aux principales affaires, j'en veux rapporter
deux aventures de leur vie. Tous deux étaient amis de M. de Chaulnes.
Courtin étant intendant en Picardie, M. de Chaulnes lui recommanda fort
ses belles terres de Chaulnes, Magny et Picquigny, qui sont d'une grande
étendue, et Courtin ne put lui refuser le soulagement qu'il demandait.
La tournée faite, M. de Chaulnes fut fort content, et il espéra que cela
continuerait de même\,; mais Courtin, venu à l'examen de ses
impositions, trouva qu'il avait fort surchargé d'autres élections de ce
qu'il avait ôté aux terres de M. de Chaulnes. Cela allait loin, le
scrupule lui en prit\,; il n'en fit pas à deux fois, il rendit du sien
ce qu'il crut avoir imposé de trop à chaque paroisse par le soulagement
qu'il avait fait à celles de M. de Chaulnes, et quitta l'intendance sans
que le roi l'y pût retenir. Le roi avait tant de confiance en lui pour
les affaires de la paix, qu'il le pressa de demeurer plénipotentiaire en
consentant que M\textsuperscript{me} de Varangeville sa fille en eût le
secret et écrivit tout sous lui, mais il ne put se résoudre au voyage ni
au travail. Avec ses yeux sa santé diminuait. Il avait été fort galant
et avait passé toute sa vie dans les affaires et dans le plus grand
monde, où il était fort goûté, et il voulut absolument mettre un
intervalle entre la vie et la mort\,; aussi ne parut-il guère depuis et
demeura fort retiré chez lui.

M. d'Harlay, avec une figure de squelette et de spectre, était galant
aussi. Le chancelier Boucherat, son beau-père, était ami intime de M. de
Chaulnes, et M. de Chaulnes, au temps de cette aventure, était aux
couteaux tirés avec M. de Pontchartrain, premier président du parlement
de Rennes tous deux en Bretagne, et tous deux remuant l'un contre
l'autre tout ce qu'ils pouvaient à la cour, à qui aurait le dessus dans
leurs prétentions. Pontchartrain était aussi fort galant, et il avait à
Paris un commerce de lettres avec une femme avec qui il était fort bien,
et qui avait la confiance de tous ses ressorts contre M. de Chaulnes. Le
diable fit qu'Harlay devint amoureux de cette même femme, et qu'elle
crut tout accommoder, en ne se rendant pas cruelle au nouvel amant pour
mieux servir l'autre. Le chancelier était instruit de tout par M. de
Chaulnes, il était déclaré pour lui contre Pontchartrain. Tout ce qui se
tramait pour l'un contre l'autre se passait sous les yeux de Boucherat,
et fort souvent par son ministère. Il aimait passionnément
M\textsuperscript{me} d'Harlay, sa fille, et ne cachait rien à Harlay
qui logeait avec lui. L'amour corrompit ce dernier jusqu'à livrer son
ami à sa maîtresse, et à lui rendre compte de tout ce qui se passait de
plus secret contre Pontchartrain.

Ce manège eut à peine duré deux ou trois mois, qu'il se présenta une
question fort importante pour les deux ennemis, sur laquelle tous les
ressorts furent mis en mouvement de part et d'autre. Au plus fort de ces
intrigues, Harlay vint de Versailles descendre chez sa dame qui trouva
son récit si important, qu'elle exigea de lui de mettre par écrit toute
sa découverte, tandis qu'elle écrirait à part à Pontchartrain pour ne
lui pas envoyer un volume sous la même enveloppe. Harlay était las, il
fallut obéir et écrire chez cette femme\,: l'écriture fut longue et
détaillée. Le cabasset s'échauffa, sa tête se remplit du nom de M. de
Chaulnes, tellement et si bien qu'il cachette sa lettre, met le dessus à
M. de Chaulnes au lieu de M. de Pontchartrain, et comme il était jour de
poste et que l'heure pressait, s'en va et la donne à un laquais pour la
mettre à la poste, et se couche très fatigué. On peut juger de la
surprise de M. de Chaulnes qui connaissait parfaitement l'écriture de M.
d'Harlay, sur l'amitié intime et le secours duquel il comptait en toute
confiance et personnellement et par rapport au chancelier, quand il se
vit trahi de la sorte, et la douleur de Pontchartrain de ne point
recevoir les avis importants d'Harlay, annoncés par la lettre de son
amie. Ils ne surent ce que la lettre était devenue, mais Harlay se
souvint de sa méprise, fut outré, mais n'osa en avertir.

Le voilà dans une peine étrange de la juste colère de M. de Chaulnes, et
de l'usage qu'il ferait de sa trahison. Il se voyait perdu auprès de son
beau-père, et pour le monde dans un prédicament à le noyer, et en même
temps bien ridicule à son tige. Son parti fut le silence et d'attendre
la bombe. M. de Chaulnes, de son côté, sut profiter d'une si lourde
méprise, et ne sut pas moins n'en faire aucun semblant. Harlay aux
écoutes tremblait à chaque ordinaire de Bretagne, et respirait jusqu'au
suivant\,; mais il transit lorsqu'il sut M. de Chaulnes en chemin de
Paris.

Il avait accoutumé, les premiers jours de ses retours à Paris, de donner
à dîner au chancelier et à sa famille avec quelques amis les plus
particuliers. Jusque-là Harlay avait caracolé pour éviter partout M. de
Chaulnes et pour l'aller chercher chez lui, lorsqu'il s'était bien
assuré de ne le trouver pas. Mais le cœur lui battait du dîner, s'il en
serait prié à l'ordinaire, s'il irait étant prié, et s'il y allait, ce
qu'il y deviendrait, et quelle scène il y pourrait essuyer devant son
beau-père. Il y fut prié, et il y alla comme un homme qu'on mène à la
potence. M. de Chaulnes avait malicieusement fait tomber ce dîner à un
jour d'ordinaire de Bretagne. La compagnie arrive, est reçue avec
l'amitié ordinaire, mais pas un mot à M. d'Harlay. Vers le moment de
servir, M. de Chaulnes regarde sa pendule, se tourne au chancelier, lui
dit qu'on va dîner, qu'il est jour d'ordinaire de Bretagne, que toutes
ses lettres sont faites, mais qu'il lui demande la permission de passer
un demi-quart d'heure dans son cabinet, parce que sa coutume est
toujours de les voir lui-même fermer, et regardant Harlay entre deux
yeux, et mettre le dessus à ses lettres pour éviter les méprises qui
arrivent quelquefois, et qui peuvent être fâcheuses, et tout de suite en
souriant et toujours regardant Harlay, va dans son cabinet. Harlay, à ce
qu'il a dit depuis à Valincourt qui me l'a conté, pensa évanouir, et se
trouva effectivement assez mal pour le craindre\,; il le cacha pourtant,
à quoi sa naturelle pâleur de mort le servit bien. Le maître d'hôtel
vint avertir M. de Chaulnes, qui riait dans son cabinet et
s'épanouissait de sa vengeance, sortit, fit passer le chancelier et les
dames, prit Harlay par la main, et souriant toujours\,: « Allons,
monsieur, et buvons ensemble\,: voilà comme je sais me venger.\,» À ces
mots l'autre pensa fondre\,; il ne put répondre une parole\,; il dîna
mal, trouva qu'on dînait longtemps, et disparut dès qu'il le put sans
trop d'affectation. Jamais il n'en a été question depuis de la part de
M. de Chaulnes, et Harlay ne sachant plus que devenir avec un homme si
offensé et si trahi, et en même temps si sage, si modéré, si maître de
soi-même, il en pensa mourir de honte et de douleur. De ces deux
plénipotentiaires il y a loin en soi, et avec le même duc de Chaulnes.

Caillières fut enfin déclaré le troisième. C'était un Normand attaché en
sa jeunesse à MM. de Matignon, pour qui il conserva toute sa vie
beaucoup de respect et de mesure. Son père avait été à eux. Il avait
beaucoup de lettres, beaucoup d'esprit d'affaires et de ressources, et
fort sobre et laborieux, extrêmement sûr et honnête homme. Je ne sais
qui le produisit pour aller secrètement en Pologne, lorsqu'il y fut
question de l'élection du comte de Saint-Paul. Il s'y conduisit fort
bien, et y lia une grande amitié avec Morstein, grand trésorier de
Pologne, qui était fort français, et avait fort travaillé pour
l'élection du comte de Saint-Paul, qui ne manqua que par la mort de ce
candidat, tué au passage du Rhin. Callières, qui se trouvait bien de
Morstein, demeura avec lui, et comme ce sénateur était tout français,
son témoignage fit employer Caillières, tout porté sur les lieux, en
plusieurs négociations obscures dans le Nord, et même en Hollande. On
fut content du compte qu'il en vint rendre plusieurs fois, et il
s'acquit plusieurs amis partout où il avait été. Morstein, s'étant
brouillé en Pologne jusqu'à craindre pour sa liberté et pour sa vie,
avait, dans l'appréhension de l'orage naissant, fait passer de gros
fonds en France, et les y suivit avec Caillières quand il crut qu'il en
était temps. Il s'établit à Paris en homme fort riche, et logea son ami
avec lui. Il n'avait qu'un fils, dont j'ai parlé sur le siège de Namur,
où il fut tué. Le père avait acquis de grandes terres, entre autres
celles de la maison de Vitry, et cherchait à appuyer son fils d'une
grande alliance. M. de Chevreuse, plus touché de la grande raison de
sans dot, dans le mauvais état de ses affaires, que du désagrément de
prendre un proscrit de Pologne tombé ici des nues pour gendre, en écouta
volontiers la proposition. Caillières en fut le négociateur pour
Morstein, et comme celui-ci était détaché de toute autre chose que de
l'alliance, l'affaire fut bientôt conclue, et Caillières s'acquit les
bonnes grâces de M. de Chevreuse. La mort du fils, puis du père,
suivirent d'assez près le mariage. Caillières se livra à la protection
de M. de Chevreuse, à qui il plut par ses lettres et par son esprit
d'affaires et de raisonnement, et par le soin qu'il prit des affaires
des deux filles que son gendre avait laissées.

C'était la vie et l'occupation de Callières, lorsque le hasard lui fit
rencontrer dans les rues de Paris un marchand hollandais fort de ses
amis et fort accrédité dans son pays, venu à Paris pour des affaires de
prises et de négoces\,; ils renouvelèrent connaissance et amitié,
parlèrent de la guerre et de la paix, et raisonnèrent tant ensemble, que
le marchand lui avoua de bonne foi le besoin et le désir qu'avait sa
république de la paix. Ils approfondirent si bien que Caillières crut en
devoir rendre compte à M. de Chevreuse. Il n'était qu'un avec le duc de
Beauvilliers, son beau-frère, qui était dans le conseil\,; il lui mena
Caillières\,; son récit fut goûté. Ces messieurs le firent voir à
Croissy, oncle de leurs femmes, et à Pomponne, leur ami, qui était aussi
ministre, et de toutes ses conversations\,: Caillières fut envoyé
secrètement en Hollande. Il revint quelques mois après, et fut encore
renvoyé, et de ce dernier voyage il conduisit les affaires au point que
lés principales difficultés se trouvèrent levées au commencement de
l'hiver, et qu'il eut ordre de paraître publiquement comme envoyé du roi
en Hollande. On a vu que Courtin s'excusa d'être plénipotentiaire pour
la paix, et que son collègue Harlay l'étant devenu, Crécy le fut
nommé\,; on l'y voulait pour sa capacité et son expérience, porté par le
P. de La Chaise et les jésuites. L'exemple d'un homme de si peu fit
mettre Caillières en troisième, qui avait seul conduit l'affaire au
point où elle était, et qui était instruit de tout à fond.

C'était un grand homme maigre, avec un grand nez, la tête en arrière,
distrait, civil, respectueux, qui, à force d'avoir vécu parmi les
étrangers, en avait pris toutes les manières, et avait acquis un
extérieur désagréable, auquel les dames et les gens du bel air ne purent
s'accoutumer, mais qui disparaissait dès qu'on l'entretenait de choses
et non de bagatelles. C'était en tout un très bon homme, extrêmement
sage et sensé, qui aimait l'État et qui était fort instruit, fort
modeste, parfaitement désintéressé, et qui ne craignait de déplaire au
roi ni aux ministres pour dire la vérité, et ce qu'il pensait et
pourquoi jusqu'au bout, et qui les faisait très souvent revenir à son
avis.

Le roi traitait une autre affaire pour laquelle il avait hâté le retour
des princes de l'armée, pour qu'il ne parût auquel d'eux il avait à
parler. L'abbé de Polignac, ambassadeur en Pologne, crut y voir jour à
l'élection en faveur de M. le prince de Conti. Il le manda, et le roi,
qui ne demandait pas mieux que de se défaire d'un prince de ce mérite si
universellement connu, et qu'il n'avait jamais pu aimer, tourna toutes
ces pensées à le porter sur ce trône. Les candidats qui s'y présentaient
étaient les électeurs de Bavière, Saxe et palatin, le duc de Lorraine\,;
et bien que les Polonais se déclarassent contre tout Piaste \footnote{On
  appelait ainsi les rois de Pologne qui étaient eux-mêmes Polonais,
  comme Jean Sobieski.}, les fils du feu roi y auraient eu grande part,
tant par une coutume assez ordinaire que par le mérite d'un aussi grand
homme que l'était J. Sobieski, si l'avarice extraordinaire de la reine,
qui avait tout vendu et rançonné, et la hauteur de ses manières n'eût
rendu ses enfants odieux à cause d'elle, et si elle eût été plus
d'accord avec eux. Jacques, l'aîné, était fort mal avec elle, niais il
était né avant l'élection de son père, ce qui le défavorisait fort\,; il
était d'ailleurs peu aimé, et son mariage avec une palatine, sœur de
l'impératrice, le rendait suspect. L'empereur le portait, sa mère le
traversait\,; elle voulait un de ses deux cadets\,; mais ses trésors lui
étaient plus chers encore. Bavière était son gendre, avait pour lui la
mémoire du feu roi et d'être homme de guerre. Saxe avait aussi cette
dernière qualité et son voisinage, qui avait fait connaître la douceur
de ses mœurs et sa libéralité. Le duc de Lorraine était fils d'une sœur
de l'empereur, qui avait été reine de Pologne, et d'un des plus grands
capitaines de son siècle, plus effectivement porté par l'empereur que
Jacques Sobieski. Enfin le prince Louis de Bade se mit aussi sur les
rangs comme un capitaine expérimenté, peut-être plus pour l'honneur d'y
prétendre que par aucune espérance d'y réussir.

La naissance du prince de Conti, si supérieure à celle de ces candidats,
ses qualités aimables et militaires, qui s'étaient fait connaître en
Hongrie, et qu'il avait si bien soutenues depuis, la qualité de neveu et
d'élève de ce fameux prince de Condé, et celle d'héritier et de cousin
germain du comte de Saint-Paul, qui était encore regretté en Pologne, et
dont il avait réuni tous les suffrages lorsqu'il mourut, firent tout
espérer à l'abbé de Polignac, qui voyait pour soi le chapeau de cardinal
pour récompense, dont les Polonais sont peu amoureux, et que leurs rois
donnent fort ordinairement à des étrangers, de la façon desquels nous en
avions en France. Le roi voulut donc voir ce que le prince de Conti
pourrait faire. Il l'entretint plusieurs fois en particulier, ce qui ne
lui arrivait guère. Il vendit pour six cent mille livres de terres à des
gens d'affaires, avec la faculté de les pouvoir reprendre dans trois ans
pour le même prix\,; cette somme fut envoyée en Pologne, et le roi
promit de la rendre si l'élection ne réussissait pas.

Pendant un temps si critique pour les candidats, les princes Alex. et
Const. Sobieski voyageaient et vinrent jusqu'à Paris pour y recevoir
l'ordre, qu'ils portaient dès avant la mort du roi leur père, qui
l'avait instamment demandé pour eux. Pour sonder les traitements qu'ils
désiraient, ils demeurèrent incognito, et néanmoins le roi leur donna
comme aux gens titrés la distinction de baiser la princesse et Madame.
M\textsuperscript{me} de Béthune, sœur de la reine leur mère, arrivait
aussi de Pologne, où son mari avait été longtemps ambassadeur, et était
mort en la même qualité en Suède. Elle avait été dame d'atours de la
reine en survivance de sa belle-mère, sœur du duc de Saint-Aignan.
C'était une femme d'esprit, hardie, entreprenante, qui, à l'abri de ses
neveux Sobieski, se mit dans la tête de faire accroire que, parce
qu'elle avait été dame d'atours de la reine, elle devait baiser les
filles de France. Madame en fut la dupe et la baisa. Avec cet exemple,
par lequel elle avait commencé, elle crut être admise au même honneur
par la princesse. Mais la duchesse du Lude, à la cour de tout temps, et
qui savait et avait vu le contraire, n'osa le prendre sur elle. Le roi,
informé de la prétention, la trouva impertinente et fausse, et fort
mauvais que Madame s'y fût laissé tromper. M\textsuperscript{me} de
Béthune, qui savait fort bien que sa prétention était une entreprise, la
laissa promptement tomber, et fut présentée à la princesse sans la
baiser.

Coetquen en arrivant épousa la seconde fille du duc de Noailles\,: il
n'avait point de père, était riche et fils de M\textsuperscript{me} de
Coetquen, célèbre par la passion de M. de Turenne, et le secret de Gand
qui lui échappa\,; elle était sœur du duc de Rohan, de
M\textsuperscript{me} de Soubise, dont la beauté a fait une si éclatante
fortune, et de la princesse d'Espinoy, tous enfants de l'héritière de
Rohan qui épousa le Chabot. Ainsi le père et les filles devinrent
célèbres par le bonheur de l'amour. Coetquen n'en tint rien\,: il
épousa, pour le crédit des Noailles, la plus laide et la plus dégoûtante
créature qu'on sût voir, et il prétendit plaisamment qu'on lui avait
fait voir la troisième qui était jolie, puis qu'on l'avait trompé et
donné l'autre. Le mariage aussi fut peu heureux.

L'année finit par deux morts et deux disgrâces\,: l'abbé Pelletier,
conseiller d'État, habile, mais fort rustre, qui mourut d'apoplexie
presqu'en sortant de dîner chez son frère, le ministre d'État, et le duc
de Roannais. Il avait perdu son père avant son grand-père, auquel il
avait succédé au gouvernement de Poitou et à sa dignité en 1642. Faute
de pairs, rares alors et dispersés dans leurs gouvernements dans ces
temps de troubles, il eut l'honneur de représenter le comte de Flandre
au sacre du roi n'ayant pas trente ans. C'était un homme de beaucoup
d'esprit et de savoir, qui tourna de fort bonne heure à la retraite et à
une grande dévotion qui l'éloigna absolument du mariage. M. de La
Feuillade en profita dans sa faveur. Il traita avec lui, lui donna gros
du duché de Roannais, épousa sa sœur en avril 1667, et sur sa démission,
en conservant le rang et les honneurs, obtint pour soi une érection
nouvelle, vérifiée au parlement en août la même année. Bientôt après, M.
de Roannais ne parut plus, prit une manière d'habit d'ecclésiastique
sans être jamais entré dans les ordres, et vécut dans une grande piété
et dans une profonde retraite, et mourut de même fort âgé à Saint-Just,
près Méry-sur-Seine.

Rubantel et M\textsuperscript{me} de Saint-Géran furent les deux
disgraciés\,: j'ai assez parlé de celle-ci pour n'avoir rien à y
ajouter. Elle était fort bien avec les princesses, et mangeuse, aimant
la bonne chère, et bonne en privé comme M\textsuperscript{me} de
Chartres et M\textsuperscript{me} la Duchesse. Cette dernière avait une
petite maison dans le parc de Versailles, auprès de la porte de Sartori
qu'elle appelait le Désert, que le roi lui avait donnée pour l'amuser et
qu'elle avait assez joliment ajustée pour s'y aller promener et faire
des collations. Les repas se fortifièrent, devinrent plus gais, et à la
fin mirent M. le Duc de mauvaise humeur, et M. le Prince en impatience.
Ils se fâchèrent inutilement, et à la fin ils portèrent leurs plaintes
au roi, qui gronda M\textsuperscript{me} la duchesse et lui défendit
d'allonger ces sortes de repas, et surtout d'y mener certaine compagnie.
Si M\textsuperscript{me} de Saint-Géran ne fut pas du nombre des
interdites, elle le dut à sa première année de deuil, pendant laquelle
le roi ne crut pas qu'elle pût être de ces parties, mais il s'expliqua
assez sur elle pour que M\textsuperscript{me} la duchesse ne pût pas
douter qu'elle n'était pas approuvée pour en être. Quelques mois se
passèrent avec plus de ménagement, et M\textsuperscript{me} la duchesse
compta que tout était oublié. Sur ce pied-là elle pressa
M\textsuperscript{me} de Saint-Géran de venir souper avec elle de bonne
heure au Désert, pour être au cabinet au sortir du souper du roi à
l'ordinaire. M\textsuperscript{me} de Saint-Géran craignit, se
défendit\,; mais, comme elle aimait à se divertir et qu'elle ne laissait
pas d'être imprudente, elle espéra qu'on ne saurait pas qu'elle y aurait
été, que sa première année de deuil détournerait même le soupçon, et que
M\textsuperscript{me} la duchesse paraissant le soir au cabinet, il n'y
aurait rien à reprendre. Elle se laissa donc aller\,; et, comme elle
était de fort bonne compagnie, elle mit si bien tout en gaieté, que
l'heure de retourner à temps pour le cabinet était insensiblement
passée, le repas et ses suites gagnèrent fort avant dans la nuit. Voilà
M. le Duc et M. le Prince aux champs, et le roi en colère, qui voulut
savoir qui était du souper. M\textsuperscript{me} de Saint-Géran fut
nommée\,; sa première année de deuil aggrava le crime\,; tout tomba sur
elle\,: elle fut exilée à vingt lieues de la cour, sans fixer le lieu,
et M\textsuperscript{me} la Duchesse bien grondée. En femme d'esprit,
M\textsuperscript{me} de Saint-Géran choisit Rouen, et dans Rouen le
couvent de Bellefonds dont une de ses parentes était abbesse. Elle dit
qu'ayant eu le malheur de déplaire au roi, il n'y avait pour elle qu'un
couvent\,; et cela fut fort approuvé.

Rubantel était un homme de peu, qui, à force d'acheter et de longueur de
temps, était devenu lieutenant-colonel du régiment des gardes et ancien
lieutenant général. Il l'était fort bon, fort entendu pour l'infanterie,
fort brave homme, fort honnête homme et fort estimé, une grande valeur
et un grand désintéressement, et vivant fort noblement à l'armée où il
était employé tous les ans comme lieutenant général. Avec ces qualités,
il était épineux, volontiers chagrin et supportait impatiemment des
vétilles et des détails du maréchal de Boufflers, dans le régiment des
gardes. Le maréchal eut beau faire pour lui adoucir l'humeur, plus
Rubantel en recevait d'avances, plus il se croyait compté, et plus il
était difficile, tant qu'à la fin la froideur succéda, et bientôt la
brouillerie et les plaintes. Rubantel, quoique difficile à vivre, était
aimé, parce qu'il avait toujours de l'argent et qu'il le prêtait fort
librement et obligeamment\,: cela lui avait attaché beaucoup de gens
dans le régiment des gardes, outre ce qui se trouve toujours dans un
grand corps de frondeurs et de mécontents qui se ralliaient à lui. À la
fin, le maréchal de Boufflers, fatigué de tout cela, proposa au roi de
tirer honnêtement Rubantel du régiment des gardes, avec lequel il n'y
avait plus moyen pour lui de demeurer. Le roi, qui de longue main
connaissait l'humeur de Rubantel, qui aimait le maréchal et qui était
jaloux de la subordination, fit dire par Barbezieux à Rubantel qu'il lui
permettait de vendre sa compagnie, lui continuait sa pension de quatre
mille livres et qu'il lui donnait le gouvernement du fort de Barreaux,
qu'il ne lui aurait pas donné sans l'instante prière de M. de Boufflers,
par le mécontentement qu'il avait de sa conduite avec ce maréchal son
colonel\,; et d'Avejan, premier capitaine aux gardes, fut
lieutenant-colonel. C'était à Versailles que Rubantel reçut ce discours.
Il en fut si outré qu'il ne voulut d'aucune grâce, s'en alla à Paris
sans voir le roi, et ne l'a jamais revu ni songé à servir depuis.

Au retour de l'armée, nous trouvâmes M\textsuperscript{me} de Castries
établie à la cour dame d'atours de M\textsuperscript{me} la duchesse de
Chartres, au lieu de M\textsuperscript{me} de Mailly. Par la bâtardise
de cette princesse, M\textsuperscript{me} de Castries était sa cousine
germaine, enfants du frère et de la sœur. L'état triste où se trouva le
cardinal Bonzi, après un fort brillant, avait fait son mariage. Il se
trouvera peut-être ailleurs occasion de parler de lui, sans faire ici
une trop longue parenthèse. Il suffit de dire qu'après s'être fort
distingué en diverses ambassades et avoir eu, du consentement du roi, la
nomination de Pologne, passé par les sièges de Béziers, Toulouse et
Narbonne, il avait été longtemps roi de Languedoc par l'autorité de sa
place, son crédit à la cour et l'amour de la province. Bâville, qui y
était intendant, second fils du premier président Lamoignon, y voulut
régner, et en sut venir à bout. L'abaissement du cardinal lui fut
insupportable\,; il tâcha de se relever, tous ses efforts furent
inutiles. Sa sœur unique, qu'il aimait tendrement, avait épousé M. de
Castries du nom de La Croix, qui était riche pour une fille qui n'avait
rien. Il était veuf, sans enfants, de la mère de M. de La Feuillade et
de M. de Metz. La faveur de son beau-frère lui procura le gouvernement
de Montpellier, ensuite une des trois lieutenances générales de
Languedoc, enfin l'ordre du Saint-Esprit, en 1661, et il fut un de ceux
que le duc d'Arpajon reçut à Pézenas, avec M. le prince de Conti, par
commission du roi. Il mourut en 1674, à soixante-trois ans, et laissa
des filles et deux fils dont l'aîné se distingua extrêmement à la guerre
par sa capacité et par des actions brillantes de valeur. C'était
d'ailleurs un homme pétri d'honneur et de vertu, doux, sage, poli, fort
aimé et de bonne compagnie. Il lutta longtemps contre sa mauvaise santé
et un asthme qu'il eut dès sa première jeunesse, mais qui fut à la fin
le plus fort, et le força, près d'être maréchal de camp, à quitter un
métier auquel il était propre, qu'il aimait avec passion et qui l'aurait
apparemment mené loin.

M. du Maine était gouverneur de Languedoc\,; le cardinal Bonzi, à bout
de douleur et de ressources, en chercha dans cet appui, et c'est ce qui
fit le mariage de son neveu. M. du Maine s'en chargea, le régla et le
conclut. Cela n'était pas difficile\,: M\textsuperscript{lle} de Vivonne
n'avait rien que sa naissance, et le cardinal et sa sœur ne cherchaient
qu'une grande alliance et un soutien domestique contre Bâville.
M\textsuperscript{me} de Montespan fit la noce en mai 1693, chez elle, à
Saint-Joseph, et se chargea de loger et nourrir les mariés. M. du Maine
promit merveilles, et, à son ordinaire, ne tint rien. Il ménageait son
crédit pour soi tout seul, et se serait bien gardé de choquer le dégoût
du roi pour la conduite du cardinal Bonzi ni ses ministres, et le goût
qu'ils lui avaient donné pour Bâville\,; mais à l'égard de la place de
dame d'atours de M\textsuperscript{me} la duchesse de Chartres peu
courue, et par des gens dont M. du Maine n'avait aucune raison de
s'embarrasser, il ne put refuser à M\textsuperscript{me} de Montespan,
quelque peu cordialement qu'ils fussent ensemble, à
M\textsuperscript{me} la duchesse de Chartres avec qui il vivait alors
intimement, et à sa propre pudeur pour des gens dont il avait fait le
mariage, et qui n'avaient trouvé en lui rien moins que ce qui l'avait
fait faire, de s'intéresser pour eux en chose si fort de leur convenance
et qui ne lui coûtait rien. Il obtint donc cette place du roi et de
M\textsuperscript{me} de Maintenon, sans laquelle ces sortes d'emplois
ne s'accordaient point, et se donna le mérite de le mander en Languedoc
où étaient M. et M\textsuperscript{me} de Castries et le cardinal Bonzi,
avant qu'ils pussent savoir que ce poste était à remplir. Ils
demeurèrent encore quelque temps chez eux à achever leurs affaires, et
puis vinrent s'établir pour toujours à la cour.

M\textsuperscript{me} de Castries était un quart de femme, une espèce de
biscuit manqué, extrêmement petite, mais bien prise, et aurait passé
dans un médiocre anneau\,; ni derrière, ni gorge, ni menton, fort laide,
l'air toujours en peine et étonné, avec cela une physionomie qui
éclatait d'esprit et qui tenait encore plus parole. Elle savait tout\,:
histoire, philosophie, mathématiques, langues savantes, et jamais il ne
paraissait qu'elle sût mieux que parler français, mais son parler avait
une justesse, une énergie, une éloquence, une grâce jusque dans les
choses les plus communes, avec ce tour unique qui n'est propre qu'aux
Mortemart. Aimable, amusante, gaie, sérieuse, toute à tous\,; charmante
quand elle voulait plaire, plaisante naturellement avec la dernière
finesse sans la vouloir être, et assenant aussi les ridicules à ne les
jamais oublier, glorieuse, choquée de mille choses avec un ton plaintif
qui emportait la pièce, cruellement méchante quand il lui plaisait, et
fort bonne amie, polie, gracieuse, obligeante en général, sans aucune
galanterie, mais délicate sur l'esprit et amoureuse de l'esprit où elle
le trouvait à son gré\,; avec cela un talent de raconter qui charmait,
et, quand elle voulait faire un roman sur-le-champ, une source de
production, de variété et d'agrément qui étonnait. Avec sa gloire, elle
se croyait bien mariée par l'amitié qu'elle eut pour son mari. Elle
l'étendit sur tout ce qui lui appartenait, et elle était aussi glorieuse
pour lui que pour elle\,; elle en recevait le réciproque et toutes
sortes d'égards et de respects.

On ajouta bientôt après une nouvelle personne à la suite, mais
intérieure, de M\textsuperscript{me} la duchesse de Chartres\,; mais
sans aller à Marly, ni paraître avec elle en public hors de son
appartement, sinon en des voyages ou en des choses familières. Ce fut
M\textsuperscript{me} de Jussac, qui avait été sa gouvernante, et qui
sut allier la plus constante confiance de M\textsuperscript{me} de
Montespan avec l'estime de M\textsuperscript{me} de Maintenon. Elle
s'appelait Saint-Just, et avait été longtemps auprès de la première
femme de mon père, qui, par confiance, la donna à ma sœur, quand elle
épousa le duc de Brissac. Les brouilleries domestiques, qui ne tardèrent
pas, l'en détachèrent. Elle entra chez M\textsuperscript{me} de
Montespan, qui, après, la mit auprès de M\textsuperscript{lle} de Blois,
dont elle fut gouvernante jusqu'à son mariage avec M. le duc de
Chartres. Son mari fut tué, écuyer de M. le duc du Maine, à la bataille
de Fleurus, en 1690. C'était une grande femme, de bonne mine, et qui
avait été fort agréable, et toujours parfaitement vertueuse. Elle était
douce, modeste, bonne, mais sage et avisée\,; qui connaissait fort le
monde et les gens\,; vraie et droite, polie, respectueuse, toujours en
sa place\,; et qui, avec la confiance et l'amitié intime de
M\textsuperscript{me} la duchesse de Chartres et de
M\textsuperscript{me} de Montespan, et depuis, avec assez de confiance
de M\textsuperscript{me} de Maintenon, ne voyait rien à l'aveugle,
discernait tout, et sut toujours se bien démêler, sans flatterie et sans
fausseté, et sans rien perdre avec elles. Elle sut aussi s'attirer une
vraie considération et des amis distingués à la cour, quand elle y fut
mise, et toujours sans sortir de son état, ni oublier avec nous ce
qu'elle y avait été. Il est très singulier qu'avec très peu de bien,
elle maria ses deux filles à deux frères, MM. d'Armentières et de
Conflans qui n'avaient rien, et que ce soient ces deux mariages qui les
aient remis au monde, et le chevalier de Conflans, leur troisième frère,
et qui les aient tirés de la poussière où l'indigence faisait languir
cette ancienne maison depuis si longtemps.

\hypertarget{chapitre-xxvi.}{%
\chapter{CHAPITRE XXVI.}\label{chapitre-xxvi.}}

1697

~

\relsize{-1}

{\textsc{Année 1697. Mort de Bignon, conseiller d'État, et de son frère,
premier président du grand conseil, dont Vertamont, son gendre, a la
place.}} {\textsc{- Caumartin, conseiller d'État, gagne sa prétention de
sa date d'intendant des finances sur les conseillers d'État
postérieurs.}} {\textsc{- La Reynie, conseiller d'État et lieutenant de
police, quitte cette dernière place à d'Argenson.}} {\textsc{- Mort de
Pussort, doyen du conseil et conseiller au conseil royal des finances.}}
{\textsc{- Cette dernière place donnée à Pomereu au refus de Courtin,
doyen du conseil.}} {\textsc{- Combat à Paris du bailli d'Auvergne et du
chevalier de Caylus\,; M\textsuperscript{lle} de Soissons exilée.}}
{\textsc{- Ruvigny et ses fils.}} {\textsc{- Harlay, premier président,
s'approprie un dépôt à lui confié par son ami Ruvigny, fait son fils
conseiller d'État et obtient vingt mille livres de pension.}} {\textsc{-
Duchesse de Valentinois brouillée et retournée avec son mari.}}
{\textsc{- Son horrible calomnie.}} {\textsc{- M\textsuperscript{me} de
L'Aigle, dame d'honneur de M\textsuperscript{me} la duchesse.}}
{\textsc{- Briord, ambassadeur à Turin, quoiqu'à M. le Prince.}}
{\textsc{- Mariage du fils de Pontchartrain avec une sœur du comte de
Roucy, après que le roi lui eut défendu celui de M\textsuperscript{lle}
de Malause.}} {\textsc{- Élévation des ministres.}} {\textsc{-
Malause.}} {\textsc{- Roucy-Roye-La Rochefoucauld.}} {\textsc{- Aventure
qui fait passer le comte et la comtesse de Roye de Danemark en
Angleterre.}} {\textsc{- Mariage du comte d'Egmont avec
M\textsuperscript{lle} de Cosnac, à qui le roi donne un tabouret de
grâce.}} \relsize{1}

~

Je perdis, au commencement de cette année 1697, M. Bignon, conseiller
d'État, si ami de mon père qu'il voulut bien être mon tuteur, quoique
sans aucune parenté, lorsqu'à la mort de M\textsuperscript{me} la
duchesse de Brissac, en 1684, elle me fit son légataire universel.
C'était un magistrat de l'ancienne roche, pour le savoir, l'intégrité,
la vertu, la modestie\,; digne du nom qu'il portait, si connu dans la
robe et dans la république des lettres\,; et qui, comme ses pères, avait
été avocat général avec grande réputation. Il était veuf de la sœur
unique de Pontchartrain, qu'il avait toujours extrêmement aimée, et qui
fit de ses enfants comme des siens. Bignon n'était point riche, et
avait, à quatre-vingts ans, la tête aussi bonne qu'à quarante. Je le
regrettai beaucoup, et je ne faisais rien dans mes affaires qu'avec son
conseil. Son frère, qui était premier président du grand conseil, et
pour qui on avait formé cette charge, le suivit huit jours après.
Celui-là était riche par un mariage\,; il n'avait qu'une fille, mariée à
Vertamont, maître des requêtes, fils du premier lit de la maréchale
d'Estrades, qui eut sa charge.

La place de l'autre, {[}au{]} conseil des parties, fut donnée à
Caumartin, proche parent et ami particulier de Pontchartrain, qui s'en
servit très principalement dans l'administration des finances, dont il
était l'un des intendants. C'était un grand homme, beau et très bien
fait, fort capable dans son métier de robe et de finance, qui savait
tout, en histoire, en généalogies, en anecdotes de cour, avec une
mémoire qui n'oubliait rien de ce qu'il avait vu ou lu, jusqu'à en citer
les pages sur-le-champ, dans la conversation. Il était fort du grand
monde, avec beaucoup d'esprit, et il était obligeant, et au fond honnête
homme. Mais sa figure, la confiance de Pontchartrain et la cour,
l'avaient gâté. Il était glorieux, quoique respectueux, avait tous les
grands airs qui le faisaient moquer et haïr encore de ceux qui ne le
connaissaient pas\,; en un mot, il portait sous son manteau toute la
fatuité que le maréchal de Villeroy étalait sous son baudrier. C'est le
premier homme de robe qui ait hasardé le velours et la soie\,: on s'en
moqua extrêmement, et {[}il{]} ne fut imité de personne.

Il prétendit une séance qui forma un procès que je rapporterai tout de
suite. Les intendants des finances, qui ne sont pas conseillers d'État,
entrent en manteau court au conseil des parties, et y ont séance du jour
de leurs provisions d'intendances des finances, et la conservent
au-dessus des conseillers d'État, qui ne le deviennent que depuis que
les intendants des finances ont acheté leurs charges. Sur ce fondement,
Caumartin, devenu conseiller d'État, prétendit précéder ceux-là, parce
qu'il les avait toujours précédés. Eux prétendaient que les intendants
des finances, qui n'étaient point conseillers d'État, n'étaient point du
conseil, quoique avec séance et voix\,; et en donnaient pour preuves
qu'ils n'y étaient ni comme maîtres des requêtes, ni comme conseillers
d'État, dont ils ne portaient pas ni l'une ni l'autre robe, et
concluaient que leur séance et voix n'étant que d'honneur, pour décorer
leurs charges, et ne les incorporant point dans le conseil, ils en
devaient prendre la queue, quand, à proprement parler, ils venaient à y
entrer comme membres, et à être faits conseillers d'État, dont alors
seulement ils revêtaient la robe et quittaient le manteau. La chose
portait directement sur Phélypeaux, frère de Pontchartrain, qui se
trouvait le premier et le plus ancien des conseillers d'État faits
depuis que Caumartin était intendant des finances. Pontchartrain
l'aimait beaucoup, et ils vivaient parfaitement en frères, et y ont
toujours vécu. Toutefois la cause financière de Caumartin l'emporta dans
l'esprit de Pontchartrain, qui lui fit gagner son procès devant le roi,
où l'affaire fut rapportée, qui fit un règlement pour l'avenir. Les
conseillers d'État en furent fort fâchés, et Phélypeaux en dit son avis
à son frère, mais sans qu'ils s'en soient refroidis.

La Reynie, conseiller d'État si connu pour avoir tiré, le premier, la
charge de lieutenant de police de Paris de son bas état naturel, pour en
faire une sorte de ministère, et fort important par la confiance directe
du roi, les relations continuelles avec la cour, et le nombre des choses
dont il se mêle, et où il peut servir ou nuire infiniment aux gens les
plus considérables, et en mille manières, obtint enfin, à quatre-vingts
ans, la permission de quitter un si pénible emploi qu'il avait le
premier ennobli par l'équité, la modestie et le désintéressement avec
lequel il l'avait rempli sans se relâcher de la plus grande exactitude,
ni faire de mal que le moins et le plus rarement qu'il lui était
possible\,: aussi était-ce un homme d'une grande vertu et d'une grande
capacité, qui, dans une place qu'il avait pour ainsi dire créée, devait
s'attirer la haine publique, {[}et{]} s'acquit pourtant l'estime
universelle. D'Argenson, maître des requêtes, fut mis en sa place. C'est
un personnage dont j'aurai lieu de parler ailleurs.

Pussort, conseiller d'État et doyen du conseil, mourut bientôt après\,;
il était aussi l'un des deux conseillers au conseil royal des finances,
et avait quatre-vingt-sept ou quatre-vingt-huit ans. M. Colbert l'avait
fait ce qu'il était\,; son mérite l'avait bien soutenu. Il était frère
de la mère de M. Colbert, et fut toute sa vie le dictateur, et, pour
ainsi dire, l'arbitre et le maître de toute cette famille si unie. Il
n'avait jamais été marié, était fort riche et fort avare, chagrin,
difficile, glorieux, avec une mine de chat fâché qui annonçait tout ce
qu'il était, et dont l'austérité faisait peur et souvent beaucoup de
mal, avec une malignité qui lui était naturelle. Parmi tout cela,
beaucoup de probité, une grande capacité, beaucoup de lumières,
extrêmement laborieux, et toujours à la tête de toutes les grandes
commissions du conseil et de toutes les affaires importantes du dedans
du royaume. C'était un grand homme sec, d'aucune société, de dure et de
difficile accès, un fagot d'épines, sans amusement et sans délassement
aucun, qui voulait être maître partout, et qui l'était parce qu'il se
faisait craindre, qui était dangereux et insolent, et qui fut fort peu
regretté. Courtin devint, par cette mort, doyen du conseil, et le roi
lui voulut donner la place du conseil des finances\,; mais les mêmes
raisons et le même esprit de retraite qui lui avaient fait refuser de
traiter la paix, le firent remercier de cette place, que Pomereu eut à
son refus. C'était un conseiller d'État fort distingué en capacité, en
lumière et en esprit, vif, actif, très intègre et laborieux, mais
brusque, plus que vif, capricieux, et que sa femme et ses domestiques ne
laissaient pas toujours voir, même à ses amis les plus intimes\,; il en
avait et savait les mériter\,; il l'était fort de mon père, et fut
toujours des miens. C'est le premier intendant qui ait été en Bretagne
avec cette qualité et ce pouvoir.

Le fils aîné du comte d'Auvergne acheva de se déshonorer de tous points
par un combat qu'il fit contre le chevalier de Caylus, au sortir duquel
il courut, éperdu, par les rues, l'épée à la main dont il s'était très
misérablement servi. La querelle était venue pour du cabaret et des
gueuses. Caylus, qui était fort jeune et qui s'était bien battu, se
sauva hors du royaume\,; et le comte d'Auvergne profita de cette triste
occasion pour que son fils n'y rentrât plus. C'était, de tous points, un
misérable, fort déshonoré, qui, à force d'aventures honteuses, fut
obligé de se laisser déshériter et de prendre la croix de Malte. Il fut
pendu en effigie à la Grève, de cette dernière-ci, avec un grand regret
de sa famille, non pas du jugement, mais de sa forme, parce que le
parlement, qui ne connaît de princes que ceux du sang, y procéda comme
pour le plus obscur gentilhomme, malgré toutes les tentatives de
distinction dont MM. de Bouillon ne purent obtenir aucune. Cet exil hors
du royaume fit depuis la fortune de Caylus. De cette même affaire,
M\textsuperscript{lle} de Soissons fut chassée de Paris.

La paix s'approchant, le roi la prévint par un trait de vengeance contre
milord Galloway, dont il n'aurait plus été temps bientôt après. Il était
fils de Ruvigny, et c'est ce qu'il faut expliquer. Ruvigny était un bon
mais simple gentilhomme, plein d'esprit, de sagesse, d'honneur et de
probité, fort huguenot, mais d'une grande conduite et d'une grande
dextérité. Ces qualités, qui lui avaient acquis une grande réputation
parmi ceux de sa religion, lui avaient donné beaucoup d'amis importants,
et une grande considération dans le monde. Les ministres et les
principaux seigneurs le comptaient et n'étaient pas indifférents à
passer pour être de ses amis, et les magistrats du plus grand poids
s'empressaient aussi à en être. Sous un extérieur fort simple, c'était
un homme qui savait allier la droiture avec la finesse de vues et les
ressources, mais dont la fidélité était si connue, qu'il avait les
secrets et les dépôts des personnes les plus distinguées. Il fut un
grand nombre d'années le député de sa religion à la cour, et le roi se
servit souvent des relations que sa religion lui donnait en Hollande, en
Suisse, en Angleterre et en Allemagne, pour y négocier secrètement, et
il y servit très utilement. Le roi l'aima et le distingua toujours, et
il fut le seul, avec le maréchal de Schomberg, à qui le roi offrit de
demeurer à Paris et à sa cour avec leurs biens et la secrète liberté de
leur religion dans leur maison, lors de la révocation de l'édit de
Nantes, mais tous deux refusèrent. Ruvigny emporta ce qu'il voulut, et
laissa ce qu'il voulut aussi, dont le roi lui permit la jouissance. Il
se retira en Angleterre avec ses deux fils. La Caillemotte, le cadet,
plus disgracié encore du côté de l'âme que de celui du corps, mourut
bientôt après. Le père ne survécut pas longtemps, et son aîné continua à
jouir des biens que son père avait laissés en France. Il s'attacha au
service du prince d'Orange, à la révolution, qui le fit comte de
Galloway en Irlande, et l'avança beaucoup. Il était bon officier. Il
avait de l'ambition\,; elle le rendit ingrat. Il se distingua en haine
contre le roi et contre la France, quoique le seul huguenot qu'on y
laissait jouir de son bien, même servant le prince d'Orange. Le roi le
fit avertir plusieurs fois du mécontentement qu'il avait de sa conduite.
Il en augmenta les torts avec plus d'éclat\,; à la fin, le roi confisqua
ses biens, et témoigna publiquement sa colère.

Le vieux Ruvigny était ami d'Harlay, lors procureur général, et depuis
premier président, et lui avait laissé un dépôt entre les mains, dans la
confiance de sa fidélité. Il la lui garda tant qu'il n'en put pas
abuser\,; mais quand il vit l'éclat, il se trouva modestement embarrassé
entre le fils de son ami et son maître, à qui il révéla humblement sa
peine\,; il prétendit que le roi l'avait su d'ailleurs et que Barbezieux
même l'avait appris, et l'avait dit au roi. Je n'approfondirai point ce
secret, mais le fait est qu'il le dit lui-même, et que, pour récompense,
le roi le lui donna comme sien confisqué, et que cet hypocrite de
justice, de vertu, de désintéressement et de rigorisme, n'eut pas honte
de se l'approprier et de fermer les yeux et les oreilles au bruit
qu'excita cette perfidie. Il en tira plus d'un parti\,; car le roi, en
colère contre Galloway, en sut si bon gré au premier président qu'il
donna à son fils fort jeune, et qui se déshonorait tous les jours dans
sa charge d'avocat général, la place de conseiller d'État, vacante par
la mort de Pussort, et que quelque temps après, il le combla par une
pension de vingt mille livres, qui est celle des ministres. Ainsi les
forfaits sont récompensés en ce monde\,; mais la satisfaction n'en dure
pas longtemps.

M. de Monaco, qui, comme on a vu plus haut, avait obtenu le rang de
prince étranger par le mariage de son fils avec la fille de M. le Grand,
trouva bientôt et son fils plus encore, qu'ils l'avaient acheté bien
cher. La duchesse de Valentinois était charmante, galante à l'avenant,
et sans esprit ni conduite, avec une physionomie fort spirituelle\,;
elle était gâtée par l'amitié de son père et de sa mère, et par les
hommages de toute la cour dans une maison jour et nuit ouverte, où les
grâces, qui étaient sa principale beauté, attiraient la plus brillante
jeunesse. Son mari, avec beaucoup d'esprit, ne se sentait pas le plus
fort\,; sa taille et sa figure lui avaient acquis le nom de Goliath. Il
souffrit longtemps les hauteurs et les mépris de sa femme et de sa
famille. À la fin, lui et son père s'en lassèrent, et ils emmenèrent
M\textsuperscript{me} de Valentinois à Monaco. Elle se désola et ses
parents aussi, comme si on l'eût menée aux Indes. On peut juger que le
voyage et le séjour ne se passèrent pas gaiement. Toutefois, elle promit
merveilles, et au bout d'une couple d'années de pénitence, elle obtint
son retour. Je ne sais qui fut son conseil, mais, sans changer de
conduite, elle songea aux moyens de se garantir de retourner à Monaco,
et pour cela fit un éclat épouvantable contre son beau-père, qu'elle
accusa non seulement de lui en avoir conté, mais de l'avoir voulu
forcer. M. le Grand, M\textsuperscript{me} d'Armagnac, leurs enfants,
prirent son parti\,; et ce fut un vacarme le plus scandaleux, mais qui
ne persuada personne. M. de Monaco n'était plus jeune. Il était fort
honnête homme et avait toujours passé pour tel\,; d'ailleurs il avait
deux gros yeux d'aveugle, éteints, et qui en effet ne distinguaient rien
à deux pieds d'eux, avec un gros ventre en pointe, qui faisait peur tant
il avançait en saillie. L'éclat ne fut pas moins grand de sa part et de
celle de son fils contre une si étrange calomnie, et la séparation
devint plus forte que jamais.

Au bout de quelques années, ils s'avisèrent qu'ils n'avaient point
d'enfants, et que M\textsuperscript{me} de Valentinois, nageant dans les
plaisirs de la cour, sous l'abri de sa famille, jouissait seule de son
crime, et se moquait d'eux. Ils prirent donc leur parti. M. de
Valentinois redemanda sa femme\,; d'abord on se moqua de lui chez
elle\,; mais bientôt l'embarras succéda. Les dévots s'en mêlèrent.
L'archevêque de Paris parla à M\textsuperscript{me} d'Armagnac, et M. de
Monaco protesta qu'il ne verrait jamais sa belle-fille, et qu'il lui
défendait de se trouver en aucun lieu où il serait. Tout cela ensemble
fut un coup de foudre. Il fallut céder, et le 27 janvier,
M\textsuperscript{me} d'Armagnac, accompagnée du prince Camille son
troisième fils, et de la princesse d'Harcourt, mena sa fille à Paris
chez le duc de Valentinois, où se trouva la maréchale de Boufflers, sa
cousine germaine. M\textsuperscript{me} de Valentinois y soupa et y
coucha, et qui pis fut, y demeura.

Elle était très souvent chez M\textsuperscript{me} la Duchesse, qui
changea en même temps de dame d'honneur. M\textsuperscript{me} de
Moreuil qui était personne d'esprit et de mérite, femme d'un original de
beaucoup d'esprit aussi, des bâtards de cette ancienne maison de Moreuil
éteinte depuis longtemps, et qui était à M. le Duc, demanda tout d'un
coup à se retirer sans qu'on pût savoir pourquoi, et le voulut
absolument. On vit depuis de quoi il était question. La pauvre femme
cachait un cancer dont elle mourut quelque temps après.
M\textsuperscript{me} de L'Aigle fut mise en sa place et s'y fit aimer
et estimer, et même considérer à la cour. C'était une femme de beaucoup
d'esprit et de monde, fille de M\textsuperscript{me} de Raré,
gouvernante des filles de M. Gaston\,; son père et sa mère étaient fort
des amis de mon père, et elle épousa le marquis de L'Aigle, à six lieues
de la Ferté, qui en était aussi beaucoup. C'est ce qui me fait remarquer
cette bagatelle. Les affaires de M. de L'Aigle étaient très mauvaises\,;
elle se mit là faute de mieux chez elle.

Il arriva une autre chose chez M. le Prince. Briord, son premier écuyer,
fut choisi pour l'ambassade de Turin. Torcy, qui était de ses amis, le
fit proposer par Pomponne, et quand l'affaire fut faite, le roi en dit
un mot d'honnêteté à M. le Prince. Le sujet était bon, mais le monde fut
surpris du lieu où on avait été chercher un ambassadeur, et je le
remarque comme une chose singulière, et tout à fait nouvelle. Au
demeurant, Briord était sage, honnête homme, et n'était pas incapable.

Pontchartrain cherchait à marier son fils. Il lui avait fait faire une
grande tournée par les ports du levant et du ponant pour lui faire voir
les choses dont il entendait parler tous les jours, et connaître les
officiers. Tout s'y passa moins en étude et en examens qu'en réceptions,
en festins et en honneurs, tels qu'on aurait pu les rendre au Dauphin.
Chacun s'y surpassa en cour et en bassesses pour le maître naissant de
son sort et de sa fortune, qui revint peu instruit, mais beaucoup plus
gâté qu'auparavant, et dans l'opinion d'être parfaitement au fait de
tout. Le père crut avoir trouvé tout ce qu'il pouvait désirer en
M\textsuperscript{lle} de Malause, qui était pensionnaire à la
Ville-l'Évêque à Paris. Sa mère, qui était Mitte, fille du marquis de
Saint-Chaumont, était morte. Son père était un homme retiré dans sa
province après avoir servi quelque temps jusqu'à être brigadier, et
s'était remarié à une Bérenger-Montmouton dont il avait deux fils. Sa
mère à lui était sœur des maréchaux de Duras et de Lorges qui avait
toujours pris soin de cette famille avec amitié.

L'alliance en plut tant à Pontchartrain qu'il traita ce mariage, et
qu'il en demanda l'agrément au roi. Sa surprise fut grande lorsqu'il
entendit le roi lui conseiller de penser à autre chose. Comme celle-là
lui convenait, il insista, tellement que le roi lui dit franchement que
cette fille portait les armes de Bourbon qui le choqueraient accolées
avec les siennes, qu'il la voulait marier à son gré, et qu'en un mot, il
désirait qu'il n'y pensât plus. La mortification fut grande. Les
ministres n'y étaient pas accoutumés. Peu à peu ils s'étaient mis de ce
règne au niveau de tout le monde. Ils avaient pris l'habit et toutes les
manières des gens de qualité. Leurs femmes étaient parvenues à manger et
à entrer dans les carrosses par M\textsuperscript{me} Colbert, sous le
prétexte de suivre M\textsuperscript{me} la princesse de Conti qu'elle
avait élevée\,; et d'ailleurs {[}elle{]} était extrêmement bien avec la
reine. Douze ou quinze ans après, M. de Louvois l'obtint pour sa femme
sous prétexte qu'elle était fille de qualité, et par l'émulation qui
était entre Colbert et lui. De là leurs belles-filles, et à cet exemple
les autres femmes des secrétaires d'État, et à la fin celle des
contrôleurs généraux. Leurs alliances les soutenaient dans ce brillant
nouveau, et leur autorité, dont tout sans exception dépendait, leur
avait acquis une supériorité et des distinctions étranges sur tout ce
qui n'était point titré, qui leur rendit bien amer et bien nouveau le
refus du roi sur une alliance dont il n'aurait pas fait difficulté avec
qui que c'eût été de la noblesse ordinaire. Pontchartrain se garda bien
de se vanter de ce qui lui était arrivé, et se hâta seulement de trouver
des prétextes de rompre. Mais le roi, si secret toujours, ne jugea pas à
propos de l'être dans cette occasion. Il parla aux maréchaux de Duras et
de Lorges, à M. de Bouillon, parce que leur mère était sœur de M. de
Turenne, et à d'autres encore, de manière que ce que Pontchartrain avait
caché fut su, et que ses confrères n'en furent pas moins mortifiés que
lui.

M\textsuperscript{lle} de Malause, unique de son lit, et ses deux frères
étaient la sixième et dernière génération, et la seule existante de
Charles, baron de Malause, sénéchal de Toulouse et de Bourbonnais,
bâtard du duc Jean II de Bourbon, connétable de France, qui ne laissa
point d'enfants légitimes, et qui était frère de Pierre, comte de
Beaujeu, mari de la célèbre M\textsuperscript{me} de Beaujeu, fille de
Louis XI, sœur et régente de la minorité de Charles VIII, qui fut duc de
Bourbon après son frère, et qui ne laissa qu'une fille héritière,
Suzanne de Bourbon, qui épousa le malheureux connétable de Bourbon si
cruellement persécuté par la mère de François Ier, et qui fut tué devant
Rome à la tête de l'armée de Charles V, après s'être trouvé à la
bataille de Pavie contre François Ier. Ils étaient frères de Louis de
Bourbon, élu évêque de Liège, qui laissa un bâtard, tige des seigneurs
de Busset qui subsistent encore. Outre ces frères légitimes, ils en
eurent un bâtard qui fut comte de Roussillon, amiral de France, et qui
figura avec sa femme, bâtarde de Louis XI et de Marguerite de Sassenage.
Mais l'amiral était bien loin alors d'être officier de la couronne, et
la marine de ce temps-là d'être sur un grand pied en France. Peu à peu
ces bâtards de Bourbon ont changé leur barre de bâtards et leurs autres
et diverses marques de bâtardise en bande comme les princes de cette
maison, et l'ont enfin raccourcie comme eux, tellement qu'il n'y a plus
aucune différence entre les armes des légitimes et des bâtards\,; et
c'est ce qui choquait si fort le roi qu'il ne voulut pas voir,
disait-il, à la chaise à porteurs de la nouvelle mariée, les armes de
Bourbon accolées à celles de Phélypeaux.

Pontchartrain eut lieu de se consoler par une alliance d'une bien autre
sorte, et à laquelle le roi consentit sans peine, car les mélanges qui
mettaient tout à l'unisson ne lui étaient point du tout désagréables en
eux-mêmes. Ce fut sur une autre nièce des maréchaux de Duras et de
Lorges, mais celle-là, fille de leur sœur, et de la maison de La
Rochefoucauld, qu'il jeta les yeux. Elle était sœur des comtes de Roucy
et de Blansac et des chevaliers de Raye et de Roucy, et elle était
élevée dans l'abbaye de Notre-Dame à Soissons. Ils étaient la troisième
génération de Charles de La Rochefoucauld, fils du comte de La
Rochefoucauld, qui fut tué à la Saint-Barthélemy, et de sa deuxième
femme, Charlotte de Roye, comtesse de Roucy, sœur de la princesse de
Condé, première femme du prince de Condé, tué à la bataille de Jarnac.
Toute cette branche de La Rochefoucauld-Roye était huguenote. Lors de la
révocation de l'édit de Nantes, le comte de Roye, père de celle dont il
s'agit, et sa femme, se retirèrent en Danemark, oh, comme il était
lieutenant général en France, il fut fait grand maréchal et commanda
toutes les troupes. C'était en 1683, et en 1686 il fut fait chevalier de
l'Éléphant. Il était là très grandement établi, et lui et la comtesse de
Roye sur un grand pied de considération.

Ces rois du Nord mangent ordinairement avec du monde, et le comte et la
comtesse de Roye avaient très souvent l'honneur d'être retenus à leur
table avec leur fille, M\textsuperscript{lle} de Roye. Il arriva à un
dîner que la comtesse de Roye, frappée de l'étrange figure de la reine
de Danemark, se tourna à sa fille, et lui demanda si elle ne trouvait
pas que la reine ressemblait à M\textsuperscript{me} Panache comme deux
gouttes d'eau. Quoiqu'elle l'eût dit en français, il arriva qu'elle
n'avait pas parlé assez bas, et que la reine, qui l'entendit, lui
demanda ce que c'était que cette M\textsuperscript{me} Panache. La
comtesse de Roye, dans sa surprise, lui répondit que c'était une dame de
la cour de France qui était fort aimable. La reine, qui avait vu sa
surprise, n'en fit pas semblant, mais, inquiète de la comparaison, elle
écrivit à Mayereron, envoyé de Danemark à Paris, et qui y était depuis
quelques années, de lui mander ce que c'était que M\textsuperscript{me}
Panache, sa figure, son âge, sa condition, et sur quel pied elle était à
la cour de France, et que surtout elle voulait absolument n'être pas
trompée et en être informée au juste. Mayereron, à son tour, fut dans un
grand étonnement. Il manda à la reine qu'il ne comprenait pas par où le
nom de M\textsuperscript{me} Panache était allé jusqu'à elle, beaucoup
moins la sérieuse curiosité qu'elle lui marquait d'être informée d'elle
exactement\,; que M\textsuperscript{me} Panache était une petite et fort
vieille créature avec des lippes et des yeux éraillés à y faire mal à
ceux qui la regardaient, une espèce de gueuse, qui s'était introduite à
la cour sur le pied d'une manière de folle, qui était tantôt au souper
du roi, tantôt au dîner de Monseigneur et de M\textsuperscript{me} la
Dauphine, ou à celui de Monsieur, à Versailles ou à Paris, où chacun se
divertissait à la mettre en colère, et qui chantait pouille aux gens à
ces dîners-là pour faire rire, mais quelquefois fort sérieusement, et
avec des injures qui embarrassaient et qui divertissaient encore plus
ces princes et ces princesses, qui lui emplissaient ses poches de viande
et de ragoûts, dont la sauce découlait tout du long de ses jupes, et que
les uns lui donnaient une pistole ou un écu, et les autres des
chiquenaudes et des croquignoles, dont elle entrait en furie, parce
qu'avec ses yeux pleins de chassie, elle ne voyait pas au bout de son
nez, ni qui l'avait frappée, et que c'était le passe-temps de la cour.

À cette réponse, la reine de Danemark se sentit si piquée qu'elle ne put
plus souffrir la comtesse de Roye, et qu'elle en demanda justice au roi
son mari. Il trouva bien mauvais que des étrangers qu'il avait comblés
des premières charges et des premiers honneurs de sa cour, avec de
grosses pensions, se moquassent d'eux d'une manière si cruelle. Il se
trouva des seigneurs du pays et des ministres jaloux de la fortune et du
grand établissement dont le comte de Roye jouissait, tellement que la
reine obtint que le roi le remercierait et lui ferait dire de se
retirer. Il ne put conjurer l'orage il vint avec sa famille à Hambourg,
en attendant qu'il sût ce qu'il pourrait devenir\,; et à la révolution
d'Angleterre il y passa, c'est-à-dire quelques mois devant. Le roi
Jacques, qui y était encore, le fit comte de Lifford et pair d'Irlande,
dont un fils qui l'avait suivi prit le nom.

Le comte de Roye était donc à Londres avec un fils et deux filles, et le
comte de Feversham, frère de sa femme, chevalier de la Jarretière et
capitaine des gardes du corps. À la révolution, ils ne se mêlèrent de
rien\,; et {[}il{]} a passé dix-huit ans en Angleterre sans charge et
sans service, et mourut aux eaux de Bath en 1690. Ses autres enfants
étaient demeurés en France\,; on les avait mis dans le service après
leur avoir fait faire abjuration, et les autres dans des collèges ou
dans des couvents. Le roi leur donna des pensions, et M. de La
Rochefoucauld avec MM. de Duras et de Lorges leur servirent de pères.

Ce fut donc principalement avec M. le maréchal de Lorges, qui aimait
extrêmement la comtesse de Roye, que le mariage se traita. On compta que
la fille n'avait rien et n'aurait jamais grand'chose\,; ce fut ce qui y
détermina, et ce qui, joint au solide du ministère, apprivoisa la
roguerie de M. de La Rochefoucauld. La comtesse de Roucy surtout fut
transportée d'un mariage dont elle comptait bien tirer un grand parti,
par la considération, et, mieux encore, par les affaires pécuniaires,
auxquelles dans la suite elle ne s'épargna pas. Les Pontchartrain furent
transportés d'aise. Le contrôleur général alla chez toute la parenté, et
ils ne firent point la petite bouche de l'honneur qu'ils recevaient de
cette alliance. La comtesse de Roucy alla chercher sa belle-sœur à
Soissons, et le mariage se fit à petit bruit à Versailles, dans la
chapelle, à minuit, par l'évêque de Soissons, Brûlard. Outre le présent
ordinaire du roi à ces mariages des ministres, il ajouta six mille
livres de pension aux quatre que la mariée avait déjà, et donna
cinquante mille écus à Pontchartrain, qui fit appeler son fils le comte
de Maurepas. Près de quatre millions que le chevalier des Augers et un
armateur prirent en ce temps-là sur les Espagnols, mirent en bonne
humeur à propos pour cette libéralité.

Le comte d'Egmont, dernier de cette grande et illustre maison, avait
quitté la Flandre depuis peu et pris le service de France. Il épousa
M\textsuperscript{lle} de Cosnac, nièce de l'archevêque d'Aix, qui
demeurait chez la duchesse de Bracciano, dont elle était aussi parente,
et le roi, par grâce, voulut bien lui donner le tabouret, les grands
d'Espagne, dont le comte d'Egmont était des premiers du temps de Charles
V, n'ayant point de rang en France.

\hypertarget{chapitre-xxvii.}{%
\chapter{CHAPITRE XXVII.}\label{chapitre-xxvii.}}

1696

~

\relsize{-1}

{\textsc{Mort de Molinos.}} {\textsc{- Continuation de l'affaire de
l'archevêque de Cambrai.}} {\textsc{- Mandements théologiques de MM. de
Paris et de Chartres.}} {\textsc{- Instruction sur les états d'oraison
de M. de Meaux.}} {\textsc{- Maximes des saints de M. de Cambrai.}}
{\textsc{- Ducs de Chevreuse et de Beauvilliers perdus auprès de
M\textsuperscript{me} de Maintenon.}} {\textsc{- M. de Cambrai se résout
à porter son affaire à Rome.}} {\textsc{- Son intime liaison avec le
cardinal de Bouillon et les jésuites.}} {\textsc{- Leurs intérêts
communs.}} {\textsc{- Cardinal de Bouillon va relever à Rome le cardinal
de Janson, et obtient pour son neveu la coadjutorerie de son abbaye de
Cluny.}} {\textsc{- Embarras des jésuites et leur adresse.}} {\textsc{-
Succès des Maximes des saints et de l'Instruction sur les états
d'oraison.}} {\textsc{- Maximes des saints mises à l'examen.}}
{\textsc{- Examinateurs.}} {\textsc{- Mort de l'évêque de Metz\,; sa
fortune.}} {\textsc{- M. de Paris commandeur de l'ordre.}} {\textsc{- M.
de Meaux conseiller d'État d'Église.}} {\textsc{- M. de Cambrai porte
son affaire à Rome.}} {\textsc{- Lettres au pape de part et d'autre.}}
{\textsc{- Réponses du pape.}} {\textsc{- M. de Cambrai exilé pour
toujours dans son diocèse.}} {\textsc{- Mort de la duchesse douairière
de Noailles.}} {\textsc{- Sa charge.}} {\textsc{- Sa famille.}}
{\textsc{- M. de Troyes\,; sa famille, sa vie, sa retraite.}} {\textsc{-
M. d'Orléans de nouveau et durement condamné contre M. de La
Rochefoucauld.}} {\textsc{- Abbé de Coislin\,; sa fortune\,; est fait
évêque de Metz.}} {\textsc{- Place décidée pour le premier aumônier
derrière le roi à la chapelle.}} {\textsc{- Réconciliation du duc de La
Rochefoucauld et de l'évêque d'Orléans.}} {\textsc{- Mort de La
Hillière, gouverneur de Rocroy, ami de mon père.}} {\textsc{- Comédiens
italiens chassés.}} \relsize{1}

~

Molinos, ce prêtre espagnol qui a passé pour le chef des quiétistes, et
pour en avoir renouvelé les anciennes erreurs, était mort à Rome dans
les prisons de l'inquisition, tout au commencement de cette année, et
cela me fait souvenir qu'il est temps de reprendre l'affaire de M. de
Cambrai. J'ai laissé M\textsuperscript{me} Guyon dans le donjon de
Vincennes, et j'ai omis bien des choses curieuses, parce qu'elles se
trouvent dans ce qui a été imprimé de part et d'autre. Il faut néanmoins
dire, pour l'intelligence de ce qui va suivre, qu'avant d'être arrêtée,
elle avait été mise entre les mains de M. de Meaux, où elle avait été
fort longtemps chez lui, ou dans les filles de Sainte-Marie de Meaux, où
ce prélat s'était instruit à fond de sa doctrine, sans avoir pu lui
persuader de changer de sentiments. On peut juger qu'elle les avait
épurés en effet, ou du moins en apparence, de tout ce qui était reproché
de sale et de honteux à cette doctrine, et à ce qui lui avait été
reproché de sa conduite avec le P. Lacombe, et de ses bizarres voyages
avec lui. Sans des précautions les plus scrupuleuses là-dessus, elle
n'aurait pu surprendre la candeur et la pureté des mœurs des ducs de
Chevreuse et de Beauvilliers, de leurs épouses, de l'archevêque de
Cambrai, et de bien d'autres personnes qui faisaient l'élite de son
petit troupeau. Mais, lasse enfin d'être comme prisonnière entre les
mains de M. de Meaux, elle avait feint d'ouvrir les yeux à sa lumière,
et avait signé une rétractation telle qu'il la lui avait présentée,
moyennant quoi lui qui était doux et de bonne foi en fut la dupe, et lui
procura la liberté, dont l'abus qu'elle fit par les assemblées secrètes
qu'elle tenait avec les plus affidés de son école la firent chasser de
Paris, puis, sur son retour secret, enfermer à Vincennes. Cette mauvaise
foi de la fausse convertie, jointe au peu de fruits des conférences
d'Issy qui sont si connues, et le célèbre tour que fit si prestement M.
de Cambrai de se confesser à M. de Meaux, pour lui fermer la bouche, mit
enfin la main de ce dernier prélat à la plume, pour exposer au public et
la doctrine, et la conduite, et les procédés de part et d'autre depuis
la naissance de cette affaire, sous le titre d'\emph{Instruction sur les
états d'oraison}.

Cet ouvrage lui parut d'autant plus nécessaire que M. de Chartres
d'abord, et M. de Paris ensuite, n'avaient traité l'affaire que d'une
manière toute théologique par leurs mandements, et qu'il crut important
de réduire au clair cette théologie assez pour être entendue de tout le
monde, et mettre en même temps au net tout ce qui s'était passé
là-dessus avec M. de Cambrai. Comme il était rempli de la matière, tant
par ce qui s'était passé à Issy avec M. de Cambrai, que par ce qu'il
avait vu des livres de M\textsuperscript{me} Guyon puis d'elle-même
tandis qu'il l'avait eue à Meaux, d'où M\textsuperscript{me} de Morstein
l'avait ramenée en triomphe dans l'équipage de la duchesse de Mortemart,
sa tante, il eut bientôt composé son ouvrage, et avant de l'imprimer le
donna à voir à M. de Chartres, aux archevêques de Reims et de Paris, et
à M. de Cambrai lui-même. Ce dernier en sentit tout le poids et la
nécessité de le prévenir. Il faut croire qu'il avait sa matière préparée
de loin et toute rédigée, parce qu'autrement la diligence de sa
composition serait incroyable, et d'une composition de ce genre. Il fit
un livre intelligible à qui n'est pas théologien versé dans le plus
mystique, qu'il intitula \emph{Maximes des saints}, et le mit en deux
colonnes\,: la première contenait les maximes qu'il donne pour
orthodoxes et pour celles des saints, l'autre les maximes dangereuses,
suspectes ou erronées, qui est l'abus qu'on a fait ou qu'on peut faire
de la bonne et saine mysticité, avec une précision qu'il donne pour
exacte de part et d'autre et qu'il propose d'un ton de maître à suivre
ou à éviter. Dans l'empressement de le faire paraître avant que M. de
Meaux pût donner le sien, il le fit imprimer avec toute la diligence
possible\,; et pour n'y perdre pas un instant, M. de Chevreuse s'alla
établir chez l'imprimeur pour en corriger chaque feuille à mesure
qu'elle fut imprimée. Aussi la promptitude et l'exactitude de la
correction répondirent-elles à des mesures si bien prises que, en très
peu de jours, il fut en état de le distribuer à toute la cour, et que
l'édition se trouva presque toute vendue.

Si on fut choqué de ne le trouver appuyé d'aucune approbation, on le fut
bien davantage du style confus et embarrassé, d'une précision si gênée
et si décidée\,; de la barbarie des termes qui faisaient comme une
langue étrangère, enfin de l'élévation et de la recherche des pensées
qui faisait perdre haleine, comme dans l'air trop subtil de la moyenne
région. Presque personne qui n'était pas théologien ne put l'entendre,
et de ceux-là encore après trois et quatre lectures. Il eut donc le
dégoût de ne recevoir de louanges de personne, et de remerciements de
fort peu, et de pur compliment\,; et les connaisseurs crurent y trouver,
sous ce langage barbare, un pur quiétisme, délié, affiné, épuré de toute
ordure, séparé du grossier, mais qui sautait aux yeux, et avec cela des
subtilités fort nouvelles et fort difficiles à se laisser entendre et
bien plus à pratiquer. Je rapporte non pas mon jugement, comme on peut
croire, de ce qui me passe de si loin, mais ce qui s'en dit alors
partout\,; et on ne parlait d'autres choses, jusque chez les dames\,; à
propos de quoi on renouvela ce mot échappé à M\textsuperscript{me} de
Sévigné lors de la chaleur des disputes sur la grave\,: «
Épaississez-moi un peu la religion, qui s'évapore toute à force d'être
subtilisée.\,»

Ce livre choqua fort tout le monde\,: les ignorants parce qu'ils n'y
entendaient rien\,; les autres par la difficulté à le comprendre, à le
suivre et à se faire à un langage barbare et inconnu\,; les prélats
opposés à l'auteur par le ton de maître sur le vrai et le faux des
maximes, et par ce qu'ils crurent apercevoir de vicieux dans celles
qu'il donnait pour vraies. Le roi surtout et M\textsuperscript{me} de
Maintenon, fort prévenus, en furent extrêmement mal contents, et
trouvèrent extrêmement mauvais que M. de Chevreuse eût fait le
personnage de correcteur d'imprimerie\,; et que M. de Beauvilliers se
fût chargé de le présenter au roi en particulier, sans en avoir rien dit
à M\textsuperscript{me} de Maintenon, et M. de Cambrai à la cour qui le
pouvait bien faire lui-même. Il craignit peut-être une mauvaise
réception devant le monde et en chargea M. de Beauvilliers qui avait des
temps plus familiers et seul avec le roi, pour faire mieux recevoir son
livre par la considération du duc, ou cacher au monde s'il était mal
reçu\,; mais ces messieurs, enchantés par les graves et par la
spiritualité du prélat, s'aliénèrent entièrement M\textsuperscript{me}
de Maintenon par ces démarches\,: l'un en se faisant le coopérateur
public, par une fonction si au-dessous de lui, d'un ouvrage qu'elle ne
pouvait agréer après avoir pris si hautement le parti contraire\,;
l'autre en lui marquant une défiance et une indépendance d'elle, qui la
blessa plus que tout, et qui la fit résoudre à travailler à les perdre
tous deux.

Parmi ces mouvements de doctrine et d'écrits, M. de Cambrai avait songé
à de plus forts secours. Ami des jésuites, il se les était attachés, et
ils étaient à lui en corps et en groupes, à la réserve de quelques
particuliers plus considérables par leur mérite que par leur poids et
par leur influence dans les secrets, la conduite et le gouvernement
intérieur de leur compagnie. Il se voyait sans ressource en France, avec
les premiers prélats en savoir, en piété, en crédit contre lui, qui,
ayant la cour déclarée pour eux, mèneraient tous les autres évêques. Il
songea donc à porter son affaire à Rome où il espéra tout par une
démarche si contraire à nos mœurs et si agréable à cette cour, qui
affecte les premiers jugements, et que toute dispute un peu considérable
soit d'abord portée devant elle sans être d'abord jugée sur les lieux.
Il y compta sur le crédit des jésuites, et la conjoncture lui présenta
une autre protection dont il ne manqua pas de s'assurer.

Le cardinal de Janson était depuis six ou sept ans à Rome\,; il y avait
très dignement et très utilement servi\,: il voulut enfin revenir. Le
cardinal de Bouillon n'avait pas moins d'envie de l'y aller relever. La
frasque ridicule qu'il avait faite sur cette terre du dauphiné
d'Auvergne, et d'autres encore, avaient diminué sa considération et
mortifié sa vanité. Il voulait une absence, et une absence causée et
chargée d'affaires, pour revenir après sur un meilleur pied. Il n'y
avait plus que deux cardinaux devant lui, et il fallait être à Rome, à
la mort du doyen, pour recueillir le décanat du sacré collège. M. de
Cambrai s'était lié d'avance avec lui, et l'intérêt commun avait rendu
cette liaison facile et sûre. Le cardinal voyait alors ce prélat dans
les particuliers intimes de M\textsuperscript{me} de Maintenon, et
maître de l'esprit des ducs de Chevreuse et de Beauvilliers qui étaient
dans la faveur et dans la confiance la plus déclarée. Bouillon et
Cambrai étaient aux jésuites, les jésuites à eux, et le prélat, dont les
vues étaient vastes, comptait de se servir utilement du cardinal, et à
la cour et à Rome. Son crédit à la cour tombé, celui de ses amis fort
obscurci, l'amitié du cardinal lui devint plus nécessaire. Ce dernier
leur avait l'obligation d'avoir vaincu la répugnance du roi pour
l'envoyer relever le cardinal de Janson, et celle encore de lui avoir
obtenu l'agrément et la protection du roi pour faire élire l'abbé
d'Auvergne, son neveu, coadjuteur de son abbaye de Cluny. C'était avoir
pris l'orgueil, qui gouvernait uniquement le cardinal, par l'endroit le
plus sensible. Il ne se démentit donc point à leur égard lorsqu'il vit
leur crédit en désarroi, et il espéra les remettre en selle par le
jugement qu'il se promettait de faire rendre à Rome. Tout l'animait en
ce dessein, le fruit d'un si grand service, et on prétendit que le
marché entre eux était fait, mais à l'insu des dues, que le crédit de
l'un ferait l'autre cardinal en lui faisant gagner sa cause, et que le
crédit de celui-ci, relevé par sa victoire et sa pourpre, serait tel en
soi et sur les deux ducs, à qui il serait alors temps de parler et sur
lesquels il pouvait tout, qu'ils feraient entrer le cardinal de Bouillon
dans le conseil, d'où Bouillon ne se promettait pas moins que de
s'élever à la place de premier ministre.

Ce dernier point du conseil n'était pas à beaucoup près si aisé à
imaginer raisonnablement que les espérances de Rome. Le roi n'avait
jamais mis d'ecclésiastique dans son conseil, et il était trop jaloux de
son autorité et de sembler tout faire, pour se résoudre jamais à un
premier ministre\,; mais Bouillon était l'homme le plus chimérique qui
ait vécu en nos jours, et le plus susceptible des chimères les plus
folles en faveur de sa vanité, dont toute sa vie a pété la preuve. Un
peu de sens aurait pu lui découvrir qu'indépendamment de la difficulté
du côté du roi, il n'était pas sûr que, si ses amis les eussent pu
vaincre, c'eût été à son profit, et que M. de Cambrai n'eût pas mieux
aimé prendre pour soi ce qu'il eût pu procurer à un autre\,; mais, outre
ces chimères, le cardinal de Bouillon haïssait personnellement les
adversaires de M. de Cambrai, et aurait peut-être plus que lui encore
triomphé de leur condamnation.

Les Bouillon et les Noailles étaient ennemis de tous les temps. Les
principales terres de Noailles étaient dans la vicomté de Turenne. Ce
joug leur était odieux, ils le voulaient secouer. Le procès en était
pendant depuis nombre d'années, et se reprenait par élans avec une
aigreur extrême et jusqu'aux injures, jusque-là que les Bouillon avaient
reproché aux Noailles dans les écritures du procès qu'un Noailles avait
été domestique d'un vicomte de Turenne de leur maison. C'était avec un
dépit extrême qu'ils voyaient briller les Noailles dans la splendeur des
dignités, des charges, des emplois et du crédit, et ce fut avec rage que
le cardinal de Bouillon vit arriver M. de Châlons à l'archevêché de
Paris, où il avait taché inutilement d'atteindre autrefois, et devenir
incessamment son confrère par le cardinalat. Les mêmes Bouillon
n'étaient pas moins ennemis des Tellier. M. de Louvois, brouillé à
l'excès avec M. de Turenne, et diverses fois humilié sous son poids,
l'avait rendu depuis à toute sa famille, et jusqu'à MM, de Duras ses
neveux, et l'inimitié s'était perpétuée. M. de Reims, dans ce grand
siège, était d'autant plus odieux au cardinal de Bouillon qu'il n'avait
pu affaiblir son crédit et sa considération. Le savoir éminent de M. de
Meaux, l'autorité qu'il lui avait acquise sur tout le clergé et dans
toutes les écoles, ses privances avec le roi, sa considération, son
estime et sa réputation au dedans et au dehors, tout cela piquait
l'émulation et l'orgueil du cardinal, et lui donnait un désir extrême de
lui voir tomber une flétrissure\,; enfin le crédit que M. de Chartres
commençait à prendre sur le roi à la faveur de cette affaire, porté par
son intimité avec M\textsuperscript{me} de Maintenon, était
insupportable à un homme qui voulait tout, et qui, dédaignant de
regarder cet évêque que comme un cuistre violet, se trouvait néanmoins
obligé à des égards et à des ménagements qui l'outraient. Toutes ces
choses ensemble étaient plus qu'il n'en fallait pour enflammer le
cardinal de Bouillon, et pour lui faire entreprendre et porter la cause
de M. de Cambrai autant et plus que la sienne propre. Je me suis étendu
sur ces motifs parce que sans cette connaissance on n'en pourrait
comprendre les suites.

M. de Cambrai ne put soutenir en face le triste succès de son livre, qui
ne trouva de louanges que dans le \emph{Journal des savants} qu'un
calviniste faisait en Hollande. Il partit pour son diocèse, où il allait
de temps en temps, et partit brusquement\,; mais aussitôt après, il
tomba malade ou le fit, et pour demeurer plus près de ses amis, se
relaissa chez Malezieux, son ami, et domestique gouvernant tout chez M.
et M\textsuperscript{me} du Maine, où il ne fut qu'à six lieues de
Versailles. Cependant les jésuites se trouvèrent embarrassés. Outre leur
liaison intime et de tout temps avec le cardinal de Bouillon, et la leur
bien affermie avec M. de Cambrai, ils haïssaient aussi ses
adversaires\,; M. de Meaux, parce qu'il ne favorisait ni leur doctrine
ni leur morale, que son crédit les contenait, et que son savoir et sa
réputation les accablaient\,; M. de Paris, par les mêmes raisons de
doctrine et de morale, mais ils frémissaient de plus de ce qu'il était
devenu archevêque de Paris sans eux, et comme malgré eux\,; M. de
Chartres, parce qu'ils haïssaient et enviaient la faveur de
Saint-Sulpice, quoique sur Rome et d'autres points dans les mêmes
sentiments, mais la jalousie détruisait toute union, et de plus ils
sentaient déjà le crédit que ce prélat prenait dans la distribution des
bénéfices, et c'était leur partie la plus sensible que d'en disposer
seuls\,; M. de Reims, qui se ralliait à ces prélats, parce qu'il ne les
ménageait en rien, et qu'ils n'avaient jamais pu ni l'adoucir ni être
soutenus contre lui en aucune occasion.

Leur partialité avait donc été aperçue\,; elle fut appréhendée\,; on
voulut les contenir\,; on en parla au roi. On lui montra l'approbation
du P. de La Chaise et du P. Valois, confesseurs des princes, au livre de
M. de Cambrai\,; on mit le roi en colère, et il s'en expliqua durement à
ces deux jésuites. Les supérieurs, inquiets des suites que cela pourrait
avoir pour le confessionnal du roi et des princes, et par conséquent
pour toute la société, en consultèrent les gros bonnets à quatre vœux\,;
et le résultat fut qu'il fallait céder ici à l'orage, sans changer les
projets pour Rome. C'était le carême\,; le P. La Rue prêchait devant le
roi\,: on fut donc tout à coup surpris que le jour de l'Annonciation,
ses trois points finis, et au moment de donner la bénédiction et de
sortir de chaire, il demanda permission au roi de dire un mot contre des
extravagants et des fanatiques qui décriaient les vies communes de la
piété autorisées par un usage constant, et approuvées de l'Église, pour
leur en substituer d'erronées, nouvelles, etc.\,; et de là prit son
thème sur la dévotion à la sainte Vierge, parla avec le zèle d'un
jésuite commis par sa société pour lui parer un coup dangereux, et fit
des, peintures d'après nature par lesquelles on ne pouvait méconnaître
les principaux acteurs pour et contre. Ce supplément dura une
demi-heure, avec fort peu de ménagements pour les expressions, et se
montra tout à fait hors d'œuvre. M. de Beauvilliers, assis derrière les
princes, l'entendit tout du long, et il essuya les regards indiscrets de
toute la cour présente. Le même jour, le fameux Bourdaloue et le P.
Gaillard firent retentir les chaires qu'ils remplissaient dans Paris des
mêmes plaintes et des mêmes instructions, et jusqu'au jésuite qui
prêchait à la paroisse de Versailles en fit autant.

La vérité est que le P. Bourdaloue, aussi droit en lui-même que pur dans
ses sermons, n'avait jamais pu goûter ce qu'alors on nommait quiétisme.
Car, que la doctrine de M. de Cambrai et de M\textsuperscript{me} Guyon,
pour la défense de laquelle il avait uniquement fait ses \emph{Maximes
des saints}, fût ou non quiétiste, ni en quel degré, ou point du tout,
c'est ce que je n'entreprends pas de décider\,; mais passant, bien ou
mal, pour telle, on lui en donnait aussi le nom, et à ceux qui lui
étaient attachés\,; et comme il faut des noms dans le langage pour
s'expliquer et pour s'entendre sans circonlocution, c'est aussi le terme
dont je me servirai avec le public pour me faire entendre, sans
prétendre qu'il ait une vraie ni une fausse application à la doctrine ou
aux gens dont il s'agit. Le P. Gaillard était encore plus loin de les
approuver\,; il était soupçonné, jusque dans sa compagnie, de n'en
porter que l'habit\,; il y a eu plus d'une fois besoin d'apologie, et il
n'y a dû son repos et les supériorités qu'il a eues, qu'à sa réputation
et au nombre d'amis illustres qu'elle lui avait faits, et encore à la
politique de la société, qui par une conduite opposée ne voulait pas
donner cette prise sur elle, en donnant force à l'opinion que le P.
Gaillard fût plus janséniste en effet que jésuite. Je dis et dirai dans
la suite janséniste et jansénisme, si l'occasion se présente de parler
de ceux qui sont réputés tels, par les mêmes raisons et avec la même
protestation que je viens d'écrire sur les quiétistes. Enfin le P. de La
Rue, jésuite de tous points, fut dirigé par ses supérieurs, et passa
toujours pour nager entre deux eaux, entre le gros de la société qui
appuyait les quiétistes, et quelques particuliers qui leur étaient
effectivement contraires. Cela fit même une espèce de scission entre
eux, dont, par politique, ils ne furent pas fâchés, mais qui embarrassa
étrangement le P. Valois et le P. de La Chaise, que l'habitude, l'amitié
et l'ancienne confiance du roi tirèrent plus d'affaire que son adresse,
et l'estime et l'affection que sa douceur, ses bons choix et toute sa
conduite lui avaient acquise, et qui avaient fait qu'il n'avait presque
point d'ennemis.

Dans ces circonstances, M. de Meaux publia son \emph{Instruction sur les
états d'oraison}, en deux volumes in-octavo, la présenta au roi, aux
principales personnes de la cour, et à ses amis. C'était un ouvrage en
partie dogmatique, en partie historique, de tout ce qui s'était passé
depuis la naissance de l'affaire jusqu'alors entre lui, M. de Paris et
M. de Chartres, d'une part\,; M. de Cambrai et M\textsuperscript{me}
Guyon, de l'autre. Cet historique très curieux, et où M. de Meaux laissa
voir et entendre tout ce qu'il ne voulut pas raconter, apprit des choses
infinies, et fit lire le dogmatique. Celui-ci, clair, net, concis,
appuyé de passages sans nombre et partout de l'Écriture et des Pères ou
des conciles, modeste, mais serré et pressant, parut un contraste du
barbare, de l'obscur, de l'ombragé, du nouveau et du ton décisif de vrai
et de faux des \emph{Maximes des saints\,;} {[}on le{]} dévora aussitôt
qu'il parut. L'un, comme inintelligible, ne fut lu que des maîtres en
Israël\,; l'autre, à la portée ordinaire, et secouru de la pointe de
l'historique, fut reçu avec avidité et dévoré de même. Il n'y eut homme
ni femme à la cour qui ne se fit un plaisir de le lire et qui ne se
piquait de l'avoir lu, de sorte qu'il fit longtemps toutes les
conversations de la cour et de la ville. Le roi en remercia publiquement
M. de Meaux. En même temps M. de Paris et M. de Chartres donnèrent
chacun une instruction fort théologique, en forme de mandement, à leur
diocèse, mais qui fut un volume, surtout celui de M. de Chartres, dont
la profondeur et la solidité l'emporta sur les deux autres, au jugement
des connaisseurs, et devint la pierre principale contre laquelle M. de
Cambrai se brisa.

Ces deux livres, si opposés en doctrine et en style, et si différemment
accueillis dans le monde, y causèrent un grand fracas. Le roi
s'interposa, et obligea M. de Cambrai à souffrir que le sien fût examiné
par les archevêques de Reims et de Paris, et par les évêques de Meaux,
Chartres, Toul, Soissons et Amiens, c'est-à-dire par ses adversaires ou
par des prélats qui leur adhéraient. Paris, Meaux et Chartres étaient
ses parties reconnues\,; Reims s'était joint à eux\,; Toul, qui a tant
fait parler de lui depuis, sous le nom de cardinal de Bissy, vivait avec
M. de Chartres comme avec un protecteur duquel il attendait sa
fortune\,; Soissons, frère de Puysieux, était un fat, mais avec de
l'esprit, du savoir, et plus d'ambition encore, qui lui avait fait
changer son évêché d'Avranches avec le savant Huet, pour être plus près
de Paris et de la cour, des volontés de laquelle il était esclave. Lui
et M. de La Rochefoucauld étaient enfants du frère et de la sœur, et
M\textsuperscript{me} de Sillery, sa mère, qui n'avait rien eu en
mariage, et dont les affaires étaient ruinées, vivait, depuis longues
années à Liancourt, chez M. de La Rochefoucauld. L'union était donc
grande entre eux, et M. de La Rochefoucauld, le plus envieux des hommes
ne pouvait souffrir les ducs de Chevreuse et de Beauvilliers, dont le
crédit et les places du dernier le désolaient, et dont la chute faisait
tous les désirs. Amiens, auparavant l'abbé de Brou et aumônier du roi,
était très savant, mais ami intime de M. de Meaux, et pensant comme lui
en tout genre de doctrine\,; c'était d'ailleurs un homme extrêmement
aimable, fort rompu au monde, goûté et recherché, mais un saint évêque,
tout appliqué à son étude et à son diocèse, dont il ne sortait que le
moins qu'il pouvait, et qui y donnait tout aux pauvres.

Je ne puis me passer de raconter ici un trait qui en deux mots le fera
connaître. Le scrupule le prit de son entrée dans l'épiscopat\,; et,
après y avoir bien réfléchi, il fut trouver le P. de La Chaise, à qui il
dit qu'il n'avait acheté une charge d'aumônier du roi que dans l'esprit
de se faire évêque\,; que c'était là une intrusion\,; qu'il lui
apportait sa démission pure et simple\,; qu'il ne demandait point
d'abbaye en quittant un évêché dans lequel il était mal entré, et qu'il
le priait de porter sa démission au roi et de lui faire nommer un
successeur. Le P. de La Chaise admira sa délicatesse, et refusa sa
démission. Ils disputèrent et se séparèrent ainsi. Quelques mois après,
M. d'Amiens lui rapporta sa démission, et voyant que ce serait avec le
même succès de la première fois, il lui déclara que, s'il ne voulait pas
s'en charger, lui-même l'allait porter au roi. Le P. de La Chaise,
voyant cette résolution si déterminée, prit sa démission, et lui promit
d'en rendre compte au roi. Il le fit en effet\,: la réponse fut prompte
et digne de tous les trois. Le confesseur dit au prélat que le roi avait
accepté sa démission, mais qu'en même temps il le nommait de nouveau
évêque d'Amiens, et lui commandait absolument d'accepter, et de cette
manière le scrupule cessa, et l'affaire fut finie\,; mais elle n'eut pas
une médiocre part au scrupule que le roi prit, à son tour, de la
vénalité des charges de ses aumôniers, et à l'attention qu'il a eue
depuis à l'éteindre.

Pour revenir d'où la parenthèse m'a distrait, M. de Cambrai souffrit
l'examen qu'il ne put éviter, et duquel il n'avait rien de bon à
attendre, pendant lequel M. de Metz mourut à Metz et qui fit vaquer un
cordon bleu et une place de conseiller d'État d'Église. M. de Metz était
frère aîné de M. de La Feuillade, leur aîné à tous deux ayant été tué à
la bataille de Lens en 1647 {[}20 août 1648{]}, seins alliance, attaché
à M. Gaston comme leur père tué au combat de Castelnaudary en 1632. M.
de Metz était un homme de beaucoup d'esprit, avec du savoir, qui avait
toujours fort été du grand monde. Il avait été un moment jésuite, à quoi
son génie vif et libre était fort peu propre. Lorsqu'en 1648 M. de
Lyonne fit nommer son père, qui était évêque de Gap, à l'archevêché
d'Embrun. Georges d'Aubusson, qui est notre M. de Metz, eut Gap, et
aussitôt après Embrun, sur le refus obstiné du père de M. de Lyonne, qui
était un saint évêque, et qui ne voulut point quitter son évêché.

M. d'Embrun brilla fort en diverses assemblées du clergé par sa capacité
et son éloquence. Il eut des abbayes et l'ambassade de Venise en 1659,
où il se soutint très dignement, avec sagesse mais fermeté, contre la
prétention du nonce Altoriti qui lui disputa l'Excellence et le rochet
découvert devant lui, parce qu'à la manière d'Italie il couvrait le sien
du mantelet. Il passa de là à l'ambassade d'Espagne en 1661, où il était
lors de l'insulte du baron de Batteville au comte depuis maréchal
d'Estrades, pour la préséance à Londres\,; et ce fut ce prélat qui fit à
Madrid toute la négociation par laquelle il fut arrêté que l'ambassadeur
d'Espagne déclarerait solennellement au roi que son maître lui cédait
partout la compétence, et qu'en aucun lieu les ambassadeurs d'Espagne ne
disputeraient le pas ni la préséance aux ambassadeurs de France\,: ce
que le marquis de La Fuentes vint exécuter à Paris comme ambassadeur
extraordinaire d'Espagne, en 1662. M. d'Embrun servit en cette occasion
avec une grande fermeté et dextérité. Pendant cette ambassade il eut
l'ordre du Saint-Esprit en la promotion de 1661. Il eut grande part à la
fortune de son frère qui lui déférait beaucoup. Il passa à Metz en 1668
avec tout ce qui lui fallut de Rome pour conserver le rang et les
honneurs d'archevêque. Le roi lui parlait toujours et plaisantait avec
lui\,; il mettait d'autres seigneurs en jeu, et cela faisait des
conversations souvent fort divertissantes. On l'attaquait fort sur son
avarice, il en riait le premier, et jamais le roi ne le put réduire à
porter un Saint-Esprit sur sa soutanelle comme les autres. Il disait que
celui du manteau suffisait, que la soutanelle était comme la soutane où
on n'en mettait point, et que la vanité avait mis cela à la mode. Les
autres lui répondaient qu'il n'en voulait point pour épargner deux écus
que cela coûtait sur chaque soutanelle\,; et c'était ainsi des prises
sur sa chère, sur son équipage, et sur tout, qu'il soutenait avec
beaucoup d'esprit, et se ruant à son tour en attaques fort plaisantes.
Il conserva un grand crédit et une grande considération jusqu'à sa mort,
et les ministres le ménageaient. Il était bon évêque, résidant et fort
appliqué à ses devoirs. Il avait quatre-vingt-cinq ans, et il y en avait
trois ou quatre qu'il était peu à peu tout à fait tombé en enfance\,: il
laissa un riche héritage à son neveu.

Cette mort arriva fort mal à propos pour M. de Cambrai. Il n'était plus
à portée de rien\,; mais il eut la douleur de voir donner l'ordre à M.
de Paris, et la place de conseiller d'État d'Église à M. de Meaux. Ce
dégoût fut suivi d'un autre. M\textsuperscript{me} de Maintenon chassa
de Saint-Cyr trois dames principales, dont une avait eu longtemps toute
sa faveur et sa confiance, et elle ne se cacha pas de dire qu'elle les
chassait à cause de leur entêtement pour M\textsuperscript{me} Guyon et
pour sa doctrine. Tout cela, avec l'examen de son livre dont il ne se
pouvait rien promettre de favorable, lui fit prendre le parti d'écrire
au pape, de porter son affaire devant lui, et de demander permission au
roi d'aller la soutenir à Rome\,; mais le roi lui défendit. M. de Meaux
là-dessus envoya son livre au pape, et M. de Cambrai eut la douleur de
recevoir une réponse sèche du pape, et de voir M. de Meaux triompher de
la sienne \footnote{Toutes trois se trouveront aux pages 1 et 2 des
  pièces (\emph{note de Saint-Simon}). Cette note, plusieurs fois
  reproduite dans le cours de ces Mémoires, prouve que Saint-Simon y
  avait joint de nombreuses pièces justificatives, elles n'ont pas été
  retrouvées. Il est probable qu'elles sont encore aux archives du
  ministère des affaires étrangères, où les manuscrits de Saint-Simon
  furent longtemps déposés.}. Rien de plus adroit, de plus insinuant, de
plus flatteur que la lettre de M. de Cambrai. L'art, la délicatesse,
l'esprit, le tour y brillaient, et, tout en ménageant certains termes
trop grossiers pour l'honneur de l'épiscopat et des maximes du royaume,
il y fit litière de l'un et de l'autre, sous prétexte de modestie et
d'humilité personnelles\,; elle ne laissa pas par cela même de faire
pour lui un bon effet dans le monde. En général on est envieux, et on
n'aime pas l'air d'oppression. Tout était déclaré contre lui\,: ses
parties, devenues ses juges par le renvoi de son livre à leur examen\,;
elles venaient de profiter des vacances de M. de Metz. On lui passa donc
les flatteries de sa lettre en faveur du tour et de la nécessité, et il
vit une lueur de retour du public.

Pour achever de suite ce qui s'en peut dire pour cette année, il ne
jouit pas longtemps de cette petite prospérité. Elle fit peur à ses
ennemis. Ils irritèrent le roi, qui, sans le vouloir voir, lui fit dire
de s'en aller sur-le-champ à Paris, et de là dans son diocèse, d'où il
n'est jamais sorti depuis. En envoyant cet ordre à M. de Cambrai, le roi
envoya chercher Mgr le duc de Bourgogne, avec lequel il fut longtemps
seul dans son cabinet, apparemment pour le déprendre de son précepteur
auquel il était fort attaché, et qu'il regretta avec une amertume que la
séparation de tant d'années n'a jamais pu affaiblir. M. de Cambrai ne
demeura que deux jours à Paris. En partant pour Cambrai, il laissa une
lettre à un de ses amis qu'on ne douta pas qu'il ne fût M. de Chevreuse
et qui incontinent après devint publique. Elle parut une espèce de
manifeste d'un homme qui, d'un langage beau, épanche sa bile et ne se
ménage plus, parce qu'il n'a plus rien à espérer. Le style haut et amer
en est d'ailleurs si plein d'esprit et à tout événement d'artifice,
qu'elle fit un extrême plaisir à lire, sans trouver d'approbateur, tant
il est vrai qu'un sage et dédaigneux silence est difficile à garder dans
les chutes.

La cour de Rome eut une extrême joie de se voir déférer cette cause à
juger en première instance par les premiers prélats d'un royaume
jusqu'alors si attachés à des maximes plus anciennes, et elle triompha
de les tenir en suppliants à ses pieds. Cette affaire y fit grand bruit.
Elle fut renvoyée à la même congrégation qui examinait un ouvrage
dogmatique du feu cardinal Sfondrat, abbé de Saint-Gall, qui avait été
déféré au saint-siège, qui, sur cette même matière et sur d'autres,
était, disait-on, fort étrange, mais que la pourpre de son auteur,
quoique mort, protégea. Il faut les laisser travailler à Rome, et y
mener le cardinal de Bouillon qui passa par Cluny, et y emporta la
coadjutorerie pour son neveu qu'il fit confirmer à Rome.

Avant de quitter les prélats, il ne faut pas oublier la mort de la
duchesse de Noailles, mère de l'archevêque de Paris. Le cardinal Mazarin
l'avait faite dame d'atours de la reine mère en 1657, qu'elle n'avait
que vingt-cinq ans, lorsque M\textsuperscript{me} d'Hautefort, dont j'ai
parlé, quitta cette charge pour épouser le maréchal-duc de Schomberg
\footnote{M\textsuperscript{me} d'Hautefort épousa le maréchal de
  Schomberg, le 6 septembre 1646, et non en 1657.}, dont elle fut la
seconde femme sans enfants. M. de Noailles ayant été fait duc et pair en
cette étrange fournée des quatorze, en 1663 \footnote{Voy. les notes à
  la fin du volume.}, sa femme, quoique devenue duchesse, n'osa
quitter\,; ce fut la première et l'unique dame d'atours duchesse, et la
demeura jusqu'à la mort de la reine mère, c'est-à-dire deux ans. C'était
une femme d'esprit, extrêmement bien avec le roi et la reine, d'une
vertu aimable, et toute sa vie dans la piété, quoique enfoncée dans la
cour et dans le plus grand monde. Elle s'appelait Boyer, et n'était
rien. Sa mère était Wignacourt, nièce et petite-nièce des deux grands
maîtres de Malte de ce nom. Les biens avaient fait le mariage de sa mère
qui n'avait rien, et le sien ensuite. Dès qu'elle fut veuve, elle se
retira peu à peu du monde, et bientôt après à Châlons auprès de son
fils, dont elle fit son directeur, et à qui tous les soirs de sa vie
elle se confessait avant de s'aller coucher. Elle l'avait suivi à Paris
et elle y mourut dans l'archevêché très saintement comme elle avait
vécu, et ce fut une grande douleur pour son fils l'archevêque. Elle
avait une sœur femme d'un marquis de Ligny, mère de la princesse de
Furstemberg, et une autre sœur femme de Tambonneau, président de la
chambre des comptes, et mère de Tambonneau qui eut la même charge, et
qui fut longtemps ambassadeur en Suisse. Cette M\textsuperscript{me}
Tambonneau était riche, bien logée et meublée, et avait trouvé le moyen
de voir chez elle la meilleure et la plus importante compagnie de la
cour et de la ville, sans donner à jouer ni à manger. Princes du sang,
grands seigneurs dans les premières charges, généraux d'armée, grandes
dames n'en bougeaient. La jeunesse en était bannie, et n'y était pas
admis qui voulait. Elle ne sortait presque point de chez elle, et s'y
faisait respecter comme une reine. Cela est si singulier que je l'ai
voulu rapporter.

Coulons à fond les prélats. M. de Troyes surprit beaucoup le monde par
sa belle et courageuse retraite. Il était fils de Chavigny, cet honnête
secrétaire d'État dont j'ai parlé, et petit-fils de Bouthillier,
surintendant des finances. Il eut des bénéfices de bonne heure, fut
aumônier du roi, devint, jeune, évêque de Troyes. Il avait du savoir et
possédait de plus les affaires temporelles du clergé mieux qu'aucun de
ce corps, en sorte qu'il était de presque toutes les assemblées du
clergé et qu'il brillait dans toutes. Il avait de plus bien de l'esprit,
et plus que tout l'esprit du monde, le badinage des femmes, le ton de la
bonne compagnie, et passa sa vie dans la meilleure et la plus distinguée
de la cour et de la ville, recherché de tout le monde, et surtout dans
le gros jeu et à travers toutes les dames. C'était leur favori\,; elles
ne l'appelaient que le Troyen, et chien d'évêque et chien de Troyen,
quand il leur gagnait leur argent. Il s'allait de temps en temps ennuyer
à Troyes, où, pour la bienséance et faute de mieux, il ne laissait pas
de faire ses fonctions\,; mais il n'y demeurait guère, et une fois de
retour, il ne se pouvait arracher.

C'est ainsi que jusqu'alors il avait passé sa vie. Cependant les
réflexions vinrent troubler ses plaisirs, puis ses amusements. Il essaya
de leur céder, il disputa avec elles, enfin l'expérience lui fit
comprendre qu'il serait toujours vaincu s'il ne rompait ses liens de
manière à ne les pouvoir renouer. Jamais il n'avait été plus gai ni de
meilleure compagnie qu'à un dîner à l'hôtel de Lorges avec M. de
Chaulnes et grand monde fort choisi, au sortir duquel il alla coucher à
Versailles, après s'être arrangé, quelques jours devant, avec le P. de
La Chaise. Le lendemain matin, au sortir du prie-Dieu, il demanda au roi
un moment d'audience\,; il l'eut dans le cabinet, avant la messe. Là il
fit sa confession avec ingénuité. Il avoua au roi le besoin qu'il avait
de retraite et de pénitence, et que jamais il n'en aurait la force tant
qu'il tiendrait au monde par quelques prétextes. Il présenta au roi la
démission de son évêché, et lui dit que, s'il le voulait combler, ce
serait de le donner à son neveu l'abbé de Chavigny qui avait de l'âge
assez et encore plus de mérite, de savoir et de vertu\,; qu'il
l'aiderait à gouverner dans ses commencements un diocèse qu'il
connaissait à fond\,; qu'il se retirerait dans sa propre maison à
Troyes\,; qu'il partagerait avec lui et qu'il y demeurerait en solitude
le reste de sa vie. L'évêché valait peu\,; le roi aimait M. de Troyes,
malgré la dissipation de sa vie\,; il lui accorda sur-le-champ sa
demande. Au sortir du cabinet, M. de Troyes gagna Paris, n'y vit
personne, et partit le lendemain pour Troyes, où il tint très exactement
tout ce qu'il s'était proposé, sans vouloir voir qui que ce soit que son
neveu et ses prêtres, encore pour affaires, et sans écrire ni avoir
aucun commerce avec personne, entièrement consacré à la prière et à la
pénitence, et à une entière solitude.

J'ai parlé plus haut de la querelle de M. de La Rochefoucauld et de M.
d'Orléans sur une place derrière le roi au sermon. J'en ai abrégé les
procédés\,: il faut dire que M. le Prince, M. le maréchal de Lorges ni
les autres amis communs n'ayant pu venir à bout de les réconcilier, le
prélat, après avoir fait un grand bruit inutile, s'en était allé à
Orléans bouder. À la fin il fallut bien revenir faire sa charge, et ses
amis et ses frères l'en pressaient depuis quelque temps, dans
l'espérance que ce retour opérerait un changement favorable dans son
affaire. Son arrivée renouvela le bruit et les plaintes. Il se jeta aux
pieds du roi avec peu de bienséance et moins de dignité, protestant
qu'il aimerait mieux être mort que voir dégrader sa charge, après
l'avoir exercée trente-quatre ans. M. de La Rochefoucauld supplia le roi
de trouver bon qu'il ne prit point la place qu'il lui avait accordée, et
qu'il avait ignoré être prétendue par le premier aumônier lorsqu'il
accepta, dont il préférait le retour de leur ancienne amitié à une place
dont il s'était bien passé toute sa vie. Le roi, qui n'aimait pas à
changer ses décisions, beaucoup moins à les voir blâmées, non seulement
tint ferme, mais il ajouta qu'après ce qu'il avait réglé, c'était son
affaire à lui, et non plus celle de M. de La Rochefoucauld\,; que le
premier aumônier n'avait point de place au sermon ni nulle part derrière
lui\,; qu'il se souvenait très bien d'avoir toujours vu M. de Meaux,
oncle et prédécesseur de M. d'Orléans, qui avait eu sa charge, ou debout
auprès de lui, ou assis sur le banc des aumôniers\,; et finit par ces
dures paroles, qui lui étaient si rares, « que si la chose était à
décider entre M. d'Orléans et un laquais, il donnerait la place au
laquais plutôt qu'à lui.\,» M. de La Rochefoucauld n'eut plus qu'à se
taire. M. d'Orléans entra dans le cabinet, à qui le roi parla tout aussi
durement\,; je l'en vis sortir l'air outré de douleur. Ni lui ni ses
parents ne la continrent pas, et il s'en retourna sur-le-champ à
Orléans, où il aurait mieux fait de demeurer que de venir presque à coup
sûr essuyer une mortification si amère pour une place qui ne lui avait
jamais appartenu, et devant connaître le roi assez pour ne pas douter
qu'après l'engagement qu'il avait pris de donner la place, c'était
s'exposer très inutilement que se hasarder à entreprendre de le faire
changer.

Mais, pour ne plus revenir à cette tracasserie, je dirai tout de suite
comment elle finit. Le roi au fond estimait et aimait M. d'Orléans, et
le montra bien par la façon si obligeante dont il lui donna sa
nomination au cardinalat, et par la considération qu'il lui avait
toujours constamment témoignée jusqu'à cette prétention de place au
sermon. Il était donc peiné du cuisant déplaisir qu'il lui avait fait,
et il l'était encore de l'irréconciliable division que cela avait mis
entre deux hommes si principaux, si anciennement amis, et si
continuellement autour de lui par leurs charges. La vacance du riche et
magnifique siège de Metz parut au roi un moyen d'apaiser M. d'Orléans,
et de finir la discorde. Il y nomma l'abbé de Coislin, sans que ni lui
ni aucun de sa famille eût osé y songer. La surprise fût extrême\,; ils
se croyaient tous bien éloignés des grâces, et l'abbé de Coislin encore
plus éloigné d'aucun évêché.

C'était un petit homme court et gros, singulier au dernier point, d'une
figure comique et de propos à l'avenant, et souvent fort indiscrets,
mêlé pourtant avec la meilleure compagnie de la cour, qu'il divertissait
en se divertissant le premier\,; avec cela dangereux et malin, et un
fort médiocre prêtre. Il se l'était fait par raison, malgré son père,
qui était pauvre et qui, voyant son aîné sans enfants, voulait marier
celui-ci. L'aîné était impuissant, celui-ci en était fort soupçonné, et
n'avait point de barbe\,; son aîné était gueux, il ne voulut pas mourir
de faim toute sa vie, et se tourna du côté des bénéfices. Dès qu'il fut
prêtre, M. d'Orléans, sans en dire mot à son frère, pour lui éviter le
chagrin d'un refus s'il en recevait un, demanda au roi sa survivance de
premier aumônier, et l'obtint sur-le-champ. Avec cet établissement, le
jeune homme ne douta plus de rien, et se livra au grand monde et à son
humeur. Le roi ne le goûta jamais, et ne le souffrit qu'à cause de son
oncle. Il eut beau le suivre à Orléans pour y travailler sous lui, cela
ne lui produisit qu'une légère abbaye, et il n'avait que celle-là seule
et point d'autre bien, lorsqu'il eut Metz. En même temps, pour finir
toute dispute, le roi donna à la charge de premier aumônier une place
derrière lui à la chapelle, au-dessous de celle de M. de La
Rochefoucauld, et la joignant. M. de Metz ne fut pas alors en termes de
la refuser, comme avait fait son oncle, à qui elle avait été offerte. M.
d'Orléans, qui allait être cardinal, et qui par là s'allait trouver hors
d'intérêts pour sa personne, et dans la joie de ce retour du roi qui
plaçait si grandement à Metz son neveu, pour lequel il n'espérait
presque plus rien, se prêta à y consentir et à se réconcilier avec M. de
La Rochefoucauld. Le roi en fut ravi\,; et tout se passa de part et
d'autre de si bonne grâce, que tout fut sincèrement oublié et qu'ils
redevinrent amis comme auparavant.

Je perdis environ dans ce temps-là le chevalier de La Hillière,
gouverneur de Rocroy. C'était un ancien ami intime de mon père, et un
des braves et des galants hommes de France, qui avait été dans la
confiance de M. Le Tellier et de beaucoup de gens très distingués de son
temps, et dans toute celle de Mademoiselle du temps de M. de Lauzun et
d'elle. Le roi le considérait, et il y avait toujours des choses
curieuses à apprendre de lui de l'ancienne cour\,; avec cela de fort
bonne et sûre compagnie.

Le roi chassa fort précipitamment toute la troupe des comédiens
italiens, et n'en voulut plus d'autre. Tant qu'ils n'avaient fait que se
déborder en ordures sur leur théâtre, et quelquefois en impiétés, on
n'avait fait qu'en rire\,; mais ils s'avisèrent de jouer une pièce qui
s'appelait \emph{la Fausse Prude}, où M\textsuperscript{me} de Maintenon
fut aisément reconnue. Tout le monde y courut, mais après trois ou
quatre représentations, qu'ils donnèrent de suite, parce que le gain les
y engagea, ils eurent ordre de fermer leur théâtre, et de vider le
royaume en un mois. Cela fit grand bruit, et si ces comédiens y
perdirent leur établissement par leur hardiesse et leur folie, celle qui
les fit chasser n'y gagna pas, par la licence avec laquelle ce ridicule
événement donna lieu d'en parler.

\hypertarget{note-i.-table-de-marbre.}{%
\chapter{NOTE I. TABLE DE MARBRE.}\label{note-i.-table-de-marbre.}}

La table de marbre, dont parle Saint-Simon, était placée dans la galerie
des prisonniers du Palais de Justice de Paris, près de la chambre de la
Tournelle. Le tribunal du grand amiral, qui se composait d'un lieutenant
civil et criminel, d'un lieutenant particulier, de cinq conseillers,
d'un procureur du roi et de plusieurs substituts, y avait son principal
siège. Voilà pourquoi le comte de Toulouse est installé, en qualité de
grand amiral, à la table de marbre. Il y avait encore d'autres
juridictions qui avaient primitivement leur siège à la table de marbre,
entre autres la connétable ou juridiction des maréchaux de France. Ce
tribunal, où siégeaient rarement les maréchaux, comprenait un lieutenant
général, un lieutenant particulier, un procureur du roi et plusieurs
autres officiers. Il connaissait principalement des actions personnelles
intentées aux gens de guerre, des contrats faits entre eux, des procès
relatifs à leur solde, des malversations commises par les trésoriers et
payeurs des compagnies, etc. Dans l'origine, le grand maître des eaux et
forêts tenait aussi sa juridiction à la table de marbre du Palais.

\hypertarget{note-ii.-conseils-du-roi.}{%
\chapter{NOTE II. CONSEILS DU ROI.}\label{note-ii.-conseils-du-roi.}}

Saint-Simon parle souvent du \emph{conseil des parties}, du
\emph{conseil des dépêches}, du \emph{conseil des finances}, de la
\emph{grande et petite direction}, et en général des \emph{conseils du
roi}. Comme le lecteur moderne n'est pas toujours familiarisé avec ces
termes, il ne sera pas inutile d'en préciser le sens.

Le \emph{conseil des parties} était présidé par le chancelier, et se
composait de conseillers d'État et de maîtres des requêtes, qui
faisaient le rapport des affaires. On y traitait des règlements de
juges, en cas de conflit ou de récusation des juges\,; des évocations ou
actes de l'autorité souveraine qui enlevaient la connaissance d'un
procès aux tribunaux ordinaires, pour l'attribuer à d'autres
juridictions\,; de la cassation d'arrêts contraires aux ordonnances,
etc. Le \emph{conseil des parties} était donc un tribunal suprême
souvent en lutte avec les parlements.

Le \emph{conseil des dépêches}, composé, comme le précédent, de
conseillers d'État et de maîtres des requêtes, était souvent présidé par
le roi. On y traitait toutes les questions relatives à l'administration
des provinces. « Audit conseil, dit l'ordonnance du 18 janvier 1630,
seront lues toutes les dépêches du dedans du royaume, et délibéré les
réponses sur icelles. Seront aussi lues les réponses et les instructions
qui seront baillées à ceux qui seront employés dans les provinces pour
les affaires de Sa Majesté. Audit conseil, tous ceux qui auront été en
commission pour le service de Sa Majesté seront tenus rendre compte de
ce qu'ils auront fait, négocié et géré en leurs voyages. Il sera traité
audit conseil de l'état des garnisons, état et payement des gens de
guerre, tant de cheval que de pied, et autres affaires de la guerre, et
généralement de toutes les affaires importantes, ainsi qu'il plaira à Sa
Majesté l'ordonner. Et afin que ce qui aura été résolu audit conseil
soit promptement et précisément exécuté, Sa Majesté ordonne que toutes
résolutions qui se prendront audit conseil en chaque journée seront
réduites par écrit par celui des secrétaires d'État qui sera en mois.\,»
Dans l'origine, les quatre secrétaires d'État assistaient pendant un
mois, à tour de rôle, aux délibérations de ce conseil, et en
transmettaient les décisions sous forme de dépêches aux intendants
chargés de les exécuter. De là venait le nom de \emph{conseil des
dépêches}. Les attributions de ce conseil répondaient en partie à celles
du ministère actuel de l'intérieur.

Le \emph{conseil des finances} se composait, comme les précédents, de
conseillers d'État, de maîtres des requêtes, et en outre des intendants
et contrôleurs des finances. On distinguait deux espèces de conseils des
finances\,: l'un, appelé \emph{conseil de grande direction}, était
présidé par le chancelier, et jugeait le contentieux financier, les
différends entre les fermiers des impôts et les particuliers, les procès
pour remboursements d'offices, adjudication des travaux publics, et
fourniture de vivres et munitions, rachat des rentes, etc. L'autre,
appelé conseil de \emph{petite direction}, était présidé par un ministre
d'État désigné par le roi, et s'occupait des impôts, de leur répartition
et de l'administration financière. C'était le duc de Beauvilliers qui, à
l'époque de Saint-Simon, présidait le conseil des finances.

\hypertarget{note-iii.-lettre-de-louis-xiv-uxe0-madame-de-maintenon-uxe0-loccasion-de-larrivuxe9e-de-la-duchesse-de-bourgogne.}{%
\chapter{NOTE III. LETTRE DE LOUIS XIV À MADAME DE MAINTENON À
L'OCCASION DE L'ARRIVÉE DE LA DUCHESSE DE
BOURGOGNE.}\label{note-iii.-lettre-de-louis-xiv-uxe0-madame-de-maintenon-uxe0-loccasion-de-larrivuxe9e-de-la-duchesse-de-bourgogne.}}

Saint-Simon dit que le roi se hâta d'envoyer un courrier à
M\textsuperscript{me} de Maintenon, pour lui mander sa joie et les
louanges de la princesse. Voici la lettre même du roi \footnote{L'original
  autographe de cette lettre se trouve à la bibliothèque du Louvre, F
  328, f° 2 et suiv. Elle a été imprimée par M. Monmerqué dans un
  recueil tiré à un petit nombre d'exemplaires pour les Bibliophiles
  français, sous ce titre\,: \emph{Lettres de Louis XIV, etc., adressées
  à M\textsuperscript{me} la marquise de Maintenon} (1 vol.~in-8, Didot,
  1822).}\,:

« Montargis, ce dimanche au soir à six heures et demie,

4 novembre 1696.

« Je suis arrivé ici devant cinq heures\,; la princesse n'est venue qu'à
près de six. Je l'ai été recevoir au carrosse. Elle m'a laissé parler le
premier, et après m'a fort bien répondu, mais avec un petit embarras qui
vous aurait plu. Je l'ai menée dans sa chambre au travers de la foule,
la faisant voir de temps en temps en approchant les flambeaux de son
visage. Elle a soutenu cette marche et ces lumières avec grâce et
modestie. Nous sommes enfin arrivés dans sa chambre, où il y avait une
foule et une chaleur qui faisaient crever \footnote{M. Monmerqué a
  mis\,: \emph{où il y avait une foule et une chaleur à faire crever}.
  J'ai suivi le manuscrit autographe.}. Je l'ai montrée de temps en
temps à ceux qui s'approchaient, et je l'ai considérée de toutes
manières, pour vous mander ce qu'il m'en semble.

« Elle a la meilleure grâce et la plus belle taille que j'aie jamais
vue, habillée à peindre, et coiffée de même, des yeux vifs et très
beaux, des paupières noires et admirables\,; le teint fort uni, blanc et
rouge, comme on peut le désirer\,; les plus beaux cheveux blonds que
l'on puisse voir, et en grande quantité. Elle est maigre, comme il
convient à son âge\,; la couche fort vermeille, les lèvres grosses, les
dents blanches, longues et très mal rangées, les mains bien faites, mais
de la couleur de son âge. Elle parle peu, au moins à ce que j'ai vu,
n'est point embarrassée qu'on la retarde, comme une personne qui a vu du
monde. Elle fait mal la révérence et d'un air un peu italien\,; elle a
quelque chose d'une Italienne dans le visage, mais elle plaît, et je
l'ai vu dans les yeux de tout le monde. Pour moi, j'en suis tout à fait
content.

« Elle ressemble fort à son premier portrait, et point du tout à
l'autre. Pour vous parles comme je fais toujours, je la trouve à
souhait, et serais fâché qu'elle fût plus belle. Je le dirai encore\,:
tout plaît hormis la révérence.

« Je vous en dirai davantage après souper\,; car je remarquerai bien des
choses que je n'ai pas pu voir encore. J'oubliais de vous dire qu'elle
est plutôt plus petite que grande pour son âge. Jusqu'à cette heure j'ai
fait merveille\,; j'espère que je soutiendrai un certain air aisé que
j'ai pris jusqu'à Fontainebleau, où j'ai grande envie de me retrouver.

« A dix heures.

« Plus je vois la princesse, plus je suis satisfait. Nous avons été dans
une conversation publique où elle n'a rien dit\,; c'est tout dire. Je
l'ai vu déshabiller\,; elle a la taille très belle, on peut dire
parfaite, et une modestie qui vous plaira. Tout s'est bien passé à
l'égard de mon frère. Il est fort chagrin\,; il dit qu'il est malade.
Nous partirons demain à dix heures et demie ou onze heures\,; nous
arriverons à cinq heures au plus tard.

« Je suis tout à fait content. {[}Rien{]} que de bien à propos en
répondant aux questions qu'on lui faisait. Elle a peu parlé, et la
duchesse du Lude m'a dit qu'elle l'avait avertie que le premier jour
elle ferait bien d'avoir une grande retenue. Nous avons soupé\,; elle
n'a manqué à rien, et est d'une politesse surprenante à toutes choses\,;
mais à moi et à mon fils, elle n'a manqué à rien, et s'est conduite
comme vous pourriez faire. J'espère que vous la serez aussi. Elle a été
regardée et observée, et tout le monde paraît satisfait de bonne foi.
L'air est noble et les manières polies et agréables. J'ai plaisir à vous
en dire du bien\,; car je trouve que, sans préoccupation et sans
flatterie, je le peux faire, et que tout m'y oblige. Ne voulant dire
tout ce que je pense, je vous donne mille bons\ldots. (deux lignes
effacées) \footnote{Ces deux lignes sont tellement biffées qu'il est
  impossible d'en apercevoir un seul mot. Il est à présumer qu'elles
  renfermaient les expressions d'une tendresse conjugale.
  M\textsuperscript{me} de Maintenon, en conservant cette lettre à cause
  de son importance historique, en a fait disparaître ce qui aurait pu
  être un indice de son union avec Louis XIV. (\emph{Note de M.
  Monmerqué}.)}.

« J'oubliais à vous dire que je l'ai vue jouer aux onchets avec une
adresse charmante. Quand il faudra un jour qu'elle représente, elle sera
d'un air et d'une grâce à charmer, et avec une grande dignité et un
grand sérieux.\,»

\hypertarget{note-iv.-ruxe9ception-des-ducs-et-pairs-au-parlement.}{%
\chapter{NOTE IV. RÉCEPTION DES DUCS ET PAIRS AU
PARLEMENT.}\label{note-iv.-ruxe9ception-des-ducs-et-pairs-au-parlement.}}

Saint-Simon parle dans ses Mémoires de ce qu'il appelle la \emph{fournée
des ducs et pairs de 1663}. Il ne sera pas inutile d'en donner ici un
récit, qui a d'autant plus d'intérêt qu'il vient d'un témoin oculaire et
qu'il est inédit. On y trouve d'ailleurs des détails importants pour
comprendre plusieurs passages des Mémoires de Saint-Simon, où il est si
souvent question des ducs et pairs et de leurs prérogatives. Voici
comment Olivier d'Ormesson retrace, dans son Journal, la réception des
ducs et pairs en 1663\,:

« Le samedi 15 décembre 1663, je fus au parlement pour voir ce qui s'y
passerait sur la réception des nouveaux ducs\,: j'y entrai facilement,
comme officier du parlement, et pris place avec des conseillers en la
place où les gens du roi se mettent aux assemblées particulières. M. le
chancelier \footnote{Pierre Séguier, chancelier de France depuis 1635,
  mort en 1672.} y était sur le banc des présidents, en robe ordinaire
de velours noir, comme tout le parlement était aussi en robes noires,
cette séance du roi n'étant point lit de justice, mais séance
particulière où le roi se trouve. Les présidents qui s'y trouvèrent
furent MM. le premier président \footnote{Guillaume de Lamoignon,
  premier président du parlement depuis 1658, mort en 1677.}, de
Nesmond, de Longueil, de Novion, de Mesmes, Le Coigneux, Champlâtreux.
Dans le parquet, sur le banc des ducs, se mirent MM. de Bonnelles, de
Bellièvre, d'Aligre, Morangis. Les maîtres des requêtes honoraires et
titulaires, à l'ordinaire. La place des ducs était sur les bancs hauts
de l'audience, mais ils ne s'y mirent que lorsque le roi arriva. Les
présidents des enquêtes n'ayant point de places, il fut mis deux bancs
dans le parquet à droite et à gauche, où ils se mirent avec quelques-uns
de la grand'chambre.

« La nouvelle étant venue par M. de Sainctot \footnote{Saint-Simon parle
  de ce maître des cérémonies dans plusieurs passages de ses Mémoires.}
que le roi était à la Sainte-Chapelle, où il entendait la messe, M. de
Nesmond et les trois présidents suivants, avec sic conseillers, furent
au-devant, à l'ordinaire. Incontinent après le roi arriva, M. le Duc, M.
le Prince et Monsieur marchant devant lui, sans le bruit des tambours ni
des trompettes, ayant seulement un capitaine des gardes, qui était M. de
Noailles, qui servait pour M. de Villequier malade. Le roi prit sa place
ordinaire avec les ducs et les carreaux accoutumés, n'ayant aucun de ses
officiers auprès de sa personne, ni capitaine des gardes, ni chambellan,
ni autres. M. le duc d'Orléans, M. le Prince et M. le duc d'Enghien en
leurs places à droite\,; au-dessous d'eux, MM. les ducs de Guise,
d'Uzès, de Beaufort, de Luynes, de Lesdiguières, de Richelieu, de Retz.
À la gauche étaient MM. les ducs de Laon et de Langres et comte de
Noyon, pairs ecclésiastiques. Le roi était vêtu de noir avec des plumes
sur son chapeau et garniture jaune, tous les autres seigneurs vêtus de
noir.

Chacun étant assis et couvert, le roi dit qu'il était venu pour faire
recevoir les nouveaux ducs. Après, M. le chancelier partit de sa place
pour aller recevoir l'ordre du roi. Étant revenu non point dans celle de
l'encoignure, comme lorsque c'est lit de justice, mais sur le banc des
présidents, il lut, en son particulier, un papier où étaient écrits les
noms des ducs à recevoir, selon l'ordre que le roi leur avait donné,
dont personne n'avait connaissance. Il demanda qui avait les lettres de
M. de Verneuil \footnote{Henri de Bourbon, duc de Verneuil, était fils
  naturel d'Henri IV. Il avait épousé Charlotte Séguier, fille du
  chancelier. Il est question plusieurs fois de cette duchesse de
  Verneuil dans Saint-Simon.} . M. Perrot La Malmaison, qui en était
rapporteur, prit la parole, et en fit la lecture nu-tête\,; tout le
préambule en fut supprimé, et on lut le dispositif. Arès, M. le
chancelier dit que le roi ordonnait le \emph{soit montré} et le
\emph{soit informé \footnote{Formule par laquelle on renvoyait une pièce
  à l'examen des gens du roi, qui remplissaient les fonctions du
  ministère public.}}, sans prendre l'avis de personne.

« Perrot étant passé au greffe, M. le chancelier fit lire ensuite celles
(les lettres) de M. le maréchal d'Estrées, puis de M. le maréchal de
Grammont, et ainsi de plusieurs autres, jusques à ce que M. Perrot fût
revenu. Alors lecture ayant été faite, par M. Perrot étant couvert,
{[}des dépositions{]} du deux témoins et des conclusions, M. le
chancelier lui demanda son avis. Il dit six lignes fort bien en faveur
de M. de Verneuil, et fut d'avis des conclusions \footnote{Les
  conclusions étaient rédigées par les gens du roi.}. M. le chancelier
demanda ensuite l'avis à tous les conseillers de la grand'chambre et des
enquêtes, suivant l'ordre ordinaire, puis aux ducs laïques et aux pairs
ecclésiastiques, sans ôter son bonnet \footnote{L'omission de cette
  formalité fut une des causes de la haine de Saint-Simon contre le
  parlement.}, puis au président, ôtant soit bonnet\,; ensuite il monta
au roi, auprès duquel se joignirent M. le duc d'Enghien, M. le Prince et
M. le duc d'Orléans, pour donner leur avis\,; et puis étant revenu dans
sa place, et ayant dit qu'on fit entre M. de Verneuil, et lui s'étant
présenté sans épée à la place des récipiendaires, il prononça\,:
\emph{Le roi tenant son parlement a ordonné et ordonne que vous serez
reçu en la charge et dignité de duc et pair de France, en prêtant par
vous le serment en tel cas requis et accoutumé. Levez la main. Vous
jurez et promettez de bien et fidèlement servir le roi, lui donner avis
dans ses plus importantes affaires, et séant au parlement rendre la
justice au pauvre comme au riche, tenir les délibérations de la cour
closes et secrètes, et vous comporter comme un ligne, vertueux et
magnanime duc et pair, officier de la couronne et conseiller en cour
souveraine doit faire. Ainsi vous le jurez et le promettez}.

« M. de Verneuil ayant répondu oui, M. le chancelier dit\,: \emph{Le roi
vous ordonne de prendre votre épée}. L'huissier, qui la portait auprès
de lui, l'ayant remise dans le baudrier, M. de Verneuil alla prendre sa
place sur le banc, et à la suite des anciens ducs.

« Cette même formalité fut observée pour chacun des autres ducs. Ils
furent reçus suivant l'ordre qui suit\,: M. de Verneuil, le premier, duc
de Verneuil\,; M. le maréchal d'Estrées, duc de Cœuvres\,; M. le
maréchal de Grammont, duc de Grammont\,; M. de La Meilleraye, duc de La
Meilleraye\,; M. de Mazarin, duc de Rethel (pairie mâle et femelle)\,;
M. de Villeroy, duc de Villeroy\,; M. de Mortemart, duc de Mortemart\,;
M. de Créqui, duc de Poix\,; M. de Saint-Aignan, duc de Saint-Aignan\,;
M. de Foix, duc de Randan (pairie mâle et femelle, à cause de
M\textsuperscript{me}s de Senecey \footnote{Marie-Catherine de La
  Rochefoucauld, mariée le 7 août 1607 à Henri de Beaufremont, marquis
  de Senecey. Elle était première dame d'honneur d'Anne d'Autriche, et
  avait été gouvernante de Louis XIV. La comtesse de Fleix, sa fille,
  était mère de Gaston de Foix, qui fut nommé duc de Randan, en 1663.}
et de Fleix)\,; M. de Liancourt, duc de La Rocheguyon\,; M. de Tresmes,
duc de Tresmes\,; M. de Noailles, duc d'Ayen\,; M. de Coislin, duc de
Cambout.

« M. de Noailles faisait ce jour la charge de capitaine des gardes en
place de M. de Villequier malade. Voyant que, pour prêter le serment, il
était de l'ordre d'ôter l'épée, il fit demander au roi, par M. le
chancelier, si lui, faisant fonction de capitaine des gardes du corps,
il devait la quitter et son bâton \footnote{Le capitaine des gardes du
  corps en quartier portait un bâton de commandement comme signe de sa
  dignité.}. Le roi répondit qu'il la devait ôter, et il l'ôta comme les
autres.

« Au sortir, le roi parla quelque temps à M. le premier président, et,
au sortir du parquet, il fit appeler à M. le procureur général
\footnote{Le procureur général était alors Achille de Harlay, qui devint
  plus tard premier président. Saint-Simon en parle souvent, surtout à
  l'occasion du procès de préséance.}, auquel il parla.\,»

\end{document}
